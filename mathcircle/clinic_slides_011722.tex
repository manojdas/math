% !TeX TXS-program:compile = txs:///pdflatex/[--shell-escape]

\documentclass[9pt]{beamer}
\usetheme{Madrid}
\usecolortheme{beaver}
\usepackage{amsmath,amssymb,amsthm,asymptote,graphicx}
\usepackage{graphics}
% \usepackage{bisvslides}

\newcounter{problem}[section]

\newenvironment{probslide}[3][]{%
    \refstepcounter{problem}\begin{frame}[#1]%
	{Problem \theproblem 
        \def\temp{#2}\ifx\temp\empty
            %
        \else
            \ - \temp%
        \fi}
    {#3}}%
	{\end{frame}}

% \newenvironment{Example}[2][Example]
%     {This is an #1. You gave #2 as an argument. The rest will be bold: \bfseries}
%     {}
% \textbf{Problem~\theproblem. #1
% \newenvironment{bsmi}{\begin{CJK}{UTF8}{bsmi}}{\end{CJK}}

\title{Math Clinic}
\subtitle{Challenge Problems from Dr. Sega Handouts}
\author{Rohan Das}
\institute{Upper School Math Circle}
\date{January 17, 2022}

%\maketitle
%~~~~~~~~~~~~~~~~~~~~~~~~~~~~~~~~~~~~~~~~~~~~~~~~~~~~~~~~~~~~~~~~~~~~~~~~~~~~~~
% Informations
%\title{TEMPLATE}

%\titlegraphic{assets/gkg.png} %change this to your preferred logo or image(the image is located on the top right corner).
%~~~~~~~~~~~~~~~~~~~~~~~~~~~~~~~~~~~~~~~~~~~~~~~~~~~~~~~~~~~~~~~~~~~~~~~~~~~~~~

\begin{document}

% Generate title page
\begin{frame}
    \titlepage        
\end{frame}
% \setbeamertemplate{footline}[miniframes Madrid]


\begin{probslide}[t]{Handout 1, \#39}{2009 Japanese Math Olympiad, Preliminary Round, \#10}
    \begin{block}{}
    Evaluate:
    \[\frac{\sqrt{10+\sqrt{1}} + \sqrt{10+\sqrt{2}} + \cdots + \sqrt{10+\sqrt{99}}}{\sqrt{10-\sqrt{1}} + \sqrt{10-\sqrt{2}} + \cdots + \sqrt{10-\sqrt{99}}}.\]

    % Answer: $\sqrt{2} + 1$
    \end{block}
\end{probslide}

\begin{probslide}[t]{Handout 1, \#40}{1961 K\"{u}rsch\'{a}k Math Competition, Hungary, \#2}
    \begin{block}{}
    Prove that if $a$, $b$, and $c$ are positive real numbers, each less than 1, then the products $(1 - a)b$, $(1 - b)c$, and $(1 - c)a$ cannot all be greater than $\frac{1}{4}$.
    \end{block}
\end{probslide}

\begin{probslide}[t]{Handout 2, \#40}{2013 USAJMO, \#1}
    \begin{block}{}
    Are there integers $a$ and $b$ such that $a^5b + 3$ and $ab^5 + 3$ are both perfect cubes of integers?
    \end{block}
\end{probslide}


\begin{probslide}[t]{Handout 4, \#36}{2015 Harvard-MIT Math Tournament, Combinatorics, \#3}
    \begin{block}{}
    Starting with the number 0, Casey performs an
    infinite sequence of moves as follows: he chooses a number from $\{1,2\}$ at random (each with probability $\frac{1}{2}$) and adds it to the current number. Let $p_m$ be the probability that Casey ever reaches the number $m$. Find $p_{20} - p_{15}$.
    % Answer: $\frac{11}{2^{20}}$
    
    \end{block}
\end{probslide}

\begin{probslide}[t]{Handout 4, \#38}{}
    \begin{block}{}
    For $n \ge 2$, consider the numbers
    \begin{align*}
        a_n&= \binom{n}{0} + \binom{n}{3} + \binom{n}{6} + \binom{n}{9} +\ldots, \\
        b_n&= \binom{n}{1} + \binom{n}{4} + \binom{n}{7} + \binom{n}{10} +\ldots, \\
        c_n&= \binom{n}{2} + \binom{n}{5} + \binom{n}{8} + \binom{n}{11} +\ldots
    \end{align*}
    Prove that the following identity holds:
    \[a^3_n +b^3_n +c^3_n - 3a_nb_nc_n =2^n.\]
    \end{block}
\end{probslide}

\begin{probslide}[t]{Handout 5, \#38}{}
    \begin{block}{}
    Suppose that $f(x)$ is a polynomial of degree 3 with leading coefficient equal to 2, $f(2014) = 2015$, and
$f (2015) = 2016$. Find the value of $f (2016) - f (2013)$. 
    % Answer: 15
    \end{block}
\end{probslide}

\begin{probslide}[t]{Handout 5, \#40}{}
    \begin{block}{}
    Let $a \ne 0$ and let the polynomial $P(x) = ax^4 +bx^3 +cx^2 -2bx+4a$ have two real roots $x_1$ and $x_2$ such that $x_1x_2 = 1$. Prove that $2b^2 + ac = 5a^2$.
    \end{block}
\end{probslide}

\begin{probslide}[t]{Handout 9, \#40}{2014 MIT Math Prize for Girls, \#20}
    \begin{block}{}
    How many complex numbers $z$ such that $|z| < 30$ satisfy the equation
    \[e^z = \frac{z-1}{z+1}?\]
% Answer: 10
    \end{block}
\end{probslide}

\end{document}
