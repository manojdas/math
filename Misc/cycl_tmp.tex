\documentclass[11pt,twoside]{scrartcl}
\usepackage{mdas}
% \usepackage[sexy, fancy, hints]{evan}


\begin{document}
\title{Cyclotomic}
% If you contribute to the handout, put your name in comment here

\author{Manoj Das}
\date{\today}
\section{1}
\begin{center}
\begin{tabular}{c|*{11}c}
     & $2^0$ & $2^1$ & $2^2$ & $2^3$ & $2^4$ & $2^5$ & $2^6$ & $2^7$ & $2^8$ & $2^9$ & $2^{10}$ \\ \hline
     1  & \textbf{1} & 2 & 4 & 8 & 16 & 32 & 64 & 39 & \textbf{78} & 67 & 45 \\
     3  & \textbf{3} & 6 & 12 & 24 & 48 & \textbf{7} & 14 & 28 & 56 & 23 & 46 \\
     5  & \textbf{5} & 10 & 20 & 40 & \textbf{80} & 71 & 53 & 17 & 34 & 68 & 47 \\
     9  & \textbf{9} & 18 & 36 & 72 & 55 & 21 & 42 & \textbf{84} & 79 & 69 & 49 \\
     11 & \textbf{11} & 22 & 44 & \textbf{88} & 87 & 85 & 81 & 73 & 57 & 25 & 50 \\
     13 & \textbf{13} & 26 & 52 & \textbf{15} & 30 & 60 & 31 & 62 & 35 & 70 & 51 \\
     19 & 19 & 38 & \textbf{76} & 63 & 37 & \textbf{74} & 59 & 29 & 58 & 27 & 54 \\
     33 & 33 & 66 & 43 & \textbf{86} & 83 & 77 & 65 & 41 & \textbf{82} & 75 & 61
\end{tabular}
    
\end{center}

\begin{center}
\begin{tabular}{c|c|c}
    $i$ & $i \mod{89}$ & coset($i$) \\ \hline
    -15 & 74 & 19 \\
    -13 & 76 & 19 \\
    -11 & 78 & 1 \\
    -9 & 80 & 5 \\
    -7 & 82 & 33 \\
    -5 & 84 & 9 \\
    -3 & 86 & 33 \\
    -1 & 88 & 11 \\
    1 & 1 & 1 \\
    3 & 3 & 3 \\
    5 & 5 & 5 \\
    7 & 7 & 3 \\
    9 & 9 & 9 \\
    11 & 11 & 11 \\
    13 & 13 & 13 \\
    15 & 15 & 13
\end{tabular}
    
\end{center}

\section*{Euclid 21}

The condition is equivalent to $a^n+a^2-1$
dividing $a^m+a-1$ as polynomials.
The big step is the following analytic one.

\begin{claim*}
	We must have $m \le 2n$.
\end{claim*}
\begin{proof}
	Assume on contrary $m > 2n$
	and let $0 < r < 1$ be the unique real number
	with $r^n+r^2 = 1$, hence $r^m+r = 1$.
	But now
	\begin{align*}
		0 &= r^m + r - 1 < r(r^n)^2 + r - 1 = r\left( (1-r^2)^2+1 \right) - 1 \\
		&= -(1-r)\left( r^4+r^3-r^2-r+1 \right).
	\end{align*}
	As $1-r > 0$ and $r^4+r^3-r^2-r+1 > 0$, this is a contradiction
\end{proof}

Now for the algebraic part.
Obviously $m > n$.
\begin{align*}
	a^n+a^2-1 &\mid a^m+a-1 \\
	\iff a^n+a^2-1 &\mid (a^m+a-1)(a+1) = a^m(a+1) + (a^2-1)  \\
	\iff a^n+a^2-1 &\mid a^m(a+1) - a^n \\
	\iff a^n+a^2-1 &\mid a^{m-n}(a+1) - 1.
\end{align*}
The right-hand side has degree $m-n+1 \le n+1$,
and the leading coefficients are both $+1$.
So the only possible situations are
\begin{align*}
	a^{m-n}(a+1) - 1 &= (a+1)\left( a^n+a^2-1 \right) \\
	a^{m-n}(a+1) + 1 &=  a^n+a^2-1.
\end{align*}
The former fails by just taking $a=-1$;
the latter implies $(m,n) = (5,3)$.
As our work was reversible, this also implies $(m,n) = (5,3)$ works, done.

\section*{Entry Combo}
\subsection{n-gon}
\begin{center}
    \begin{asy}
        unitsize (5cm);

        pair[] P;

        for (int i = 0; i < 11; ++i) {
            P[i] = dir(i*360/11);

            dot ("$P_{" + (string)i + "}$", P[i], dir(P[i]));
            if (i > 0) draw(P[i-1]--P[i]);
        }
        draw (P[0]--P[10]);
        draw (P[0]--P[6]);
    \end{asy}
\end{center}

\subsection{grid}
\begin{center}
    \begin{asy}
        unitsize (1cm);

        for (int x = 0; x < 8; ++x) 
        for (int y = 0; y < 4; ++y) {
            if ((x % 2) == (y % 2)) {
                fill ((x,y)--(x,y+1)--(x+1,y+1)--(x+1,y)--cycle, gray(0.75));
            } else {
                string s = (x % 2 == 0) ? "A" : "B";
                label (s, (x+0.5, y+0.5));
            }
        }
        for (int x = 0; x < 9; ++x) {
            draw ((x, 0) -- (x, 4), (x % 2 == 0) ? black+2 : black);
        }
        for (int y = 0; y < 5; ++y) {
            draw ((0, y) -- (8, y), (y % 2 == 0) ? black+2 : black);
        }
    \end{asy}
\end{center}

\section*{Fn Eqn}
\subsection{3}
The answers are $f(x) \equiv x$ and $f(x) \equiv 1/x$.
These work, so we show they are the only ones.

First, setting $(t,t,t,t)$ gives $f(t^2) = f(t)^2$.
In particular, $f(1) = 1$.
Next, setting $(t, 1, \sqrt t, \sqrt t)$ gives
\[ \frac{f(t)^2 + 1}{2f(t)} = \frac{t^2 + 1}{2t} \]
which as a quadratic implies $f(t) \in \{t, 1/t\}$.

Now assume $f(a) = a$ and $f(b) = 1/b$.
Setting $(\sqrt a, \sqrt b, 1, \sqrt{ab})$ gives
\[ \frac{a + 1/b}{f(ab) + 1} = \frac{a+b}{ab+1}. \]
One can check the two cases on $f(ab)$ each imply
$a=1$ and $b=1$ respectively.
Hence the only answers are those claimed.

\subsection{4}
The function  $f \colon {\mathbb Z}_{>0} \to {\mathbb Z}_{>0}$ satisfies
\[f(n+1) > f(f(n))\]
for every positive integer $n$. Show that $f$ is the identity.

Hint 1: Prove that $f(n)$ is the nth smallest output of $f$.

Hint 2: Focus on first proving $f(1)$ is the smallest output of $f$. From there, it is easy to generalize to all  using induction.


Consider the sentence
\[ P(n,k) : f(n) \ge k \quad \text{for all } n \ge k. \]
Note that $P(n,k)$ is true by induction on $k$.
Indeed, for $k = 1$ this is obvious.
For $k \ge 2$, note that if $n \ge k+1$ then $n-1 \ge k$, so
\[ f(n) > f(f(n-1)) = f(\text{something at least $k$}) \ge k
	\implies f(n) \ge k \]
by applying the inductive hypothesis twice.

In particular, $P(n,n)$ gives $f(n) \ge n$ for every $n$.
Thus $f(n+1) > f(n)$.
Consequently, $f$ is increasing,
and then it follows that $f$ is the identity.
\begin{remark*}
This is a special case of the key lemma in Shortlist 2011 A4.
\end{remark*}
\end{document}