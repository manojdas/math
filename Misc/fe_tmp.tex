\documentclass[11pt,twoside]{scrartcl}
\usepackage{mdas}
% \usepackage[sexy, fancy, hints]{evan}


\begin{document}
\title{Cyclotomic}


\author{Manoj Das}
\date{\today}
\section{7 - 21EGMO3}

\begin{soln}
    The answers are $f(x) \equiv +x$
and $f(x) \equiv -x$ which work.
To show they are the only ones,
we follow an approach similar to \texttt{SnowPanda}.

First, by setting $x=y=1$ we get $\boxed{f(1) \neq 0}$.

Note that by iterating the functional equation
multiple times, we have more generally that
\begin{align*}
	f(xf(x) + m \cdot y) &= x^2 + m \cdot f(y) \\
	f(n \cdot xf(x) + y) &= n \cdot x^2 + f(y).
\end{align*}
In particular, replacing $x=1$ everywhere gives:
\begin{align*}
	f(f(1) + m \cdot y) &= 1 + m \cdot f(y) \\
	f(n \cdot f(1) + y) &= n + f(y).
\end{align*}
Hence, we have the following sentence for any $y \in \QQ$
and $m,n \in \ZZ$:
\[ f(1) + my = n f(1) + y \implies 1 + mf(y) = n + f(y). \]
If we choose $m = kf(1)+1 \neq 1$ and $n = ky+1$
for some suitable nonzero $k$,
the first equation is true automatically;
we hence conclude $f$ is linear.
From here one quickly recovers the solution set.

\end{soln}
\section{20 - 10SLA6}

    \begin{problem}[Shortlist 2010 A6]
        Prove that if two functions $f, g \colon \NN \to \NN$
        obey
        \[ f(g(n)) = f(n)+1 \quad\text{ and }\quad g(f(n)) = g(n)+1 \]
        for each positive integer $n$, then $f = g$.
    \end{problem}

    Hints:
    \begin{enumerate}
        \item It may be useful to first solve the FE $h(h(n)) = h(n) + 1$.
        \item IMO 1977/6 is worth solving before attempting this problem as well. (Did it already)
        \item Let the images of $f$ and $g$ be $\{A,A+1,\ldots\}$ and 
        $\{B,B+1,\ldots\}$ respectively. If $A \le B$, try to prove that  $g(n) \ge f(n) + (B-A)$ to start.
        \item To prove the main estimate in the previous hint, show the sentence $n \ge A+k$ implies $g(n) \ge B+k+1$ by induction on k.
    \end{enumerate}
    \begin{soln}
        Let $P(n)$ and $Q(n)$ denote the two given assertions, respectively.

\begin{claim*}
	[Contiguous images]
	The images of $f$ and $g$ are of the form
	\begin{align*}
		f(\NN) &= \left\{ A, A+1, \dots \right\} \\
		g(\NN) &= \left\{ B, B+1, \dots \right\}
	\end{align*}
	for some $A$ and $B$.
\end{claim*}
\begin{proof}
	If $f(n) \in f(\NN)$, then $f(n)+1 = fg(n) \in f(\NN)$, by $P(n)$.
	Similarly for $g$.
\end{proof}

\begin{claim*}
	For any nonnegative integer $k$, we have
	$g(f(n)+k) = g^{k+1}(n)+1$.
\end{claim*}
\begin{proof}
	Both sides are equal to $gfg^k(n)$.
\end{proof}

Let us assume WLOG that $A \le B$.
The main lemma is the following (see IMO 1977/6 for a similar argument):
\begin{lemma*}
	[Main estimate]
	We have $g(n) \ge f(n) + (B-A)$ for every $n \ge 1$.
\end{lemma*}
\begin{proof}
	We prove the sentence
	\[ S(k) \colon \quad \text{If } n \ge A+k \text{ then } g(n) \ge B+k+1 \]
	by induction on $k$.
	For $k = 0$ this follows by varying $n$ in assertion $Q$.
	Now note that
	\begin{align*}
		g(n) &\ge B \\
		\implies gg(n) &\ge B+1 &\text{by $S(0)$} \\
		\implies ggg(n) &\ge B+2 &\text{by $S(1)$} \\
		& \vdotswithin\ge \\
		\implies g^{k}(n) &\ge B+(k-1) &\text{ by $S(k-2)$} \\
		\implies g^{k+1}(n) &\ge B+k &\text{ by $S(k-1)$}.
	\end{align*}
	And thus $g(f(n)+k) = g^{k+1}(n) + 1 \ge (B+k)+1$ which implies $S(k)$.

	Now for any particular $m \ge A$, we have $g(m) \ge m+(B-A+1)$.
	Replacing $m$ with $f(n)$ gives
	$1 + g(n) = g(f(n)) \ge f(n) + (B-A+1)$ as needed.
\end{proof}

\begin{claim*}
	[$f$ has no eventual cycles]
	For any $k > 0$ and $n \ge 1$,
	we have $f^k(n) \neq n$.
\end{claim*}
\begin{proof}
	Otherwise, $g(n) + k = (g \circ f^k)(n) = g(n) \implies k = 0$, contradiction.
\end{proof}

\begin{claim*}
	[We found all functions!]
	For $n \ge A$, we have
	\[ f(n) = n+1 \qquad g(n) = n+(B-A+1). \]
\end{claim*}
\begin{proof}
	By induction on $n \ge A$.
	For the base step, pick $x_0$ such that $g(x_0) = B$.
	By the main lemma, we have
	$B = g(x_0) \ge f(x_0) + (B-A) \implies f(x_0) = A$.
	Hence, $Q(x_0) \implies g(A) = B+1$.
	Finally,
	\[ B+1 = g(A) \ge f(A) + (B-A) \implies f(A) \le A+1. \]
	Since $f(A) \neq A$, but also $f(A) \ge A$ from the range of $f$,
	it follows $f(A) = A+1$.

	The inductive step is almost the same:
	assume we know $f$ and $g$ up to $n-1$.
	Then
	\[ g(n) = g(f(n-1)) = g(n-1) + 1 = n + (B-A+1). \]
	Now by the main lemma,
	\[ n + (B-A+1) = g(n) \ge f(n) + (B-A) \implies f(n) \le n+1. \]
	But since $f$ has no eventual cycles,
	and we already have a chain
	\[ A \xmapsto{f} (A+1) \xmapsto{f} (A+2)
		\xmapsto{f} \dots \xmapsto{f} (n-1) \xmapsto{f} n \]
	we must have $f(n) = n+1$ since
	it can't map to any values in $\{A , \dots, n\}$.
\end{proof}

Now $P(n)$ reads $f(n)+1 = f(g(n)) = g(n)+1$ since $g(n) \ge A$.
Thus $f = g$.

\begin{remark*}
	From here it easy to characterize all functions.
	We have $A=B$ and $f(n) = g(n) = n+1$ for all $n \ge A$.
	The values of $f(n) = g(n)$ may be arbitrary for $n < A$,
	as long as they are at least $A$.
\end{remark*}
    \end{soln}
\end{document}