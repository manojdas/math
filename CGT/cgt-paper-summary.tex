\documentclass[11pt,twoside]{scrartcl}
\usepackage{mdas}

% \usepackage{cgt}
\newcommand{\mcN}{\mathcal{N}}
\newcommand{\mcP}{\mathcal{P}}
\newcommand{\msN}{\mathscr{N}}
\newcommand{\msP}{\mathscr{P}}

\title{Impartial Games}
\subtitle{Paper Outline}
\author{Rohan Das}

\date{\today}

\begin{document}

\maketitle
\section{Fibonacci NIM}
\subsection{Introduction}
I will describe the rules of the game. I will elaborate with some examples.
\subsection{Winning Strategy}
I will come up with a winning strategy. I will prove key theorems and elaborate with examples.
\begin{theorem}{Zeckendorf ’s Theorem}
    Let $n$ be a positive integer. Then there is a unique increasing sequence $(c_i)_{i=0}^k$ such that $c_i \ge 2$ and $c_{i+1} > c_i + 1$ for $i \ge 0$, and that
    \[n = \sum_{i=0}^k F_{c_i}.\]
\end{theorem}

\begin{theorem}
    Initial game is a $\mcP$ position if and only if the number of stones is a Fibonacci number.
\end{theorem}
\subsection{Grundy Values}
I will compute some Grundy values.

\subsection{Two Pile Fibonacci NIM}
I will extend to the two pile game.

\subsection{Game with Lucas Numbers and Catalan Numbers}
If I can get some \textbf{hints} on how to solve Chapter 4, problem 16, I would like to solve it and see if it makes sense to add here.

\section{WYTHOFF}
\subsection{Introduction}
I will describe the game and the rules. I will elaborate with some examples.
\subsection{First few Grundy values}
I will compute the first few Grundy values.

\subsection{$\mathscr{P}$ positions}
I will discuss the $\mathscr{P}$ positions and try to prove the key theorem.
\begin{theorem}
    The $n^{\text{th}}$ $\mathscr{P}$-position of \textsc{Wythoff} is given by
    \[(a_n,b_n) = (\floor{n\phi}), \floor{n\phi^2})\] 
\end{theorem}

I will compute some of the $\mathscr{P}$ positions.

\subsection{Generalized Grundy values}
I will discuss the patterns in generalized Grundy values and prove the key theorem.
\begin{theorem}
    For a fixed value of $a$, the sequence $b \mapsto \mathscr{G}(a,b)$ is arithematic periodic.
\end{theorem}
\end{document}