\documentclass[11pt,twoside]{scrartcl}
\usepackage{mdas}
\usepackage{diagbox}

% \usepackage{cgt}
\newcommand{\mcN}{\mathcal{N}}
\newcommand{\mcP}{\mathcal{P}}
\newcommand{\mcG}{\mathcal{G}}
\newcommand{\mcA}{\mathcal{A}}
\newcommand{\mcB}{\mathcal{B}}

\newcommand{\msN}{\mathscr{N}}
\newcommand{\msP}{\mathscr{P}}
\newcommand{\TODO}[1]{\emph{\textcolor{red}{#1\ }}}

\DeclareMathOperator{\mex}{mex}

\title{Impartial Games}
\subtitle{Paper Outline}
\author{Rohan Das}

\date{\today}

\begin{document}

\maketitle
\section{Introduction}
Quick introduction to impartial games, subtraction games, and NIM.

Outline of the rest of the paper.
\section{Fibonacci NIM}
\subsection{Introduction}
I will describe the rules of the game. I will elaborate with some examples.
\subsection{Zeckendorf’s Representation}
The winning strategy and analysis for Fibonacci NIM depends on the Zeckendorf's represetation. We briefly review it here.

\begin{theorem}{Zeckendorf’s Theorem}
    Let $n$ be a positive integer. Then there is a unique increasing sequence $(c_i)_{i=0}^k$ such that $c_i \ge 2$ and $c_{i+1} > c_i + 1$ for $i \ge 0$, and that
    \[n = \sum_{i=0}^k F_{c_i}.\]
\end{theorem}
Note sure if you want to prove this. You can either get the proof from or just cite \cite{henderson}. 

\TODO{Add example(s)}.

\begin{corollary}
    If $F_{k+1} > n \ge F_k$ , then $F_k$ is the largest number in the Zeckendorff representation of $n$.
\end{corollary}

If there is time, \TODO{prove this.} See page 3 of \cite{cody}.

\begin{definition}
    Let $n = F_{i_r} +F{i_{r-1}} + \cdots +F_{i_1}$ where $r,i, n \in \NN$. We define the \emph{tail}, $T(n) = F_{i_1}$.

    That is, $T(n)$ is the smallest number in the Zeckendorff representation.
\end{definition}

\TODO{Review and change}. If we don't end up using $j-$length tails, then we can remove the next definition. In that case, we should also use $z_1(n)$ as Simon uses that notation.

\begin{definition}
    Let $n = F_{i_r} +F{i_{r-1}} + \cdots +F_{i_1}$ where $r,i, n \in \NN$. We define the length $j$ tail to be the specific sum of $j$ consecutive Fibonacci numbers in the
    Zeckendorff representation of n beginning with the smallest number, $F_{i_1}$. We set
    $T_1(n) = T (n)$ for consistency. Then, 
    \[T_j(n) = T (n) + T_{i-1}(n - T (n)).\]
    
\end{definition}

\subsection{Winning Strategy}
I will come up with a winning strategy. I will prove key theorems and elaborate with examples.

\begin{theorem}\label{thm-fibp}
    Initial game is a $\mcP$ position if and only if the number of stones is a Fibonacci number.
\end{theorem}
\TODO{Give} intuition of this first and then prove.

This lemma may be useful for the proof.
\begin{lemma}
    For every $i \in N$ where $i \ge 3$, $2F_{i-1} \ge F_i$ and $F_{i+1} > 2F_{i-1}$.
\end{lemma}
For proof \TODO{see page 5 of \cite{cody}}.

\begin{remark}
    Simon uses different notation. The winning strategy for one-pile Fibonacci nim was described by Whinihan. Consider the position $(n;r)$. If $z_1(n) \le r$, then $(n;r)$ is an $\msN$ position, and removing $z_1(n)$ stones is a winning move. If $z_1(n) > r$, then $(n;r)$ is a $P$ position, and there are no winning moves. \TODO{Change to match?}
\end{remark}
\begin{claim}\label{claim-part}
    $\msP$ is the set of positions $(n, r)$, such that $T(n) > r$. $\msN$ is the set of positions not in $\msP$, that is $T(n) \le r$.
\end{claim}

The next Lemma, Lemma \ref{part-1}, says that every move from $\msP$ is to $\msN$.
\begin{lemma}\label{part-1}
    Let $n \in \NN$. For any $p$ with $T (n) > p$, $(n-p, 2p) \in \msN$.
\end{lemma}
\begin{proof}
    \TBD see page 5, lemma 11 of \cite{cody}.
\end{proof}

\vspace{6pt}
The next Lemma, Lemma \ref{part-2}, says from $\msN$ there is a move to $\msP$.
\begin{lemma}\label{part-2}
    Let $n \in \NN$. Let $p := T(n)$. Then $(n - p, 2p) \in \msP$.
\end{lemma}
\begin{proof}
    \TBD see page 5, lemma 12 of \cite{cody}.
\end{proof}

\vspace{6pt}
\begin{proof}[Proof of Claim \ref{claim-part}]
    \TBD partition theorem.
\end{proof}

\vspace{6pt}
\begin{proof}[Proof of Theorem \ref{thm-fibp}]
    \TBD should be trivial from the claim above.
\end{proof}

\subsection{Grundy Values}
\TODO{Describe} what Grundy values, $\mcG$ are and their use in multi-pile games. Define what mex is and the formula for $\mcG$.

Now compute some $\mcG$ values - in addition to some small ones that are built ground up, do a few that are built using the table below.

Refer to the table \ref{tab:gvals}.

\begin{table}[h]
    \centering
    \begin{tabular}{c|*{21}c}
      $n \backslash r$ & 0 & 1 & 2 & 3 & 4 & 5 & 6 & 7 & 8 & 9 & 10 & 11 & 12 & 13 & 14 & 15 & 16 & 17 & 18 & 19 & 20  \\\hline
    0  & 0 &   &   &   &   &   &   &   &   &   &    &    &    &    &    &    &    &    &    &    &     \\
    1  & 0 & 1 &   &   &   &   &   &   &   &   &    &    &    &    &    &    &    &    &    &    &     \\
    2  & 0 & 0 & 2 &   &   &   &   &   &   &   &    &    &    &    &    &    &    &    &    &    &     \\
    3  & 0 & 0 & 0 & 3 &   &   &   &   &   &   &    &    &    &    &    &    &    &    &    &    &     \\
    4  & 0 & 1 & 1 & 3 & 3 &   &   &   &   &   &    &    &    &    &    &    &    &    &    &    &     \\
    5  & 0 & 0 & 0 & 0 & 0 & 4 &   &   &   &   &    &    &    &    &    &    &    &    &    &    &     \\
    6  & 0 & 1 & 1 & 1 & 1 & 4 & 4 &   &   &   &    &    &    &    &    &    &    &    &    &    &     \\
    7  & 0 & 0 & 2 & 2 & 2 & 4 & 4 & 4 &   &   &    &    &    &    &    &    &    &    &    &    &     \\
    8  & 0 & 0 & 0 & 0 & 0 & 0 & 0 & 0 & 5 &   &    &    &    &    &    &    &    &    &    &    &     \\
    9  & 0 & 1 & 1 & 1 & 1 & 1 & 1 & 1 & 5 & 5 &    &    &    &    &    &    &    &    &    &    &     \\
    10 & 0 & 0 & 2 & 2 & 2 & 2 & 2 & 2 & 5 & 5 & 5  &    &    &    &    &    &    &    &    &    &     \\
    11 & 0 & 0 & 0 & 3 & 3 & 3 & 3 & 5 & 5 & 5 & 5  & 5  &    &    &    &    &    &    &    &    &     \\
    12 & 0 & 1 & 1 & 3 & 3 & 3 & 3 & 3 & 6 & 6 & 6  & 6  & 6  &    &    &    &    &    &    &    &     \\
    13 & 0 & 0 & 0 & 0 & 0 & 0 & 0 & 0 & 0 & 0 & 0  & 0  & 0  & 6  &    &    &    &    &    &    &     \\
    14 & 0 & 1 & 1 & 1 & 1 & 1 & 1 & 1 & 1 & 1 & 1  & 1  & 1  & 6  & 6  &    &    &    &    &    &     \\
    15 & 0 & 0 & 2 & 2 & 2 & 2 & 2 & 2 & 2 & 2 & 2  & 2  & 2  & 6  & 6  & 6  &    &    &    &    &     \\
    16 & 0 & 0 & 0 & 3 & 3 & 3 & 3 & 3 & 3 & 3 & 3  & 3  & 7  & 7  & 7  & 7  & 7  &    &    &    &     \\
    17 & 0 & 1 & 1 & 3 & 3 & 3 & 3 & 3 & 3 & 3 & 3  & 3  & 3  & 7  & 7  & 7  & 7  & 7  &    &    &     \\
    18 & 0 & 0 & 0 & 0 & 0 & 4 & 4 & 4 & 4 & 4 & 4  & 7  & 7  & 7  & 7  & 7  & 7  & 7  & 7  &    &     \\
    19 & 0 & 1 & 1 & 1 & 1 & 4 & 4 & 4 & 4 & 4 & 4  & 4  & 7  & 7  & 7  & 7  & 7  & 7  & 7  & 7  &     \\
    20 & 0 & 0 & 2 & 2 & 2 & 4 & 4 & 4 & 4 & 4 & 4  & 4  & 4  & 7  & 7  & 7  & 7  & 7  & 7  & 7  & 7  \\
    \hline
    \end{tabular}
    \caption{\label{tab:gvals}Gundy Values for Fibonacci NIM.}
    \end{table}

\TODO{Refer to} \textit{Larsson and Rubinstein-Salzedo \cite{simon1}}
\begin{remark}
    \TODO{Delete Later.} The paper is at 
    
    \url{https://drive.google.com/file/d/1UmXLhttps://drive.google.com/file/d/1UmXL36N0uQWlUq4OaCruBifv6yJ0m7Fp/view?usp=sharing}
\end{remark}

\subsection{Two Pile Fibonacci NIM}
I will extend to the two pile game.

\textit{Larsson and Rubinstein-Salzedo \cite{simon2}}
\begin{remark}
    \TODO{Delete Later.} The paper is at 
    
    \url{https://drive.google.com/file/d/1UmXL36N0uQWlUq4OaCruBifv6yJ0m7Fp/view?usp=sharing}
\end{remark}
\subsection{Game with Lucas Numbers and Catalan Numbers}
If I can get some \textbf{hints} on how to solve Chapter 4, problem 16, I would like to solve it and see if it makes sense to add here.

\section{WYTHOFF}
\subsection{Introduction}
I will describe the game and the rules. I will elaborate with some examples.

\TODO{Consider} also describing the interpretation of queen.

\subsection{First few Grundy values}
I will compute the first few Grundy values.

\begin{table}[h]
    \centering
    \begin{tabular}{c|*{11}c}
    \diagbox{$a$}{$b$}& 0  & 1  & 2  & 3  & 4  & 5  & 6  & 7  & 8  & 9  & 10  \\ \hline
    0   & 0  & 1  & 2  & 3  & 4  & 5  & 6  & 7  & 8  & 9  & 10  \\
    1   & 1  & 2  & 0  & 4  & 5  & 3  & 7  & 8  & 6  & 10 & 11  \\
    2   & 2  & 0  & 1  & 5  & 3  & 4  & 8  & 6  & 7  & 11 & 9   \\
    3   & 3  & 4  & 5  & 6  & 2  & 0  & 1  & 9  & 10 & 12 & 8   \\
    4   & 4  & 5  & 3  & 2  & 7  & 6  & 9  & 0  & 1  & 8  & 13  \\
    5   & 5  & 3  & 4  & 0  & 6  & 8  & 10 & 1  & 2  & 7  & 12  \\
    6   & 6  & 7  & 8  & 1  & 9  & 10 & 3  & 4  & 5  & 13 & 0   \\
    7   & 7  & 8  & 6  & 9  & 0  & 1  & 4  & 5  & 3  & 14 & 15  \\
    8   & 8  & 6  & 7  & 10 & 1  & 2  & 5  & 3  & 4  & 15 & 16  \\
    9   & 9  & 10 & 11 & 12 & 8  & 7  & 13 & 14 & 15 & 16 & 17  \\
    10  & 10 & 11 & 9  & 8  & 13 & 12 & 0  & 15 & 16 & 17 & 14 \\ \hline
    \end{tabular}
    \caption{\label{tab:gvals-wyth}$\mcG$ values for \textsc{Wythoff} game.}
\end{table}

\subsection{$\msP$ positions}
I will discuss the $\mathscr{P}$ positions and try to prove the key theorem.

\begin{lemma}\label{lemma-wyth-p}
    The \textsc{Wythoff} $\msP$ positions are the pairs of the form $(a_n, b_n)$, or $(b_n, a_n)$, where $(a_n, b_n) \in \NN^2$ satisfy
    \begin{align*}
        a_n &= \mex\{a_i, b_i : i < n\}, \\
        b_n &= a_n + n.
    \end{align*}
\end{lemma}
\begin{proof}
    \TODO{Page 199} of Siegal book.
\end{proof}

Some of the early values for $A_n, B_n$ are shown in table \ref{tab:wyth-p}.

\begin{table}[h]
    \centering
    \begin{tabular}{c|*{11}c}
    $n$    & 0 & 1 & 2 & 3 & 4  & 5  & 6  & 7  & 8  & 9  & 10  \\ \hline
    $A_n$ & 0 & 1 & 3 & 4 & 6  & 8  & 9  & 11 & 12 & 14 & 16  \\
    $B_n$ & 0 & 2 & 5 & 7 & 10 & 13 & 15 & 18 & 20 & 23 & 16 
    \end{tabular}
    \caption{\label{tab:wyth-p}$\msP$ positions for \textsc{Wythoff}.}

    \end{table}

\begin{definition}
    Let $\mcA, \mcB \subset \NN^+$. We say $\mcA$ and $\mcB$ are complementary if $B = \NN^+ \setminus A$, that is, if $\mcA \cap \mcB = \emptyset$ and $\mcA \cup \mcB = \NN^+$.
\end{definition}

\begin{lemma}\label{lemma-wyth-comp}
    Fix irrational real numbers $\alpha, \beta > 1$ such that
    \[\frac{1}{\alpha} + \frac{1}{\beta} = 1.\]
    Then the sets 
        \[\{\floor{n\alpha}: n \in \NN^+\} \quad\text{and}\quad \{\floor{n\beta}: n \in \NN^+\}\]
        are complementary.
\end{lemma}
\begin{theorem}\label{thm-wyth}
    The $n^{\text{th}}$ $\mathscr{P}$-position of \textsc{Wythoff} is given by
    \[(a_n,b_n) = (\floor{n\phi}), \floor{n\phi^2})\] 
\end{theorem}

I will compute some of the $\mathscr{P}$ positions.

\subsection{Generalized Grundy values}
\TODO{May be skip this.}
I will discuss the patterns in generalized Grundy values and prove the key theorem.
\begin{theorem}
    For a fixed value of $a$, the sequence $b \mapsto \mathscr{G}(a,b)$ is arithematic periodic.
\end{theorem}

\section{Temp}
\begin{center}
    \begin{asy}
        import graph;

        unitsize(0.75cm);
        picture pic;
        path star;
        int n = 10;

        pen border = gray(0.7);
        pen safe = blue;
        pen covered = red;

        void drawStep(int x, int y) {
            int n1;

            draw(pic, shift(x+0.5, y+0.5)*scale(0.35)*rotate(18)*star, safe);
            draw(pic,(x+0.5,y+1)--(x+0.5,n-0.5), covered, EndArrow(4));
            draw(pic,(x+1,y+0.5)--(n-0.5,y+0.5), covered, EndArrow(4));

            if (x > y) {
                n1 = n - x;
            } else {
                n1 = n - y;
            }
            draw(pic,(x+1,y+1)--(n1+x-0.5,n1+y-0.5), covered, EndArrow(4));
        }
        for (int i = 0; i <= n; ++i) {
            draw(pic, (0,i)--(n,i), border);
            draw(pic, (i,0)--(i,n), border);
        }
        star=expi(0)--(scale((3-sqrt(5))/2)*expi(pi/5))--expi(2*pi/5)--
            (scale((3-sqrt(5))/2)*expi(3*pi/5))--expi(4pi/5)--
            (scale((3-sqrt(5))/2)*expi(5*pi/5))--expi(6*pi/5)--
            (scale((3-sqrt(5))/2)*expi(7*pi/5))--expi(8*pi/5)--
            (scale((3-sqrt(5))/2)*expi(9*pi/5))--cycle;

        drawStep(0,0);
        add (pic);

        drawStep(2,1);
        drawStep(1,2);


        add(shift((n+2)*right)*pic);

        drawStep(3,5);
        drawStep(5,3);

        add(shift((n+2)*down)*pic);

        drawStep(7,4);
        drawStep(4,7);
        add(shift((n+2)*right)*shift((n+2)*down)*pic);

    \end{asy}
\end{center}

\begin{lstlisting}[language=iPython, caption=Python to Generate $\msP$ Positions.]

def mex(mex_set):
    ret = 0
    while ret in mex_set:
        ret += 1
    return ret

def wyth_p_positions(max_n = 10):
    mex_set = set()
    ret = []
    for n in range(max_n+1):
        a_n = mex(mex_set)
        b_n = a_n + n

        mex_set.add(a_n)
        mex_set.add(b_n)

        ret.append((a_n, b_n))
        ret.append((b_n, a_n))

    return ret
\end{lstlisting}

\begin{figure}[h]
\begin{tikzpicture}
    [
    level 1/.style = {red, sibling distance = 2cm},
    level 2/.style = {blue, sibling distance = 2cm},
    level 3/.style = {red, sibling distance = 2cm},
    level 4/.style = {blue, sibling distance = 1.5cm},
    level 5/.style = {red, sibling distance = 2cm},
    level 6/.style = {blue, sibling distance = 2cm},
    level 7/.style = {red, sibling distance = 2cm},    
    ]
    \tikzstyle{hollow node}=[circle,draw,inner sep=2.5]
    \tikzstyle{solid node}=[circle,draw,inner sep=2.5,fill=black]
    \tikzstyle{win node}=[rectangle,draw=blue,inner sep=2.5]

    \node[blue]{(10)}
    child{node[]{(8,4)} 
        child{node[win node]{(4,4)}}
        child{node[win node]{(5,5)}}
        child{node[]{(6,4)}
            child{node[]{(5,2)}
                child{node[win node]{(3,3)}}
                child{node[]{(4,2)}
                    child{node[]{(3,2)}
                        child{node[win node]{(2,2)}}
                        child{node[win node]{(1,1)}}
                    }
                }    
            }
        }
        child{node[right=5.5cm]{(7,2)}
            child{node[]{(5,4)}
                child{node[]{(4,2)}
                    child{node[]{(3,2)}
                        child{node[win node]{(2,2)}}
                        child{node[win node]{(1,1)}}
                    }
                }    
                child{node[win node]{(3,3)}}
                child{node[win node]{(2,2)}}
                child{node[win node]{(1,1)}}
            }
        }
    }
    ;
\end{tikzpicture}
    \caption{\label{fig:fibnim-ex1} Fibonnaci NIM Example 1}
\end{figure}

\begin{figure}[h]
    \begin{tikzpicture}
        [
        level 1/.style = {red, sibling distance = 2cm},
        level 2/.style = {blue, sibling distance = 2cm},
        level 3/.style = {red, sibling distance = 1.5cm},
        level 4/.style = {blue, sibling distance = 1.5cm},
        level 5/.style = {red, sibling distance = 1.5cm},
        level 6/.style = {blue, sibling distance = 1cm},
        level 7/.style = {red, sibling distance = 1cm},    
        ]
        \tikzstyle{hollow node}=[circle,draw,inner sep=2.5]
        \tikzstyle{solid node}=[circle,draw,inner sep=2.5,fill=black]
        \tikzstyle{win node}=[rectangle,draw=blue,inner sep=2.5]
    
        \node[blue]{(8,7)}
        child{node[left=3cm]{(7,2)}
            child{node[]{(5,4)}
                child{node[]{(4,2)}
                    child{node[]{(3,2)}
                        child{node[win node]{(2,2)}}
                        child{node[win node]{(1,1)}}
                    }
                }    
                child{node[win node]{(3,3)}}
                child{node[win node]{(2,2)}}
                child{node[win node]{(1,1)}}
            }
        }
        child{node[]{(6,4)}
            child{node[]{(5,2)}
                child{node[win node]{(3,3)}}
                child{node[]{(4,2)}
                    child{node[]{(3,2)}
                        child{node[win node]{(2,2)}}
                        child{node[win node]{(1,1)}}
                    }
                }    
            }
        }
        child{node[win node]{(5,5)}}
        child{node[win node]{(4,4)}}
        child{node[win node]{(3,3)}}
        child{node[win node]{(2,2)}}
        child{node[win node]{(1,1)}}
        ;
    \end{tikzpicture}
        \caption{\label{fig:fibnim-ex1} Fibonnaci NIM Example 2}
    \end{figure}
    

\clearpage
\sectionmark{References}
\begin{thebibliography}{99}
    \bibitem{simoncgt} Rubinstein-Salzedo, S. Combinatorial Game Theory. 
    \bibitem{simon1} Larsson, U., Rubinstein-Salzedo, S. Grundy values of Fibonacci nim. Int J Game Theory 45, 617–625 (2016). \url{https://doi.org/10.1007/s00182-015-0473-y}
    \bibitem{simon2} Larsson, U., Rubinstein-Salzedo, S. Global Fibonacci nim. Int J Game Theory 47, 595–611 (2018). 
    
    \url{https://doi.org/10.1007/s00182-017-0574-x}

    \bibitem{cody} Allen, C., Ponomarenko, V. (2014), Fibonacci Nim and a full characterization of winning moves, 
    
    Involve, \href{https://msp.org/involve/2014/7-6/p08.xhtml}{doi:10.2140/involve.2014.7.807}
    \bibitem{wiki} Fibonacci nim, Wikipedia. 5 June 2021, \url{https://en.wikipedia.org/wiki/Fibonacci_nim}.
    \bibitem{siegel} Aaron N. Siegel, Combinatorial Game Theory

    \bibitem{henderson} Henderson, N. What is Zeckendorf’s Theorem?

    \url{https://math.osu.edu/sites/math.osu.edu/files/henderson_zeckendorf.pdf}

    \bibitem{beatty}Beatty sequence, Wikipedia. 5 June 2021, \url{https://en.wikipedia.org/wiki/Beatty_sequence}
\end{thebibliography}



\end{document}