\documentclass[11pt,twoside]{scrartcl}
\usepackage{mdas}
% \usepackage[sexy, fancy, hints]{evan}

\lstset{
numbers=left,
numberstyle=\small,
numbersep=8pt,
frame = single,
language=C,
framexleftmargin=15pt}

\begin{document}
\title{Using Asymptote}
% If you contribute to the handout, put your name in comment here

\author{Manoj Das}
\org{Manoj Math Notes}
\date{December 8, 2020}

\maketitle
\section{Triangles}
\subsection{Drawing Using Angles}
\subsubsection{Option 1}
Use \textit{dir()} to get an unit vector in the desired direction.
\\
\begin{asy}
import geometry;
size(4cm);

pair A,B,C;

point A=(-.5*sqrt(37),0);
point B=(.5*sqrt(37),0);

draw(A--B);
dot(A^^B);
//dot(origin,red);
point C_help=shift(A)*dir(50);
line line_1=line(A,C_help);
line line_2=line(origin,(0,1));
pair C=intersectionpoint(line_1,line_2);
dot(C);
draw(A--B--C--cycle);
draw("$50^\circ$",arc(A,.6,0,50));
draw("$50^\circ$",arc(B,.6,180-50,180));
draw("$80^\circ$",arc(C,.6,degrees(A-C),degrees(B-C)));
label("$A$",A,S);
label("$B$",B,S);
label("$C$",C,N);
\end{asy}
\begin{lstlisting}
import geometry;
...
point A=(-.5*sqrt(37),0);
point B=(.5*sqrt(37),0);
...
point C_help=shift(A)*dir(50);
line line_1=line(A,C_help);
line line_2=line(origin,(0,1));
pair C=intersectionpoint(line_1,line_2);
...
\end{lstlisting}
\subsubsection{Option 2}
Another version using \textit{rotate}. Also, usage of \textit{markangle}.

\begin{asy}
size(6 cm);
import geometry;

pair A = (3,0), B = (4,6);
pair rotA = rotate(-50,B)*A, rotB = rotate(50,A)*B; //Rotate A 50 degrees clockwise around B, rotate B 50 degrees counterclockwise around A.
pair C = extension(A,rotB,B,rotA); //Standard function for the intersection of two lines.

draw(A--B--C--cycle);
dot(A); dot(B); dot(C);
label("$A$",A,-unit(B+C-2*A)); //Puts the label a unit distance away, opposite the side BC.
label("$B$",B,-unit(C+A-2*B));
label("$C$",C,-unit(A+B-2*C));
markangle("$50^{\circ}$",B,A,C); //These angle marks are the only pieces that actually use the import.
markangle("$50^{\circ}$",C,B,A);
markangle("$80^{\circ}$",A,C,B);
\end{asy}
\begin{lstlisting}
...
//Rotate A 50 degrees clockwise around B,
//rotate B 50 degrees counterclockwise around A.
pair rotA = rotate(-50,B)*A, rotB = rotate(50,A)*B;
...
markangle("$50^{\circ}$",B,A,C);
...
\end{lstlisting}
\subsection{Circumcircle, Incircle}
Using \textit{olympiad} package's functions.
\\
\vspace{5 pt} \\
\begin{asy}
import olympiad;
unitsize(50);
pair A,B,C,O,I;
A=origin; B=2*right; C=1.5*dir(70);
O=circumcenter(A,B,C); // olympiad - circumcenter
I=incenter(A,B,C); // olympiad - incenter
draw(A--B--C--cycle);
dot(O);
dot(I);
draw(circumcircle(A,B,C)); // olympiad - circumcircle
draw(incircle(A,B,C)); // olympiad - incircle
label("$I$",I,W);
label("$O$",O,S);
\end{asy}
\begin{lstlisting}
import olympiad;
...
O = circumcenter(A,B,C);
I = incenter(A,B,C);
...
draw (circumcircle (A,B,C));
draw (incircle (A,B,C));
...
\end{lstlisting}

%-----------------------------------------------
\section{Circles}
\subsection{Circles at Tangents}
We use a parabola to find where the center of the tangent circle will be. Let $R$, and $r$ be the radius of the large and the small circle. Then the center of the small circle will be $R + r$ from the center of the large circle. If we let the \textit{Directerix} be at $x$ from the center of the large circle. Then we have $x - R  + r = R + r \implies x = 2 R$.
\\
\vspace{5 pt} \\
\begin{asy}
size(6 cm);
import geometry;
pair O = (0,0);
line Direc = line((0,2),(1,2));
//draw (Direc, red + dashed, L = "Directerix");
parabola Parab = parabola(O,Direc);
draw (Parab, red + dashed);
line Loc = line((0,.5),(1,.5));
draw (Loc, red + dashed);
pair P = intersectionpoints(Parab,Loc)[1];
pair Q = (P.x,-P.y);
draw(circle(O,1));
draw(circle(P,.5));
draw(circle(Q,.5));
draw(box((-1,1),(P.x+.5,-1)));
dot (O, red);
dot (P, red);
dot (Q, red);
label("O", O, S, red);
label("P", P, S, red);
label("Q", Q, S, red);
\end{asy}
\begin{lstlisting}
...
line Direc = line((0,2),(1,2));
parabola Parab = parabola(O,Direc);
line Loc = line((0,.5),(1,.5));
pair P = intersectionpoints(Parab,Loc)[1];
pair Q = (P.x,-P.y);
...
\end{lstlisting}

\section{More 2D Examples}
\subsection{Rotations, Parallel Lines}
\begin{example}[HMMT November 2014, Team Round \#10] \label{hmmt_nov14_team_10}
    Let $ABCDEF$ be a convex hexagon with the following properties.
    \begin{enumerate}[a)]
        \item $\overline{AC}$ and $\overline{AE}$ trisect $\angle BAF$.
        \item $\overline{BE} \parallel \overline{CD}$ and $\overline{CF} \parallel \overline{DE}$.
        \item $AB = 2AC = 4AE = 8AF$.
    \end{enumerate}
\end{example}
\begin{figure}[ht!]
    \centering
    \begin{asy}
        unitsize(1cm);
        pair A, B, C, D, E, F, C1, E1;
        real theta = 162, s = 1;

        A = (0, 0);
        B = (s*8, 0);
        C = s*4*dir(theta/3);
        E = s*2*dir(2*theta/3);
        F = s*dir(theta);

        C1 = rotate(degrees(B-E), C)*(C.x-1,C.y);
        //draw(C--C1, red);

        E1 = rotate(degrees(C-F), E)*(E.x+1, E.y);
        //draw(E--E1, red);

        D = extension(C, C1, E, E1);

        draw(E--F--A--B--C--D--cycle);
        draw(A--C^^A--E^^B--E^^C--F, yellow);
        
        

    \end{asy}
\end{figure}

\begin{thebibliography}{99}
    \bibitem{staats} Charles Staats III, An Asymptote Tutorial, available at: \\ \url{https://asymptote.sourceforge.io/asymptote_tutorial.pdf} 
\end{thebibliography}

\end{document}