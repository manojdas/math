\documentclass[11pt,twoside]{scrartcl}
\usepackage{mdas}
\usepackage{ytableau}
% \usepackage[sexy, fancy, hints]{evan}

\newcommand{\TA}{\begin{ytableau}
    \\
    \\
    &
\end{ytableau}}

\newcommand{\TB}{\begin{ytableau}
    \none & \\
    \none & \\
    &
\end{ytableau}}

\newcommand{\TC}{\begin{ytableau}
    \ & \ \\
    \\
    & \none
\end{ytableau}}

\newcommand{\TD}{\begin{ytableau}
    \ & \ \\
    \none & \\
    \none &
\end{ytableau}}

\newcommand{\TE}{\begin{ytableau}
    \\
    & &
\end{ytableau}}

\newcommand{\TF}{\begin{ytableau}
    \ & & \\
    \
\end{ytableau}}

\newcommand{\TG}{\begin{ytableau}
    \none & \none & \\
    & & 
\end{ytableau}}

\newcommand{\TI}{\begin{ytableau}
    \ & & \\
    \none & \none &
\end{ytableau}}

\begin{document}
\title{Young Tableau Examples}
% If you contribute to the handout, put your name in comment here

\author{Manoj Das}
% \org{Manoj Math Notes}
\date{November, 2022}

This is a ytableau:
\begin{ytableau}
    a & d & f \\
    b & e & g \\
    c
\end{ytableau}

\vspace{12pt}
TA is \TA; TB is \TB ; TC is \TC; TD is \TD .

\vspace{12pt}
TE is \TE; TF is \TF ; TG is \TG; TI is \TI .


 
\end{document}