\documentclass[11pt,twoside]{scrartcl}
\usepackage{mdas}
% \usepackage[sexy, fancy, hints]{evan}


\begin{document}
\title{Algebra - AIME Basic}
% If you contribute to the handout, put your name in comment here

\author{Manoj Das}
\org{Manoj Math Notes}
\date{Fall, 2020}

\maketitle
\section{MC 40 - Fall 2019}
\begin{problem}
Alice, Bob, and Charlie are consuming snacks. Alice consumes snacks for six hours. Halfway through the six hours, Bob starts to consume snacks as well. In total, Alice and Bob consume 309 snacks. Alice continues to consume snacks for two hours and again is joined halfway through the two hours by Charlie. Together, Alice and Charlie consume 101 snacks. Working together, Charlie and Bob can consume 150 more snacks in four hours than Alice can consume in two hours. Assuming Alice, Bob, and Charlie each consume at their own constant rate, how many snacks can Alice, Bob, and Charlie consume in 9 hours?
\begin{sketch}
    \boxed{837}
\end{sketch}
\end{problem}

\begin{problem}
    \begin{sketch}
        
    \end{sketch}
\end{problem}
The sequence 𝑣𝑛 satisfies the following: 𝑣0=1, 𝑣1=2, 𝑣2=3, 𝑣3=4, and 𝑣𝑛=𝑣𝑛−2−𝑣𝑛−4 for all integer 𝑛≥4. Calculate ∑785𝑛=780𝑣𝑛.

\begin{problem}
    \begin{sketch}
        
    \end{sketch}
\end{problem}

\newpage

\section{Solution Sketches}
\makehints

\end{document}