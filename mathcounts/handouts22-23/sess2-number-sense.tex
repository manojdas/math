% !TeX TXS-program:compile = txs:///pdflatex/[--shell-escape]

\documentclass[9pt]{beamer}
\usetheme{Madrid}
\usecolortheme{beaver}
\usepackage{amsmath,amssymb,amsthm,asymptote,graphicx}
\usepackage{graphics}
% \usepackage{bisvslides}
\usepackage{arcs}
\graphicspath{{./images}}

\newcounter{problem}[section]

\newenvironment{probslide}[3][]{%
    \refstepcounter{problem}\begin{frame}[t]%
	{Problem \thesection.\theproblem 
        \def\temp{#2}\ifx\temp\empty
            %
        \else
            \ - \temp%
        \fi}
    {#3}}%
	{\end{frame}}

% \newenvironment{Example}[2][Example]
%     {This is an #1. You gave #2 as an argument. The rest will be bold: \bfseries}
%     {}
% \textbf{Problem~\theproblem. #1
% \newenvironment{bsmi}{\begin{CJK}{UTF8}{bsmi}}{\end{CJK}}

\title{Number Sense}
\subtitle{Mathcounts 22 - Session 2}
\author{Krishnaveni P, Derrick L, Shivaani V, Irene W}
\institute{BISV Mathcounts Club 22-23}
\date{December 6, 2022}

%\maketitle
%~~~~~~~~~~~~~~~~~~~~~~~~~~~~~~~~~~~~~~~~~~~~~~~~~~~~~~~~~~~~~~~~~~~~~~~~~~~~~~
% Informations
%\title{TEMPLATE}

%\titlegraphic{assets/gkg.png} %change this to your preferred logo or image(the image is located on the top right corner).
%~~~~~~~~~~~~~~~~~~~~~~~~~~~~~~~~~~~~~~~~~~~~~~~~~~~~~~~~~~~~~~~~~~~~~~~~~~~~~~

\begin{document}

% Generate title page
\begin{frame}
    \titlepage        
\end{frame}
% \setbeamertemplate{footline}[miniframes Madrid]

\setcounter{section}{6}

\section{Beginner Practice Problems}
\begin{probslide}[t]{}{}
    \begin{block}{} [Mathcounts]
    What is the sum of all two-digit multiples of three that have units digit 7?
    
    \end{block}
\end{probslide}
\begin{probslide}[t]{}{}
    \begin{block}{} [Mathcounts]
        Using the digits 4, 5, 6, 7, 8 and 9 to form two three-digit numbers, with no digit being used more than once, what is the greatest possible positive difference that can be obtained from the two numbers?
    
    \end{block}
\end{probslide}
\begin{probslide}[t]{}{}
    \begin{block}{} Find the last digit of $7^{100}$.

    \end{block}
\end{probslide}
\begin{probslide}[t]{}{}
    \begin{block}{}
    Find the value of GCD(LCM(8, 832), LCM(12, 560), where GCD$(a,b)$ denotes the greatest common divisor of $a$ and $b$ and LCM$(a,b)$ denotes the least common multiple of $a$ and $b$.
    
    \end{block}
\end{probslide}
\begin{probslide}[t]{}{}
    \begin{block}{} (Mathcounts) What is the sum of all positive integer values of $n$ for which $\frac{n+8}{n}$ is an integer?

    \end{block}
\end{probslide}
\begin{probslide}[t]{}{}
    \begin{block}{} (Mathcounts) What is the least natural number, greater than $1$, that is a factor of $11000+1100+11$?

    \end{block}
\end{probslide}

\section{Intermediate Practice Problems}
\begin{probslide}[t]{}{}
    \begin{block}{} (1992 AHSME, \#17). The two-digit integers form 19 to 92 are written consecutively to form the large integer \[
    N = 192021 \cdots 909192.
\]
Suppose that $3^k$ is the highest power of 3 that is a factor of $N$. What is $k$?

    \end{block}
\end{probslide}
\begin{probslide}[t]{}{}
    \begin{block}{} (2000 AMC 12, \#18). In year $N$, the 300th day of the year is a Tuesday. In year $N$+1, the 200th day is also a Tuesday. On what day of the week did the 100th day of the year $N-1$ occur?
    
    \end{block}
\end{probslide}
\begin{probslide}[t]{}{}
    \begin{block}{} (2000 AMC 12, \#9) Mrs. Walter gave an exam in a mathematics class of five students. She
entered the scores in random order into a spreadsheet, which recalculated the class average
after each score was entered. Mrs. Walter noticed that after each score was entered, the
average was always an integer. The scores (listed in ascending order) were 71, 76, 80, 82, and 91.
What was the last score Mrs. Walter entered?

    \end{block}
\end{probslide}
\begin{probslide}[t]{}{}
    \begin{block}{} (Zoller) Find the number of integers $n, 1 \leq n \leq 25$ such that $n^2 + 3n + 2$ is divisible by 6.

    \end{block}
\end{probslide}
\begin{probslide}[t]{}{}
    \begin{block}{} (Zoller) If $n!$ denotes the product of the integers 1 through $n$, what is the remainder when
$(1! + 2! + 3! + 4! + 5! + 6! + ...)$ is divided by 9?

    \end{block}
\end{probslide}
\begin{probslide}[t]{}{}
    \begin{block}{} (Zoller) Find the sum of all possible values of $a+b$, where substituting digits $a$ and $b$ in $30a0b03$ results in the integer being divisible by 13.

    \end{block}
\end{probslide}
\begin{probslide}[t]{}{}
    \begin{block}{} (Zoller) When 30! is computed, it ends in 7 zeroes. Find the digit that immediately proceeds these zeroes.

    \end{block}
\end{probslide}

\begin{probslide}[t]{}{}
    \begin{block}{} (Mathcounts State Sprint Round, 2014) In base 5, what is the value of $27_{10} \cdot 314_{5}$?

    \end{block}
\end{probslide}

\begin{probslide}[t]{}{}
    \begin{block}{} (Mathcounts State Sprint Round, 2014) How many digits are in the integer representation of $2^{30}$?

    \end{block}
\end{probslide}

\begin{probslide}[t]{}{}
    \begin{block}{} (Mathcounts State Sprint Round, 2014) What is the remainder when $11^{12}$ is divided by 13?

    \end{block}
\end{probslide}
\begin{probslide}[t]{}{}
    \begin{block}{} (Mathcounts State Sprint Round, 2018) The base four representation of $p = 3 + \frac{0}{4} + \frac{2}{4^2} + \frac{1}{4^3}$ is $p = 3.021_{4}$. In base eight, $p = 3.AB_{8}$. What is the value of $A + B$? 

    \end{block}
\end{probslide}


\begin{probslide}[t]{}{}
    \begin{block}{} (Mathcounts Chapter Sprint Round, 2013) What fraction of the first 100 triangular numbers are evenly divisible by 7? Express your answer as a common fraction.

    \end{block}
\end{probslide}

\begin{probslide}[t]{}{}
    \begin{block}{} (Mathcounts State Sprint Round, 2013) If $12_{3} + 12_{5} + 12_{7} + 12_{9} + 12_{x} = 101110_{2}$, what is the value of $x$, the base of the fifth term?

    \end{block}
\end{probslide}


% Topic: Modular Arithmetic, Difficulty: 6, Answer: 59
\begin{probslide}[t]{}{}
    \begin{block}{}(Mathcounts Handbook 2015-16)
    When organizing her pens, Faith notices that when she puts them in groups of 3, 4, 5, or 6, she is always one pen short of being able to make full groups. If Faith has between 10 and 100 pens, how many pens does she have?
     
    \end{block}
\end{probslide}

%----------------------------------------------------------
\section{Challenge Practice Problems}
\begin{probslide}[t]{}{}
    \begin{block}{} (2015 AIME I, \#3) There is a prime number $p$ such that $16p+1$ is the cube of a positive integer. Find $p$.
    
    \end{block}
\end{probslide}
\begin{probslide}[t]{}{}
    \begin{block}{} (2014 AIME I, \#8) The positive integers $N$ and $N^2$ both end in the same sequence of four digits $abcd$ when written in base 10, where digit $a$ is nonzero. Find the three-digit number $abc$.
  
    \end{block}
\end{probslide}
\begin{probslide}[t]{}{}
    \begin{block}{} (2004 AIME 2, \#10) Let $S$ be the set of integers between 1 and $2^{40}$ that contain two 1’s when written in base 2. If the probability that a random integer from $S$ is divisible by 9 can be expressed as $\frac{a}{b}$, where $a$ and $b$ are relatively prime, find $a+b$.

    \end{block}
\end{probslide}

\begin{probslide}[t]{}{}
    \begin{block}{} (2010 AIME II, \#3) Let $K$ be the product of all factors $(b-a)$ (not necessarily distinct) where $a$ and $b$ are integers satisfying $1 \leq a \le b \leq 20$. Find the greatest positive integer $n$ such that $2^{n}$ divides $K$.

    \end{block}
\end{probslide}



\end{document}
