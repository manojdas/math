\documentclass[11pt,twoside]{scrartcl}
\usepackage{mdas}
\usepackage{adjustbox}
\setlist{nosep}

\begin{document}
\title{Burnside Lemma}

\author{Manoj Das}
\org{Manoj Math Notes}
\date{\today}

\maketitle
\section{Introduction}

\section{Definitions}
\begin{example}
    How many ways are there to color the vertices of a square using red and blue colors?
\end{example}

\begin{table}[h!]
    \centering
\begin{tabular}{|c|c|}
    \hline
    Class 1 &
    \begin{adjustbox}{margin = 0 5 0 5, valign=c}
    \begin{asy}
        unitsize(1 cm);
        picture pic1;
        real r = 0.1;
        
        draw(pic1, (0,0)--(0,1)--(1,1)--(1,0)--cycle, gray(0.7));
        fill(pic1, circle((0,0), r), blue);
        fill(pic1, circle((0,1), r), blue);
        fill(pic1, circle((1,0), r), blue);
        fill(pic1, circle((1,1), r), blue);
        
        add(pic1);
        label("$C_1$", (0.5,-0.5));
    \end{asy}
    \end{adjustbox}

    \\
    \hline
    Class 2 & 
    \begin{adjustbox}{margin = 0 5 0 5, valign=c}
        \begin{asy}
            unitsize(1 cm);
            picture pic1;
            real r = 0.1;
            
            draw(pic1, (0,0)--(0,1)--(1,1)--(1,0)--cycle, gray(0.7));
            fill(pic1, circle((0,0), r), red);
            fill(pic1, circle((0,1), r), blue);
            fill(pic1, circle((1,0), r), blue);
            fill(pic1, circle((1,1), r), blue);
            
            add(pic1);
            add(shift(1*up)*shift(2.5*right)*rotate(270)*pic1);
            
            add(shift(1*up)*shift(6*right)*rotate(180)*pic1);
            
            add(shift(8.5*right)*rotate(90)*pic1);
            label("$C_2$", (0.5,-0.5));
            label("$C_3$", (3.0,-0.5));
            label("$C_4$", (5.5,-0.5));
            label("$C_5$", (8,-0.5));
            
            \end{asy}
            
    \end{adjustbox} 


    \\
    \hline
    Class 3 & 
    \begin{adjustbox}{margin = 0 5 0 5, valign=c}
        \begin{asy}
            unitsize(1 cm);
            picture pic1;
            real r = 0.1;
            
            draw(pic1, (0,0)--(0,1)--(1,1)--(1,0)--cycle, gray(0.7));
            fill(pic1, circle((0,0), r), red);
            fill(pic1, circle((0,1), r), blue);
            fill(pic1, circle((1,0), r), red);
            fill(pic1, circle((1,1), r), blue);
            
            add(pic1);
            add(shift(1*up)*shift(2.5*right)*rotate(270)*pic1);
            
            add(shift(1*up)*shift(6*right)*rotate(180)*pic1);
            
            add(shift(8.5*right)*rotate(90)*pic1);
            label("$C_6$", (0.5,-0.5));
            label("$C_7$", (3.0,-0.5));
            label("$C_8$", (5.5,-0.5));
            label("$C_9$", (8,-0.5));
            
            \end{asy}
            
    \end{adjustbox} 

    \\
    \hline
    Class 4 & 
    \begin{adjustbox}{margin = 0 5 0 5, valign=c}
        \begin{asy}
            unitsize(1 cm);
            picture pic1;
            real r = 0.1;
            
            draw(pic1, (0,0)--(0,1)--(1,1)--(1,0)--cycle, gray(0.7));
            fill(pic1, circle((0,0), r), red);
            fill(pic1, circle((0,1), r), blue);
            fill(pic1, circle((1,0), r), blue);
            fill(pic1, circle((1,1), r), red);
            
            add(pic1);
            add(shift(1*up)*shift(2.5*right)*rotate(270)*pic1);
            
            label("$C_{10}$", (0.5,-0.5));
            label("$C_{11}$", (3.0,-0.5));
            
            \end{asy}
            
    \end{adjustbox} 

    \\
    \hline
    Class 5 & 
    \begin{adjustbox}{margin = 0 5 0 5, valign=c}
        \begin{asy}
            unitsize(1 cm);
            picture pic1;
            real r = 0.1;
            
            draw(pic1, (0,0)--(0,1)--(1,1)--(1,0)--cycle, gray(0.7));
            fill(pic1, circle((0,0), r), red);
            fill(pic1, circle((0,1), r), blue);
            fill(pic1, circle((1,0), r), red);
            fill(pic1, circle((1,1), r), red);
            
            add(pic1);
            add(shift(1*up)*shift(2.5*right)*rotate(270)*pic1);
            
            add(shift(1*up)*shift(6*right)*rotate(180)*pic1);
            
            add(shift(8.5*right)*rotate(90)*pic1);
            label("$C_{12}$", (0.5,-0.5));
            label("$C_{13}$", (3.0,-0.5));
            label("$C_{14}$", (5.5,-0.5));
            label("$C_{15}$", (8,-0.5));
            
            \end{asy}
            
    \end{adjustbox} 

    \\
    \hline
    Class 6 &
    \begin{adjustbox}{margin = 0 5 0 5, valign=c}
    \begin{asy}
        unitsize(1 cm);
        picture pic1;
        real r = 0.1;
        
        draw(pic1, (0,0)--(0,1)--(1,1)--(1,0)--cycle, gray(0.7));
        fill(pic1, circle((0,0), r), red);
        fill(pic1, circle((0,1), r), red);
        fill(pic1, circle((1,0), r), red);
        fill(pic1, circle((1,1), r), red);
        
        add(pic1);
        label("$C_{16}$", (0.5,-0.5));
    \end{asy}
    \end{adjustbox} \\
    \hline

\end{tabular}

\caption{\label{tab:square-colors}$2^4$ red-blue colorings of a square vertices. Colorings within a class are equivalent with respect to rotations and reflections.}
\end{table}

Table \ref{tab:square-colors} lists all the different colorings. We have organized the colorings into different classes where elements of class can be converted into one another using rotations; in other words, are equivalent with respect to rotations and reflections. For example, $C_2$ when rotated by $90^\circ, 180^\circ, 270^\circ,$ yields $C_3, C_4, C5$ respectively.

Next, let's consider the various symmetries in this example:
\begin{enumerate}
    \item Identity: initial position or no action. Let us denote this as $e$.
    \item Rotations: we can rotate by $90^\circ, 180^\circ,$ or $270^\circ$. Let's denote these as $\pi_1, \pi_2, \pi_3$.
    \item Reflections: we can reflect horizontally, vertically, across the north-east diagonal, or across the north-west diagonal. Let's denote these as $r_1, r_2, r_3, r_4$ respectively.
\end{enumerate}
Let's illustrate how these symmetries or actions work:
\begin{itemize}
    \item $e(C_x) = C_x$ for all $x$.
    \item $\pi_1(C_2) = C_3, \pi_1(C_3) = C_4, \ldots$, $p_1(C_{10}) = C_{11}$, $\pi_1(C_{12}) = C_{13}, \pi_1(C_{13}) = C_{14}, \ldots$.
    \item $\pi_2(C_2) = C_4, \pi_2(C_3) = C_5, \ldots$, $\pi_2(C_{12}) = C_{14}, \pi_2(C_{13}) = C_{15}, \ldots$.
    \item $r_1(C_6) = C_8$, $r_2(C_7) = C_9$.
    \item $r_3(C_3) = C_5$, $r_5(C_2) = C_4$.
    \item $r_3(C_{10}) = C_{10}$, $r_4(C_{10}) = C_{10}$.  
\end{itemize}

Let's use this example to define some terminology.

\begin{definition}[Set $X$]
    Let's denote the different configurations by $x$, and the set of all configurations by $X$.
\end{definition}
In this example, each coloring $C_0, \ldots, C_{16}$ is a configuration and $X = \{C_0, C_1, C_2, \ldots, C_{16}\}$.

\begin{definition}[Group of Actions/Symmetries, $G$]
We call the group of actions or symmetries as group of actions or group of symmetries and denote it as $G$.    
\end{definition}

In this example, $G = \{e, \pi_1, \pi_2, \pi_3, r_1, r_2, r_3, r_4 \}$. The operation $\cdot$ for this group is the composition operation. That is $g_1 \cdot g_2$ is apply $g_2$ to the result of applying $g_1$ to $x$.

This is a Group because:
\begin{itemize} 
    \item \textbf{Identity: } $e$ is identity because composing it with any other action gives the same action.
    \item \textbf{Associativity: } We can verify that $(g_1 \cdot g_2) \cdot g_3 = g_1 \cdot (g_2 \cdot g_3)$.
    \item \textbf{Inverse: } We can verify that each action has an inverse. For example, $\pi_3 = \pi_1^{-1}$.
\end{itemize}

\begin{definition}[Orbit]
    An Orbit of $x$, denoted as $O_x$, is all the configurations $x'$ that can be reached from $x$ by applying a combination of actions in $G$. That is
    \[O_x = \{x' \mid \exists g \in G, g \cdot x = x' \}. \]
\end{definition}
In this example, the orbit of $C_2$, $O_{C_2} = \{C_2, C_3, C_4, C_5\}.$ In fact, each class in Table \ref{tab:square-colors} is an Orbit. That leads us to the following observation:
\begin{remark}
    The number of distinct configurations (w.r.t symmetry) is same as the number of orbits. In our example, Class 1 to Class 6 each correspond to a distinct coloring as well as to an orbit.
\end{remark}

It is easy to see that an orbit is a closed group, because:
\begin{itemize}
    \item If $g_1 \cdot x_a = x_b$ and $g_2 \cdot x_b = x_c$, then $g_2 \cdot g_1 \cdot x_a = x_c$. In other words if $a, b$ are in the same orbit and $b, c$ are in the same orbit then $a, c$ are in the same orbit.
    \item If $a, b$ are in the same orbit, and $a, c$ are not in the same orbit, then $b, c$, can not be in the same orbit. Otherwise, we should be able to go from $a$ to $b$ and then $b$ to $c$.
\end{itemize}

\begin{definition}[Fixed Point]
    A configuration $x$ is a fixed point of an action $g$, if $g \cdot x = x$. The set of fixed points for $g$ is denoted as $fix(g)$. That is 
    \[fix(g) = \{x \mid g\cdot x = x\}. \]
\end{definition}

In this example,
\begin{itemize}
    \item $fix(\pi_1) = \{C_1, C_{16}\}$.
    \item $fix(\pi_2) = \{C_1, C_{10}, C_{11}, C_{16}\}$. 
\end{itemize}

Table \ref{tab:square-colors-1} lists all the fixed points for all the actions.

\begin{table}[ht!]
    \centering
    \begin{tabular}{|c|c|c|}
        \hline
        $g$ & $fix(g)$ & $|fix(g)|$ \\
        \hline
        e & $C_1, C2, C_3, \ldots, C_{16}$ & 16 \\
        \hline
        $\pi_1$ & $C_1, C_{16}$ & 2 \\
        $\pi_2$ & $C_1, C_{10}, C_{11}, C_{16}$ & 4 \\
        $\pi_3$ & $C_1, C_{16}$ & 2 \\
        \hline
        $r_1$ & $C_1, C_7, C_9, C_{16}$& 4\\
        $r_2$ & $C_1, C_6, C_8, C_{16}$& 4\\
        $r_3$ & $C_1, C_2, C_4, C_{10}, C_{11}, C_{13}, C_{15}, C_{16}$& 8\\
        $r_4$ & $C_1, C_3, C_5, C_{10}, C_{11}, C_{12}, C_{14}, C_{16}$& 8\\
        \hline
        \multicolumn{2}{|r|}{Total}&48\\
        \hline

    \end{tabular}
    \caption{\label{tab:square-colors-1} Fixed points for the various actions.}
\end{table}
\begin{definition}[Stabilizer]
    A stabilizer for $x$ is a an action $g$ that fixes it, that is $g \cdot x = x$. The set of actions that fixes $x$ is called the Stabilizer for $x$ and denoted as $S_x$. That is
    \[S_x = \{g \mid g \cdot x = x\}. \]

\end{definition}

In this example,
\begin{itemize}
    \item $S_{C_2} = \{e, r_3\}$ because $C_2$ is unchanged by reflection about the north-west axis; and other than $e$ every other action changes it.
    \item $fix(\pi_2) = \{C_1, C_{10}, C_{11}, C_{16}\}$. 
\end{itemize}


Table \ref{tab:square-colors-2} lists all the Stabilizers for all the configurations.

\begin{table}[h!]
    \centering
    \begin{tabular}{|c|c|c|}
        \hline
        $x$ & Orbit $O_x$ & Stabilizers $S_x$ \\
        \hline
        $C_1$ & $C_1$ & $e, \pi_1, \pi_2, \pi_3, r_1, r_2, r_3, r_4$ \\
        \hline
        $C_2$ & $C_2, C_3, C_4, C_5$ & $e, r_3$ \\
        $C_3$ & $C_2, C_3, C_4, C_5$ & $e, r_4$ \\
        $C_4$ & $C_2, C_3, C_4, C_5$ & $e, r_3$ \\
        $C_5$ & $C_2, C_3, C_4, C_5$ & $e, r_4$ \\
        \hline
        $C_6$ & $C_6, C_7, C_8, C_9$ & $e, r_2$ \\
        $C_7$ & $C_6, C_7, C_8, C_9$ & $e, r_1$ \\
        $C_8$ & $C_6, C_7, C_8, C_9$ & $e, r_2$ \\
        $C_9$ & $C_6, C_7, C_8, C_9$ & $e, r_1$ \\
        \hline
        $C_{10}$ & $C_{10}, C_{11}$ & $e, \pi_2, r_3, r_4$ \\
        $C_{11}$ & $C_{10}, C_{11}$ & $e, \pi_2, r_3, r_4$ \\
        \hline
        $C_{12}$ & $C_{12}, C_{13}, C_{14}, C_{15}$ & $e, r_4$ \\
        $C_{13}$ & $C_{12}, C_{13}, C_{14}, C_{15}$ & $e, r_3$ \\
        $C_{14}$ & $C_{12}, C_{13}, C_{14}, C_{15}$ & $e, r_4$ \\
        $C_{15}$ & $C_{12}, C_{13}, C_{14}, C_{15}$ & $e, r_3$ \\
        \hline
        $C_{16}$ & $C_{16}$ & $e, \pi_1, \pi_2, \pi_3, r_1, r_2, r_3, r_4$ \\
        \hline
    \end{tabular}
    \caption{\label{tab:square-colors-2} Orbits and Stabilizers for the various configurations.}
\end{table}
\section{Orbit Stabilizer Theorem}
If we look at Table \ref{tab:square-colors-2} we notice some interesting things:
\begin{itemize}
    \item All $x$ in one orbit, have the same number of stabilizers,
    \item this number when multiplied by the size of that orbit is same across all orbits,
    \item and it is the same as the size of the group $G$.
\end{itemize}
This happens to be no coincidence.
\begin{theorem}[Orbit Stabilizer Theorem]
    For all $x$, the product of the size of its orbit and the size of stabilizers is equal to the size of the group. That is 
    \[|O_x| \cdot |S_x| = |G|.\]
\end{theorem}
This theorem can be proved using group theory. However, we will not go into it. 

This theorem is need to prove Burnside Lemma. It may also be useful to quickly count the number of symmetries.

\begin{example}
    How many rotational symmetries are there for a cube?
\end{example}
\begin{proof}[Solution 1: Using Orbit Stabilizer Theorem]
    Let's use a simple configuration where we label the faces and see which face is on the top. It is easy to see that all 6 faces can be rotated into top position starting from any top face $x$. Therefore, $|O_x| = 6$. Next, let's see which $g$ stabilizes $x$, that is leave $x$ as the top face. It's easy to see that only the rotation around the axis that passes from center of the top face to center of bottom face, fixes $x$. There are 3 such rotations: $90, 180, 270$. These plus $e$ are $S_x$. So $|S_x| = 4$. Then from the Orbit Stabilizer theorem, we have
    \[ |G| = |S_x| \cdot |O_x| = 4 \cdot 6 = \boxed{24}\].
\end{proof}

\begin{proof}[Solution 2: Direct Counting]
    \quad
    \begin{center}
        \begin{tabular}{|l|c|l|c|}
            \hline
            Axis& \# Axes& Symmetries&\# symmetries \\
            \hline
            Identity & & & 1 \\
            \hline
            Face-centers&$\frac{6}{2}=3$ & $90^\circ, 180^\circ, 270^\circ$ & $3\cdot3 = 9$ \\
            \hline
            Edge-midpoints&$\frac{12}{2}=6$ & $180^\circ$ & $6\cdot 1 = 6$ \\
            \hline
            Long-diagonals&$\frac{8}{2}=4$ & $120^\circ, 240^\circ$ & $4\cdot 2 = 8$ \\
            \hline
            \multicolumn{3}{|r|}{Total} & \boxed{24} \\
            \hline
        \end{tabular}
    \end{center}
\end{proof}
\section{Burnside Lemma}
\begin{theorem}[Burnside Lemma]
    Let $X / G$ be the set of orbits of $X$. Then
    \[|X / G| = \frac{1}{|G|} \sum_{g \in G} fix(g). \]
\end{theorem}

In other words, the number of orbits, which is the same as number of distinct configurations, can be computed by summing the $fix(g)$ over all $g \in G$ and then dividing by the size of $G$.

\begin{proof}
    We prove this by double counting the number of fix-edges between $g \in G$ and $x \in X$. First, we count over $g \in G$. This gives us $sum_{g \in G} fix(g)$.

    Next, we count over the $x \in X$. However, we group the $x$ in orbits for counting. From the Orbit Stabilizer theorem, we have that $|O_x| \cdot |S_x| = |G|$. Therefore, each orbit contributes $|G|$ edges. And since we have $|X / G|$ orbits, the total number of edges is $|X / G| \cdot |G|$ edges.

    Equating the two and dividing both sides by $|G|$ we get the desired result.
\end{proof}

\begin{example}
    How many ways are there to color the vertices of a square using red and blue colors that are distinct with respect to rotations and reflections?
\end{example}

While we have seen the answer to this already above, let's finish our example using Burnside Lemma. From Table \ref{tab:square-colors-1} we have the total number of fixed points as 48 and we have $|G| = 8$. Therefore, the number of distinct colorings is
\[ \frac{1}{|G|} \cdot \sum fix(g) = \frac{1}{8} 48 = \boxed{6}, \]
which is what we had listed in Table \ref{tab:square-colors}.

\begin{center}
    \begin{asy}
        unitsize(1cm);
        int edge[][] = {
            {1,1,1,1,1,1,1,1},
            {1,0,0,0,0,0,1,0},
            {1,0,0,0,0,0,0,1},
            {1,0,0,0,0,0,1,0},
            {1,0,0,0,0,0,0,1},
            {1,0,0,0,0,1,0,0},
            {1,0,0,0,1,0,0,0},
            {1,0,0,0,0,1,0,0},
            {1,0,0,0,1,0,0,0},
            {1,0,1,0,0,0,1,1},
            {1,0,1,0,0,0,1,1},
            {1,0,0,0,0,0,0,1},
            {1,0,0,0,0,0,1,0},
            {1,0,0,0,0,0,0,1},
            {1,0,0,0,0,0,1,0},
            {1,1,1,1,1,1,1,1},
        };
        int cl[] = {0, 1, 5, 9, 11, 15, 16};

        string g[] = {"$e$", "$\pi_1$", "$\pi_2$", "$\pi_3$", "$r_1$", "$r_2$", "$r_3$", "$r_4$"};
        real r = 0.35;
        real cx = 0.40;
        real y = 4;

        for (int i = 0; i < 6; ++i) {
            draw((cl[i]-cx, -0.5)--((cl[i+1]-1)+cx, -0.5)--((cl[i+1]-1)+cx, 0.5)--(cl[i]-cx, 0.5)--cycle);
        }
        for (int i = 0; i < 16; ++i) {
            for (int j = 0; j < 8; ++j) {
                if (edge[i][j] == 1) {
                    draw((i,0)--(2*j,y));
                }
            }
        }
        for (int i = 0; i < 16; ++i) {
            filldraw(circle((i, 0), r), white, black);
            label("$C_{" + (string)(i+1) + "}$", (i, 0));
        }
        for (int i = 0; i < 8; ++i) {
            filldraw(circle((2*i, y), r), white, black);
            label(g[i], (2*i, y));
        }
    \end{asy}
\end{center}
\clearpage

\section{Problems}
\begin{problem}
    How many ways the faces of a cube can be colored using 3 colors, red, blue, green, that are distinct with respect to rotations.

    \begin{sketch}
        Let's count the $fix(g)$.
        \begin{center}
            \begin{tabular}{|l|l|p{5cm}|c|}
                \hline
                Axis & Rotations &  & Fixes \\
                \hline
                None & None & Freely choose all faces & $1 \cdot 3^6$ \\
                \hline
                Face-Centers & $90^\circ, 270^\circ$ & Say top-bottom axis: can choose top, bottom, and one side face. Other side faces are determined. & $3 \cdot 2 \cdot 3^3$ \\
                \hline
                Face-Centers & $180^\circ$ & Say top-bottom axis: can choose top, bottom, and two adjacent side faces. The other side faces are determined. & $3 \cdot 1 \cdot 3^4$ \\
                \hline
                Edge-Midpoints & $180^\circ$ & If we choose a color for a face adjacent to the edge, the other 3 faces that it can be rotated into are determined. Can freely choose the 2 side faces. & $6 \cdot 1 \cdot 3^3$ \\
                \hline
                Long-Diagonal & $120^\circ, 240^\circ$ & There are 2 set of faces that rotate into each other, so we have 2 choices. & $4 \cdot 2 \cdot 3^2$ \\
                \hline
                \multicolumn{3}{|r|}{total} & 1368 \\
                \hline
            \end{tabular}
        \end{center}
        We have $|G| = 24$. Therefore, the number of distinct colorings is
        \[\frac{1}{24} \cdot 1368 = \boxed{57}.\]
    \end{sketch}
\end{problem}

\begin{problem}
    How many ways can color the squares of a $4 \times 4$ chessboard using black and white that are distinct with respect to rotations and reflections.
    \begin{sketch}
        First let's look at the symmetries. We have the identity, $e$, 3 rotations around the center by $90^\circ, 180^\circ, 270^\circ$, $\pi_1, \pi_2, pi_3$, and four reflections about the horizontal, vertical, north-east and north-west diagonals, $r_1, r_2, r_3, r_4$.

        Now, let's count the fixes.

        \begin{itemize}
            \item $e$: We can choose a color for each square freely. So $2^{16}$ fixes.
            \item $\pi_1 (90^\circ \text{ rotation}), \pi_3 (270^\circ \text{ rotation})$: $\pi_3$ is symmetrical version of $\pi_1$, so let's just consider $\pi_1$. As we can see in the diagram each square goes to 4 other square and we can choose 4 colors independently. So $2\cdot 2^4$ fixes.
            \begin{center}
                \begin{asy}
                    unitsize(1cm);
                    string l[][] = {
                        {"1", "3", "2", "1"},
                        {"2", "4", "4", "3"},
                        {"3", "4", "4", "2"},
                        {"1", "2", "3", "1"},

                    };
                    for (int i = 0; i < 4; ++i)
                    for (int j = 0; j < 4; ++j) {
                        draw((i,0)--(i,4)^^(0,j)--(4,j));
                        label(l[j][i], (i+0.5, j+0.5));
                    }
                    draw((4,0)--(4,4)^^(0,4)--(4,4));
                \end{asy}
            \end{center}
            \item $\pi_2 (180^\circ \text{ rotation})$: As we can see in the diagram below, each square goes to another and so we can choose colors for 8 independently. So the number of fixes is $1 \cdot 2^8$.
            \begin{center}
                \begin{asy}
                    unitsize(1cm);
                    string l[][] = {
                        {"4", "3", "2", "1"},
                        {"8", "7", "6", "5"},
                        {"5", "6", "7", "8"},
                        {"1", "2", "3", "4"},

                    };
                    for (int i = 0; i < 4; ++i)
                    for (int j = 0; j < 4; ++j) {
                        draw((i,0)--(i,4)^^(0,j)--(4,j));
                        label(l[j][i], (i+0.5, j+0.5));
                    }
                    draw((4,0)--(4,4)^^(0,4)--(4,4));
                \end{asy}
            \end{center}
            \item $r_1 (\text{ horizontal reflextion}), r_3 (\text{ vertical reflextion})$: As we can see in the diagram below for $r_1$, each square goes to another and so we can choose colors for 8 independently. So the number of fixes is $2 \cdot 2^8$.
            \begin{center}
                \begin{asy}
                    unitsize(1cm);
                    string l[][] = {
                        {"1", "2", "3", "4"},
                        {"5", "6", "7", "8"},
                        {"5", "6", "7", "8"},
                        {"1", "2", "3", "4"},

                    };
                    for (int i = 0; i < 4; ++i)
                    for (int j = 0; j < 4; ++j) {
                        draw((i,0)--(i,4)^^(0,j)--(4,j));
                        label(l[j][i], (i+0.5, j+0.5));
                    }
                    draw((4,0)--(4,4)^^(0,4)--(4,4));
                    draw((-0.5,2)--(4.5,2), red+dashed);
                \end{asy}
            \end{center}
            \item $r_3 (\text{ north-east reflection}), r_4 (\text{ north-west reflextion})$: As we can see in the diagram below for $r_3$, the triangle above the diagonal maps below the diagonal, allowing us to choose 6 colors independently; also, we can choose the 4 colors on the diagonal independently. So the number of fixes is $2 \cdot 2^{10}$.
            \begin{center}
                \begin{asy}
                    unitsize(1cm);
                    string l[][] = {
                        {"7", "6", "4", "1"},
                        {"6", "8", "5", "2"},
                        {"4", "5", "9", "3"},
                        {"1", "2", "3", "10"},

                    };
                    for (int i = 0; i < 4; ++i)
                    for (int j = 0; j < 4; ++j) {
                        draw((i,0)--(i,4)^^(0,j)--(4,j));
                        label(l[j][i], (i+0.5, j+0.5));
                    }
                    draw((4,0)--(4,4)^^(0,4)--(4,4));
                    draw((-0.5,-0.5)--(4.5,4.5), red+dashed);

                \end{asy}
            \end{center}
        \end{itemize}
        Then by Burnside Lemma, the number of distinct colorings is 
        \[\frac{1}{8}(2^{16} + 2\cdot 2^4 + 1 \cdot 2^8 + 2 \cdot 2^8 + 2 \cdot 2^{10}) = \frac{68384}{8} = \boxed{8548}.\]
    \end{sketch}
\end{problem}

\begin{problem}
    How many ways we can color the vertices of an equilateral triangle using 4 colors that are distinct with respect to reflections and rotations.
    \begin{sketch}
    First, let's list the symmetries: we have identity, $e$, rotations by $120^\circ, 240^\circ$, $\pi_1, \pi_2$, and reflections over the medians, $r_1, r_2, r_3$.
    
    Now, let's count the fixes:
    \begin{itemize}
        \item $e$: $4^3$.
        \item $\pi_1, \pi_2$: Each vertex gets rotated into every every other vertex. So we can independently choose only 1. Therefore, $2 \cdot 4$.
        \item $r_1, r_2, r_3$: The 2 vertices are not on the median get reflected into one other. So we independenltly choose color for the vertex on the median and one of the other vertices. Therefore, $3 \cdot 4^2$.
    \end{itemize}
    So by Burnside Lemma, the number of colorings is
    \[\frac{1}{6}(4^3 + 2 \cdot 4 + 3 \cdot 4^2) = \boxed{20} .\]
    \end{sketch}
\end{problem}
\begin{problem}[PUMaC 2012, Combinatorics A, \# 6]
    Two white pentagonal pyramids, with side lengths all the same, are glued to each other
at their regular pentagon bases. Some of the resulting 10 faces are colored black. How many
rotationally distinguishable colorings may result?

\begin{sketch}
There is 1 identity rotation, 4
$(72^\circ, 144^\circ, 216^\circ, 288^\circ)$ rotations about the apex vertices, and 5 $180^\circ$
rotations about the nonapex vertices that send the shape to itself, and these form the symmetry group of the figure.
Respectively, these fix 10, 2 (can select color for top and bottom independently), and 5 orbits (each face rotates to another only) of faces. By the Burnside’s lemma, the answer
is 
\[ \frac{1}{10}(2^{10}+4\cdot2^2+5\cdot2^5)  = \boxed{120} .\]

\end{sketch}
\end{problem}


\begin{problem}[2006 AIME II, \#8]
    There is an unlimited supply of congruent equilateral triangles made of colored paper. Each triangle is a solid color with the same color on both sides of the paper. A large equilateral triangle is constructed from four of these paper triangles. Two large triangles are considered distinguishable if it is not possible to place one on the other, using translations, rotations, and/or reflections, so that their corresponding small triangles are of the same color.


    Given that there are six different colors of triangles from which to choose, how many distinguishable large equilateral triangles may be formed?

    \begin{center}
        \begin{asy}
            unitsize(1cm);
            pair A,B; 
            A=(0,0); 
            B=(2,0); 
            pair C=rotate(60,A)*B; 
            pair D, E, F; 
            D = (1,0); 
            E=rotate(60,A)*D; 
            F=rotate(60,C)*E; 
            draw(C--A--B--cycle); 
            draw(D--E--F--cycle);
        \end{asy}
    \end{center}
\begin{sketch}
    There are 6 symmetries: identity, $e$, rotations of $120^\circ, 240^\circ$, let's call them $\pi_1, \pi_2$, reflections about the medians, let's call them $r_1, r_2, r_3$.

    Let's count the fixes for each:
    \begin{itemize}
        \item $e$: Each trinagle can be selected independently, that is $6^4$ ways.
        \item $\pi_1, \pi_2$: The triangle with constraints for $\pi_1$ is:
            \begin{center}
                \begin{asy}
                    unitsize(1cm);
                    pair A,B; 
                    A=(0,0); 
                    B=(2,0); 
                    pair C=rotate(60,A)*B; 
                    pair D, E, F; 
                    D = (1,0); 
                    E=rotate(60,A)*D; 
                    F=rotate(60,C)*E; 
                    draw(C--A--B--cycle); 
                    draw(D--E--F--cycle);

                    label("A", (A+D+E)/3);
                    label("A", (B+D+F)/3);
                    label("A", (C+E+F)/3);

                    label("B", (D+E+F)/3);
                \end{asy}
            \end{center}
            Therfore, for each there are $6^2$ fixes.
        \item $r_1, r_2, r_3$: The triangle with constraints for $r_1$ is:
        \begin{center}
            \begin{asy}
                unitsize(1cm);
                pair A,B; 
                A=(0,0); 
                B=(2,0); 
                pair C=rotate(60,A)*B; 
                pair D, E, F; 
                D = (1,0); 
                E=rotate(60,A)*D; 
                F=rotate(60,C)*E; 
                draw(C--A--B--cycle); 
                draw(D--E--F--cycle);
                draw((A-(F-A)*0.1)--(F + (F-A)*0.1), red+dashed);

                label("A", (A+D+E)/3);
                label("C", (B+D+F)/3);
                label("C", (C+E+F)/3);

                label("B", (D+E+F)/3);
            \end{asy}
        \end{center}  
        Therefore, for each there are $6^3$ fixes.      
    \end{itemize}
    By Burnside's lemma, the number of ways is
    \[\frac{1}{6} (6^4 + 2 \cdot 6^2 + 3 \cdot 6^3) = \boxed{336}.\]
\end{sketch}
\end{problem}
\begin{problem}[Osman Nal YouTube]
    Consider a $3\times 3$ square grid. How many different coloring patterns (upto rotation and reflection) with \textcolor{red}{$2$ red}, \textcolor{ForestGreen}{$3$ green}, and \textcolor{blue}{$4$ blue} squares?

\begin{sketch}
    First, let's count the fixes.
    \begin{itemize}
        \item $e$: Each cell can be colored independent of the other, so there are $ \binom{9}{2} \cdot \binom{7}{3} = 1260$ fixes.
        \item $\pi_1, \pi_3$: The grid with the constraints is:
            \begin{center}
                \begin{asy}
                    unitsize(1cm);
                    string l[][] = {
                        {"A", "B", "A"},
                        {"B", "C", "B"},
                        {"A", "B", "A"},
                    };
                    for (int i = 0; i < 3; ++i)
                    for (int j = 0; j < 3; ++j) {
                        draw((i,0)--(i,3)^^(0,j)--(3,j));
                        label(l[j][i], (i+0.5, j+0.5));
                    }
                    draw((3,0)--(3,3)^^(0,3)--(3,3));
                    // draw((-0.5,-0.5)--(3.5,3.5), red+dashed);
        
                \end{asy}
        \end{center}
        There are $4 A$s and $4 B$s, where as we have only one color, \textcolor{blue}{blue} with 4 squares. So its not possible to get a grid with these constraints and the number of fixes is $0$.
            \item $\pi_2$: The grid with the constraints is:
            \begin{center}
                \begin{asy}
                    unitsize(1cm);
                    string l[][] = {
                        {"A", "B", "C"},
                        {"D", "E", "D"},
                        {"A", "B", "C"},
                    };
                    for (int i = 0; i < 3; ++i)
                    for (int j = 0; j < 3; ++j) {
                        draw((i,0)--(i,3)^^(0,j)--(3,j));
                        label(l[j][i], (i+0.5, j+0.5));
                    }
                    draw((3,0)--(3,3)^^(0,3)--(3,3));
                    // draw((-0.5,-0.5)--(3.5,3.5), red+dashed);
        
                \end{asy}
        \end{center}
        Since, \textcolor{ForestGreen}{green} is the only color with odd number of squares, $E$ must be \textcolor{ForestGreen}{green}. Then we have 4 pairs of $A, B, C, D$ and we have to choose from \textcolor{blue}{$2$ blue}, \textcolor{red}{$1$ red}, \textcolor{ForestGreen}{$1$ green} pairs. This can be done in $\binom{4}{2} \cdot \binom{2}{1} = 12$ ways. So the number of fixes is $12$.
        
        \item $r_1, r_2$: The grid with the constraints for $r_1$ is:
        \begin{center}
            \begin{asy}
                unitsize(1cm);
                string l[][] = {
                    {"A", "B", "C"},
                    {"D", "E", "F"},
                    {"A", "B", "C"},
                };
                for (int i = 0; i < 3; ++i)
                for (int j = 0; j < 3; ++j) {
                    draw((i,0)--(i,3)^^(0,j)--(3,j));
                    label(l[j][i], (i+0.5, j+0.5));
                }
                draw((3,0)--(3,3)^^(0,3)--(3,3));
                // draw((-0.5,-0.5)--(3.5,3.5), red+dashed);
    
            \end{asy}
        \end{center}

        Now there are two cases:
        \begin{itemize}
            \item $D, E, F$ are all \textcolor{ForestGreen}{green}:  The reamining 3 pairs of $A, B, C$ can be chosen from \textcolor{blue}{$2$ blue}, \textcolor{red}{$1$ red} in $\binom{3}{2}$ ways.
            \item One of $D, E, F$ is \textcolor{ForestGreen}{green}: This one can be chosen in 3 ways and then one of the remaiining three pairs of $A, B, C$ can be chosen for green in 3 ways and then the colors for remaining 3 pairs can be chosen in $\binom{3}{2}$ ways. That is the number of ways is $3 \cdot 3 \cdot \binom{3}{2}$ ways.
        \end{itemize}
            Then the total number of ways is $\binom{3}{2} + 3 \cdot 3 \cdot \binom{3}{2} = 30$. And since $r_2$ is symmetric to $r_1$, it will also have same $30$ fixes. Therefore, the total fixes for $r_1, r_2 = 60$.
        
        \item $r_3, r_4$: The grid with the constraints for $r_3$ is:
        \begin{center}
            \begin{asy}
                unitsize(1cm);
                string l[][] = {
                    {"D", "C", "A"},
                    {"C", "E", "B"},
                    {"A", "B", "F"},
                };
                for (int i = 0; i < 3; ++i)
                for (int j = 0; j < 3; ++j) {
                    draw((i,0)--(i,3)^^(0,j)--(3,j));
                    label(l[j][i], (i+0.5, j+0.5));
                }
                draw((3,0)--(3,3)^^(0,3)--(3,3));
                draw((-0.5,-0.5)--(3.5,3.5), red+dashed);
    
            \end{asy}
        \end{center}
        As we can quickly see, the choices here are similar to the choices in the previous case, and so we have a total of $60$ fixes.
    \end{itemize}
    By Burnside's lemma, the number of distinct colorings upto rotations and reflextions is 
    \[\frac{1}{8} (1260 + 12 + 60 + 60) = \boxed{174}. \]
\end{sketch}
\end{problem}

\begin{problem}[2010 AIME II, \#10]
    Find the number of second-degree polynomials $f(x)$ with integer coefficients and integer zeros for which $f(0)=2010$.

    \begin{sketch}
        Let $f(x) = a(x-r)(x-s)$. Then $ars=2010=2\cdot3\cdot5\cdot67$. Finding distinct polynomials is to find the distinct triplets $(a, r, s)$ with $r, s$ being swappable.

        We can use Burnside's lemma. The two permutations that take a valid solution and yield another valid solution is the identity permutation, $e$ or $(a)(r)(s)$, and the permutation that swaps $r, s$, $(a)(rs)$.

        Let's count the number of fixes for both:
        \begin{itemize}
            \item $e$: First, the 4 primes can be divided among $a, r, s$ in $3^4$ ways. Then there are $1 + \binom{3}{2} = 4$ ways of having positive and negatives such that the overall product is positive. So we have $4 \cdot 3^4$ fixes.
            \item $(a)(r)(s)$: The only way we can have $r = s$ is to have $r = s = \pm 1$. Therefore, there are $2$ fixes.
        \end{itemize}
        By Burnside's lemma the number of distinct triples (with $r, s$ swappable) is
        \[\frac{1}{2}(4 \cdot 3^4 + 2) = \boxed{163}.\]
    \end{sketch}
\end{problem}

\begin{problem}
    How many ways are there to color the 8 regions of a three-set Venn Diagram with 3 colors such that each color is used at least once? Two colorings are considered the same if one can be reached from the other by rotation and/or reflection.

    \begin{sketch}
        First, note that the symmetry actions on the three-set Venn Diagram is the same as that on the equilateral trinagle connecting the centers of the 3 circles. They are identity, $e$, rotations of $120^\circ, 240^\circ$, $\pi_1, \pi_2$, and reflections about the 3 medians, $r_1, r_2, r_3$.

        \begin{center}
        \begin{asy}
            unitsize(1cm);
            real r = 1.75;
            pair A,B; 
            A=(0,0); 
            B=(2,0); 
            pair C=rotate(60,A)*B; 
            draw(C--A--B--cycle, gray(0.75)); 

            draw(circle(A, r));
            draw(circle(B, r));
            draw(circle(C, r));

        \end{asy}
        \end{center}
    Let's count the number of fixes with $n$ colors:
    \begin{itemize}
        \item $e$: $n^8$ fixes.
        \item $\pi_1, \pi_2$: The constraints are:
        \begin{center}
        \begin{asy}
            unitsize(1cm);
            real r = 1.75;
            pair A,B; 
            A=(0,0); 
            B=(2,0); 
            pair C=rotate(60,A)*B; 
            draw(C--A--B--cycle, gray(0.75)); 

            draw(circle(A, r));
            draw(circle(B, r));
            draw(circle(C, r));

            label("A", (C - (C - (A + B)/2)*(2.25)));
            label ("B", (A+B+C)/3);

            label("C", (C - (C - (A + B)/2)*(1.25)));
            label("C", (B - (B - (A + C)/2)*(1.25)));
            label("C", (A - (A - (C + B)/2)*(1.25)));

            label("D", (C + (C - (A + B)/2)*(1/2)));
            label("D", (B + (B - (A + C)/2)*(1/2)));
            label("D", (A + (A - (C + B)/2)*(1/2)));

        \end{asy}
        \end{center}
        Therefore, the number of fixes for each is $n^4$.
        \item $r_1, r_2, r_3$: The constraints for $r_1$ are:
        \begin{center}
            \begin{asy}
                unitsize(1cm);
                real r = 1.75;
                pair A,B; 
                A=(0,0); 
                B=(2,0); 
                pair C=rotate(60,A)*B; 
                draw(C--A--B--cycle, gray(0.75)); 
    
                draw(circle(A, r));
                draw(circle(B, r));
                draw(circle(C, r));
    
                label("A", (C - (C - (A + B)/2)*(2.25)));
                label ("B", (A+B+C)/3);
    
                label("D", (C - (C - (A + B)/2)*(1.25)));
                label("D", (B - (B - (A + C)/2)*(1.25)));
                label("C", (A - (A - (C + B)/2)*(1.25)));
    
                label("F", (C + (C - (A + B)/2)*(1/2)));
                label("F", (B + (B - (A + C)/2)*(1/2)));
                label("E", (A + (A - (C + B)/2)*(1/2)));
    
                draw((A + (A - (C + B)/2)*(1.1))--(A - (A - (C + B)/2)*(2)), red+dashed);
            \end{asy}
            \end{center}   
            Therefore, the number of fixes for each is $n^6$.     
    \end{itemize}
    So by Burnside lemma, the number of ways to color with at most $n$ colors accounting for rotation and reflection symmetries is
    \[f(n) = \frac{1}{6}(n^8 + 2 \cdot n^4 + 3 \cdot n^6) .\]

    We have $f(3) = 1485$, $f(2) = 80$, and $f(1) = 1$.

    By PIE, the number of ways to do this such that each color is used at least once is 
    \[f(3) - \binom{3}{2} \cdot f(2) + \binom{3}{1} \cdot f(1) = 1485 - 3\cdot80 + 3 = \boxed{1248}.\]
    \end{sketch}
\end{problem}
\clearpage
\section{Solution Sketches}
\makehints
\newpage
\section{References}
\begin{itemize}
    \item \url{https://www.youtube.com/watch?v=wvofB5tZz3Y}
    \item \url{https://www.youtube.com/watch?v=6kfbotHL0fs}
    \item \url{https://youtu.be/zQZ4kTiaqQI}
\end{itemize}
\end{document}
