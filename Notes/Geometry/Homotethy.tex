\documentclass[11pt,twoside]{scrartcl}
\usepackage{mdas}
% \usepackage[sexy, fancy, hints]{evan}


\begin{document}
\title{Homothety}
% If you contribute to the handout, put your name in comment here

\author{Manoj Das}
\org{Manoj Math Notes}
\date{December 8, 2020}

\maketitle
\section{Homothety}
\begin{definition}[Homothety]
    A \emph{homothety} (or dilation) is a mapping that takes a figure and creates an expanded or contracted version of that figure. More formally, givan a point $H$ and number $k, k \neq 0$, a homothety with center $H$ and coefficient $k$ maps point $A$ to a point $A'$ such that:
    \begin{enumerate}[(a)]
        \item Points $H$, $A$, and $A'$ are collinear.
        \item $\overrightarrow{HA'} = k \cdot \overrightarrow{HA}$.
    \end{enumerate}
    We denote such homothety by $\mathcal{H}(H, k)$.
\end{definition}
\begin{figure}[h]
    \begin{asy} 
        unitsize(0.05cm);
        pair A=(15,15),B=(30,15),C=(30,30),D=(15,30),a=(60,60),b=(120,60),c=(120,120),d=(60,120);
    
        dot((0,0));
        draw(d--(0,0)--b,dotted);
        draw(c--(0,0),dotted);
        label("$O$",(0,0),SW);
        draw(A--B--C--D--A);
        dot(A);
        label("$A$",A,SW);
        dot(B);
        label("$B$",B,SE);
        dot(C);
        label("$C$",C,NE);
        dot(D);
        label("$D$",D,NW);
        draw(a--b--c--d--a);
        dot(a);
        label("$A'$",a,SW);
        dot(b);
        label("$B'$",b,SE);
        dot(c);
        label("$C'$",c,NE);
        dot(d);
        label("$D'$",d,NW);
    \end{asy}        
\end{figure}
\begin{remark}
    Observe that $k = 1$ corresponds to identity (nothing happens) and $k = -1$ to central symmetry.
\end{remark}
\subsection{Properties of Homotheties}
\subsubsection{Collinearity}
If $\mathcal{H}(A) = A'$, then $H, A, A'$ are collinear.
\subsubsection{Preserves Angles}
A Homothety preserves angles. If $\mathcal{H}(A, B, C) = A', B', C'$, then $ \angle ABC = \angle A'B'C'$.
\begin{note}
    Can be easily proved using similar triangles. May be prove the next property first.
\end{note}

\subsubsection{Preserves Ratios}
A Homothety preserves ratios. If $\mathcal{H}(A, B, C) = A', B', C'$, then $AB:BC = A'B':B'C'$.
\begin{proof}
    From similar triangles $\triangle HAB$ and $\triangle HA'B'$, we get $A'B' = k \cdot AB$, and the result follows.
\end{proof}

\subsubsection{Maps Line to Parallel Line}
A Homothety maps a line to another line parallel to the first. If $\mathcal{H}(A, B) = A', B'$, then $\overline{AB} \parallel \overline{A'B'}$.
\begin{proof}
    Follows from similar triangles $\triangle HAB$ and $\triangle HA'B'$.
\end{proof}

\subsubsection{Maps Circles to Circles}
A Homothety maps a circle to another circle.
\begin{remark}
    Proof \TBD.
\end{remark}
\begin{remark}
    See \url{https://mathworld.wolfram.com/HomotheticCenter.html} for an animation of Homothety for circles.
\end{remark}

\subsubsection{Preserves Tangency}
A Homothety preserves tangency.
\begin{remark}
    Proof \TBD.
\end{remark}

\subsubsection{Preserves Figures}
In general, a homothety maps a figure (i.e. a triangle with its orthocenter) into a similar figure (similar triangle with its orthocenter).

\subsubsection{Existence}
For every pair of similar, ``equally oriented'' figures (i.e. traingles with corresponding sides parallel) there exists a homothety mapping one of them to the other.

\begin{note}
    For an applications of this in a problem see Example \ref{hmmt_nov14_team_10}.
\end{note}

\subsubsection{Inverse}
Every homothety has an inverse transformation. If $\mathcal{H}$ is a homothety with center $H$ and coefficient $k$, then $\mathcal{H}^{-1}$ is the homothety with same center $H$ and coefficient $\frac{1}{k}$.

\subsection{Examples}
\begin{example}
    Let $P$ be a point inside a square $ABCD$. Prove that the centroids of the triangle $PAB$, $PBC$, $PBC$, $PCD$, $PDA$ form a square.
\end{example}
\begin{soln}
    Draw the medians $PM_1, PM_2, PM_3, PM_4$ as shown below. Note that $M_1M_2M_3M_4$ is a square.
    \begin{figure}[h!]
        \centering        
        \begin{asy}
            import cse5;
            unitsize(1cm);
            pair A, B, C, D, P, A1, B1, C1, D1, A2, B2, C2, D2;

            A = (0,0);
            B = (5,0);
            C = (5,5);
            D = (0,5);
            P = (3,4);

            A1 = (A + B)/2;
            B1 = (B + C)/2;
            C1 = (C + D)/2;
            D1 = (D + A)/2;

            draw(A--B--C--D--cycle);
            draw(A--P--B^^C--P--D);

            draw(A1--P--B1^^C1--P--D1, blue);

            A2 = P + (A1-P)*2/3;
            B2 = P + (B1-P)*2/3;
            C2 = P + (C1-P)*2/3;
            D2 = P + (D1-P)*2/3;

            draw(A1--B1--C1--D1--cycle, lightcyan);
            draw(A2--B2--C2--D2--cycle, cyan);

            dot(P);
            dot(A2);
            dot(B2);
            dot(C2);
            dot(D2);
            
            label("$P$", P, S);
            label("$A$", A, S);
            label("$B$", B, S);
            label("$C$", C, N);
            label("$D$", D, N);

            label("$M_1$", A1, S);
            label("$M_2$", B1, E);
            label("$M_3$", C1, N);
            label("$M_4$", D1, W);

            label("$G_1$", A2, SE);
            label("$G_2$", B2, NE);
            label("$G_3$", C2, NE);
            label("$G_4$", D2, NW);

        \end{asy}
        \caption{Point inside a square example.}
    \end{figure}
    Define the homothety $\mathcal{H}(P, \frac{2}{3})$. Then $\mathcal{H}(M_1,M_2,M_3,M_4) = G_1, G_2, G_3, G_4$. Therefore, $G_1G_2G_3G_4$ is a square too.
\end{soln}
% \newpage
\section{Common Homotethies}
\subsection{Triangles with Parallel Corresponding Sides}
\begin{theorem}
    For two triangles $ABC$ and $A'B'C'$, which have parallel corresponding sides, the
lines $AA'$, $BB'$ and $CC'$, which join the vertices with the corresponding equal angles, either pass through a common point and the triangles are homothetic, or are parallel and the triangles are
congruent.
\end{theorem}
\begin{figure}[h!]
    \centering
    \begin{asy}
        import cse5;
        unitsize(1cm);

        picture pic1, pic2;
        pair O, A, B, C;
        real k1, k2, d;

        O = (0,0);
        A = (1,3);
        B = (1.15, 1.75);
        C = (2, 1);

        k1 = 2;
        d = 0.2;

        draw(pic1, (-d*A)--((k1+d)*A), blue+dashed);
        draw(pic1, (-d*B)--((k1+d)*B), blue+dashed);
        draw(pic1, (-d*C)--((k1+d)*C), blue+dashed);

        draw(pic1, A--B--C--cycle);
        draw(pic1, (k1*A)--(k1*B)--(k1*C)--cycle);
        label(pic1, "$A$", A, NW);
        label(pic1, "$A'$", k1*A, NW);
        label(pic1, "$B$", B, NW);
        label(pic1, "$B'$", k1*B, NW);
        label(pic1, "$C$", C, SE);
        label(pic1, "$C'$", k1*C, SE);

        dot(pic1, O, blue);
        label(pic1, "$O$", O, SE);


        k1 = -1.25;
        d = 0.2;

        draw(pic2, ((1+d)*A)--((k1-d)*A), blue+dashed);
        draw(pic2, ((1+d)*B)--((k1-d)*B), blue+dashed);
        draw(pic2, ((1+d)*C)--((k1-d)*C), blue+dashed);

        draw(pic2, A--B--C--cycle);
        draw(pic2, (k1*A)--(k1*B)--(k1*C)--cycle);
        label(pic2, "$A$", A, NW);
        label(pic2, "$A'$", k1*A, SE);
        label(pic2, "$B$", B, NW);
        label(pic2, "$B'$", k1*B, SE);
        label(pic2, "$C$", C, SE);
        label(pic2, "$C'$", k1*C, NW);

        dot(pic2, O, blue);
        label(pic2, "$O$", O, SE);

        add(shift(3*up)*shift(10*right)*pic2);
        add(pic1);
    \end{asy}
    \caption{Homothetic Triangles}
\end{figure}
\begin{proof}
    \TBD
\end{proof}
\subsection{Tangential Circles}
\begin{figure}[h!]
    \centering
    \begin{asy}
        import cse5;
        unitsize(0.5cm);
        pair H, O1, O2, A1, A2;
        real k, r, d;

        d = 1;
        H = (0,0);
        r = 2;
        O1 = (r, r);
        k = 2.5;
        O2 = k*O1;

        A1 = O1 + (0, r*sqrt(2));
        A2 = O2 + (0, k*r*sqrt(2));

        draw(circle(O1, r*sqrt(2)));
        draw(circle(O2, k*r*sqrt(2)));
        draw((A1.x-d,A1.y)--(A1.x+d,A1.y), blue);
        draw((A2.x-d*k,A2.y)--(A2.x+d*k,A2.y), blue);

        draw(H--O2--A2--cycle^^O1--A1);
        dot (O1);
        dot (O2);
        dot (A1);
        dot (A2);

        dot (H);
        label ("$H$", H, SW);
        label ("$O$", O1, SE);
        label ("$O'$", O2, SE);
        label ("$A$", A1, NW);
        label ("$A'$", A2, NW);
        label("$l_1$",(A1.x-d,A1.y), W);
        label("$l_2$",(A2.x-k*d,A2.y), W);

    \end{asy}
    \caption{Homotethic Circles}
\end{figure}
A common homothety arises when two circles touch at a point $H$. Then the ray from $H$ meeting the
two circles at two corresponding points is a homothety that maps the circles to each other.

In the above figure, one can show using similar triangles that $\frac{HA'}{HA} = \frac{HO'}{HO} = \frac{O'A'}{OA} = \frac{r'}{r} = k$, where $r$ and $r'$ are radii of the circles. As $OA$ is parallel to $O'A'$, we have $l$ is parallel to $l'$. Thus the homothety also maps $l$ to $l'$.

\begin{note}
    Similar to the top point of the circle shown in the figure, the homotethy also maps the bottom point of one to the other.
\end{note}

\TBD
\clearpage

\section{Nine Point Circle or Euler Circle}
\begin{lemma}[Reflecting the Orthocenter]
    Let $H$ be the orthocenter of $\triangle ABC$, and $X$ be the reflection of $H$ over $\overline{BC}$, and $Y$ be the reflection of $H$ over the midpoint of $\overline{BC}$. Then
    \begin{enumerate}
        \item $X$ is on $(ABC)$ (that is, the circumcircle of $\triangle ABC$).
        \item $AY$ is the diameter of $(ABC)$.
    \end{enumerate}
\end{lemma}
\begin{figure}[h]
    \centering
    \begin{asy}
        import cse5;
        import olympiad;

        unitsize(0.75cm);

        pair A, B, C, H, O, O1;

        B = (0,0);
        A = (2,5);
        C = (8,0);

        draw (A--B--C--cycle);
        path c1;
        c1 = circumcircle(A, B, C);
        O = circumcenter(A, B, C);
        draw (c1, yellow);

        pair Ap, Ap2, Bp, Bp2, Cp, Cp2;
        Ap = foot(A, B, C);
        Bp = foot (B, A, C);
        Cp = foot (C, A, B);

        H = extension(A, Ap, B, Bp);

        // Ap2 = 

        Ap2 = H + 2 (Ap - H);
        draw (A--Ap2, blue);
        draw (B--Bp, blue);
        draw (C--Cp, blue);

        dot(Ap, red);
        dot(Ap2, red);

        pair A1, A1p;

        A1 = (B + C) /2;

        A1p = H + 2 (A1 - H);

        draw (H--A1p, green); 
        dot(A1, darkgreen);
        dot(A1p, darkgreen);




        dot(H, red);


        label("$H$", H, NW);
        label("$X$", Ap2, S);

        label("$Y$", A1p, S);

        label("$A$", A, N);
        label("$B$", B, SW);
        label("$C$", C, SE);

    \end{asy}
    \caption{Reflecting the Orthocenter}
\end{figure}

\begin{proof}[Proof of Part 1]
    We can prove part 1 by extending $AH$ to meet the circle $(ABC)$ at $A''$ and show that $A''$ is the reflection of $H$ over $\overline{BC}$.

    We can do this by noting that $\measuredangle XBC = \measuredangle XAC = 90^\circ - \measuredangle C = \measuredangle A'BH$. Therefore $\triangle A''BA' \cong HBA'$, and $HA' = A'A''$.

    \begin{figure}[h]
        \centering
        \begin{asy}
            import cse5;
            import olympiad;
    
            unitsize(0.75cm);
    
            pair A, B, C, H, O, O1;
    
            B = (0,0);
            A = (2,5);
            C = (8,0);
    
            draw (A--B--C--cycle);
            path c1;
            c1 = circumcircle(A, B, C);
            O = circumcenter(A, B, C);
            draw (c1, yellow);
    
            pair Ap, Ap2, Bp, Bp2, Cp, Cp2;
            Ap = foot(A, B, C);
            Bp = foot (B, A, C);
            Cp = foot (C, A, B);
    
            H = extension(A, Ap, B, Bp);
    
            // Ap2 = 
    
            Ap2 = H + 2 (Ap - H);
            draw (A--Ap2--B, blue);
            draw (B--Bp, blue);
            draw (C--Cp, blue);
    
            dot(Ap, red);
            dot(Ap2, red);
    
            pair A1, A1p;
    
            A1 = (B + C) /2;
    
            A1p = H + 2 (A1 - H);
    
            // draw (H--A1p, green); 
            // dot(A1, darkgreen);
            // dot(A1p, darkgreen);
    
    
            draw(anglemark(Ap2, B, C, 10), red);

            draw(anglemark(Ap2, A, C, 19,20), orange);
            draw(anglemark(Ap2, A, C, 10), red);

            draw(anglemark(Ap, B, H, 19,20), orange);

            dot(H, red);
        
            label("$H$", H, NW);
            label("$A'$", Ap, SE);
            label("$A''$", Ap2, SE);
    
    
            label("$A$", A, N);
            label("$B$", B, SW);
            label("$C$", C, SE);
    
        \end{asy}
        \caption{Intersection of $AH$ with $(ABC)$ is the reflection of $H$ over $\overline{BC}$}
    \end{figure}
    
\end{proof}

\begin{proof}[Proof of Part 2]
    We will prove this by constructing the diameter $AD$ of the circle $(ABC)$ and showing that $D$ is the reflection of $H$ over the midpoint of $\overline{AB}$.

    \begin{figure}[h]
        \centering
        \begin{asy}
            import cse5;
            import olympiad;
    
            unitsize(0.75cm);
    
            pair A, B, C, H, O, O1;
    
            B = (0,0);
            A = (2,5);
            C = (8,0);
    
            draw (A--B--C--cycle);
            path c1;
            c1 = circumcircle(A, B, C);
            O = circumcenter(A, B, C);
            draw (c1, yellow);
    
            pair Ap, Ap2, Bp, Bp2, Cp, Cp2;
            Ap = foot(A, B, C);
            Bp = foot (B, A, C);
            Cp = foot (C, A, B);
    
            H = extension(A, Ap, B, Bp);
    
            // Ap2 = 
    
            Ap2 = H + 2 (Ap - H);
            //draw (A--Ap, blue);
            draw (B--Bp, blue);
            draw (C--Cp, blue);
    
            //dot(Ap, red);
            //dot(Ap2, red);
    
            pair A1, A1p;
    
            A1 = (B + C) /2;
    
            A1p = H + 2 (A1 - H);
    
            draw (H--A1p, green); 

            draw(A--A1p, purple);
            dot(A1, darkgreen);
            dot(A1p, darkgreen);
    
    
    
    
            dot(H, red);
            dot(O, purple);
            draw(C--A1p--B--H--cycle, fuchsia);
            draw(B--C, green);

            draw (rightanglemark(A1p,B,A, 10), red);
            draw (rightanglemark(C,Cp,A, 10), red);
            draw (rightanglemark(A1p,C,A, 10), red);
            draw (rightanglemark(B,Bp,A, 10), red);

            label("$H$", H, NW);
    
            label("$D$", A1p, S);
    
            label("$A$", A, N);
            label("$B$", B, SW);
            label("$C$", C, SE);
    
        \end{asy}
        \caption{For diameter $AD$, $D$ is reflection of $H$ over midpoint of $\overline{BC}$ }
    \end{figure}
    
    $\overline{CD} \perp \overline{CA}$ and $\overline{BH} \perp \overline{CA}$, therefore $\overline{CD} \parallel \overline{BH}$. Similarly, $\overline{BD} \parallel \overline{CH}$. Therefore, $BHCD$ is a parallelogram and its diagonals bisect. Thereore, $D$ is the reflection of $H$ over the midpoint of $\overline{BC}$.
\end{proof}

\begin{lemma}[Nine Point Circle]
Let $A_p, B_p, C_p$ be the foot of altitudes from $A, B, C$ to $\overline{BC}, \overline{AC}, \overline{AB}$, $A_m, B_m, C_n$ be the midpoints of $\overline{BC}, \overline{AC}, \overline{AB}$ and  $A_{\frac{H}{2}}, B_{\frac{H}{2}}, C_{\frac{H}{2}}$ be the midpoints of $\overline{AH}, \overline{BH}, \overline{CH}$ respectively. Then all these nine points lie on a circle. Further, the center of the circle $N_9$ is the midpoint of $\overline{OH}$, where $O$ is the circumcenter of $\triangle ABC$.  
\end{lemma}
\begin{figure}[h!]
    \centering
    \begin{asy}
        import cse5;
        import olympiad;

        unitsize(1cm);

        pair A, B, C, H, O, O1;

        B = (0,0);
        A = (2,5);
        C = (8,0);

        draw (A--B--C--cycle);
        path c1;
        c1 = circumcircle(A, B, C);
        O = circumcenter(A, B, C);
        draw (c1, yellow);

        pair Ap, Ap2, Bp, Bp2, Cp, Cp2;
        Ap = foot(A, B, C);
        Bp = foot (B, A, C);
        Cp = foot (C, A, B);

        H = extension(A, Ap, B, Bp);

        // Ap2 = 

        Ap2 = H + 2 (Ap - H);
        Bp2 = H + 2 (Bp - H);
        Cp2 = H + 2 (Cp - H);
        draw (A--Ap2, blue);
        draw (B--Bp2, blue);
        draw (C--Cp2, blue);

        dot(Ap, red);
        dot(Bp, red);
        dot(Cp, red);
        dot(Ap2, red);
        dot(Bp2, red);
        dot(Cp2, red);

        pair A1, A1p, B1, B1p, C1, C1p;

        A1 = (B + C) /2;
        B1 = (A + C) / 2;
        C1 = (A + B) / 2;

        A1p = H + 2 (A1 - H);
        B1p = H + 2 (B1 - H);
        C1p = H + 2 (C1 - H);

        draw (H--A1p, green); 
        dot(A1, darkgreen);
        dot(A1p, darkgreen);

        draw (H--B1p, green); 
        dot(B1, darkgreen);
        dot(B1p, darkgreen);

        draw (H--C1p, green); 
        dot(C1, darkgreen);
        dot(C1p, darkgreen);

        pair A3, B3, C3;
        A3 = A + (H - A)/2;
        B3 = B + (H - B)/2;
        C3 = C + (H - C)/2;

        dot(A3, purple);
        dot(B3, purple);
        dot(C3, purple);

        path c2 = circumcircle (Ap, Bp, Cp);
        O1 = circumcenter(Ap, Bp, Cp);
        draw (c2, orange);

        draw (H--O, cyan);

        dot(H, red);
        dot (O, purple);
        dot (O1, purple);

        label("$O$", O, S);
        label("$N_9$", O1, S);

        label("$H$", H, NW);
        label("$A_p$", Ap, SW);
        label("$A_p'$", Ap2, SW);

        label("$A_m$", A1, SW);
        label("$A_m'$", A1p, SW);

        label("$A_\frac{H}{2}$", A3, SW);
        
        label("$A$", A, N);
        label("$B$", B, SW);
        label("$C$", C, SE);

    \end{asy}
    \caption{The 9-point circle}
\end{figure}

\begin{proof}
    From the previous lemma, we know that $A_p'$ and $A_m'$ lie on the circle $(ABC)$. 

    Consider the Homothety $\mathcal{H}(H, \frac{1}{2})$. $\mathcal{H}(A_p') = A_p$, $\mathcal{H}(A_m') = A_m$, and $\mathcal{H}(A) = A_{\frac{H}{2}}$. Since the original 3 points are on a circle, the Homothety is also a circle and therefore, $A_p, A_m, A_{\frac{H}{2}}$ are on a circle.

    The center of the circle will also be the Homothety of the original circle, that is $\mathcal{H}(O) = N_9$, therfore, $HN_9 = \frac{1}{2}HO$.
\end{proof}

\section{More Applications}
\subsection{Euler Line}
\begin{lemma}[Euler Line]
    Let $ABC$ be a triangle, and $H, G, O$ be its orthocenter, centroid, and circumcenter, respectively. Then the points $H, G, O$ lie on a a single line (called the \emph{Euler line} of $\triangle ABC$) in this order and $HG = 2 \cdot GO$.
\end{lemma}
\begin{proof}
    First note that the orthocenter of the medial triangle $A_mB_mC_m$ is the same as the circumcenter $O$ of the trinagle $ABC$. This is becuse $C_mB_m \parallel BC$, and therefor $OA_m \perp BC \implies OA_m \perp C_mB_m$. 
    \begin{figure}[h!]
        \centering
        \begin{asy}
            import cse5;
            import olympiad;
    
            unitsize(1cm);
    
            pair A, B, C, H, O, G;
    
            B = (0,0);
            A = (2,5);
            C = (8,0);
    
            draw (A--B--C--cycle);

            O = circumcenter(A, B, C);
            G = (A + B + C) / 3;

            pair Ap, Ap2, Bp, Bp2, Cp, Cp2;
            Ap = foot(A, B, C);
            Bp = foot (B, A, C);
            Cp = foot (C, A, B);
    
            H = extension(A, Ap, B, Bp);
    
    
            draw (A--Ap, palecyan);
            draw (B--Bp, palecyan);
            draw (C--Cp, palecyan);
    
            //dot(Ap, red);
            //dot(Bp, red);
            //dot(Cp, red);
    
            pair A1, A1p, B1, B1p, C1, C1p;
    
            A1 = (B + C) /2;
            B1 = (A + C) / 2;
            C1 = (A + B) / 2;
    
            A1p = extension(A1,O,B1,C1);
            B1p = extension(B1,O,A1,C1);
            C1p = extension(C1,O,A1,B1);

            draw (A--A1^^B--B1^^C--C1, paleblue);
            draw (A1--B1--C1--cycle, blue);
            draw (A1--A1p^^B1--B1p^^C1--C1p, green);
            draw (H--O, orange);

            dot(A1, darkgreen);
    
            dot(B1, darkgreen);
            dot(C1, darkgreen);
    
    
    
            dot(H, red);
            dot (O, red);
            dot(G, red);

            label("$O$", O, S);
            label("$G$", G, S);

            label("$H$", H, NW);
            //label("$A_p$", Ap, SW);
    
            label("$A_m$", A1, S);
            label("$B_m$", B1, E);
            label("$C_m$", C1, W);
    
            
            label("$A$", A, N);
            label("$B$", B, SW);
            label("$C$", C, SE);
    
        \end{asy}
        \caption{The Euler Line}
    \end{figure}
    Consider the homotethy $\mathcal{H}(G, -\frac{1}{2})$ (centered at $G$ with a coefficient $-\frac{1}{2}$). $\mathcal{H}(A, B, C) = A_m, B_m, C_m$. That is $\mathcal{H}$ maps $\triangle ABC$ to its medial  $\triangle A_mB_mC_m$. Therefore, $\mathcal{H} = H_m$. But as we noted earlier, $H_m = O$, therefore $\mathcal{H} = O$. By collinearity of homothety, $H, G, O$ are collinear. And by the contraction factor of $\mathcal{H}$, $GO = \frac{1}{2} HG$.
\end{proof}

\subsection{Circles Inscribed in Segments}
\begin{problem}
    Let $\Omega$ be a circle with center $O$ and a chord $\overline{AB}$, and consider a circle $\omega$ tangent internally to $\Omega$ at $T$ and $\overline{AB}$ at $K$. Let $M$ denote the midpoint of $\arc{AB}$ not containing $T$. Show that there is a homotethy mapping $K$ to $M$ and in particular $T$, $K$, and $M$ are all collinear. 
    \textit{Evan Chen EGMO, \#4.31}
\end{problem}
\begin{figure}[h!]
    \centering
    \begin{asy}
        import olympiad;
        import cse5;
        unitsize(0.35cm);

        pair O, O1, O2, A, B, T1, K1, T2, K2, M, CT1, CT11, CT2;
        real r = 8, r1, r2, x, y = 1, xd = 3;

        x = sqrt(r*r-y*y);

        O = (0,0);
        A = (-x,-y);
        B = (x,-y);
        M = (0,r);

        K1 = (x - xd, -y);

        path circ = circle(O, r);
        pair [] t1c =intersectionpoints(circ, M--(M+(M-K1)*-5));

        if (t1c[0].y != M.y) {
            T1 = t1c[0];
        } else {
            T1 = t1c[1];
        }

        r1 = r * (distance(K1,T1)/distance(M,T1));

        O1 = (x - xd, -y -r1);

        pair Ap, Bp;
        Ap = (A.x, M.y);
        Bp = (B.x,M.y);
        draw (Ap--Bp, blue + dashed);
        draw (circle(O1, r1), red);

        draw (M--T1, blue);
        draw (A--M--B--T1, cyan);
        draw (T1--O1--K1^^O1--O--M, yellow);
        draw(circ);
        draw(A--B);

        draw(anglemark(B, M, Bp, 40,45), orange);
        draw(anglemark(B, A, M, 40,45), orange);

        draw(anglemark(B, M, Bp, 20), red);
        draw(anglemark(M, B, A, 20), red);


        dot(O);
        dot(O1);
        dot(M);
        dot(K1);
        dot(T1);
        label("$O$", O, N);
        label("$A$", A, W);
        label("$B$", B, E);
        label("$A'$", Ap, W);
        label("$B'$", Bp, E);
        label("$T$", T1, SE);
        label("$K$", K1, NE);
        label("$M(K')$", M, N);

    \end{asy}
    \caption{Homothethy of Circle Inscribed in Segment \textit{(\cite{echen} EGMO 4.31)}}
\end{figure}
\begin{soln}
    As the centers of $\Omega$ and $\omega$ are collinear with $T$ due to tangency, it follows that there is a homotethy centered at $T$ mapping $\omega$ to $\Omega$.

\begin{note}
    We noted this earlier in the common homotethy situations section as well.
\end{note}
    To prove that this homotethy maps $K$ to $M$, we will take the mapping of $K$ as $K'$ and show that $K'$ is the midpoint of the $\arc{AB}$ and is therefore same as $M$.

    By homotethy, $A'B'$ is parallel to $AB$ and therefore $\angle B'K'B = \angle K'BA$. Also, $OK'$ is tangent at $K'$, so $\angle B'K'B = \angle K'AB$. Therefore, $\angle K'BA = \angle K'AB$. That is $\triangle K'AB$ is isoceles and $K'O$ is median in addition to being altitude. Therefore $K'$ is the midpoint of $\arc{AB}$.
\end{soln}
\begin{problem}
    Further, show that $\triangle TMB \sim \triangle BMK$. \textit{\cite{echen} EGMO, \#4.32}
\end{problem}
\begin{soln}
    $\angle MAB = \angle MTB$ (same inscribed angles), and therefore $\angle MTB = \angle MBK $ and $\angle KMB$ is common. Therefore by AA similarity $ \triangle TMB \sim BMK$.
\end{soln}
\begin{remark}
    The above implies that $\frac{TM}{MB} = \frac{MB}{MK}$, that is $MB^2 = MK \cdot MT$, that is the power of $M$ w.r.t circle $\omega$ is $MB^2$.
\end{remark}

\begin{lemma}[Circles inscribed in Segments]
    Let $\overline{AB}$ be a chord of circle $\Omega$. Let $\omega$ be a circle tangent to $AB$ at $K$ and internally tangent to $\Omega$ at $T$. Then ray $TK$ passes through the midpoint $M$ of $\arc{AB}$ not containing $T$. 

    Moreover, $MA^2 = MB^2$ is the power of $M$ with respect to $\omega$.
\end{lemma}
\begin{proof}
    Proved in the previous two problems.
\end{proof}

\begin{problem}\label{prob_7_9_0_wong}
    Let $\omega_1, \omega_2$ be two cicrles internally tangent to $\Omega$ at $T_1, T_2$, and externally tangent to each other. Suppose their common tangent meets the circle $\Omega$ at $P$. Let $K_1K_2$ be another common tangent, tangent at points $K_1, K_2$ to $\omega_1, \omega_2$. Show that the rays $T_1K_1$ and $T_2K_2$ also meet $\Omega$ at $P$.
\end{problem}
\begin{figure}[h!]
    \centering
    \begin{asy}
        import olympiad;
        import cse5;
        unitsize(0.5cm);

        real vectRatio(pair v1, pair v2) {
            return sqrt(v1.x*v1.x + v1.y*v1.y)/sqrt(v2.x*v2.x + v2.y*v2.y);
        }
        pair O, O1, O2, A, B, T1, K1, T2, K2, M, CT1, CT11, CT2;
        real r = 8, r1, r2, x, y = 1, xd = 3;

        x = sqrt(r*r-y*y);

        O = (0,0);
        A = (-x,-y);
        B = (x,-y);
        M = (0,r);

        K1 = (x - xd, -y);

        path circ = circle(O, r);
        pair [] t1c =intersectionpoints(circ, M--(M+(M-K1)*-5));

        if (t1c[0].y != M.y) {
            T1 = t1c[0];
        } else {
            T1 = t1c[1];
        }

        //r1 = r*vectRatio((K1-T1),(M-T1));
        r1 = r * (distance(K1,T1)/distance(M,T1));
        //label((string)r1, O);

        O1 = (x - xd, -y -r1);
        CT1 = tangent(M, O1, r1, 1);
        t1c = intersectionpoints(circ, M--(M+(M-CT1)*-5));
        CT11 = t1c[1];

        pair tmp = extension(M,CT1,A,B);
        pair bp = bisectorpoint(A, tmp, CT1);

        //draw(tmp--bp,red);

        O2 = extension(O1,CT1, tmp, bp);
        r2 = distance(O2, CT1);
        K2 = foot (O2, A, B);
        t1c =intersectionpoints(circ, M--(M+(M-K2)*-5));
        T2 = t1c[1];

        //draw(O1--O2, blue);

        draw (M--CT11);

        draw (circle(O1, r1), red);
        draw (circle(O2, r2), red);
        //dot (O2, purple);

        draw (M--T1^^M--T2, blue);
        draw(circ);
        draw(A--B);
        dot(O);
        dot(O1);
        dot(O2);
        dot(M);
        dot(K1);
        dot(K2);
        dot(T1);
        dot(T2);

        label("$\Omega$", O, NE);
        label("$\omega_1$", O1, NW);
        label("$\omega_2$", O2, NE);

        label("$A$", A, W);
        label("$B$", B, E);
        //label("$A'$", Ap, W);
        //label("$B'$", Bp, E);
        label("$T_1$", T1, SE);
        label("$K_1$", K1, NE);
        label("$T_2$", T2, SE);
        label("$K_2$", K2, NE);
        label("$P$", M, N);


    \end{asy}
    \caption{Solution for Problem \ref{prob_7_9_0_wong}}
\end{figure}
\begin{proof}
    By the previous lemma, we know that there is a homothety centered at $T_1$ that sends $K_1$ to $K_1'$, such that $K_1'$ is the midpoint of $\arc{AB}$, and power of $K_1'$ w.r.t circle $\omega_2$  is $PB^2 = PA^2$. Similarly, there is homothety centered at $T_2$ that sends $K_2$ to $K_2'$, such that $K_2'$ is the midpoint of $\arc{AB}$, and power of $K_2'$ w.r.t circle $\omega_2$ is $PB^2 = PA^2$.

    However, if $K_1', K_2'$ are both the midpoint of $\arc{AB}$, then they are the same point $K'$. Then $K'$ has the same power w.r.t circles $\omega_1, \omega_2$, and therefore is on the radical axis, that is on the common tangent. Therefore, $K'$ and $P$ are the same points.
\end{proof}

\begin{problem}{Wong, MA2219 \cite{wong}, Exercise 7.9.}\label{prob_7_9_wong}
    Let $\omega_1, \omega_2, \omega_3$ be three circles inside and tangent to a circle $\Omega$ such that $\omega_2$ is tangent to $\omega_1$ and $\omega_3$ externally; and $\omega_1, \omega_3$ lie outside each other. Suppose the common tangent between $\omega_1, \omega_2$
    and the common tangent between $\omega_1, \omega_2$ meet at a point $P$ on $\Omega$. Prove that $\omega_1, \omega_2, \omega_3$ touch a line.
    \begin{figure}[h!]
        \centering
        \begin{asy}
            import olympiad;
            import cse5;
            unitsize(0.5cm);
    
            pair O, O1, O2, O3, A, B, T1, K1, T2, K2, T3, K3, M, CT1, CT11, CT2, CT21;
            real r = 8, r1, r2, r3, x, y = 1, xd = 3;
    
            x = sqrt(r*r-y*y);
    
            O = (0,0);
            A = (-x,-y);
            B = (x,-y);
            M = (0,r);
    
            K1 = (x - xd, -y);
    
            path circ = circle(O, r);
            pair [] t1c =intersectionpoints(circ, M--(M+(M-K1)*-5));
    
            if (t1c[0].y != M.y) {
                T1 = t1c[0];
            } else {
                T1 = t1c[1];
            }
    
            //r1 = r*vectRatio((K1-T1),(M-T1));
            r1 = r * (distance(K1,T1)/distance(M,T1));
            //label((string)r1, O);
    
            O1 = (x - xd, -y -r1);
            CT1 = tangent(M, O1, r1, 1);
            t1c = intersectionpoints(circ, M--(M+(M-CT1)*-5));
            CT11 = t1c[1];
    
            pair tmp = extension(M,CT1,A,B);
            pair bp = bisectorpoint(A, tmp, CT1);
    
            //draw(tmp--bp,red);
    
            O2 = extension(O1,CT1, tmp, bp);
            r2 = distance(O2, CT1);
            K2 = foot (O2, A, B);
            t1c =intersectionpoints(circ, M--(M+(M-K2)*-5));
            T2 = t1c[1];
    
            CT2 = tangent(M, O2, r2, 1);
            //dot(CT2, red);
            CT21 = intersectionpoints(circ, M--(M+(M-CT2)*-5))[1];
            //dot(CT21, red);
    
            tmp = extension(M,CT2,A,B);
            bp = bisectorpoint(A, tmp, CT2);
    
            O3 = extension(O2,CT2, tmp, bp);
            r3 = distance(O3, CT2);
            K3 = foot (O3, A, B);
            t1c =intersectionpoints(circ, M--(M+(M-K3)*-5));
            T3 = t1c[1];
    
            //draw(O1--O2, blue);
    
            draw (M--CT11^^M--CT21);
    
            draw (circle(O1, r1));
            draw (circle(O2, r2));
            draw (circle(O3, r3));
    
            //dot (O2, purple);
    
            //draw (M--T1^^M--T2, blue);
            draw(circ);
            draw((A.x+1, A.y)--(B.x-1,B.y));
            dot(O);
            dot(O1);
            dot(O2);
            dot(O3);
            dot(M);
            
            label("$\Omega$", O, NE);
            label("$\omega_1$", O1, N);
            label("$\omega_2$", O2, N);
            label("$\omega_3$", O3, N);
    
            label("$P$", M, N);
    
    
        \end{asy}
        \caption{Problem \ref{prob_7_9_wong}}
    \end{figure}
\end{problem}


\begin{figure}[h!]
    \centering
    \begin{asy}
        import olympiad;
        import cse5;
        unitsize(0.5cm);

        pair O, O1, O2, O3, A, B, T1, K1, T2, K2, T3, K3, M, CT1, CT11, CT2, CT21;
        real r = 8, r1, r2, r3, x, y = 1, xd = 3;

        x = sqrt(r*r-y*y);

        O = (0,0);
        A = (-x,-y);
        B = (x,-y);
        M = (0,r);

        K1 = (x - xd, -y);

        path circ = circle(O, r);
        pair [] t1c =intersectionpoints(circ, M--(M+(M-K1)*-5));

        if (t1c[0].y != M.y) {
            T1 = t1c[0];
        } else {
            T1 = t1c[1];
        }

        //r1 = r*vectRatio((K1-T1),(M-T1));
        r1 = r * (distance(K1,T1)/distance(M,T1));
        //label((string)r1, O);

        O1 = (x - xd, -y -r1);
        CT1 = tangent(M, O1, r1, 1);
        t1c = intersectionpoints(circ, M--(M+(M-CT1)*-5));
        CT11 = t1c[1];

        pair tmp = extension(M,CT1,A,B);
        pair bp = bisectorpoint(A, tmp, CT1);

        //draw(tmp--bp,red);

        O2 = extension(O1,CT1, tmp, bp);
        r2 = distance(O2, CT1);
        K2 = foot (O2, A, B);
        t1c =intersectionpoints(circ, M--(M+(M-K2)*-5));
        T2 = t1c[1];

        CT2 = tangent(M, O2, r2, 1);
        //dot(CT2, red);
        CT21 = intersectionpoints(circ, M--(M+(M-CT2)*-5))[1];
        //dot(CT21, red);

        tmp = extension(M,CT2,A,B);
        bp = bisectorpoint(A, tmp, CT2);

        O3 = extension(O2,CT2, tmp, bp);
        r3 = distance(O3, CT2);
        K3 = foot (O3, A, B);
        t1c =intersectionpoints(circ, M--(M+(M-K3)*-5));
        T3 = t1c[1];

        //draw(O1--O2, blue);

        draw (M--CT11^^M--CT21);

        draw (circle(O1, r1));
        draw (circle(O2, r2));
        draw (circle(O3, r3));

        //dot (O2, purple);

        draw (M--T1^^M--T2^^M--T3, blue);
        draw(circ);
        draw(A--B);
        dot(O);
        dot(O1);
        dot(O2);
        dot(O3);
        dot(M);
        dot(K1);
        dot(K2);
        dot(K3);

        dot(T1);
        dot(T2);
        dot(T3);

        label("$\Omega$", O, NE);
        label("$\omega_1$", O1, NW);
        label("$\omega_2$", O2, NE);
        label("$\omega_3$", O3, NW);

        label("$A$", A, W);
        label("$B$", B, E);
        //label("$A'$", Ap, W);
        //label("$B'$", Bp, E);
        label("$T_1$", T1, SE);
        label("$K_1$", K1, NE);
        label("$T_2$", T2, SE);
        label("$K_2$", K2, NE);
        label("$T_3$", T3, SW);
        label("$K_3$", K3, NW);
        label("$P$", M, N);


    \end{asy}
    \caption{Solution for problem \ref{prob_7_9_wong}}
\end{figure}
\vspace{0.1in}

    Let $T_1, T_2, T_3$ be the points of tangencies for the 3 circles with $\Omega$ as previous problems.

    Let's first consider the pair $\omega_1, \omega_2$. Let their common tangency be $K_1, K_{21}$ with $K_1$ being the tangency point with $\omega_1$ and $K_{21}$  the tangency point with $\omega_2$. From the previous problem, we know that $T_1K_1$ and $T_2K_{21}$ intersect with $\Omega$ and their common tangent at a point, say $P_{12}$.

    Similarly, if we consider $\omega_2, \omega_3$ and let their common tangent be $K_{23}K_3$, we know that they intersect with $\Omega$ and their common tangent at a point, say $P_{23}$.

    However, the 2 common tangents meet $\Omega$ at the same point $P$. Therefore, $P = P_{12} = P_{23}$. This implies that $K_{21} = K_{23}$ and so the two tangent lines $K_1K_{21}$ and $K_{23}K_3$ are the same.

\clearpage
\section{More Examples}
\begin{example}[2007 AIME II, \# 15]\label{aime_07_2_15}
    Four circles $\omega, \omega_A, \omega_B,$ and $\omega_C$ with the same radius are drawn in the interior of triangle $ABC$ such that $\omega_A$ is tangent to sides $AB$ and $AC$, $\omega_B$ to $BC$ and $BA$, $\omega_C$ to $CA$ and $CB$, and $\omega$ is externally tangent to $\omega_A, \omega_B,$ and $\omega_C$. If the sides of triangle $ABC$ are 13, 14, and 15, the radius of $\omega$ can be represented in the form $\frac{m}{n}$, where $m$ and $n$ are relatively prime positive integers. Find $m + n$.
\end{example}

\begin{figure}[ht!]
    \centering
    \begin{asy}
        import olympiad;

        unitsize(0.75cm);
        pair A, B, C, Oa, Ob, Oc, Od, O, I;
        path circ1, circ2;

        // Homotethy factor - backplugged from solution
        real k = 64/129;
        real r = 260/129;

        B = (0, 0);
        C = (14, 0);

        circ1 = circle(B, 13);
        circ2 = circle(C, 15);

        A = intersectionpoints(circ1, circ2)[0];
        I = incenter(A, B, C);

        Oa = (65*A + 64*I)/129;
        Ob = (65*B + 64*I)/129;
        Oc = (65*C + 64*I)/129;

        Od = circumcenter(Oa, Ob, Oc);
        O = circumcenter(A, B, C);

        draw(circle(Oa, r));
        draw(circle(Ob, r));
        draw(circle(Oc, r));
        draw(circle(Od, r));

        draw(incircle(Oa, Ob, Oc)^^incircle(A, B, C)^^I--foot(I, A, C), green);
        draw(A--B--C--cycle);
        draw(Oa--Ob--Oc--cycle, blue);
        draw(A--I--B^^I--C, blue);
        draw(Oa--foot(Oa, A, C)^^Oc--foot(Oc, A, C), blue);
        draw(rightanglemark(Oa, foot(Oa, A, C), C)^^rightanglemark(Oc, foot(Oc, A, C), A));
        dot(I);
        dot(Oa);
        dot(Ob);
        dot(Oc);
        dot(Od);
        dot(O, red);

        label("$A$", A, N);
        label("$B$", B, S);
        label("$C$", C, S);
        label("$I$", I, S);
        label("$O_A$", Oa, NW);
        label("$O_B$", Ob, SW);
        label("$O_C$", Oc, SE);
        label("$O_\omega$", Od, N);
        label("$O$", O, SE, red);

    \end{asy}
    \caption{Example \ref{aime_07_2_15}}
    \label{aime_07_2_15_fig}
\end{figure}

\begin{soln}
    First, we notice that $O_AO_C \parallel AC$, $O_AO_B \parallel AB$, and $O_BO_C \parallel BC$ (because same radius circles tangent to sides, refer to Figure \ref{aime_07_2_15_fig}). Therefore, not only are triangles $ABC$ and $O_AO_BO_C$ similar but also there is a Homotethy $\mathcal{H}$ mapping one to the other. 

    Where is the center of $\mathcal{H}$ and what is the dilation factor, $k$? Since $\mathcal{H}$ maps $A, B, C$ to $O_A, O_B, O_C$, its center is at the point of intersection of $AO_A, BO_B, CO_C$. Since the circles are tangents to two sides, their centers lie on the angle bisectors and therefore, the center of $\mathcal{H}$ is the incenter $\mathcal{I}$ of $\triangle ABC$.

    Therefore, $\mathcal{H}$ maps the incircle of $\triangle ABC$ to the incircle of $\triangle O_AO_BO_C$. Let $r_\omega, r, r'$ be the radius of circle $\omega$, inradius of $\triangle ABC$, and inradius of $\triangle O_AO_BO_C$, respectively. Then $r' = r - r_\omega$ and the dilation factor is $k = \frac{r'}{r} = \frac{r - r_\omega}{r}$.

    Since we have the side lengths of $\triangle{ABC}$, we can calulate its inradius $r = \frac{[ABC]}{s} = \frac{\sqrt{21\cdot8\cdot7\cdot6}}{21} = 4$. Therefore, dilation factor $k = \frac{4-r_\omega}{4}$.

    Next, we note that the center of the circle $\omega$, $O_\omega$ is at a distance $2r_\omega$ from $O_A, O_B, O_C$. Therefore, it is the circumcenter of $\triangle O_AO_BO_C$, and is the mapping of the circumcenter $O$ of $\triangle ABC$.

    Let $R, R'$ be the circumradius of $\triangle ABC, O_AO_BO_C$. By Homotethy $\mathcal{H}$, $\frac{R'}{R} = k = \frac{4-r_\omega}{4}$. But $R' = 2r_\omega$. Therefore, $\frac{2r_\omega}{R} = \frac{4-r_\omega}{4}$.

    We have $R = \frac{a\cdot b\cdot c}{4[ABC]} = {13 \cdot 14 \cdot 15}{4 \cdot 84} = \frac{65}{8}$. Putting it all together $\frac{2r_\omega}{\frac{65}{8}} = \frac{4-r_\omega}{4}$, and $r_w = \boxed{\frac{260}{129}}.$
\end{soln}


\begin{example}[HMMT February 2002, Guts \#32] \label{hmmt_feb02_guts_32}
    Two circles have radii 13 and 30, and their centers are 41 units apart. The line
through the centers of the two circles intersects the smaller circle at two points; let $A$ be the
one outside the larger circle. Suppose $B$ is a point on the smaller circle and $C$ a point on
the larger circle such that $B$ is the midpoint of $AC$. Compute the distance $AC$.
\end{example}
\begin{figure}[ht!]
    \centering
    \begin{asy}
        unitsize(0.15cm);
        pair O, O2, O1, A, B, C;

        O  = (0,0);
        O1 = (41, 0);
        A = (-13, 0);
        O2 = (13, 0);
        draw(circle(O, 13));
        draw(circle(O1, 30));
        draw(O--O1);
        dot(A);
        dot(O);
        dot(O1);

        draw (circle(O2, 26), red);
        dot(O2, red);

        C = intersectionpoints(circle(O1, 30), circle(O2, 26))[0];
        B = intersectionpoints(circle(O, 13), O--C)[0];
        draw(O--C);

        draw(O2--C--O1, blue);
        //draw(Label("$2\cdot13=26$", Rotate(dir(O2--C))), O2--C, red);

        dot(C);
        dot(B);

        label("$A$", A, W);
        label("$O$", O, S);
        label("$O_1$", O1, S);
        label("$O_2$", O2, SE, red);

        label("$B$", B, N);
        label("$C$", C, N);

        label(Label("$30$", Relative(0.5), Rotate(dir(C--O1))), C--O1, NE, blue);
        label(Label("$2\cdot13=26$", Relative(0.4), Rotate(dir(O2--C))), O2--C, NW, blue);
        label("$41-13=28$", O2--O1, N, blue);

    \end{asy}
    \caption{Example \ref{hmmt_feb02_guts_32}}
\end{figure}
\begin{soln}
    The key insight here is where will point $C$ and $B$ lie such that $AC = 2 \cdot AB$? 

    Realize that if we consider the homothety $\mathcal{H}(A, 2)$ it will send the small circle to another circle such that if we take any point $X$ on the new circle and the point $Y$ on the intersection of $AX$ with the original circle, the $X$ is the mapping of $Y$ under the homothety $\mathcal{H}$ and therefore $AX = 2 AY$.

    Therefore, if we draw the circle that is the mapping of the smaller circle under the homothety $\mathcal{H}(A, 2)$ and mark the point of intersection of this circle with bigger circle as $C$ and the intersection of $AC$ with the smaller circle as $B$, then we have the desired $B, C$.

    Now the mapping $O_2$ under $\mathcal{H}(A, 2)$ of the center $O$ is such that $A, O, O_2$ are collinear and $AO_2 = 2 \cdot AO$, that is $O_2$ is the diameterically opposite point of $A$ and is therefore, on the line $OO_1$.
    
    \begin{note}
        If we constructed a circle of twice the radius of the smaller circle at the diameterically opposite point of $A$, we could solve this problem the same. However, homotethy explains why we want to construct that circle.
    \end{note}

    Rest of the problem is straightforward. First, $O_2C = 2 \cdot 13 = 26$, $O_2O_1 = 41 - 13 = 28$, and $O_1C = 30$. Therefore, we have a $13-14-15$ triangle in $O_2CO_1$. Therefore, if we drop the foot, $D$, from $C$ to $O_2O_1$, $CD = 2\cdot12$ and $O_2D = 2\cdot5$. Finally, $AC^2 = AD^2 + CD^2 = (26+2\cdot5)^2 + (2\cdot12)^2$, and $AC = \boxed{12\sqrt{13}}$.
\end{soln}

\begin{example}[HMMT November 2014, Team Round \#10] \label{hmmt_nov14_team_10}
    Let $ABCDEF$ be a convex hexagon with the following properties.
    \begin{enumerate}[a)]
        \item $\overline{AC}$ and $\overline{AE}$ trisect $\angle BAF$.
        \item $\overline{BE} \parallel \overline{CD}$ and $\overline{CF} \parallel \overline{DE}$.
        \item $AB = 2AC = 4AE = 8AF$.
    \end{enumerate}
    Suppose that quadrilaterals $ACDE$ and $ADEF$ have areas 2014 and 1400, respectively. Find the area of quadrilateral $ABCD$

\end{example}

\begin{soln}

\begin{figure}[ht!]
    \centering
    \begin{asy}
        unitsize(1cm);
        import patterns;

        add("hatch",hatch());
        add("hatchback",hatch(NW));
        add("crosshatch",crosshatch(3mm));

        pair A, B, C, D, E, F, C1, E1;
        real theta = 162, s = 1;

        A = (0, 0);
        B = (s*8, 0);
        C = s*4*dir(theta/3);
        E = s*2*dir(2*theta/3);
        F = s*dir(theta);

        C1 = rotate(degrees(B-E), C)*(C.x-1,C.y);
        //draw(C--C1, red);

        E1 = rotate(degrees(C-F), E)*(E.x+1, E.y);
        //draw(E--E1, red);

        D = extension(C, C1, E, E1);

        //axialshade(A--C--D--E--cycle, green+opacity(0.4), A, blue+opacity(0.4), D);
        //axialshade(A--D--E--F--cycle, green+opacity(0.4), A, blue+opacity(0.4), E);
        fill(A--C--D--E--cycle, blue+opacity(0.25));
        fill(A--D--E--F--cycle, green+opacity(0.25));

        draw(E--F--A--B--C--D--cycle);
        draw(B--E^^C--F, yellow);
        
        draw(A--C--D--E--cycle, red);
        draw(A--D--E--F--cycle, red);
        label("$A$", A, S);
        label("$B$", B, S);
        label("$C$", C, NE);
        label("$D$", D, N);
        label("$E$", E, NW);
        label("$F$", F, SW);
    \end{asy}
    \caption{Example \ref{hmmt_nov14_team_10} Problem Statement}
\end{figure}

First, Since $\angle BAC = \angle CAE = \angle EAF$ and $\frac{AC}{AB} = \frac{AE}{AC} = \frac{AF}{AE} =2$, $\triangle ABC \sim \triangle ACE \sim \triangle AEF$. Therefore, $[ABC] = 16[AEF]$. Also, $AFEC \sim AECB$.

\begin{figure}[ht!]
    \centering
    \begin{asy}
        unitsize(1cm);
        pair P, Q, T;

        pair A, B, C, D, E, F, C1, E1;
        real theta = 162, s = 1;

        A = (0, 0);
        B = (s*8, 0);
        C = s*4*dir(theta/3);
        E = s*2*dir(2*theta/3);
        F = s*dir(theta);

        C1 = rotate(degrees(B-E), C)*(C.x-1,C.y);
        //draw(C--C1, red);

        E1 = rotate(degrees(C-F), E)*(E.x+1, E.y);
        //draw(E--E1, red);

        D = extension(C, C1, E, E1);

        P = extension(B, E, A, C);
        Q = extension(A, E, C, F);
        T = extension(C, F, B, E);

        draw(E--F--A--B--C--D--cycle);
        draw(B--E^^C--F, yellow);
        
        //draw(A--C--D--E--cycle, red);
        //draw(A--D--E--F--cycle, red);
        draw(A--F--E--cycle^^A--E--C--cycle^^A--C--B--cycle, red);
        draw(C--F^^B--E, blue);

        dot(P);
        dot(Q); 
        dot(T);

        label("$A$", A, S);
        label("$B$", B, S);
        label("$C$", C, NE);
        label("$D$", D, N);
        label("$E$", E, NW);
        label("$F$", F, SW);

        label("$P$", P, SW);
        label("$Q$", Q, SE);
        label("$T$", T, NW);
    \end{asy}
    \caption{Example \ref{hmmt_nov14_team_10} - Similar Triangles and Quadrilaterals}
\end{figure}

Let $P = BE \cap AC, Q = AE \cap FC, T = BE \cap FC$. Then, since the quadrilaterals $AFEC$ and $AECB$ are similar to one another, we have $AP : PC = AQ : QE$. Therefore, $PQ \parallel EC$.

\begin{figure}[ht!]
    \centering
    \begin{asy}
        unitsize(1cm);
        pair P, Q, T;

        pair A, B, C, D, E, F, C1, E1;
        real theta = 162, s = 1;

        A = (0, 0);
        B = (s*8, 0);
        C = s*4*dir(theta/3);
        E = s*2*dir(2*theta/3);
        F = s*dir(theta);

        C1 = rotate(degrees(B-E), C)*(C.x-1,C.y);
        //draw(C--C1, red);

        E1 = rotate(degrees(C-F), E)*(E.x+1, E.y);
        //draw(E--E1, red);

        D = extension(C, C1, E, E1);

        P = extension(B, E, A, C);
        Q = extension(A, E, C, F);
        T = extension(C, F, B, E);

        draw(E--F--A--B--C--D--cycle);
        draw(B--E^^C--F, yellow);
        
        //draw(A--C--D--E--cycle, red);
        //draw(A--D--E--F--cycle, red);
        //draw(A--F--E--cycle^^A--E--C--cycle^^A--C--B--cycle, red);
        //draw(C--F^^B--E, blue);
        draw(Q--P--T--cycle^^E--C--D--cycle, red);
        draw(A--E^^A--D^^A--C, blue+dashed);
        dot(P);
        dot(Q); 
        dot(T);

        label("$A$", A, S);
        label("$B$", B, S);
        label("$C$", C, NE);
        label("$D$", D, N);
        label("$E$", E, NW);
        label("$F$", F, SW);

        label("$P$", P, SW);
        label("$Q$", Q, SE);
        label("$T$", T, NW);
    \end{asy}
    \caption{Example \ref{hmmt_nov14_team_10} - Homothety implies $A, T, D$ Collinear}
\end{figure}


$\triangle TQP$ and $\triangle DEC$ are similar triangles with all corresponding sides similar. Therefore, there exists a \textbf{\textcolor{red}{homotethy}} mapping one triangle to the other and $EQ, DT, PC$ are collinear at the center of that homotethy. Now since $A$ is the interesection of $EQ$ and $CP$, $D, T, A$ are collinear.

\begin{figure}[ht!]
    \centering
    \begin{asy}
        unitsize(1cm);
        pair P, Q, T;

        pair A, B, C, D, E, F, C1, E1;
        real theta = 162, s = 1;

        A = (0, 0);
        B = (s*8, 0);
        C = s*4*dir(theta/3);
        E = s*2*dir(2*theta/3);
        F = s*dir(theta);

        C1 = rotate(degrees(B-E), C)*(C.x-1,C.y);
        //draw(C--C1, red);

        E1 = rotate(degrees(C-F), E)*(E.x+1, E.y);
        //draw(E--E1, red);

        D = extension(C, C1, E, E1);

        P = extension(B, E, A, C);
        Q = extension(A, E, C, F);
        T = extension(C, F, B, E);

        draw(E--F--A--B--C--D--cycle);
        draw(B--E^^C--F, yellow);
        
        //draw(A--C--D--E--cycle, red);
        //draw(A--D--E--F--cycle, red);
        //draw(A--F--E--cycle^^A--E--C--cycle^^A--C--B--cycle, red);
        //draw(C--F^^B--E, blue);
        //draw(Q--P--T--cycle^^E--C--D--cycle, red);
        //draw(A--E^^A--D^^A--C, blue+dashed);
        draw(E--T--C--D--cycle, blue);
        draw(C--E, blue);
        draw(A--E--D--C--A^^A--D, red);
        dot(P);
        dot(Q); 
        dot(T);

        label("$A$", A, S);
        label("$B$", B, S);
        label("$C$", C, NE);
        label("$D$", D, N);
        label("$E$", E, NW);
        label("$F$", F, SW);

        //label("$P$", P, SW);
        //label("$Q$", Q, SE);
        label("$T$", T, NW);
    \end{asy}
    \caption{Example \ref{hmmt_nov14_team_10} - Finish Off}
\end{figure}
Since $ETCD$ is a parallelogram, $TD$ bisects $EC$, and since $A, T, D$ are collinear, $AD$ bisects $EC$. This implies that \[[ADE] = [ACD] = \frac{1}{2}[ACDE] = 1007.\]
Also,\[ [AEF] = [ADEF] - [ADE] = 1400 - 1007 = 393. \]
Finally, \[[ABCD] = [ABC] + [ACD] = 16\cdot [AEF] + [ACD] = 16\cdot 393 + 1007 = \boxed{7295}.\]
\end{soln}

\clearpage
\section{Exercises}
\begin{problem}[2018 AMC 12 B, \#13]
    Square $ABCD$ has side length $30$. Point $P$ lies inside the square so that $AP = 12$ and $BP = 26$. The centroids of $\triangle{ABP}$, $\triangle{BCP}$, $\triangle{CDP}$, and $\triangle{DAP}$ are the vertices of a convex quadrilateral. What is the area of that quadrilateral?    
    \begin{sketch}
        $\big(\frac{2}{3}\big)^2(15\sqrt{2})^2 = \boxed{200}$.
    \end{sketch}
\end{problem}

\begin{problem}[2015 HMMT Geometry, \#2]
    Let $ABC$ be a triangle with orthocenter $H$; suppose that $AB = 13,BC = 14,CA = 15$. Let $G_A$ be the centroid of triangle $HBC$, and define $G_B$, $G_C$ similarly. Determine the area of triangle $G_AG_BG_C$.
    \begin{sketch}
        $G_AG_BG_C$ is a homotethy of the medial triangle. Therefore, $[G_AG_BG_C] = \big(\frac{2}{3})^2\big(\frac{1}{2})^2[ABC] = \boxed{\frac{28}{3}}$.
    \end{sketch}
\end{problem}


\begin{problem}
    \textit{Homotethy and Collinearity} Let $ABCD$ be a trapezoid. Construct equilateral triangles $ABE$ and $CDF$ externally. If $X$ is the intersection of $AC$ and $BD$, then prove that $E, F, X$ are collinear.
    \begin{sketch}
        The homotethy cenetered at $X$ sends $A$ to $C$ and $B$ to $D$. Therefore, it must send $E$ to $F$ (preserve triangle similarity). 
    \end{sketch}
\end{problem}

\begin{problem}
    Let $X$ be a point inside a rectangle $ABCD$ and denote by $K, L, M, N$ the reflections of $X$ about the midpoints of the sides $AB, BC, CD, DA$, respectively. Prove that $KLMN$ is a rhombus.
    \begin{sketch}
        The midpoints of the rectangle form a rohombus. And $\mathcal{H}(X, 2)$ maps each midpoint to $K, L, M, N$. Therefore, $KLMN$ is a rhombus.
    \end{sketch}
\end{problem}

\begin{problem}
    Given a fixed point $P$ and circle $\omega$, find the locus of the points $R$ on the segment $PQ$ where $PR:RQ = 2$ and $Q$ runs over the circle $\omega$.
    \begin{sketch}
        Let $\mathcal{H}$ be the homotethy $\mathcal{H}(P, 2)$, then $\mathcal{H}(Q) = R$, and therefore the locus of $R$ as $Q$ over the circle $\omega$ is a circle $\omega_2$ with radius $r_2 = 2\cdot r$ and center $O_2$ at the line segment $PO$ such that $PO_2 = 2 \cdot PO$, where $O$ and $r$ are the center and radius of $\omega$. 
    \end{sketch}
\end{problem}

\begin{problem}
    Points $B, C$ are fixed while $A$ is a variable point on a fixed circle $\omega$ of center $O$ and radius $R$. Find the locus of the centroid of the triangle $ABC$.
    \begin{sketch}
        Since centroid is $\frac{a+b+c}{3}$, $\mathcal{H}(\frac{b+c}{3}, \frac{1}{3})$ maps $A$ to the centroid. Hence the locus is a circle of radius $\frac{1}{3}\cdot R$, centered at $\frac{b+c}{3}$, where $b, c$ are the vectors for point $B, C$.
    \end{sketch}
\end{problem}

\begin{problem}
    Let $X$ be a point inside a quadrilateral $ABCD$. Prove that the centroids of the triangles $XAB, XBC, XCD, XDA$ form a parallelogram.
    \begin{sketch}
        \TBD 
    \end{sketch}
\end{problem}

\begin{problem}
    Suppose that $ABCD$ is a trapezoid with $AB \parallel CD$. Suppose that $M, N$ are the midpoints of $AB,CD$, and let $AC \cap BD = X$ and let $BC \cap AD = Y$. Prove that $X, Y, M, N$ are collinear.
    \begin{sketch}
        \TBD 
    \end{sketch}
\end{problem}

\begin{problem}
    In a triangle $ABC$, let $M, N, P$ be the midpoints of the sides $BC, CA, AB$. Denote the incenters of the three small triangles $APN, BMP, CNM$ by $I, J, K$, respectively. Prove that $MI, NJ, PK$ are concurrent.
    \begin{sketch}
        \TBD 
    \end{sketch}
\end{problem}

\begin{problem}
    Given a fixed circle with center $O$ and a point $P$ in the plance, consider a point $Q$ on the circle and let $OL$ be the angle bisector of $ \angle POQ $ with $ L \in PQ $. Find the locus of points $L$ when $Q$ runs over the circle.
    \begin{sketch}
        \TBD 
    \end{sketch}
\end{problem}

\begin{problem}
    \textit{(USAMO 1992)} Chords $AA', BB'$, and $CC'$ of a sphere $\Omega$ meet at an interior point $P$ but are not all in a plane. The sphere through $A, B, C, P$ is a tangent to the sphere through $A', B', C', P$. Prove that $AA' = BB' = CC'$.
    \begin{sketch}
        \TBD 
    \end{sketch}
\end{problem}

\begin{problem}
    Points $A$ and $B$ on a circle are fixed. Point $C$ runs along the circle. Find the set of the intersection points of the medians of triangles $ABC$. \textit{(Prasolov \cite{pgeo}, 7.26)}
    \begin{sketch}
        Let $O$ be the midpoint of segment $AB$, and $M$ the intersection point of the medians of triangle $ABC$. The homothety with center $O$ and coefficient $\frac{1}{3}$ sends point $C$ to point $M$.
        Therefore, the intersection point of the medians of triangle $ABC$ lies on circle $S$ which is
        the image of the initial circle under the homothety with center $O$ and coefficient $\frac{1}{3}$. To get the desired locus we have to delete from $S$ the images of points $A$ and $B$.
    \end{sketch}
\end{problem}

\begin{problem}
    Triangle $ABC$ is given. Find the locus of the centers of rectangles $PQRS$ whose
    vertices $Q$ and $P$ lie on side $AC$ and vertices $R$ and $S$ lie on sides $AB$ and $BC$, respectively. \textit{(Prasolov \cite{pgeo}, 7.27)}
    \begin{sketch}
    Let $O$ be the midpoint of height $BH$; let $M$, $D$ and $E$ be the midpoints of segment $AC$, and sides $RQ$ and $PS$, respectively (Fig. \TBD).

    Points $D$ and $E$ lie on lines $AO$ and $CO$, respectively. The midpoint of segment $DE$ is the center of rectangle $PQRS$. Clearly, this midpoint lies on segment $OM$. The locus in question is segment $OM$ without its endpoints.
\end{sketch}
\end{problem}

\begin{problem}
    Two circles intersect at points $A$ and $B$. Through point $A$ a line passes. It intersects the circles for the second time at points $P$ and $Q$. What is the line plotted by the midpoint of segment $PQ$ while the intersecting line rotates about point $A$. \textit{(Prasolov \cite{pgeo}, 7.28)}
    \begin{sketch}
    Let $O1$ and $O2$ be the centers of the given circles (point $P$ lies on the circle centered at $O1$); $O$ the midpoint of segment $O1O2$; $P', Q'$ and $O'$
    the projections of points $O1, O2$ and $O$ to line $P Q$. As line $P Q$ rotates, point $O'$ runs the circle $S$ with diameter $AO$. Clearly, the
    homothety with center $A$ and coefficient $2$ sends segment $P'Q'$
    to segment $P Q$, i.e., point $O'$ turns into the midpoint of segment $P Q$. Hence, the locus in question is the image of circle
    $S$ under this homothety.    
    \end{sketch}
\end{problem}

\begin{problem}
    Points $A$, $B$ and $C$ lie on one line; $B$ is between $A$ and $C$. Find the locus of points $M$ such that the radii of the circumscribed circles of triangles $AMB$ and $CMB$ are equal. \textit{(Prasolov \cite{pgeo}, 7.29)}
    \begin{sketch}
    Let $P$ and $Q$ be the centers of the circumscribed circles of triangles $AMB$ and
    $CMB$. Point $M$ belongs to the locus to be found if $BPMQ$ is a rhombus, i.e., point $M$ is the image of the midpoint of segment $P Q$ under the homothety with center $B$ and coefficient 2. Since the projections of points $P$ and $Q$ to line $AC$ are the midpoints of segments $AB$ and $BC$, respectively, the midpoints of all segments $PQ$ lie on one line. (The locus to be found is the above-obtained line without the intersection point with line $AC$.)    
    \end{sketch}
\end{problem}

\begin{remark}
    \TBD See also Problems 19.10, 19.21, 19.38.
\end{remark}



Prove that if no side of a quadrilateral is parallel to any other side, then the midpoint of the segment that connects the intersection points of the opposite sides lies on the line that connects the midpoints of the diagonals. (The Gauss line.)

\begin{problem}
    \TBD
    \begin{sketch}
        \TBD 
    \end{sketch}
\end{problem}

\begin{problem}
    \TBD
    \begin{sketch}
        \TBD 
    \end{sketch}
\end{problem}


\clearpage
\section{Solutions}
\makehints

\clearpage

\begin{thebibliography}{99}
    \bibitem{echen} Evan Chen, Euclidean Geometry in Mathematical Olympiads (EGMO)
    \bibitem{wong} Wong Yan Loi, Notes MA2219, available at: \\ \url{https://cpb-us-w2.wpmucdn.com/blog.nus.edu.sg/dist/2/12659/files/2019/12/Notes_MA2219.pdf} 
    \bibitem{pgeo} Prasolov, Problems in Plane and Solid Geometry, available at: \\ \url{http://e.math.hr/afine/planegeo.pdf}
	\bibitem{paris} Paris Pamfilos, Homotheties and Similiarities : \\ \url{http://users.math.uoc.gr/~pamfilos/eGallery/problems/Similarities.pdf}
\end{thebibliography}

\end{document}