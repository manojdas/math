\documentclass[11pt,twoside]{scrartcl}
\usepackage{mdas}
% \usepackage[sexy, fancy, hints]{evan}


\begin{document}
\title{Advanced AIME - MC 45G}
% If you contribute to the handout, put your name in comment here

\author{Manoj Das}
\org{Manoj Math Notes}
\date{Fall, 2020}

\maketitle

\section{Chapter 1 - Angles}
\begin{example}[BMT-2012-Tournament-Round5-P4]\label{bmt-12-r5-p4}
    Right triangle ABC with right angle B and
    integer side lengths has BD as the altitude. E and F are the incenters of triangles
    ADB and BDC respectively. Line EF is extended and intersects BC at G, and AB
    at H. If AB = 15 and BC = 8, find the area of triangle BGH.    
\end{example}
\begin{soln}
\begin{figure}[ht!]
    \centering
    \begin{asy}
        unitsize(0.75cm);
        import cse5;
        import olympiad;

        pair A, B, C, D, E, F, G, H, M, K;

        B = (0,0);
        A = (0,15);
        C = (8,0);
        D = foot(B, A, C);

        draw(A--B--C--cycle^^B--D);
        label("A", A, N);
        label("B", B, S);
        label("C", C, S);
        label("D", D, NE);

        E = incenter(A, D, B);
        F = incenter(C, D, B);
        dot(E);
        dot(F);
        label("E", E, SW);
        label("F", F, SW);
        draw(incircle(A, D, B));
        draw(incircle(B, C, D));

        G = extension(E, F, B, C);
        H = extension(F, E, B, A);
        draw(G--H);
        label("G", G, S);
        label("H", H, W);

        M = foot(F, B, C);
        K = foot(F, B, D);
        label("M", M, S, blue);
        label("K", K, NW, blue);
        draw (M--F--K, blue);

        draw(A--E--B^^E--D^^B--F--D, orange);

        draw(MarkAngle(s="\frac{\alpha}{2}", E, A, D, 2, red));
        draw(MarkAngle(s="\frac{\alpha}{2}", F, B, D, 3, red));
        draw(MarkAngle(s="45^\circ", A, D, E, 0.8, red));
        draw(MarkAngle(s="45^\circ", B, D, F, 0.8, red));


    \end{asy}
    \caption{Example \ref{bmt-12-r5-p4}}
\end{figure}

$\triangle AED \sim \triangle BDF$ implying $\frac{AD}{BD} = \frac{ED}{FD}$. 

Rearranging, we get $\frac{AD}{ED}
= \frac{BD}{FD}$. Also, $\angle EDF = \angle ABC = 90^\circ$. Therefore, $\triangle EDB \sim \triangle ABC$.

Implying, $\angle FED = \alpha$.

$\angle BFD = \angle BFE + \angle EFD = 180 - 45 - \frac{\alpha}{2}$, but $\angle EFD = 90 - \angle FED = 90 - \alpha$. That is $ \angle BFE + 90 - \alpha = 135 - \frac{\alpha}{2}$, or $\angle BFE = 45^\circ + \frac{\alpha}{2}$.

But $\angle BFE = \angle BGF + \frac{\alpha}{2}$. That is $45^\circ + \frac{\alpha}{2} = \angle BGF + \frac{\alpha}{2}$, implying $\angle BGF = 45^\circ$.

Therefore, $\triangle FMG \cong \triangle FKD$, and therefore $MG = KD$. 

Now $BG = BM + MG = BK + KD = BD$. 

$\triangle HBG$ is isosceles, so $BH = BG = BD$, so $[HBG] = \frac{1}{2}BD^2$.

    
\end{soln}

\clearpage
\section{Chapter 6 - Lengths II}

\begin{remark}
    This is instructor provided lemma and is not in book.
\end{remark}
\begin{lemma}\label{extra_sess6_lemma}
    Let $D, E, F$ be the points where the incircle of $ \triangle ABC$ touches $BC, AC, AB$, respectively, and $M$ be the midpoint of $AB$. Then $AB, EF,$ and the diameter of the incircle at $D$ are all concurrent.
\end{lemma}
\begin{figure}[ht!]\label{extra_sess6_lemma_fig}
    \centering
    \begin{asy}
        import cse5;
        import olympiad;
        unitsize(0.5cm);

        real a = 14, b = 13, c = 15;
        path c1, c2;
        pair A, B, C, D, E, F, D, D1, M, I, N, P, Q;

        B = (0,0);
        C = (a, 0);
        M = (B + C)/2;

        c1 = circle(B, c);
        c2 = circle(C, b);

        A = intersectionpoints(c1, c2)[0];
        I = incenter(A, B, C);
        D = foot(I, B, C);
        E = foot(I, A, C);
        F = foot(I, A, B);
        D1 = rotate(180,I)*D;
        N = intersectionpoints(D--D1, E--F)[0];


        draw(A--B--C--cycle);
        draw(incircle(A,B,C));
        draw(D--D1^^E--F^^A--M);

        //dot(I);
        //dot(D);
        dot(N, red);

        label("$A$", A, NE);
        label("$B$", B, SW);
        label("$C$", C, SE);
        label("$M$", M, S);
        label("$D$", D, S);
        label("$E$", E, NE);
        label("$F$", F, NW);
        //label("$N$", N, SE);
    \end{asy}
    \caption{Lemma \ref{extra_sess6_lemma}}
\end{figure}
\begin{proof}
    Let $N = DI \cap DF$ and $PQ \parallel BC$ and passing through $N$ as shown in Figure \ref{extra_sess6_lemma_proof_fig}.

    We want to show that $N$ defined this way and $A$ and $D$ are collinear. Sufficient to show that $N$ is the midpoint of $PQ$ because then since $PQ \parallel BC$ extension of $AN$ to $BC$ will be the midpoint of $BC$, that is $M$.

    Now $IF \perp AB$ and $PN \perp DI$ so $I, F, P, N$ are concyclic. This implies that $ \angle EFI = \angle QPI$. Similarly, $\angle PQI = \angle FEI$. Therefore, since $IF = IE$, $IP = IQ$ and $N$ is midpoint of $AD$.
    
\begin{figure}[ht!]\label{extra_sess6_lemma_proof_fig}
    \centering
    \begin{asy}
        import cse5;
        import olympiad;
        unitsize(0.5cm);

        real a = 14, b = 13, c = 15;
        path c1, c2;
        pair A, B, C, D, E, F, D, D1, M, I, N, P, Q;

        B = (0,0);
        C = (a, 0);
        M = (B + C)/2;

        c1 = circle(B, c);
        c2 = circle(C, b);

        A = intersectionpoints(c1, c2)[0];
        I = incenter(A, B, C);
        D = foot(I, B, C);
        E = foot(I, A, C);
        F = foot(I, A, B);
        D1 = rotate(180,I)*D;
        N = intersectionpoints(D--D1, E--F)[0];
        P = extension(N, (N.x-1, N.y), A, B);
        Q = extension(N, (N.x+1, N.y), A, C);

        draw(A--B--C--cycle);
        draw(incircle(A,B,C));
        draw(D--D1^^E--F^^A--M);
        draw(P--Q--I--cycle^^F--I--E, blue);

        dot(I);
        //dot(D);
        dot(N, red);


        label("$A$", A, NE);
        label("$B$", B, SW);
        label("$C$", C, SE);
        label("$M$", M, S);
        label("$D$", D, S);
        label("$E$", E, SW);
        label("$F$", F, NW);
        label("$N$", N, SE);
        label("$P$", P, SE);
        label("$Q$", Q, NE);
        label("$I$", I, SE);

    \end{asy}
    \caption{Proof for lemma \ref{extra_sess6_lemma}}
\end{figure}
\end{proof}

\begin{example}[HMMT Feb 2007, Guts \#36]\label{hmmt_feb07_guts_36}
    The Marathon. Let $\omega$ denote the incircle of triangle $ABC$. The segments $BC, CA,$ and $AB$ are
    tangent to $\omega$ at $D, E,$ and $F$, respectively. Point $P$ lies on $EF$ such that segment $PD$ is perpendicular
    to $BC$. The line $AP$ intersects $BC$ at $Q$. The circles $\omega_1$ and $\omega_2$ pass through $B$ and $C$, respectively,
    and are tangent to $AQ$ at $Q$; the former meets $AB$ again at $X$, and the latter meets $AC$ again at $Y$ .
    The line $XY$ intersects $BC$ at $Z$. Given that $AB = 15, BC = 14$, and $CA = 13$, find $\floor{XZ \cdot YZ}$.    
\end{example}


\begin{soln}
    Solution outline:
    \begin{enumerate}
        \item Use previous lemma, lemma \ref{extra_sess6_lemma}, to know thar $Q$ is the midpoint.
        \item Use Stewart's to find the length of the median.
        \item Use power of A w.r.t $\omega$ to find $AX, BY$.
        \item $BXYC$ is cyclic (see Claim \ref{hmmt_feb07_guts_36_claim}); use power of $Z$ w.r.t this circle
        \item Use Menelaus on $\triangle ABC$ with $Z$. 
    \end{enumerate}
    
    \begin{figure}[ht!]\label{hmmt_feb07_guts_36_fig}
        \centering
        \begin{asy}
            import cse5;
            import olympiad;
            unitsize(0.5cm);
    
            real a = 14, b = 13, c = 15;
            path c1, c2, w1, w2;
            pair A, B, C, D, E, F, D, D1, Q, I, N, P, Q, X, Y, Z;
    
            B = (0,0);
            C = (a, 0);
            Q = (B + C)/2;
    
            c1 = circle(B, c);
            c2 = circle(C, b);
    
            A = intersectionpoints(c1, c2)[0];
            I = incenter(A, B, C);
            D = foot(I, B, C);
            E = foot(I, A, C);
            F = foot(I, A, B);
            D1 = rotate(180,I)*D;
            P = intersectionpoints(D--D1, E--F)[0];
    
            //These numbers come from soln (Stewart's theorem)
            X = ((148/15)*B + (15-148/15)*A)/15;
            Y = ((148/13)*C + (13-148/13)*A)/13;
            w1 = circumcircle(Q, X, B);
            w2 = circumcircle(Q, Y, C);

            Z = extension (X, Y, B, C);

            //dot(X, red);
            //dot(Y, red);

            draw(A--B--C--cycle);
            draw(incircle(A,B,C));
            draw(D--P^^E--F^^A--Q^^X--Z--C);
            draw(w1^^w2);

            //draw(P--Q--I--cycle^^F--I--E, blue);
            
            //N = circumcenter(B, X, C);
            //dot(N, red);
    
    
            label("$A$", A, NE);
            label("$B$", B, SW);
            label("$C$", C, SE);
            label("$Q$", Q, SW);
            label("$D$", D, S);
            label("$E$", E, NE);
            label("$F$", F, NW);
            label("$P$", P, SE);
            label("$X$", X, NW);
            label("$Y$", Y, NE);
            label("$Z$", Z, S);

        \end{asy}
        \caption{Example \ref{hmmt_feb07_guts_36}}
    \end{figure}


    By Stewart's theorem (and using that AQ is median), we have 
    \[AQ^2 = \frac{2\cdot13^{2}+2\cdot15^{2}-14^{2}}{4} = 148.\]


    Using power of $A$ w.r.t $\omega_1, \omega_2$ we have 
    \[AX \cdot AB = AY \cdot AC = AQ^2 = 148.\]
    Therefore, $AX = \frac{148}{15}$ and $AY = \frac{148}{13}$. 

    Using power of $Z$ on $(BXYC)$, we have
    \[ZY \cdot ZX = ZC \cdot ZB.\]
    
    Using Menelaus on $\triangle ABC$ and $Z$, we have
    \[ \frac{ZB}{ZC} \cdot \frac{YC}{AY} \cdot \frac{AX}{BX} = 1.\]

    $YC = 13 - \frac{148}{13} = \frac{21}{13}$, and $BX = 15 - \frac{148}{15} = \frac{77}{15}$. Then we get 
    \[\frac{ZB}{ZC} = \frac{AY}{YC} \cdot \frac{BX}{AX} = \frac{\frac{148}{13}}{\frac{21}{13}} \cdot \frac{\frac{77}{15}}{\frac{148}{15}} = \frac{11}{3}.\]

    Since, $ZC$ lies outside of segment $BC$, this means $\frac{ZC + 14}{ZC} = \frac{11}{3}$, giving $ZC = \frac{21}{4}$. Therefore,
    \[\floor{ZC \cdot ZB} = \floor{\frac{11}{3}\Big(\frac{21}{4}\Big)^2} = \boxed{101}.\]
\end{soln}
\begin{claim}\label{hmmt_feb07_guts_36_claim}
    $BXYC$ is cyclic.
\end{claim}
\begin{remark}
    This seems like a very general configuration. May be there is an existing lemma for it. Seems the claim could be more generally stated as: Let $AQ$ be the common tangent to $\omega_1, \omega_2$ at $Q$, and the segment $BQC$ intersect the circles at $B, C$. Let $X, Y$ be the other intersection of $AB, AC$ with the circles $\omega_1, \omega_2$. Then $AXQY$ and $BCYX$ are cyclic.

    \TBD verify with Sanjana. In particular, that this does not rely on $Q$ being the midpoint of $BC$.
\end{remark}
\begin{proof}
    Let $\angle AQC = \theta_1$. By \emph{Tagent Cord Angle} $\angle BXQ = \theta_1$, and similarly, $\angle QXC = 180 - \theta_1$, and therefore, $\angle AYQ = theta_1$. 

    Now, we have $\triangle AQY \sim \triangle ACQ$, and hence $\angle AQY = \angle ACQ = \gamma$. Similarly, $\angle AQX = \beta$. Then $\angle XAY + \angle XQY = \alpha + \beta + \gamma = 190^\circ$, and $AXQY$ is cyclic.

    Since $AXQY$ is cyclic, $\angle AXY = \angle AQY = \gamma$, and therefore, $BXYC$ is cyclic.

    \begin{figure}[ht!]\label{hmmt_feb07_guts_36_fig_2}
        \centering
        \begin{asy}
            import cse5;
            import olympiad;
            unitsize(0.5cm);
    
            real a = 14, b = 13, c = 15;
            path c1, c2, w1, w2;
            pair A, B, C, D, E, F, D, D1, Q, I, N, P, Q, X, Y, Z;
    
            B = (0,0);
            C = (a, 0);
            Q = (B + C)/2;
    
            c1 = circle(B, c);
            c2 = circle(C, b);
    
            A = intersectionpoints(c1, c2)[0];
            I = incenter(A, B, C);
            D = foot(I, B, C);
            E = foot(I, A, C);
            F = foot(I, A, B);
            D1 = rotate(180,I)*D;
            P = intersectionpoints(D--D1, E--F)[0];
    
            //These numbers come from soln (Stewart's theorem)
            X = ((148/15)*B + (15-148/15)*A)/15;
            Y = ((148/13)*C + (13-148/13)*A)/13;
            w1 = circumcircle(Q, X, B);
            w2 = circumcircle(Q, Y, C);

            Z = extension (X, Y, B, C);

            //dot(X, red);
            //dot(Y, red);

            draw(A--B--C--cycle);
            //draw(incircle(A,B,C));
            draw(A--Q^^X--Y);
            draw(w1^^w2);

            draw(X--Q--Y, blue+dashed);
            //draw(P--Q--I--cycle^^F--I--E, blue);
            
            //N = circumcenter(B, X, C);
            //dot(N, red);
    
    
            label("$A$", A, NE);
            label("$B$", B, SW);
            label("$C$", C, SE);
            label("$Q$", Q, SW);
            //label("$D$", D, S);
            //label("$E$", E, NE);
            //label("$F$", F, NW);
            //label("$P$", P, SE);
            label("$X$", X, NW);
            label("$Y$", Y, NE);
            //label("$Z$", Z, S);

            markangle("$\theta_1$", radius=8, C,Q,A);
            markangle("$\theta_1$",radius=8, B,X,Q);
            markangle("$180-\theta_1$",radius=8, Q,Y,C);
            //markangle("$\frac{\alpha}{2}$", radius=12, X, A, Q);
            //markangle("$\frac{\alpha}{2}$", radius=12, Q, A, C);
            markangle("$\gamma$", radius=25, Y, Q, A, red);
            markangle("$\beta$", radius=15, A, Q, X, red);
            markangle("$\gamma$", radius=15, Y, X, A, orange);
        \end{asy}
        \caption{Claim \ref{hmmt_feb07_guts_36_claim} - $BXYC$ is cyclic (and so is $AXQY$)}
    \end{figure}
\end{proof}
\clearpage
\section{Instructor Insights}
\begin{enumerate}
    \item Advanced Geometry is lot about recognizing configurations.
\end{enumerate}
\end{document}