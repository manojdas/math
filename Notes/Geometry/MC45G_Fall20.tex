\documentclass[11pt,twoside]{scrartcl}
\usepackage{mdas}
% \usepackage[sexy, fancy, hints]{evan}


\begin{document}
\title{Advanced AIME - MC 45G}
% If you contribute to the handout, put your name in comment here

\author{Manoj Das}
\org{Manoj Math Notes}
\date{Fall, 2020}

\maketitle

\section{Chapter 1 - Angles}
\begin{example}[BMT-2012-Tournament-Round5-P4]\label{bmt-12-r5-p4}
    Right triangle ABC with right angle B and
    integer side lengths has BD as the altitude. E and F are the incenters of triangles
    ADB and BDC respectively. Line EF is extended and intersects BC at G, and AB
    at H. If AB = 15 and BC = 8, find the area of triangle BGH.    
\end{example}
\begin{soln}
\begin{figure}[ht!]
    \centering
    \begin{asy}
        unitsize(0.75cm);
        import cse5;
        import olympiad;

        pair A, B, C, D, E, F, G, H, M, K;

        B = (0,0);
        A = (0,15);
        C = (8,0);
        D = foot(B, A, C);

        draw(A--B--C--cycle^^B--D);
        label("A", A, N);
        label("B", B, S);
        label("C", C, S);
        label("D", D, NE);

        E = incenter(A, D, B);
        F = incenter(C, D, B);
        dot(E);
        dot(F);
        label("E", E, SW);
        label("F", F, SW);
        draw(incircle(A, D, B));
        draw(incircle(B, C, D));

        G = extension(E, F, B, C);
        H = extension(F, E, B, A);
        draw(G--H);
        label("G", G, S);
        label("H", H, W);

        M = foot(F, B, C);
        K = foot(F, B, D);
        label("M", M, S, blue);
        label("K", K, NW, blue);
        draw (M--F--K, blue);

        draw(A--E--B^^E--D^^B--F--D, orange);

        draw(MarkAngle(s="\frac{\alpha}{2}", E, A, D, 2, red));
        draw(MarkAngle(s="\frac{\alpha}{2}", F, B, D, 3, red));
        draw(MarkAngle(s="45^\circ", A, D, E, 0.8, red));
        draw(MarkAngle(s="45^\circ", B, D, F, 0.8, red));


    \end{asy}
    \caption{Example \ref{bmt-12-r5-p4}}
\end{figure}

$\triangle AED \sim \triangle BDF$ implying $\frac{AD}{BD} = \frac{ED}{FD}$. 

Rearranging, we get $\frac{AD}{ED}
= \frac{BD}{FD}$. Also, $\angle EDF = \angle ABC = 90^\circ$. Therefore, $\triangle EDF \sim \triangle ADB$.

Implying, $\angle FED = \alpha$.

$\angle BFD = \angle BFE + \angle EFD = 180 - 45 - \frac{\alpha}{2}$, but $\angle EFD = 90 - \angle FED = 90 - \alpha$. That is $ \angle BFE + 90 - \alpha = 135 - \frac{\alpha}{2}$, or $\angle BFE = 45^\circ + \frac{\alpha}{2}$.

But $\angle BFE = \angle BGF + \frac{\alpha}{2}$. That is $45^\circ + \frac{\alpha}{2} = \angle BGF + \frac{\alpha}{2}$, implying $\angle BGF = 45^\circ$.

Therefore, $\triangle FMG \cong \triangle FKD$, and therefore $MG = KD$. 

Now $BG = BM + MG = BK + KD = BD$. 

$\triangle HBG$ is isosceles, so $BH = BG = BD$, so $[HBG] = \frac{1}{2}BD^2$.

    
\end{soln}

\clearpage
\section{Chapter 4 - Special Points of a Triangle}

\subsection{Examples}
\begin{example}[2007 AIME II, \# 15]\label{aime_07_2_15}
    Four circles $\omega, \omega_A, \omega_B,$ and $\omega_C$ with the same radius are drawn in the interior of triangle $ABC$ such that $\omega_A$ is tangent to sides $AB$ and $AC$, $\omega_B$ to $BC$ and $BA$, $\omega_C$ to $CA$ and $CB$, and $\omega$ is externally tangent to $\omega_A, \omega_B,$ and $\omega_C$. If the sides of triangle $ABC$ are 13, 14, and 15, the radius of $\omega$ can be represented in the form $\frac{m}{n}$, where $m$ and $n$ are relatively prime positive integers. Find $m + n$.
\end{example}

\begin{figure}[ht!]
    \centering
    \begin{asy}
        import olympiad;

        unitsize(0.75cm);
        pair A, B, C, Oa, Ob, Oc, Od, O, I;
        path circ1, circ2;

        // Homotethy factor - backplugged from solution
        real k = 64/129;
        real r = 260/129;

        B = (0, 0);
        C = (14, 0);

        circ1 = circle(B, 13);
        circ2 = circle(C, 15);

        A = intersectionpoints(circ1, circ2)[0];
        I = incenter(A, B, C);

        Oa = (65*A + 64*I)/129;
        Ob = (65*B + 64*I)/129;
        Oc = (65*C + 64*I)/129;

        Od = circumcenter(Oa, Ob, Oc);
        O = circumcenter(A, B, C);

        draw(circle(Oa, r));
        draw(circle(Ob, r));
        draw(circle(Oc, r));
        draw(circle(Od, r));

        draw(incircle(Oa, Ob, Oc)^^incircle(A, B, C)^^I--foot(I, A, C), green);
        draw(A--B--C--cycle);
        draw(Oa--Ob--Oc--cycle, blue);
        draw(A--I--B^^I--C, blue);
        draw(Oa--foot(Oa, A, C)^^Oc--foot(Oc, A, C), blue);
        draw(rightanglemark(Oa, foot(Oa, A, C), C)^^rightanglemark(Oc, foot(Oc, A, C), A));
        dot(I);
        dot(Oa);
        dot(Ob);
        dot(Oc);
        dot(Od);
        dot(O, red);

        label("$A$", A, N);
        label("$B$", B, S);
        label("$C$", C, S);
        label("$I$", I, S);
        label("$O_A$", Oa, NW);
        label("$O_B$", Ob, SW);
        label("$O_C$", Oc, SE);
        label("$O_\omega$", Od, N);
        label("$O$", O, SE, red);

    \end{asy}
    \caption{Example \ref{aime_07_2_15}}
    \label{aime_07_2_15_fig}
\end{figure}

\begin{soln}
    First, we notice that $O_AO_C \parallel AC$, $O_AO_B \parallel AB$, and $O_BO_C \parallel BC$ (because same radius circles tangent to sides, refer to Figure \ref{aime_07_2_15_fig}). Therefore, not only are triangles $ABC$ and $O_AO_BO_C$ similar but also there is a Homotethy $\mathcal{H}$ mapping one to the other. 

    Where is the center of $\mathcal{H}$ and what is the dilation factor, $k$? Since $\mathcal{H}$ maps $A, B, C$ to $O_A, O_B, O_C$, its center is at the point of intersection of $AO_A, BO_B, CO_C$. Since the circles are tangents to two sides, their centers lie on the angle bisectors and therefore, the center of $\mathcal{H}$ is the incenter $\mathcal{I}$ of $\triangle ABC$.

    Therefore, $\mathcal{H}$ maps the incircle of $\triangle ABC$ to the incircle of $\triangle O_AO_BO_C$. Let $r_\omega, r, r'$ be the radius of circle $\omega$, inradius of $\triangle ABC$, and inradius of $\triangle O_AO_BO_C$, respectively. Then $r' = r - r_\omega$ and the dilation factor is $k = \frac{r'}{r} = \frac{r - r_\omega}{r}$.

    Since we have the side lengths of $\triangle{ABC}$, we can calulate its inradius $r = \frac{[ABC]}{s} = \frac{\sqrt{21\cdot8\cdot7\cdot6}}{21} = 4$. Therefore, dilation factor $k = \frac{4-r_\omega}{4}$.

    Next, we note that the center of the circle $\omega$, $O_\omega$ is at a distance $2r_\omega$ from $O_A, O_B, O_C$. Therefore, it is the circumcenter of $\triangle O_AO_BO_C$, and is the mapping of the circumcenter $O$ of $\triangle ABC$.

    Let $R, R'$ be the circumradius of $\triangle ABC, O_AO_BO_C$. By Homotethy $\mathcal{H}$, $\frac{R'}{R} = k = \frac{4-r_\omega}{4}$. But $R' = 2r_\omega$. Therefore, $\frac{2r_\omega}{R} = \frac{4-r_\omega}{4}$.

    We have $R = \frac{a\cdot b\cdot c}{4[ABC]} = {13 \cdot 14 \cdot 15}{4 \cdot 84} = \frac{65}{8}$. Putting it all together $\frac{2r_\omega}{\frac{65}{8}} = \frac{4-r_\omega}{4}$, and $r_w = \boxed{\frac{260}{129}}.$
\end{soln}

\begin{example}[HMMT Feb 2019, Geometry 8]\label{hmmt_feb_19_geo_8}
    In triangle $ABC$ with $AB < AC$, let $H$ be the orthocenter and $O$ be the circumcenter. Given that the midpoint of $OH$ lies on $BC$, $BC = 1$, and the perimeter of $ABC$ is 6, find the area of $ABC$.
\end{example}

\begin{figure}[ht]
    \centering
    \begin{asy}
        unitsize(0.25cm);
        pair A, B, C, N, Ma, Mb, Mc, M2;

        A = (-25, 24);
        B = (-7, 0);
        C = (7, 0);

        N = (-25/2, 0);
        Ma = (0, 0);
        Mb = (A + C) / 2;
        Mc = (A + B) / 2;

        M2 = (Mb + Mc)/2;

        draw((M2.x, M2.y+5)--(M2.x, -0.5), blue+dashed);
        draw(A--B--C--cycle);
        draw (Ma--Mb--Mc--cycle, blue);

        draw(Ma--N^^Mb--N^^Mc--N, green);

        dot(N, red);
        dot(Ma);
        dot(Mb);
        dot(Mc);

        label("$A$", A, NE);
        label("$B$", B, S);
        label("$C$", C, S);
        label("$N_9$", N, S);
        label("$M_a$", Ma, S);
        label("$M_b$", Mb, NE);
        label("$M_c$", Mc, NW);

    \end{asy}
    \caption{Figure \ref{hmmt_feb_19_geo_8}}
    \label{hmmt_feb_19_geo_8_fig}
\end{figure}
\begin{remark}
    The key observations in this problem are:
    \begin{itemize}
        \item The midpoint of $O$ and $H$ is $N_9$, the center of the nine-point circle.
        \item The ninepoint circle is the circumcircle of the medial triangle.
    \end{itemize}
    After this, the problem may be solved using coordinate bashing or trig bashing to get $\boxed{\frac{6}{7}}$.
\end{remark}
\clearpage
\section{Chapter 5 - Lengths I}
\subsection{Additional Examples}
\begin{example}[HMMT Feb 2009, Geometry \#10]
    Points $A, B$ lie on a circle $\omega$. $P$ lies on extension of $AB$ past $B$. Line $l$ passes through $P$ and is tangent to $\omega$. The tangents to $\omega$ at $A$ and $B$ intersect $l$ at $D$ and $C$, respectively. If $AB=7, BC=2, AD=3$, find $BP$.
\end{example}

\begin{note}[Hint]
    Set up Menelaus such that end up with only $CP$ and numerical values.
\end{note}

\begin{soln}
    \TBD \ldots
    $l$ tangent to $\omega$ at $T$. 
    \[CD = CT + DT = BC + AD = 5.\]
    Solve use Menelaus and Power of Point. 
    \ldots $\boxed{9}$.
\end{soln}
\clearpage
\section{Chapter 6 - Lengths II}
\subsection{Concepts}
\begin{enumerate}
    \item Power of a point
    \item Radical axis and radical center
    \item Ptolemy's theorem
\end{enumerate}

\begin{remark}
    The proof of Ptolemy's is very instructive (how an auxilary line is used).
\end{remark}
\subsection{Additional Examples}
\begin{example}\label{ch6_add_1}
    Triangle $ABC$ has side lengths $AB = 65, BC = 33,$ and $AC = 56$. Find the radius of the circle tangent to sides $AC$ and $BC$ and to the circumcircle of $\triangle ABC$.
\end{example}
\begin{note}
    $\triangle ABC$ is right-angled $\triangle$.
\end{note}
\begin{figure}[ht!]
    \centering
    \begin{asy}
        //ch6_add_1
        import olympiad;

        unitsize(0.10cm);
        pair A, B, C, D, I, E, O;
        path c1;
        pair []OE;
        real r = 24;

        A = (0, 56);
        B = (33, 0);
        C = (0, 0);
        I = incenter(A, B, C);
        c1 = circumcircle(A, B, C);
        O = circumcenter(A, B, C);

        D = intersectionpoints(C--I*5, c1)[1];
        E = (r, r);
        OE = intersectionpoints((O-(E-O)*4)--(O+(E-O)*4), c1);

        draw (c1);
        draw (circle(E, r));
        draw(A--C--B--cycle);
        draw(C--D^^A--D--B, blue);
        //dot (I);
        dot(D);
        dot (E);
        dot (O);
        dot(OE[1]);

        draw(OE[0]--OE[1], red);

        label("$A$", A, N);
        label("$C$", C, S);
        label("$B$", B, S);
        label("$E$", E, S);
        label("$D$", D, NE);
        label("$O$", O, SW);
        label("$T$", OE[1], SE);
    \end{asy}

    \caption{Example \ref{ch6_add_1}}
\end{figure}
\begin{soln}
    Let $E$ be the center of the second circle. Extend $CE$ to intersect $(ABC)$ at $D$. Since $CD$ is angle bisector of $\angle ACB$ (tangent to two sides), $D$ is the midpoint of $\arc{ADB}$.

    Therefore, $\triangle ADB$ is isosceles triangle, and specifically 45-45-90 triangle. Therefore $AD = BD = \frac{AB}{\sqrt{2}}$.

    First, note that by Homothety, the common point of tangency, $T$, the first circle's center, $O$, and the second circle's center, $E$ are all collinear. Therefore, the power of $E$ on the bigger circle evaluated along the diameter $OT$ is $r \cdot (65 - r)$. And evaluated along $CD$, it is $CE \cdot ED$. Therefore,
    \[CE \cdot ED = r \cdot (65 - r).\]
    But $CE = \sqrt{2} r$, and $ED = CD - CE$; plugging in the above, $\sqrt{2} r (CD - \sqrt{2}r) = r \cdot (65 - r)$, which simplifies to
    \[\sqrt{2} CD - r = 65.\]
    Applying \emph{Ptolemy} to $ADBC$ we get,
    \begin{align*}
        AC \cdot BD + AD \cdot BC &= AB \cdot CD, \\
        AC \cdot \frac{AB}{\sqrt{2}} + \frac{AB}{\sqrt{2}} \cdot BC &= AB \cdot CD, \\
        CD &= \frac{AC + BC}{\sqrt{2}}, \\
        CD &= \frac{89\sqrt{2}}{2}.
    \end{align*}
    Plugging in the previous equation we have $89 - r = 65$ and $r = \boxed{24}$.
\end{soln}
\begin{remark}
    This is instructor provided lemma and is not in book.
\end{remark}
\begin{lemma}\label{extra_sess6_lemma}
    Let $D, E, F$ be the points where the incircle of $ \triangle ABC$ touches $BC, AC, AB$, respectively, and $M$ be the midpoint of $AB$. Then $AB, EF,$ and the diameter of the incircle at $D$ are all concurrent.
\end{lemma}
\begin{figure}[ht!]
    \centering
    \begin{asy}
        import cse5;
        import olympiad;
        unitsize(0.5cm);

        real a = 14, b = 13, c = 15;
        path c1, c2;
        pair A, B, C, D, E, F, D, D1, M, I, N, P, Q;

        B = (0,0);
        C = (a, 0);
        M = (B + C)/2;

        c1 = circle(B, c);
        c2 = circle(C, b);

        A = intersectionpoints(c1, c2)[0];
        I = incenter(A, B, C);
        D = foot(I, B, C);
        E = foot(I, A, C);
        F = foot(I, A, B);
        D1 = rotate(180,I)*D;
        N = intersectionpoints(D--D1, E--F)[0];


        draw(A--B--C--cycle);
        draw(incircle(A,B,C));
        draw(D--D1^^E--F^^A--M);

        //dot(I);
        //dot(D);
        dot(N, red);

        label("$A$", A, NE);
        label("$B$", B, SW);
        label("$C$", C, SE);
        label("$M$", M, S);
        label("$D$", D, S);
        label("$E$", E, NE);
        label("$F$", F, NW);
        //label("$N$", N, SE);
    \end{asy}
    \caption{Lemma \ref{extra_sess6_lemma}}
    \label{extra_sess6_lemma_fig}
\end{figure}
\begin{proof}
    Let $N = DI \cap DF$ and $PQ \parallel BC$ and passing through $N$ as shown in Figure \ref{extra_sess6_lemma_proof_fig}.

    We want to show that $N$ defined this way and $A$ and $D$ are collinear. Sufficient to show that $N$ is the midpoint of $PQ$ because then since $PQ \parallel BC$ extension of $AN$ to $BC$ will be the midpoint of $BC$, that is $M$.

    Now $IF \perp AB$ and $PN \perp DI$ so $I, F, P, N$ are concyclic. This implies that $ \angle EFI = \angle QPI$. Similarly, $\angle PQI = \angle FEI$. Therefore, since $IF = IE$, $IP = IQ$ and $N$ is midpoint of $AD$.
    
\begin{figure}[ht!]
    \centering
    \begin{asy}
        import cse5;
        import olympiad;
        unitsize(0.5cm);

        real a = 14, b = 13, c = 15;
        path c1, c2;
        pair A, B, C, D, E, F, D, D1, M, I, N, P, Q;

        B = (0,0);
        C = (a, 0);
        M = (B + C)/2;

        c1 = circle(B, c);
        c2 = circle(C, b);

        A = intersectionpoints(c1, c2)[0];
        I = incenter(A, B, C);
        D = foot(I, B, C);
        E = foot(I, A, C);
        F = foot(I, A, B);
        D1 = rotate(180,I)*D;
        N = intersectionpoints(D--D1, E--F)[0];
        P = extension(N, (N.x-1, N.y), A, B);
        Q = extension(N, (N.x+1, N.y), A, C);

        draw(A--B--C--cycle);
        draw(incircle(A,B,C));
        draw(D--D1^^E--F^^A--M);
        draw(P--Q--I--cycle^^F--I--E, blue);

        dot(I);
        //dot(D);
        dot(N, red);


        label("$A$", A, NE);
        label("$B$", B, SW);
        label("$C$", C, SE);
        label("$M$", M, S);
        label("$D$", D, S);
        label("$E$", E, SW);
        label("$F$", F, NW);
        label("$N$", N, SE);
        label("$P$", P, SE);
        label("$Q$", Q, NE);
        label("$I$", I, SE);

    \end{asy}
    \caption{Proof for lemma \ref{extra_sess6_lemma}}
    \label{extra_sess6_lemma_proof_fig}
\end{figure}
\end{proof}

\begin{example}[HMMT Feb 2007, Guts \#36]\label{hmmt_feb07_guts_36}
    The Marathon. Let $\omega$ denote the incircle of triangle $ABC$. The segments $BC, CA,$ and $AB$ are
    tangent to $\omega$ at $D, E,$ and $F$, respectively. Point $P$ lies on $EF$ such that segment $PD$ is perpendicular
    to $BC$. The line $AP$ intersects $BC$ at $Q$. The circles $\omega_1$ and $\omega_2$ pass through $B$ and $C$, respectively,
    and are tangent to $AQ$ at $Q$; the former meets $AB$ again at $X$, and the latter meets $AC$ again at $Y$ .
    The line $XY$ intersects $BC$ at $Z$. Given that $AB = 15, BC = 14$, and $CA = 13$, find $\floor{XZ \cdot YZ}$.    
\end{example}


\begin{soln}
    Solution outline:
    \begin{enumerate}
        \item Use previous lemma, lemma \ref{extra_sess6_lemma}, to know thar $Q$ is the midpoint.
        \item Use Stewart's to find the length of the median.
        \item Use power of A w.r.t $\omega$ to find $AX, BY$.
        \item $BXYC$ is cyclic (see Claim \ref{hmmt_feb07_guts_36_claim}); use power of $Z$ w.r.t this circle
        \item Use Menelaus on $\triangle ABC$ with $Z$. 
    \end{enumerate}
    
    \begin{figure}[ht!]
        \centering
        \begin{asy}
            import cse5;
            import olympiad;
            unitsize(0.5cm);
    
            real a = 14, b = 13, c = 15;
            path c1, c2, w1, w2;
            pair A, B, C, D, E, F, D, D1, Q, I, N, P, Q, X, Y, Z;
    
            B = (0,0);
            C = (a, 0);
            Q = (B + C)/2;
    
            c1 = circle(B, c);
            c2 = circle(C, b);
    
            A = intersectionpoints(c1, c2)[0];
            I = incenter(A, B, C);
            D = foot(I, B, C);
            E = foot(I, A, C);
            F = foot(I, A, B);
            D1 = rotate(180,I)*D;
            P = intersectionpoints(D--D1, E--F)[0];
    
            //These numbers come from soln (Stewart's theorem)
            X = ((148/15)*B + (15-148/15)*A)/15;
            Y = ((148/13)*C + (13-148/13)*A)/13;
            w1 = circumcircle(Q, X, B);
            w2 = circumcircle(Q, Y, C);

            Z = extension (X, Y, B, C);

            //dot(X, red);
            //dot(Y, red);

            draw(A--B--C--cycle);
            draw(incircle(A,B,C));
            draw(D--P^^E--F^^A--Q^^X--Z--C);
            draw(w1^^w2);

            //draw(P--Q--I--cycle^^F--I--E, blue);
            
            //N = circumcenter(B, X, C);
            //dot(N, red);
    
    
            label("$A$", A, NE);
            label("$B$", B, SW);
            label("$C$", C, SE);
            label("$Q$", Q, SW);
            label("$D$", D, S);
            label("$E$", E, NE);
            label("$F$", F, NW);
            label("$P$", P, SE);
            label("$X$", X, NW);
            label("$Y$", Y, NE);
            label("$Z$", Z, S);

        \end{asy}
        \caption{Example \ref{hmmt_feb07_guts_36}}
        \label{hmmt_feb07_guts_36_fig}
    \end{figure}


    By Stewart's theorem (and using that AQ is median), we have 
    \[AQ^2 = \frac{2\cdot13^{2}+2\cdot15^{2}-14^{2}}{4} = 148.\]


    Using power of $A$ w.r.t $\omega_1, \omega_2$ we have 
    \[AX \cdot AB = AY \cdot AC = AQ^2 = 148.\]
    Therefore, $AX = \frac{148}{15}$ and $AY = \frac{148}{13}$. 

    Using power of $Z$ on $(BXYC)$, we have
    \[ZY \cdot ZX = ZC \cdot ZB.\]
    
    Using Menelaus on $\triangle ABC$ and $Z$, we have
    \[ \frac{ZB}{ZC} \cdot \frac{YC}{AY} \cdot \frac{AX}{BX} = 1.\]

    $YC = 13 - \frac{148}{13} = \frac{21}{13}$, and $BX = 15 - \frac{148}{15} = \frac{77}{15}$. Then we get 
    \[\frac{ZB}{ZC} = \frac{AY}{YC} \cdot \frac{BX}{AX} = \frac{\frac{148}{13}}{\frac{21}{13}} \cdot \frac{\frac{77}{15}}{\frac{148}{15}} = \frac{11}{3}.\]

    Since, $ZC$ lies outside of segment $BC$, this means $\frac{ZC + 14}{ZC} = \frac{11}{3}$, giving $ZC = \frac{21}{4}$. Therefore,
    \[\floor{ZC \cdot ZB} = \floor{\frac{11}{3}\Big(\frac{21}{4}\Big)^2} = \boxed{101}.\]
\end{soln}
\begin{claim}\label{hmmt_feb07_guts_36_claim}
    $BXYC$ is cyclic.
\end{claim}
\begin{remark}
    This seems like a very general configuration. May be there is an existing lemma for it. Seems the claim could be more generally stated as: Let $AQ$ be the common tangent to $\omega_1, \omega_2$ at $Q$, and the segment $BQC$ intersect the circles at $B, C$. Let $X, Y$ be the other intersection of $AB, AC$ with the circles $\omega_1, \omega_2$. Then $AXQY$ and $BCYX$ are cyclic.

    \TBD verify with Sanjana. In particular, that this does not rely on $Q$ being the midpoint of $BC$.
\end{remark}
\begin{proof}
    Let $\angle AQC = \theta_1$. By \emph{Tagent Cord Angle} $\angle BXQ = \theta_1$, and similarly, $\angle QXC = 180 - \theta_1$, and therefore, $\angle AYQ = theta_1$. 

    Now, we have $\triangle AQY \sim \triangle ACQ$, and hence $\angle AQY = \angle ACQ = \gamma$. Similarly, $\angle AQX = \beta$. Then $\angle XAY + \angle XQY = \alpha + \beta + \gamma = 190^\circ$, and $AXQY$ is cyclic.

    Since $AXQY$ is cyclic, $\angle AXY = \angle AQY = \gamma$, and therefore, $BXYC$ is cyclic.

    \begin{figure}[ht!]
        \centering
        \begin{asy}
            import cse5;
            import olympiad;
            unitsize(0.5cm);
    
            real a = 14, b = 13, c = 15;
            path c1, c2, w1, w2;
            pair A, B, C, D, E, F, D, D1, Q, I, N, P, Q, X, Y, Z;
    
            B = (0,0);
            C = (a, 0);
            Q = (B + C)/2;
    
            c1 = circle(B, c);
            c2 = circle(C, b);
    
            A = intersectionpoints(c1, c2)[0];
            I = incenter(A, B, C);
            D = foot(I, B, C);
            E = foot(I, A, C);
            F = foot(I, A, B);
            D1 = rotate(180,I)*D;
            P = intersectionpoints(D--D1, E--F)[0];
    
            //These numbers come from soln (Stewart's theorem)
            X = ((148/15)*B + (15-148/15)*A)/15;
            Y = ((148/13)*C + (13-148/13)*A)/13;
            w1 = circumcircle(Q, X, B);
            w2 = circumcircle(Q, Y, C);

            Z = extension (X, Y, B, C);

            //dot(X, red);
            //dot(Y, red);

            draw(A--B--C--cycle);
            //draw(incircle(A,B,C));
            draw(A--Q^^X--Y);
            draw(w1^^w2);

            draw(X--Q--Y, blue+dashed);
            //draw(P--Q--I--cycle^^F--I--E, blue);
            
            //N = circumcenter(B, X, C);
            //dot(N, red);
    
    
            label("$A$", A, NE);
            label("$B$", B, SW);
            label("$C$", C, SE);
            label("$Q$", Q, SW);
            //label("$D$", D, S);
            //label("$E$", E, NE);
            //label("$F$", F, NW);
            //label("$P$", P, SE);
            label("$X$", X, NW);
            label("$Y$", Y, NE);
            //label("$Z$", Z, S);

            markangle("$\theta_1$", radius=8, C,Q,A);
            markangle("$\theta_1$",radius=8, B,X,Q);
            markangle("$180-\theta_1$",radius=8, Q,Y,C);
            //markangle("$\frac{\alpha}{2}$", radius=12, X, A, Q);
            //markangle("$\frac{\alpha}{2}$", radius=12, Q, A, C);
            markangle("$\gamma$", radius=25, Y, Q, A, red);
            markangle("$\beta$", radius=15, A, Q, X, red);
            markangle("$\gamma$", radius=15, Y, X, A, orange);
        \end{asy}
        \caption{Claim \ref{hmmt_feb07_guts_36_claim} - $BXYC$ is cyclic (and so is $AXQY$)}
        \label{hmmt_feb07_guts_36_fig_2}
    \end{figure}
\end{proof}
\clearpage
\section{Instructor Insights}
\begin{enumerate}
    \item Advanced Geometry is lot about recognizing configurations.
\end{enumerate}
\end{document}