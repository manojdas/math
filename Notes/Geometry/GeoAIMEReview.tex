\documentclass[11pt,twoside]{scrartcl}
\usepackage{mdas}
\usepackage{adjustbox}
\DeclareCaptionType{myfig}[Figure][List of myfig]
\newenvironment{myfigenv}{}{}

\setlist{nosep}

\begin{document}
% \listofmyfigs

\title{AIME Geometry Review}

\author{Manoj Das}
\org{Manoj Math Notes}
\date{\today}

\maketitle
\section{Theorems and Lemmas}
\subsection{Fact 5}
\begin{lemma}
	Let $ABC$ be a triangle with incenter $I$, $A$-excenter $I_A$, and denote by $L$ the midpoint of arc $BC$.
	Show that $L$ is the center of a circle through $I$, $I_A$, $B$, $C$.
\end{lemma}
\begin{myfigenv}
\begin{center}
	\begin{asy}
        import olympiad;
        import cse5;
        unitsize(2cm);
		pair A = dir(110);
		pair B = dir(210);
		pair C = dir(330);
		pair I = incenter(A, B, C);
		draw(A--B--C--cycle);
		draw(unitcircle);
		pair L = dir(270);
		pair I_A = 2*L-I;
		draw(CP(L, I), blue);

		draw(A--I_A, dotted);

		dot("$A$", A, dir(A));
		dot("$B$", B, dir(190));
		dot("$C$", C, dir(-10));
		dot("$I$", I, dir(60));
		dot("$L$", L, dir(225));
		dot("$I_A$", I_A, dir(I_A));

		/* Source generated by TSQ */
	\end{asy}
\end{center}
    \captionof{myfig}{Incenter Excenter Lemma, AKA Fact 5}
    \label{fact5}            

\end{myfigenv}

\subsection{Nine Point Circle}
\begin{lemma}[Reflecting the Orthocenter]
    Let $H$ be the orthocenter of $\triangle ABC$, and $X$ be the reflection of $H$ over $\overline{BC}$, and $Y$ be the reflection of $H$ over the midpoint of $\overline{BC}$. Then
    \begin{enumerate}
        \item $X$ is on $(ABC)$ (that is, the circumcircle of $\triangle ABC$).
        \item $AY$ is the diameter of $(ABC)$.
    \end{enumerate}
\end{lemma}
    See Figure \ref{9-point-circle}, Nine Point Circle.

\begin{lemma}[Nine Point Circle]
    Let $A_p, B_p, C_p$ be the foot of altitudes from $A, B, C$ to $\overline{BC}, \overline{AC}, \overline{AB}$, $A_m, B_m, C_n$ be the midpoints of $\overline{BC}, \overline{AC}, \overline{AB}$ and  $A_{\frac{H}{2}}, B_{\frac{H}{2}}, C_{\frac{H}{2}}$ be the midpoints of $\overline{AH}, \overline{BH}, \overline{CH}$ respectively. Then all these nine points lie on a circle. Further, the center of the circle $N_9$ is the midpoint of $\overline{OH}$, where $O$ is the circumcenter of $\triangle ABC$.  
    \end{lemma}
    \begin{myfigenv}
    \begin{center}
        \begin{asy}
            import cse5;
            import olympiad;
    
            unitsize(1cm);
    
            pair A, B, C, H, O, O1;
    
            B = (0,0);
            A = (2,5);
            C = (8,0);
    
            draw (A--B--C--cycle);
            path c1;
            c1 = circumcircle(A, B, C);
            O = circumcenter(A, B, C);
            draw (c1, yellow);
    
            pair Ap, Ap2, Bp, Bp2, Cp, Cp2;
            Ap = foot(A, B, C);
            Bp = foot (B, A, C);
            Cp = foot (C, A, B);
    
            H = extension(A, Ap, B, Bp);
    
            // Ap2 = 
    
            Ap2 = H + 2 (Ap - H);
            Bp2 = H + 2 (Bp - H);
            Cp2 = H + 2 (Cp - H);
            draw (A--Ap2, blue);
            draw (B--Bp2, blue);
            draw (C--Cp2, blue);
    
            dot(Ap, red);
            dot(Bp, red);
            dot(Cp, red);
            dot(Ap2, red);
            dot(Bp2, red);
            dot(Cp2, red);
    
            pair A1, A1p, B1, B1p, C1, C1p;
    
            A1 = (B + C) /2;
            B1 = (A + C) / 2;
            C1 = (A + B) / 2;
    
            A1p = H + 2 (A1 - H);
            B1p = H + 2 (B1 - H);
            C1p = H + 2 (C1 - H);
    
            draw (H--A1p, green); 
            dot(A1, darkgreen);
            dot(A1p, darkgreen);
    
            draw (H--B1p, green); 
            dot(B1, darkgreen);
            dot(B1p, darkgreen);
    
            draw (H--C1p, green); 
            dot(C1, darkgreen);
            dot(C1p, darkgreen);
    
            pair A3, B3, C3;
            A3 = A + (H - A)/2;
            B3 = B + (H - B)/2;
            C3 = C + (H - C)/2;
    
            dot(A3, purple);
            dot(B3, purple);
            dot(C3, purple);
    
            path c2 = circumcircle (Ap, Bp, Cp);
            O1 = circumcenter(Ap, Bp, Cp);
            draw (c2, orange);
    
            draw (H--O, cyan);
    
            dot(H, red);
            dot (O, purple);
            dot (O1, purple);
    
            label("$O$", O, S);
            label("$N_9$", O1, S);
    
            label("$H$", H, NW);
            label("$A_p$", Ap, SW);
            label("$A_p'$", Ap2, SW);
    
            label("$A_m$", A1, SW);
            label("$A_m'$", A1p, SW);
    
            label("$A_\frac{H}{2}$", A3, SW);
            
            label("$A$", A, N);
            label("$B$", B, SW);
            label("$C$", C, SE);
    
        \end{asy}
    \end{center}
    \captionof{myfig}{Nine Point Circle}
    \label{9-point-circle}            
    \end{myfigenv}

\section{Common Configurations}
\subsection{Orthic Triangle}
\begin{myfigenv}

\begin{center}
    \begin{asy}
        import cse5;
        import olympiad;

        unitsize(1cm);

        pair A, B, C, H, O, Am, J;

        B = (0,0);
        A = (2,5);
        C = (8,0);
        Am = (B + C)/2;

        draw (A--B--C--cycle);
        path c1;
        c1 = circumcircle(A, B, C);
        O = circumcenter(A, B, C);
        J = rotate(180,O)*C;

        draw (c1, red);

        pair Ap, Bp, Cp;
        Ap = foot(A, B, C);
        Bp = foot (B, A, C);
        Cp = foot (C, A, B);

        H = extension(A, Ap, B, Bp);

        draw (A--Ap^^B--Bp^^C--Cp, blue);

        draw(Ap--Bp--Cp--cycle);
        draw(B--O--C^^O--Am);
        draw(B--J--O, dashed);

        // dot(Ap, red);
        // dot(Bp, red);
        // dot(Cp, red);


        // draw (H--O, cyan);

        dot(H, red);
        dot (O, red);

        label("$O$", O, N);

        label("$H$", H, NW);
        label("$A_p$", Ap, SW);
        label("$B_p$", Bp, NE);
        label("$C_p$", Cp, NW);


        label("$A$", A, N);
        label("$B$", B, SW);
        label("$C$", C, SE);
        label("$A_m$", Am, S);
        label("$J$", J, NW);

        MarkAngle("A", B, A, C, 0.5);
        MarkAngle("A", B, J, C, 0.5);
        MarkAngle("A", Am, O, C, 0.5);

        MarkAngle("\alpha", Bp, B, A, 0.85);
        MarkAngle("\alpha", A, C, Cp, 0.85);

        MarkAngle("\alpha", Am, B, O, 0.85);
        MarkAngle("\alpha", O, C, Am, 0.85);

        MarkAngle("\alpha", A, Ap, Cp, 0.85);
        MarkAngle("\alpha", Bp, Ap, A, 0.85);

    \end{asy}
\end{center}
\captionof{myfig}{Orthic Triangle}
\label{orthic-triangle}            
\end{myfigenv}

In the above, $\angle \alpha = 90^\circ - \angle A$.

\paragraph{Cyclic Quadrilaterals} We have the following cyclic quadrilaterals:
\begin{itemize}
    \item $BC_pB_pC$, and counterparts
    \item $BA_pHC_p$, and counterparts
\end{itemize}

\begin{theorem}
    The orthocenter, $H$, of an \emph{acute angled} triangle is the incenter of its orthic triangle.
\end{theorem}
Proof is obvious from Figure \ref{orthic-triangle}.

\subsection{Medial Triangle and Euler Line}
\begin{myfigenv}

    \begin{center}
        \begin{asy}
            import cse5;
            import olympiad;
    
            unitsize(1cm);
    
            pair A, B, C, H, O, G, Am, Bm, Cm;
    
            B = (0,0);
            A = (2,5);
            C = (8,0);
            Am = (B + C)/2;
            Bm = (A + C)/2;
            Cm = (A + B)/2;
            G = (A + B + C)/3;
            O = circumcenter(A, B, C);

            draw (A--B--C--cycle);
    
            pair Ap, Bp, Cp;
            Ap = foot(A, B, C);
            Bp = foot (B, A, C);
            Cp = foot (C, A, B);
    
            H = extension(A, Ap, B, Bp);
    
            draw(Am--Bm--Cm--cycle);

            draw (A--Ap^^B--Bp, green);
            draw (A--Am^^B--Bm^^C--Cm, blue);

            draw (O--G--H, red);
    
        
            dot(H, red);
            dot (O, red);
            dot (G, red);

            label("$O$", O, N);
    
            label("$H$", H, NW);
            label("$G$", G, NW);

            label("$A_p$", Ap, SW);
            label("$B_p$", Bp, NE);    
    
            label("$A$", A, N);
            label("$B$", B, SW);
            label("$C$", C, SE);
            label("$A_m$", Am, S);
            label("$B_m$", Bm, NE);
            label("$C_m$", Cm, NW);
    
        \end{asy}
    \end{center}
    \captionof{myfig}{Medial Triangle}
    \label{medial-triangle}            
    \end{myfigenv}
    
    \begin{lemma}
        The medial triangle $A_mB_mC_m$ is a Homotethy $\mathcal{H}(G, -\frac{1}{2})$ of the triangle $ABC$.
    \end{lemma}

    \begin{proof}
        Each of the sides of the medial triangle is parallel to the sides of the corresponding sides of the original triangle and is $\frac{1}{2}$. Also, the centroid of the medial triangle is the same as the centroid of the original triangle.
    \end{proof}

    \begin{theorem}[Euler Line]
        The orthocenter, centroid, and circumcenter of any triangle is collinear and the centroid divides the distance from the orthocenter to the circumcenter in the ratio 2:1. That is $H, G, O$ are collinear and $HG = 2 GO$.
    \end{theorem}
\end{document}