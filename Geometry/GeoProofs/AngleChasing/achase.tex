\documentclass[11pt,twoside]{scrartcl}
\usepackage{mdas}

\begin{document}
\section{Centers of the Triangle}
\begin{claim}
    Let $I$ be the incenter, $H$ be the orthocenter, and $O$ be the circumcenter. Then we have the following:
    \begin{enumerate}
        \item $\angle BOC = 2 \angle A$.
        \item $\angle BIC = 90 + \frac{\angle A}{2}$.
        \item $\angle BHC = \angle B + \angle C$.
    \end{enumerate}
\end{claim}
\begin{center}
    \begin{asy}
        import olympiad;
        unitsize (2cm);
        pair O, A, B, C, I, H, Ap;
    
        B = (-2, 0);
        C = (2, 0);
        A = (-1, 3);

        O = circumcenter(A, B, C);
        H = orthocenter(A, B, C);
        I = incenter(A, B, C);
        Ap = foot(A, B, C);

        dot(O);
        dot(I);
        dot(H);

        draw(A--B--C--cycle);
        draw(A--Ap);
        draw(A--I^^B--I^^C--I);
        draw(B--foot(B, A, C), blue);
        draw(C--foot(C, A, B), blue);
        
        draw(rightanglemark(B, Ap, A));
        draw(rightanglemark(B, foot(B, A, C), C));
        draw(rightanglemark(B, foot(C, A, B), C));

        label("$O$", O, N);
        label("$I$", I, S);
        label("$H$", H, NW);
        label("$B$", B, W);
        label("$C$", C, E);
        label("$A$", A, dir(A-C));

    \end{asy}
    
\end{center}
\begin{proof}
    By inscribed angle theorem, $\angle BOC = 2 \angle A$.

    \[\angle{BIC} = 180 - \angle{IBC} - \angle{ICB} = 180 - \frac{\angle B}{2} - \frac{\angle C}{2} = 90 + \frac{\angle A}{2}.\]

    \begin{align*}
        \angle{BHC} &= 180 - \angle{HBC} - \angle{HCB}, \\
        &= 180 - (180 - 90 - \angle C) - (180 - 90 - \angle B), \\
        &= \angle C + \angle B.
    \end{align*}

\end{proof}

\subsection{Excenters}
\begin{claim}
    Let $I_A, I_B, I_C$ be the excenters of the triangle $ABC$, and $M$ be the second intersection of $AI_A$ with $(ABC)$. Then,
    \begin{enumerate}
        \item $I_BB \perp I_AI_C$.
        \item $MB = MI_A = MC$, that is $M$ is the circumcenter of $(BIC)$ and $I_A$ is on $(BIC)$.   
        
    \end{enumerate}

\end{claim}
\begin{center}
    \begin{asy}
        import geometry;
        import olympiad;
        unitsize (1cm);


        pair O, A, B, C, I, Ma, Mb, Mc, Ia, Ib, Ic, Bc, Ba;
    
        B = (-2, 0);
        C = (2, 0);
        A = (-1, 3);

        Ba = B + dir(B-A);
        Bc = B + dir(B-C);

        O = circumcenter(A, B, C);
        I = incenter(A, B, C);

        Ma = circumcenter(B, C, I);
        Mb = circumcenter(A, C, I);
        Mc = circumcenter(A, B, I);

        Ia = 2*Ma - I;
        Ib = 2*Mb - I;
        Ic = 2*Mc - I;

        dot(O);
        dot(I);
        dot(Ia);
        dot(Ib);
        dot(Ic);
        dot(Ma);

        draw(A--B--C--cycle);
        draw(circumcircle(A, B, C));
        draw(A--Ia^^B--Ib^^C--I);
        draw(Ia--Ic^^Ia--C);
        draw(B--Ba^^B--Bc);
        
        draw (circumcircle(B, I, C), blue+dashed);
        draw(C--Ma, blue+dashed);

        markangle("$\alpha$", radius=10, Ic,B,Bc, red);
        markangle("$\alpha$", radius=10, A,B,Ic, red);
        markangle("$\alpha$", radius=10, Ia,B,C, red);
        markangle("$\alpha$", radius=10, Ba,B,Ia, red);

        markangle("$\frac{B}{2}$", radius=15, C,B,I, blue);
        markangle("$\frac{B}{2}$", radius=15, I,B,A, blue);
        markangle("$B$", radius=15, C,Ma,I, blue);
        label("$O$", O, N);
        label("$I$", I, S);
        label("$I_A$", Ia, S);
        label("$I_B$", Ib, NW);
        label("$I_C$", Ic, NE);

        label("$M$", Ma, SE);

        label("$B$", B, W);
        label("$C$", C, E);
        label("$A$", A, dir(A-C));

    \end{asy}
    
\end{center}
\begin{proof}[Proof of Part 1]
    Considering the extension of $BC$, we have $2 \alpha + 2 \frac{\angle B}{2} = 180$, and therefore, 
    \[ \alpha + \frac{\angle B}{2} = 90^\circ.\]
    That is $$\angle I_CBI = \angle{IBI_A} = 90^\circ.$$
    This implies that $I_C, B, I_A$ are collinear. Also, $I_B \exists  \angle{B} \text{ bisector}$. Therefore, $I_BB\perp I_CI_A$.
\end{proof}

\begin{proof}[Proof of Part 2]
    By exterior angle theorem, 
    \[\angle{MIC} = \angle{IAC} + \angle{ICA} = \frac{\angle A}{2} + \frac{\angle C}{2}.\]
    
    Considering the angles subtended by the $\arc{AC}$, we have 
    \[\angle IMC = \angle AMC = \angle B.\]

    Therefore, 
    \begin{align*}
        \angle ICM = 180^\circ - \angle IMC - \angle{MIC}, \\
        &= 180^\circ - \angle B - (\frac{\angle A}{2} + \frac{\angle C}{2}), \\
        &= 180^\circ - (180^\circ - \angle A - \angle C) - (\frac{\angle A}{2} + \frac{\angle C}{2}), \\
        &= \frac{\angle A}{2} + \frac{\angle C}{2}, \\
        &= \angle MIC.
    \end{align*}
    Therefore, $MI = MC$. Similarly, we can show $MI = MB$.

    Considering the angle subtended by $\arc{MC}$, and since $MB = MC$, we 
    \[\angle BCM = \angle MBC = \angle MAC = \frac{\angle A}{2}.\]

   From part 1, we have $\triangle ICI_A$ is right angled at $C$, therefore, 
   \[\angle MI_AC = \angle II_AC = 90^\circ - \angle MIC = 90^\circ - \frac{\angle A}{2} + \frac{\angle C}{2} = \frac{\angle B}{2}.\]

   
    From the exterior angle of $\triangle CMB$ at $M$,
    \[ \angle CMI = \angle B = \angle MI_AC + \angle MCI_A = \frac{\angle B}{2} + \angle MCI_A.\]
    Therefore,
    \[ \angle MCI_A = \frac{\angle B}{2} = \angle MI_AC,\]
    and $MI_A = MC$.

\end{proof}
\section{Nine Point Circle}
\begin{center}
    \begin{asy}
        import olympiad;
        import cse5;

        unitsize(5cm);

        pathpen = black + linewidth(0.7);
        pointpen = black;
        pen s = fontsize(8);

        pair A = dir(110), B = dir(-10), C = dir(260);
        pair E = (A+C)/2, F = (A+B)/2, D = (C+B)/2;
        pair Y = foot(B,A,C), X = foot(A,C,B), Z = foot(C,A,B);
        pair H = orthocenter(A,B,C);
        pair K = midpoint(H--A), L = midpoint(H--B), M = midpoint(H--C);
        draw(MP("A",A,N,s)--MP("B",B,rotate(180)*W,s)--MP("C",C,SW,s)--cycle);
        draw(MP("E",E,W,s)--MP("F",F,NE,s)--MP("D",D,S,s)--cycle);
        draw(circumcircle(E,F,D),heavygreen);
        draw(B--MP("Y",Y,W,s)^^A--MP("X",X,SE,s)^^C--MP("Z",Z,NE,s),lightred);
        draw(rightanglemark(A,Y,B,1.5)^^rightanglemark(C,Z,B,1.5)^^rightanglemark(A,X,C,1.5));
        MP("K",K,rotate(20)*NE,s);MP("M",M,rotate(15)*S,s);MP("L",L,NE,s);MP("H",H,N,s);
    \end{asy}
\end{center}
\section{Examples}   
\begin{example}[IMO 2010, \#4]
    Let $P$ be a point interior to triangle $ABC$ (with $CA \neq CB$). The lines $AP$, $BP$ and $CP$ meet again its circumcircle $\Gamma$ at $K$, $L$, respectively $M$. The tangent line at $C$ to $\Gamma$ meets the line $AB$ at $S$. Show that from $SC = SP$ follows $MK = ML$.
\end{example}
\begin{center}
    \begin{asy}
    import olympiad;
    unitsize (1cm);
    pair O, A, B, C, P, K, L, M, S, Sb;

    B = (-2, 0);
    C = (2, 0);
    A = (-1, 4);

    path w = circumcircle (A, B, C);
    O = circumcenter (A, B, C);

    S = extension(A, B, C, rotate(90,C)*O);


    P = rotate(10, S)*C;

    K = intersectionpoints (w, P--P+dir(P-A)*10)[0];
    L = intersectionpoints (w, P--P+dir(P-B)*10)[0];
    M = intersectionpoints (w, P--P+dir(P-C)*10)[0];


    dot(K);
    dot(L);
    dot(M);

    dot (P);
    dot(S);

    // draw(B--(P-B)*2^^C--P+dir(P-C)*2.5^^A--(P-A)*2, red);

    // draw(P--P+dir(P-B)*4, red);
    draw(A--B--C--cycle);
    draw(w);
    draw(A--K^^L--B--S--C--M^^P--S);

    // , # ticks, where, gap, size
    add(pathticks(P--S, 1, 0.5, 0, 10));
    add(pathticks(C--S, 1, 0.55, 0, 10));
    // arrow((C+S)/2, dir(C-S), length=5, arrow = Arrow(TeXHead), green);
    label("$B$", B, W);
    label("$C$", C, E);
    label("$A$", A, dir(A-C));
    label("$P$", P, N);
    label("$K$", K, NW);
    label("$L$", L, dir(L-B));
    label("$M$", M, dir(M-C));
    label("$S$", S, dir (S-K));

    \end{asy}        
\end{center}

\begin{remark}
    To show $MK = ML$, we need to show $\arc{MK} = \arc{ML}$. There is one hard angle chase here, for which we use Power-of-Point. Since $S$, $P$ are the only 2 points not on the circle, they are candidates for Power-of-Point. Power-of-point for $S$ gives us the needed insight.
\end{remark}

\begin{soln}
    We will show $\arc{MK} = \arc{ML}$ using angle chase. 
    \begin{claim}
        $SP$ is tangent to $(ABP)$.
    \end{claim}
    \begin{proof}
        We have 
        \[\text{Pow}_\Gamma (S) = SA\cdot SB = SC^2,\]
        and we are given that $SC = SB$. Therefore,
        \[SA \cdot SB = SC^2 = SP^2.\]
        By inverse power of point, this implies that $SP$ is tangent to $(ABP)$. 
    \end{proof}
\end{soln}

\begin{center}
    \begin{asy}
    import olympiad;
    unitsize (1cm);
    pair O, A, B, C, P, K, L, M, S, Sb;

    B = (-2, 0);
    C = (2, 0);
    A = (-1, 4);

    path w = circumcircle (A, B, C);
    O = circumcenter (A, B, C);

    S = extension(A, B, C, rotate(90,C)*O);


    P = rotate(10, S)*C;

    K = intersectionpoints (w, P--P+dir(P-A)*10)[0];
    L = intersectionpoints (w, P--P+dir(P-B)*10)[0];
    M = intersectionpoints (w, P--P+dir(P-C)*10)[0];


    dot(K);
    dot(L);
    dot(M);

    dot (P);
    dot(S);

    // draw(B--(P-B)*2^^C--P+dir(P-C)*2.5^^A--(P-A)*2, red);

    // draw(P--P+dir(P-B)*4, red);
    draw(A--B--C--cycle);
    draw(w);
    draw(A--K^^L--B--S--C--M^^P--S);
    
    // , # ticks, where, gap, size
    add(pathticks(P--S, 1, 0.5, 0, 10));
    add(pathticks(C--S, 1, 0.55, 0, 10));
    // arrow((C+S)/2, dir(C-S), length=5, arrow = Arrow(TeXHead), green);
    label("$B$", B, W);
    label("$C$", C, E);
    label("$A$", A, dir(A-C));
    label("$P$", P, N);
    label("$K$", K, NW);
    label("$L$", L, dir(L-B));
    label("$M$", M, dir(M-C));
    label("$S$", S, dir (S-K));

    draw(P--P+dir(P-S)*2, green+dashed);
    draw(circumcircle(A, B, P), green + dashed);
    draw(anglemark(B, A, P, 20), red);
    draw(anglemark(B, P, S, 20), red);

    draw(anglemark(P, B, A, 22, 23), purple);
    draw(anglemark(S, P, K, 22, 23), purple);


    draw(anglemark(B, A, P, 20), red);
    draw(anglemark(B, P, S, 20), red);

    draw(anglemark(S, P, C, 15));
    draw(anglemark(P, C, S, 15));

    \end{asy}        
\end{center}
    Since $SP$ is tangent to $(ABP)$, we have
    \begin{align*}
        \angle{BPS} &= \angle{BAP}, \text{ and} \\
        \angle{ABP} &= \angle{SPK}.
    \end{align*}

    From the given conditions, we also have
    \[\angle{SPC} = \angle{PCS} = \frac{\arc{MC}}{2}.\]

    Considering the intersecting chords $AK$ and $MC$, we have $\angle{MPA} = \angle{CPK} = \frac{\arc{MA} + \arc{CK}}{2},$ that is 
    \[\arc{MA} = 2 \angle{CPK} - \arc{CK}.\]

    Combining the above,
    \begin{align*}
        \arc{ML} &= \arc{MA} + \arc{AL}, \\
        &= (2 \angle{CPK} - \arc{CK}) + 2\angle{ABL}, \\
        &= 2 (\angle{CPK} + \angle{ABL}) - \arc{CK}, \\
        &= 2 (\angle{CPK} + \angle{SPK}) - \arc{CK}, \\
        &= 2 \angle{SPC} - \arc{CK}, \\
        &= \arc{MC} - \arc{CK}, \\
        &= \arc{MK}.
    \end{align*}
\end{document}