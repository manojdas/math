%-- Message Achilleas ( moderator )
Now that we're in our second week, I want to take a few minutes before we get started today to ask you the following question:

%------------------
%-- Message Achilleas ( moderator )
What should you do if you feel like you're stuck on a homework problem?

%------------------
%-- Message bryanguo ( user )
% Ask people on the message board

%------------------
%-- Message dragondemon ( user )
% ask a question on the message board

%------------------
%-- Message MathJams ( user )
% ask for help on the message board!

%------------------
%-- Message mark888 ( user )
% Ask questions on the message board!

%------------------
%-- Message aaravdodhia ( user )
% Ask questions and/or view pre-existing discussions on that problem

%------------------
%-- Message dxs2016 ( user )
% ask on the message board

%------------------
%-- Message Achilleas ( moderator )
\textbf{Ask for help on the message board}: This is a great idea! This way, I, our TAs, other AoPS staff, and even your fellow students can try to help guide you to a good starting point for answering the problem. Our goal will be to help you without "giving too much away" about the problem: we want to get you started so that you can still do most of the work on the problem yourself.

%------------------
%-- Message MTHJJS ( user )
% First, message board! Second, parents. Third, consult a book

%------------------
%-- Message Achilleas ( moderator )
\textbf{Ask for help from parents, tutor, or classmate}: Yes, this is fine! Just make sure that you are only asking for hints and guidance, instead of the full solution. You should have them read the AoPS Honor Code too, to make sure they are following it while they help you. If you do get help this way on a writing problem, make sure to credit this help in your final submission by adding a comment like “I got help from my mom.”

%------------------
%-- Message MathJams ( user )
% go over the transcript

%------------------
%-- Message Yufanwang ( user )
% Read the transcript

%------------------
%-- Message Achilleas ( moderator )
\textbf{Read the class transcript} or \textbf{Read the textbook}: Another great idea! If you read (or re-read!) the class transcript and the week's section of the textbook, you might notice something that gives you a good idea as to how to start working on the problem.

%------------------
%-- Message Achilleas ( moderator )
\textbf{Set the problem aside and come back to it later}: This is a fantastic suggestion! Your brain is a complicated organ. Even when you're not actively thinking about a problem, your brain might be working on the problem in the background on its own, without you even being consciously aware of it. (Have you ever had the experience where you can't remember some fact, like somebody's name, but then some time later it just "hits you" and you suddenly remember?) If you set aside a problem and come back to it later, your brain might have subconsciously come up with an idea that you can use to solve the problem.

%------------------
%-- Message Achilleas ( moderator )
\textbf{Dive into your bag of problem-solving tactics}: If you're taking this course, you're not a newbie anymore! You've done a fair amount of problem-solving in your life and you've got some tactics at your disposal. Here are some general ideas:
- Try to solve a simpler problem.
- Work backwards: start at the end of the problem and see if you can get back to the beginning.
- If you're trying to prove something, maybe try for a proof by contradiction.
- Try to come up with examples of what you're trying to prove. Exploring an example may help with the more general proof.
- If you're able to make partial progress, but get stuck somewhere in the middle, ask yourself: "What information have I not used yet?"
Most importantly: try something! Anything! Just staring at a problem is usually not all that helpful: you have to actually \textbf{do} something.

%------------------
%-- Message Achilleas ( moderator )
A not-so-good idea is to search the internet for help. For one thing, the "help" you find might not be correct! (Not everything you read on the internet is true!) But more importantly, if you just look up and copy something that you found on the internet, that's an example of \textbf{plagiarism}. If you look up a solution to the exact problem you're trying to solve, and use that solution as a model for your own, that's an example of plagiarism.

%------------------
%-- Message Achilleas ( moderator )
Plagiarism means that you're presenting someone else's work as your own. It's a form of academic dishonesty, or in simpler language, cheating. It's not fair to the person whose work you're plagiarizing. It also means that you're misrepresenting to us that you know how to solve a problem when you really don't.

%------------------
%-- Message Achilleas ( moderator )
More importantly, the process of solving hard problems you haven't seen before is how you learn new things, and to me it's also what makes math fun! Math would be boring if I could just compute the right answer every time without having to solve problems. Computers can do that. What you and I can do, that computers don't do so well, is be creative and solve problems. But this is a brain muscle that requires exercise to make strong! Plagiarizing is like paying someone else to go to the gym and lift weights for you -- it doesn't do anything to strengthen your brain muscles.

%------------------
%-- Message Achilleas ( moderator )
So, to be clear, you should \textbf{not} try to look things up on the internet to help you solve a homework problem. We consider this plagiarism, and there are consequences for submitting plagiarized work for AoPS Online homework assignments.

%------------------
%-- Message Achilleas ( moderator )
Please read our Honor Code and our Academic Integrity policy for more details about what we consider to be plagiarism, and why.


\url{https://artofproblemsolving.com/school/handbook/current/honorcode}

\url{https://artofproblemsolving.com/school/handbook/current/academicintegrity}


Remember, you all agreed to abide by our Honor Code when you started this class.

%------------------
%-- Message Achilleas ( moderator )
If you have any questions about this policy, feel free to ask on the message board after class or reach out to student-services@aops.com.

%------------------
%-- Message Achilleas ( moderator )
We'll start with a couple warm-ups, and then we'll dive into some very challenging problems.

%------------------
%-- Message Achilleas ( moderator )
$\triangle ABC$ is acute; $BD$ and $CE$ are altitudes. Points $F$ and $G$ are the feet of perpendiculars $BF$ and $CG$ to line $DE$. Prove that $EF = DG$.

%------------------
%-- Message Achilleas ( moderator )
How do start?

%------------------
%-- Message MathJams ( user )
% draw a good diagram!

%------------------
%-- Message TomQiu2023 ( user )
% draw a diagram

%------------------
%-- Message trk08 ( user )
% draw picture

%------------------
%-- Message pritiks ( user )
% Draw a diagram

%------------------
%-- Message dragondemon ( user )
% draw a diagram

%------------------
%-- Message Celwelf ( user )
% Draw a diagram

%------------------
%-- Message Achilleas ( moderator )


\begin{center}
\begin{asy}
import cse5;
import olympiad;
unitsize(3cm);

import markers;
size(200); 
pathpen = black + linewidth(0.7);
pointpen = black; 
pen s = fontsize(8); 
pair O=origin, A=dir(-50), B=dir(100), C=dir(205), D=foot(B,A,C), E=foot(C,A,B), F = foot(B,D,E), G = foot(C,E,D);
draw(MP("A",A,SE,s)--MP("B",B,N,s)--MP("C",C,SW,s)--cycle);
draw(MP("F",F,ESE,s)--B--MP("D",D,SE,s));
draw(MP("E",E,rotate(30)*ESE,s)--C--MP("G",G,ESE,s));
draw((D+E)/2+scale(2.5)*(-(D+E)/2+D)--(D+E)/2+scale(2.5)*(-(D+E)/2+E));
markscalefactor = 0.008;
draw(rightanglemark(B,F,E));
draw(rightanglemark(C,E,A));
draw(rightanglemark(C,D,B));
draw(rightanglemark(C,G,D));

\end{asy}
\end{center}





%------------------
%-- Message Achilleas ( moderator )
What's plenty on this diagram?

%------------------
%-- Message Yufanwang ( user )
% right angles everywhere

%------------------
%-- Message SmartZX ( user )
% Right angles

%------------------
%-- Message dragondemon ( user )
% right angles

%------------------
%-- Message silver_maple ( user )
% right angles

%------------------
%-- Message christopherfu66 ( user )
% right angles

%------------------
%-- Message pritiks ( user )
% lots of right angles

%------------------
%-- Message AOPS81619 ( user )
% Right angles

%------------------
%-- Message mark888 ( user )
% right triangles

%------------------
%-- Message Vishaln2024 ( user )
% right angles

%------------------
%-- Message Trollyjones ( user )
% right angles

%------------------
%-- Message TomQiu2023 ( user )
% right angles

%------------------
%-- Message ca981 ( user )
% right angles

%------------------
%-- Message Achilleas ( moderator )
Right angles all over, so what do we do?

%------------------
%-- Message MTHJJS ( user )
% cyclic quadrilaterals

%------------------
%-- Message AOPS81619 ( user )
% Cyclic quads maybe

%------------------
%-- Message chardikala2 ( user )
% think about cyclic quadrilaterals

%------------------
%-- Message bryanguo ( user )
% cyclic quadrilaterals

%------------------
%-- Message Bimikel ( user )
% try to find cyclic quads

%------------------
%-- Message ww2511 ( user )
% check for cyclic quads

%------------------
%-- Message TomQiu2023 ( user )
% cyclic quadrilaterals

%------------------
%-- Message Vishaln2024 ( user )
% cyclic quad/similar triangles

%------------------
%-- Message MeepMurp5 ( user )
% consider similar triangles or cyclic  quads

%------------------
%-- Message Ezraft ( user )
% look for cyclic quadrilaterals

%------------------
%-- Message coolbluealan ( user )
% look for cyclic quadrilaterals

%------------------
%-- Message Wangminqi1 ( user )
% Look for cyclic quadrilaterals

%------------------
%-- Message Achilleas ( moderator )
We look for cyclic quadrilaterals.

%------------------
%-- Message Achilleas ( moderator )
Do you see any?

%------------------
%-- Message AOPS81619 ( user )
% $CDEB$ is cyclic I think

%------------------
%-- Message dxs2016 ( user )
% CDEB

%------------------
%-- Message renyongfu ( user )
% CDEB

%------------------
%-- Message coolbluealan ( user )
% BCDE

%------------------
%-- Message Vishaln2024 ( user )
% CBED

%------------------
%-- Message ca981 ( user )
% CDEB

%------------------
%-- Message bryanguo ( user )
% $BEDC$ is cyclic

%------------------
%-- Message Ezraft ( user )
% $BEDC$

%------------------
%-- Message ay0741 ( user )
% CDEB

%------------------
%-- Message SmartZX ( user )
% CBED

%------------------
%-- Message MeepMurp5 ( user )
% BEDC

%------------------
%-- Message Bimikel ( user )
% BCDE

%------------------
%-- Message MTHJJS ( user )
% CBED

%------------------
%-- Message MathJams ( user )
% CDEB

%------------------
%-- Message aaravdodhia ( user )
% BCDE

%------------------
%-- Message Wangminqi1 ( user )
% BCDE

%------------------
%-- Message Achilleas ( moderator )
$BCDE$ is cyclic since $\angle BDC =  \angle BEC$.

%------------------
%-- Message Achilleas ( moderator )
We draw our circle.

%------------------
%-- Message Achilleas ( moderator )



\begin{center}
\begin{asy}
import cse5;
import olympiad;
unitsize(4cm);

import markers;
size(200); 
pathpen = black + linewidth(0.7);
pointpen = black; 
pen s = fontsize(8); 
pair O=origin, A=dir(-50), B=dir(100), C=dir(205), D=foot(B,A,C), E=foot(C,A,B), F = foot(B,D,E), G = foot(C,E,D);
draw(MP("A",A,SE,s)--MP("B",B,N,s)--MP("C",C,SW,s)--cycle);
draw(MP("F",F,ESE,s)--B--MP("D",D,SE,s));
draw(MP("E",E,rotate(30)*ESE,s)--C--MP("G",G,ESE,s));
draw((D+E)/2+scale(2.5)*(-(D+E)/2+D)--(D+E)/2+scale(2.5)*(-(D+E)/2+E));
markscalefactor = 0.01;
draw(rightanglemark(B,F,E));
draw(rightanglemark(C,E,A));
draw(rightanglemark(C,D,B));
draw(rightanglemark(C,G,D));

/* Part 2 */
draw(circumcircle(B,C,E),heavygreen);

\end{asy}
\end{center}





%------------------
%-- Message Achilleas ( moderator )
Now what?

%------------------
%-- Message dxs2016 ( user )
% angle chasing?

%------------------
%-- Message Achilleas ( moderator )
Before we go diving into the other cyclic quadrilateral ($EAD$ and the intersection of $BD$ and $EC$), are there any useful angle equalities from our first cyclic quadrilateral?

%------------------
%-- Message Achilleas ( moderator )
Instead of listing all our angle equalities, we focus on what we want (one great thing about geometry proofs is you get to work in two directions). We want something involving $EF$ and $GD,$ so we focus on $\triangle BEF$ and $\triangle CDG,$ since they include these lengths. Each of these is a right triangle, so we know we only have to find one more angle equality to show that these triangles are similar to one of the many right triangles in the diagram. Where is it?

%------------------
%-- Message Achilleas ( moderator )
How about $\angle BCD?$

%------------------
%-- Message Achilleas ( moderator )
(please use "angle" for an angle and "triangle" for a triangle)

%------------------
%-- Message MathJams ( user )
% Equal to <BEF

%------------------
%-- Message bryanguo ( user )
% we know $\angle BCD = \angle BEF$ so triangles $BEF$ and $BCD$ are similar

%------------------
%-- Message Gamingfreddy ( user )
% angle BCD = angle BEF?

%------------------
%-- Message xyab ( user )
% $\angle BCD = \angle BEF$

%------------------
%-- Message SlurpBurp ( user )
% it is equal to $\angle BEF$

%------------------
%-- Message Wangminqi1 ( user )
% $\angle BCD=\angle BEF$

%------------------
%-- Message Achilleas ( moderator )
$\angle BCD = \angle BEF$ because both are supplementary to $\angle BED.$ Thus $\triangle BEF \sim \triangle BCD$.

%------------------
%-- Message Achilleas ( moderator )
How about $\angle CBE?$

%------------------
%-- Message MathJams ( user )
% equal to <CDG

%------------------
%-- Message dxs2016 ( user )
% angle CBE = angle CDG

%------------------
%-- Message Ezraft ( user )
% $\angle CBE = \angle CDG$

%------------------
%-- Message AOPS81619 ( user )
% $\angle CDG=\angle CBE$, so $\triangle CDG\sim\triangle CBE$

%------------------
%-- Message aaravdodhia ( user )
% <CBE = <ADE = <CDG

%------------------
%-- Message bryanguo ( user )
% $\angle CBE = \angle CDG$

%------------------
%-- Message mustwin_az ( user )
% $\angle CBE = \angle CDG$

%------------------
%-- Message Gamingfreddy ( user )
% angle CBE = angle CDG

%------------------
%-- Message bigmath ( user )
% angle CBE = angle CDG because they are both supplementary to CDE

%------------------
%-- Message Lucky0123 ( user )
% It's equal to $\angle CDG$

%------------------
%-- Message MeepMurp5 ( user )
% $\angle CBE = \angle CDG$

%------------------
%-- Message bigmath ( user )
% angle CBE = angle CDG because they are both supplementary to angle CDE

%------------------
%-- Message vsar0406 ( user )
% angle CBE is congruent to angle CDG

%------------------
%-- Message Achilleas ( moderator )
Similarly, $\angle CBE = \angle CDG$ and $\triangle CDG \sim \triangle CBE.$ Now what?

%------------------
%-- Message Achilleas ( moderator )
What do we usually do after we find a pair of similar triangles?

%------------------
%-- Message MathJams ( user )
% similarity ratios

%------------------
%-- Message coolbluealan ( user )
% write out the ratios

%------------------
%-- Message dxs2016 ( user )
% ratios

%------------------
%-- Message ay0741 ( user )
% ratios

%------------------
%-- Message xyab ( user )
% write a ratio

%------------------
%-- Message ww2511 ( user )
% ratios

%------------------
%-- Message bryanguo ( user )
% ratios!

%------------------
%-- Message apple.xy ( user )
% write ratios of side lengths

%------------------
%-- Message Riya_Tapas ( user )
% Find some ratios

%------------------
%-- Message MeepMurp5 ( user )
% write ratios

%------------------
%-- Message michaellikemath ( user )
% ratios

%------------------
%-- Message TomQiu2023 ( user )
% write out ratio of their side lengths

%------------------
%-- Message mustwin_az ( user )
% side length ratio

%------------------
%-- Message Gamingfreddy ( user )
% Use side ratios

%------------------
%-- Message pritiks ( user )
% write out the ratios

%------------------
%-- Message Yufanwang ( user )
% Side ratios?

%------------------
%-- Message leoouyang ( user )
% Make ratios

%------------------
%-- Message Achilleas ( moderator )
What ratios do we get from $\triangle BEF \sim \triangle BCD$?

%------------------
%-- Message silver_maple ( user )
% BE / BC = BF / BD = EF / CD

%------------------
%-- Message bryanguo ( user )
% EF/CD=BE/BC

%------------------
%-- Message MathJams ( user )
% BE/BC=BF/BD=EF/CD

%------------------
%-- Message dxs2016 ( user )
% BE/BC = EF/CD = BF/BD

%------------------
%-- Message ay0741 ( user )
% BC/CD = BE/EF

%------------------
%-- Message Ezraft ( user )
% $\frac{BE}{BC} = \frac{EF}{CD}$

%------------------
%-- Message aaravdodhia ( user )
% BE/EF = BC/CD

%------------------
%-- Message TomQiu2023 ( user )
% EF / CD = BE / BC

%------------------
%-- Message Riya_Tapas ( user )
% BE/BC = BF/BD = EF/CD

%------------------
%-- Message Trollyjones ( user )
% EF/CD=BF/BD=BE/BC

%------------------
%-- Message mustwin_az ( user )
% $BE/BC=EF/CD=BF/BD$

%------------------
%-- Message Gamingfreddy ( user )
% EF/CD = BE/BC

%------------------
%-- Message AOPS81619 ( user )
% $$\frac{BE}{BC}=\frac{BF}{BD}=\frac{EF}{CD}$$

%------------------
%-- Message TomQiu2023 ( user )
% EF / CD = BE / BC = BF / BD

%------------------
%-- Message ca981 ( user )
% EF/CD=BE/BC=BF/BD

%------------------
%-- Message Trollface60 ( user )
% $\frac{BC}{BE} = \frac{BF}{BD} = \frac{EF}{cD}$

%------------------
%-- Message apple.xy ( user )
% BD/BF = DC/EF = BC/BE

%------------------
%-- Message chardikala2 ( user )
% $\frac{BE}{BC} = \frac{EF}{CD}$

%------------------
%-- Message Achilleas ( moderator )
From $\triangle BEF \sim \triangle BCD$, we have $\dfrac{CD}{EF} = \dfrac{BC}{BE}$.

%------------------
%-- Message Achilleas ( moderator )
What ratios do we get from$\triangle CDG \sim \triangle CBE$?

%------------------
%-- Message TomQiu2023 ( user )
% DG / BE = CG / CE = CD / CB

%------------------
%-- Message dxs2016 ( user )
% CD/CB = DG/BE = CG/CE

%------------------
%-- Message Trollyjones ( user )
% DG/BE=CD/CB=CG/CE

%------------------
%-- Message Riya_Tapas ( user )
% CD/CB = CG/CE = DG/BE

%------------------
%-- Message AOPS81619 ( user )
% $$\frac{CD}{CB}=\frac{CG}{CE}=\frac{DG}{BE}$$

%------------------
%-- Message apple.xy ( user )
% CD/CB = DG/BE = CG/CE

%------------------
%-- Message silver_maple ( user )
% CD / CB = CG / CE = DG / BE

%------------------
%-- Message aaravdodhia ( user )
% DG/CD = BE/BC

%------------------
%-- Message Trollface60 ( user )
% $\frac{CD}{CB} = \frac{CG}{CE} = \frac{BE}{GD}$

%------------------
%-- Message Ezraft ( user )
% $\frac{CD}{CB} = \frac{DG}{BE}$

%------------------
%-- Message shausa ( user )
% $\frac{CD}{CB} = \frac{DG}{BE} = \frac{CG}{CE}$

%------------------
%-- Message Bimikel ( user )
% CD/CB=CG/CE=DG/BE

%------------------
%-- Message mark888 ( user )
% $\frac{CG}{CE}=\frac{DG}{EB}=\frac{CD}{CB}$

%------------------
%-- Message mkannan ( user )
% CD/CB=DG/BE=CG/CE

%------------------
%-- Message bryanguo ( user )
% GD/BD=CG/CE=CD/BC

%------------------
%-- Message MathJams ( user )
% CD/CB=CG/CE=DG/BE

%------------------
%-- Message Achilleas ( moderator )
From $\triangle CDG \sim \triangle CBE$, we have $\dfrac{BC}{BE} = \dfrac{CD}{DG}$.

%------------------
%-- Message Achilleas ( moderator )
Putting these together gives $$ \dfrac{CD}{EF }= \dfrac{BC}{BE} = \dfrac{CD}{DG}. $$  What does this give us?

%------------------
%-- Message bryanguo ( user )
% $EF=DG$

%------------------
%-- Message Trollyjones ( user )
% EF=DG

%------------------
%-- Message Lucky0123 ( user )
% $EF = DG$

%------------------
%-- Message leoouyang ( user )
% EF = DG

%------------------
%-- Message pritiks ( user )
% EF=DG

%------------------
%-- Message dxs2016 ( user )
% DG=EF

%------------------
%-- Message Yufanwang ( user )
% $EF=DG$ as desired

%------------------
%-- Message ca981 ( user )
% EF=DG , done!

%------------------
%-- Message michaellikemath ( user )
% EF=DG

%------------------
%-- Message mark888 ( user )
% EF=DG

%------------------
%-- Message RP3.1415 ( user )
% $EF=DG$

%------------------
%-- Message MathJams ( user )
% EF=DG

%------------------
%-- Message silver_maple ( user )
% EF = DG

%------------------
%-- Message christopherfu66 ( user )
% EF=DG

%------------------
%-- Message TomQiu2023 ( user )
% CD / EF = CD / DG, EF = DG

%------------------
%-- Message apple.xy ( user )
% CD/EF = CD/DG, so EF=DG

%------------------
%-- Message xyab ( user )
% EF = DG

%------------------
%-- Message Riya_Tapas ( user )
% EF = DG

%------------------
%-- Message Trollface60 ( user )
% $\frac{1}{EF} = \frac{1}{DG}, EF = DG$

%------------------
%-- Message coolbluealan ( user )
% EF=DG

%------------------
%-- Message Achilleas ( moderator )
So $EF = DG$.

%------------------
%-- Message sae123 ( user )
% Woohoo!

%------------------
%-- Message TomQiu2023 ( user )
% 

%------------------
%-- Message Achilleas ( moderator )
As a quick summary, here's our proof: $\angle BDC = \angle BEC$, so $BEDC$ is cyclic. $\angle BCD = 180^\circ - \angle BED = \angle BEF$ and $\angle BDC = \angle BFE$, so $\triangle BCD \sim \triangle BEF$, so $\dfrac{CD}{EF} = \dfrac{BC}{BE}$. Similarly, we have $\triangle CDG \sim \triangle CBE$, so $\dfrac{CD}{DG}  = \dfrac{BC}{BE} = \dfrac{CD}{EF}$, which means $EF = DG.$

%------------------
%-- Message Ezraft ( user )
% 

%------------------
%-- Message Yufanwang ( user )
% 

%------------------
%-- Message MTHJJS ( user )
% That is very short

%------------------
%-- Message bryanguo ( user )
% Nice warmup 

%------------------
%-- Message Achilleas ( moderator )
Note how short this proof is. Yet we easily could have gone all over the diagram doing fruitless things. By staying focused on what we wanted to prove, we were able to get straight to the solution. We wanted something about EF and DG, so we focused on those. They're conveniently only sides of one triangle each, and those triangles are right triangles, to boot. With all those right angles in the problem, we know we'll find something similar to them. Then, spotting the cyclic quadrilateral gives us a clear target for where to go hunting for the similar triangles.

%------------------
%-- Message Achilleas ( moderator )
Here is another quick solution: The midpoint of BC is the circumcenter of circle BCDE, so it projects to the midpoint of DE. On the other hand, the midpoint of BC projects to the midpoint of FG, since BFGC is a trapezoid. It follows that DE and GF have the same midpoint, so DG=EF.

%------------------
%-- Message Achilleas ( moderator )
Think about it after class, if you need to. 

%------------------
%-- Message Achilleas ( moderator )
Next problem:

%------------------
%-- Message Achilleas ( moderator )
Points $E$ and $F$ are on side $BC$ of convex quadrilateral $ABCD$ with $BE < BF.$ Given that $\angle BAE = \angle CDF$ and $\angle EAF = \angle FDE$, prove that $\angle FAC = \angle EDB$.

%------------------
%-- Message Achilleas ( moderator )
(This comes from Russia...with love)

%------------------
%-- Message xyab ( user )
% draw a diagram

%------------------
%-- Message bryanguo ( user )
% construct a diagram

%------------------
%-- Message Ezraft ( user )
% draw a diagram

%------------------
%-- Message Yufanwang ( user )
% Draw a diagram? 

%------------------
%-- Message Achilleas ( moderator )



\begin{center}
\begin{asy}
import cse5;
import olympiad;
unitsize(4cm);

import markers;
size(180); 
pathpen = black + linewidth(0.7);
pointpen = black; 
pen s = fontsize(8); 
real r = 1.8;
pair O=origin, A=dir(130), E=dir(-130), F=dir(-70), D=dir(40);
pair B = (F+E)/2+scale(r)*(-(F+E)/2+E);
pair C = intersectionpoints(E--(F+E)/2+scale(2)*(-(F+E)/2+F),circumcircle(A,B,D))[0];
draw(MP("A",A,NW,s)--MP("E",E,SW,s)--MP("F",F,SSE,s)--MP("D",D,NE,s)--cycle);
draw(A--MP("C",C,SE,s)--D--MP("B",B,SW,s)--cycle);
draw(A--F--C);
draw(D--E--B);

\end{asy}
\end{center}





%------------------
%-- Message Achilleas ( moderator )
What should we start looking for?

%------------------
%-- Message bryanguo ( user )
% cyclic quadrilaterals

%------------------
%-- Message Trollface60 ( user )
% cyclic quads

%------------------
%-- Message MathJams ( user )
% cyclic quads

%------------------
%-- Message MeepMurp5 ( user )
% cyclci quads

%------------------
%-- Message myltbc10 ( user )
% cyclic quadrilaterals

%------------------
%-- Message bigmath ( user )
% cyclic quads

%------------------
%-- Message Ezraft ( user )
% cyclic quadrilaterals

%------------------
%-- Message leoouyang ( user )
% Cyclic quadrilaterals

%------------------
%-- Message coolbluealan ( user )
% cyclic quadrilaterals

%------------------
%-- Message Bimikel ( user )
% cyclic quads

%------------------
%-- Message xyab ( user )
% cyclic quaddrilaterals

%------------------
%-- Message trk08 ( user )
% cyclic quadliraterals

%------------------
%-- Message SlurpBurp ( user )
% cyclic quadrilateral!!

%------------------
%-- Message Trollyjones ( user )
% cyclic quadraliterals

%------------------
%-- Message Achilleas ( moderator )
Equal angles, quadrilaterals. We look for cyclic quadrilaterals. Where do we find them?

%------------------
%-- Message Vishaln2024 ( user )
% AEFD

%------------------
%-- Message Lucky0123 ( user )
% ADFE is cyclic

%------------------
%-- Message Trollface60 ( user )
% ADFE

%------------------
%-- Message dxs2016 ( user )
% ADFE

%------------------
%-- Message bryanguo ( user )
% i see $AEFD$ is cyclic

%------------------
%-- Message christopherfu66 ( user )
% AEFD

%------------------
%-- Message ca981 ( user )
% AEFD cyclic

%------------------
%-- Message MathJams ( user )
% EADF is cyclic from whats given

%------------------
%-- Message michaellikemath ( user )
% AEFD

%------------------
%-- Message coolbluealan ( user )
% EADF

%------------------
%-- Message MeepMurp5 ( user )
% ADFE is cyclic

%------------------
%-- Message Lucky0123 ( user )
% $\angle EAF = \angle EDF,$ so $ADFE$ is cyclic

%------------------
%-- Message ay0741 ( user )
% AEFD

%------------------
%-- Message chardikala2 ( user )
% ADFE

%------------------
%-- Message RP3.1415 ( user )
% $EFDA$ is cyclic

%------------------
%-- Message bigmath ( user )
% EFDA

%------------------
%-- Message myltbc10 ( user )
% ADFE using angles

%------------------
%-- Message TomQiu2023 ( user )
% so ADFE is a cyclic quadrilateral

%------------------
%-- Message bryanguo ( user )
% We're given $\angle EAF = \angle FDE,$ so $AEFD$ since the angles lie on same arc $EF$

%------------------
%-- Message Achilleas ( moderator )
$\angle EAF = \angle FDE$ tells us that $AEFD$ is cyclic.

%------------------
%-- Message Achilleas ( moderator )
What could we show to imply that $\angle FAC = \angle EDB$ (what we're looking for)?

%------------------
%-- Message bryanguo ( user )
% $ABCD$ is cyclic!

%------------------
%-- Message mustwin_az ( user )
% that ADCB is cyclic

%------------------
%-- Message Ezraft ( user )
% $ABCD$ is cyclic

%------------------
%-- Message Wangminqi1 ( user )
% we could show that ABCD is cyclic

%------------------
%-- Message MeepMurp5 ( user )
% Show $ABCD$ is cyclic

%------------------
%-- Message MathJams ( user )
% ABCD is cyclic

%------------------
%-- Message Riya_Tapas ( user )
% Show that ABCD is cyclic

%------------------
%-- Message ca981 ( user )
% If ABCD cyclic

%------------------
%-- Message BigBrain23567 ( user )
% ADBC is a cyclic quadrelaterial?

%------------------
%-- Message Achilleas ( moderator )
We know that $\angle BAF = \angle CDE$ (adding the two angle equalities we are given), so if we show that $\angle BAC = \angle BDC$, we will be done, since then we can subtract $\angle BAF = \angle CDE$ and have the desired $\angle FAC = \angle EDB$.

%------------------
%-- Message Achilleas ( moderator )
What do we have to show to get $\angle BAC = \angle BDC?$

%------------------
%-- Message Achilleas ( moderator )
We want to show that $ABCD$ is cyclic. Is it?

%------------------
%-- Message trk08 ( user )
% yes

%------------------
%-- Message ca981 ( user )
% Yes

%------------------
%-- Message TomQiu2023 ( user )
% yes

%------------------
%-- Message Achilleas ( moderator )
Which are we more likely to use to show that $ABCD$ is cyclic: opposite angles being supplementary or equal inscribed angles?

%------------------
%-- Message TomQiu2023 ( user )
% opposite angles being supplementary

%------------------
%-- Message mathlogic ( user )
% opposite angles being supplementary

%------------------
%-- Message Achilleas ( moderator )
This is a bit of a subtle point. When we use cyclic quadrilaterals, it usually goes like this:

%------------------
%-- Message Achilleas ( moderator )
"Opposite angles sum to 180 -> cyclic quad -> inscribed angles equal"

%------------------
%-- Message Achilleas ( moderator )
or

%------------------
%-- Message Achilleas ( moderator )
"Inscribed angles equal -> cyclic quad -> Opposite angles sum to 180"

%------------------
%-- Message Achilleas ( moderator )
(We did the latter on the first problem.)

%------------------
%-- Message Achilleas ( moderator )
The "inscribed angles" involve diagonals, and the "opposite angles" involve only the angles of the quadrilateral. If the cyclic quadrilateral is to be a key step in our proof, we will likely be going from "diagonal angles" to "opposite angles" or vice-versa in our reasoning (in an angle-chasing problem; power of a point also gives us ways to use cyclic quadrilaterals).

%------------------
%-- Message Achilleas ( moderator )
We want to use $ABCD$ to prove something about the potentially equal inscribed angles $\angle BAC$ and $\angle BDC$, so we'll try to prove $ABCD$ is cyclic by showing that the opposite angles sum to $180^\circ.$ That is, we want to prove $\angle ABC + \angle ADC = 180^\circ $.

%------------------
%-- Message Achilleas ( moderator )
What are we likely to have to use in our proof that $\angle ABC + \angle ADC = 180^\circ$?

%------------------
%-- Message Achilleas ( moderator )
(I do not see how similar triangles help here)

%------------------
%-- Message MeepMurp5 ( user )
% and the cyclic quad from earlier

%------------------
%-- Message RP3.1415 ( user )
% $EFDA$ being cyclic

%------------------
%-- Message sae123 ( user )
% AEFD is cyclic

%------------------
%-- Message xyab ( user )
% existing cyclic quad

%------------------
%-- Message Achilleas ( moderator )
We're probably going to use the fact that $AEFD$ is cyclic. How can we get to that quadrilateral from $\angle ABC + \angle ADC$?

%------------------
%-- Message bryanguo ( user )
% we can split $\angle ADC$ into two angles ?

%------------------
%-- Message Achilleas ( moderator )
Sure! How?

%------------------
%-- Message Achilleas ( moderator )
(use "angle" or $\angle$)

%------------------
%-- Message xyab ( user )
% $\angle ADF and \angle FDC$

%------------------
%-- Message TomQiu2023 ( user )
% angle ADF and angle FDC

%------------------
%-- Message bryanguo ( user )
% $\angle FDC + \angle ADF$

%------------------
%-- Message dxs2016 ( user )
% angle ADF and angle FDC

%------------------
%-- Message Ezraft ( user )
% $\angle FDA + \angle CDF$

%------------------
%-- Message christopherfu66 ( user )
% Angle ADF + Angle FDC = Angle ADC

%------------------
%-- Message Trollyjones ( user )
% angle ADF and angle FDC

%------------------
%-- Message xyab ( user )
% $\angle ADF$ and $\angle FDC$

%------------------
%-- Message MeepMurp5 ( user )
% $\angle ADC = \angle ADF + \angle FDC$

%------------------
%-- Message Gamingfreddy ( user )
% angle ADF + angle FDC

%------------------
%-- Message apple.xy ( user )
% <ADC = <ADF + <CDF?

%------------------
%-- Message mustwin_az ( user )
% $\angle ADF+\angle CDF$

%------------------
%-- Message vsar0406 ( user )
% angles ADF and FDC?

%------------------
%-- Message coolbluealan ( user )
% $\angle ADF$ and $\angle FDC$

%------------------
%-- Message Bimikel ( user )
% angle ADF and angle FDC

%------------------
%-- Message AOPS81619 ( user )
% $\angle ADF+\angle FDC$

%------------------
%-- Message RP3.1415 ( user )
% $\angle ADC = \angle FDC + \angle FDA$

%------------------
%-- Message Achilleas ( moderator )
We write $\angle ADC$ as $\angle ADF + \angle FDC$, so we have

%------------------
%-- Message Achilleas ( moderator )
$$ \angle ABC + \angle ADC = \angle ABC + \angle ADF + \angle FDC. $$

%------------------
%-- Message Achilleas ( moderator )
What next? How are we going to get to $180?$

%------------------
%-- Message pritiks ( user )
% use the cyclic quadrilateral AEFD

%------------------
%-- Message Achilleas ( moderator )
We'll probably get to 180 through cyclic quadrilateral AEFD, so since we have $\angle ADF$, we want to get $\angle AEF$ in there. How can we do that?

%------------------
%-- Message coolbluealan ( user )
% $\angle ABC=\angle AEF-\angle BAE$

%------------------
%-- Message bryanguo ( user )
% $\angle ABC = \angle AEF - \angle BAE$

%------------------
%-- Message Achilleas ( moderator )
We have $\angle ABC = \angle AEF - \angle BAE$ (since $\angle AEF$ is an exterior angle of triangle $BAE$).

%------------------
%-- Message pritiks ( user )
% angle ADF = 180 - angle AEF

%------------------
%-- Message Bimikel ( user )
% angle ADF equals 180-angle AEF

%------------------
%-- Message Wangminqi1 ( user )
% $\angle ADF=180^\circ-\angle AEF$

%------------------
%-- Message xyab ( user )
% $\angle AEF + \angle ADF = 180$

%------------------
%-- Message Achilleas ( moderator )
Now we're home:

%------------------
%-- Message Achilleas ( moderator )
\begin{align*}  
\angle ABC + \angle ADC &= \angle ABC + \angle ADF + \angle FDC\\  
&=\angle AEF -\angle BAE + \angle ADF +\angle FDC\\  
&=\angle AEF + \angle ADF\\  
&=180^\circ  
\end{align*}

%------------------
%-- Message Achilleas ( moderator )
Note that we used the fact that $\angle FDC = \angle BAE$ as a step in those equalities.

%------------------
%-- Message Achilleas ( moderator )
Hence, $ABCD$ is cyclic, so $\angle BAC = \angle BDC.$ Subtracting $\angle BAF = \angle EDC$ from this gives $\angle FAC = \angle EDB$ and we are done.

%------------------
%-- Message chardikala2 ( user )
% cool 

%------------------
%-- Message RP3.1415 ( user )
% Nice!

%------------------
%-- Message Achilleas ( moderator )
Notice that once again we did a little working forwards and a little working backwards.

%------------------
%-- Message Achilleas ( moderator )
In the end, we write our proof forwards. Since $\angle EAF = \angle FDE$, $AEFD$ is a cyclic quadrilateral, so $\angle AEF + \angle ADF = 180$. Since $\angle AEF$ is an exterior angle of triangle $ABE$, we have $\angle AEF = \angle ABC + \angle BAE$. We are given that $\angle BAE = \angle CDF$, so $$ \angle AEF = \angle ABC + \angle CDF$$ and $$ \angle ABC = \angle AEF - \angle CDF. $$Therefore, we have $$ \angle ABC + \angle ADC = \angle AEF - \angle CDF + \angle ADC. $$

%------------------
%-- Message Achilleas ( moderator )
Since $\angle ADC -  CDF = \angle ADF$, we have $\angle ABC + \angle ADC = \angle AEF + \angle ADF = 180$. So, $ABCD$ is cyclic, which gives us $\angle BAC = \angle BDC.$ Since $\angle BAE = \angle CDF$ and $\angle EAF = \angle FDE$, we have $\angle BAC - \angle BAE - \angle EAF = \angle BDC - \angle CDF - \angle FDE$, so $\angle FAC = \angle EDB$, as desired.

%------------------
%-- Message Achilleas ( moderator )
(one could label some equal angles on the diagram to follow the solution easier)

%------------------
%-- Message Achilleas ( moderator )
Notice that this proof is in almost the exact opposite order as our work towards finding the solution. This is not uncommon, and it's part of the reason reading solutions to hard geometry problems can feel so fruitless. You probably find yourself thinking "How would I ever have thought of this?" Often, you can answer this question by reading the solution in reverse order!

%------------------
%-- Message Achilleas ( moderator )
Next problem:

%------------------
%-- Message Achilleas ( moderator )
A convex quadrilateral $ABCD$ is given for which $\angle ABC + \angle BCD <  180^\circ.$ $AB$ and $CD$ extended meet at $E.$ Prove that $\angle ABC = \angle ADC$ if and only if $AC^2 = CD \cdot CE - AB \cdot AE.$

%------------------
%-- Message xyab ( user )
% draw a diagram

%------------------
%-- Message TomQiu2023 ( user )
% Draw a diagram

%------------------
%-- Message ca981 ( user )
% Draw diagram

%------------------
%-- Message Trollface60 ( user )
% diagram

%------------------
%-- Message Riya_Tapas ( user )
% Diagram!

%------------------
%-- Message BigBrain23567 ( user )
% Draw a diagram?

%------------------
%-- Message RP3.1415 ( user )
% We need to make a diagram.

%------------------
%-- Message Ezraft ( user )
% draw a diagram

%------------------
%-- Message Achilleas ( moderator )
We are far from solving an olympiad problem without a diagram. If you reach that point, I'm done teaching this class! 

%------------------
%-- Message MTHJJS ( user )
% 

%------------------
%-- Message TomQiu2023 ( user )
% lol

%------------------
%-- Message Achilleas ( moderator )
We start by sketching a diagram, careful to make $\angle ABC + \angle BCD < 180^\circ$.

%------------------
%-- Message Achilleas ( moderator )



\begin{center}
\begin{asy}
import cse5;
import olympiad;
unitsize(4cm);

size(180); 
pathpen = black + linewidth(0.7);
pointpen = black; 
pen s = fontsize(8); 
pair O = origin, B=dir(-150), C=dir(-25), E=dir(110);
draw(MP("B",B,SW,s)--MP("C",C,SE,s)--MP("E",E,N,s)--cycle);
draw(C--MP("A",point(B--E,.65),W,s)--MP("D",point(E--C,.3),rotate(-60)*E,s));

\end{asy}
\end{center}





%------------------
%-- Message Achilleas ( moderator )
Before we go anywhere, look closely at the diagram. Notice anything that might cause us problems?

%------------------
%-- Message SmartZX ( user )
% The conditions aren't met

%------------------
%-- Message Achilleas ( moderator )
Which condition isn't met?

%------------------
%-- Message Achilleas ( moderator )
(I find it hard checking the last equation by looking at the diagram  )

%------------------
%-- Message leoouyang ( user )
% Angle ABC doesn't look like it is equal to Angle ADC

%------------------
%-- Message mustwin_az ( user )
% angle ABC = angle ADC does not look right

%------------------
%-- Message Trollface60 ( user )
% $\angle ABC = \angle ADC$

%------------------
%-- Message SmartZX ( user )
% Angle ABC = Angle ADC

%------------------
%-- Message AOPS81619 ( user )
% We want a diagram where $\angle ABC=\angle ADC$

%------------------
%-- Message Bimikel ( user )
% $\angle ADC=\angle ABC$

%------------------
%-- Message chardikala2 ( user )
% angle ABC and angle ADC don't look equal according to this diagram

%------------------
%-- Message Gamingfreddy ( user )
% angle ABC = angle ADC

%------------------
%-- Message TomQiu2023 ( user )
% angle ABC = angle ADC

%------------------
%-- Message Riya_Tapas ( user )
% Visually, the specified angles don't look congruent?

%------------------
%-- Message ca981 ( user )
% ∠ABC=∠ADC , not met

%------------------
%-- Message Achilleas ( moderator )
We want to show $\angle ABC = \angle ADC$ if and only if something. Our diagram has these two angles not only very different in measure, but one is acute while the other is obtuse.

%------------------
%-- Message Achilleas ( moderator )
If you continue the problem with this diagram starting with the equation $AC^2 = CD \cdot CE - AB \cdot AE$, you'll end up proving that a length is negative. You then have to go back to the problem and diagram and algebra and try to find your mess up. Rather than take you through all that pain, we'll fix the diagram now.

%------------------
%-- Message Achilleas ( moderator )



\begin{center}
\begin{asy}
import cse5;
import olympiad;
unitsize(4cm);

size(180); pathpen = black + linewidth(0.7);
pointpen = black; 
pen s = fontsize(8); 
pair O = origin, B=dir(-140), C=dir(-40);
pair D = (0.1,0.6);
pair A = intersectionpoint(B--(B + rotate(75)*(4*C-4*B)),D--( D + rotate(-75)*(4*C-4*D)));
pair E = intersectionpoint(D -- (D + D-C), A--(A + 5*(A-B)));
pair F = intersectionpoints(circumcircle(A,D,E), circumcircle(B,C,E))[1];
draw(MP("B",B,SW,s)--MP("C",C,SE,s)--MP("E",E,N,s)--cycle);
draw(C--MP("A",A,SW,s)--MP("D",D,rotate(-60)*E,s));

\end{asy}
\end{center}





%------------------
%-- Message bryanguo ( user )
% ok this diagram is much better

%------------------
%-- Message Achilleas ( moderator )
You \textbf{will} make this mistake at least once in your life, but if you get in the habit of re-reading the question after drawing the diagram, you will save yourself a lot of pain. I know people who have spent 2 hours on the USAMO puzzled, convinced that a problem was asking them to prove something that wasn't true, just because they started with an incorrect diagram.

%------------------
%-- Message Achilleas ( moderator )
Where do we start?

%------------------
%-- Message Achilleas ( moderator )
(angles or lengths?)

%------------------
%-- Message mustwin_az ( user )
% lengths

%------------------
%-- Message bryanguo ( user )
% lengths

%------------------
%-- Message TomQiu2023 ( user )
% lengths

%------------------
%-- Message Ezraft ( user )
% lengths

%------------------
%-- Message mark888 ( user )
% lengths

%------------------
%-- Message SmartZX ( user )
% Lengths

%------------------
%-- Message Riya_Tapas ( user )
% Lengths

%------------------
%-- Message Wangminqi1 ( user )
% lengths

%------------------
%-- Message Achilleas ( moderator )
It's not so clear what the angle equality means for us, so let's start with the lengths equation and try to reason from there. What does that equation make us think about?

%------------------
%-- Message Achilleas ( moderator )
(waiting for more powerful answers)

%------------------
%-- Message Lucky0123 ( user )
% Power of a point

%------------------
%-- Message AOPS81619 ( user )
% Power of point

%------------------
%-- Message SlurpBurp ( user )
% power of a point

%------------------
%-- Message andromeda7 ( user )
% power of a point

%------------------
%-- Message xyab ( user )
% power of a point

%------------------
%-- Message aaravdodhia ( user )
% PoP

%------------------
%-- Message Ezraft ( user )
% power of a point

%------------------
%-- Message pike65_er ( user )
% power of a point

%------------------
%-- Message bigmath ( user )
% Power of a point

%------------------
%-- Message Wangminqi1 ( user )
% power of a point

%------------------
%-- Message christopherfu66 ( user )
% power of point

%------------------
%-- Message bryanguo ( user )
% power of a point

%------------------
%-- Message coolbluealan ( user )
% power of a point

%------------------
%-- Message Trollface60 ( user )
% power of a point

%------------------
%-- Message robertfeng ( user )
% power of a point

%------------------
%-- Message TomQiu2023 ( user )
% power of point

%------------------
%-- Message dxs2016 ( user )
% PoP or ptolemy?

%------------------
%-- Message SmartZX ( user )
% Power of a point

%------------------
%-- Message ay0741 ( user )
% power of a point?

%------------------
%-- Message MathJams ( user )
% power of a point

%------------------
%-- Message RP3.1415 ( user )
% Power of a Point

%------------------
%-- Message pritiks ( user )
% power of a point?

%------------------
%-- Message shausa ( user )
% Power of point

%------------------
%-- Message Achilleas ( moderator )
That side equation looks like power of a point or even Ptolemy. But then we look at our diagram, and what do we realize is the problem with using those right away?

%------------------
%-- Message Trollface60 ( user )
% no circles

%------------------
%-- Message TomQiu2023 ( user )
% we don't have a circle lol

%------------------
%-- Message aidan0626 ( user )
% there is no circle

%------------------
%-- Message apple.xy ( user )
% we don't have the circle for power of a point

%------------------
%-- Message xyab ( user )
% there is no circle

%------------------
%-- Message shausa ( user )
% There is no circles

%------------------
%-- Message AOPS81619 ( user )
% There's no circle

%------------------
%-- Message BigBrain23567 ( user )
% no circles

%------------------
%-- Message mustwin_az ( user )
% no circles

%------------------
%-- Message Achilleas ( moderator )
No circles. We don't have any circles in the problem. So, what do we do?

%------------------
%-- Message xyab ( user )
% find a cyclic quad

%------------------
%-- Message mustwin_az ( user )
% look for cyclic quads

%------------------
%-- Message ay0741 ( user )
% find one?

%------------------
%-- Message TomQiu2023 ( user )
% find cyclic quadrilaterals

%------------------
%-- Message aaravdodhia ( user )
% hunt for cyclic quad to build a circle

%------------------
%-- Message robertfeng ( user )
% find cyclic quadrilaterals

%------------------
%-- Message MathJams ( user )
% find them!

%------------------
%-- Message Trollyjones ( user )
% try to find one

%------------------
%-- Message pike65_er ( user )
% look for cyclic quads?

%------------------
%-- Message trk08 ( user )
% make one

%------------------
%-- Message Achilleas ( moderator )
We go circle hunting. Are there any obvious cyclic quadrilaterals?

%------------------
%-- Message Achilleas ( moderator )
Our only real option is $ABCD$, which by definition is not cyclic. But we really want a circle. What circle does the equation $AC^2 = CD \cdot CE - AB \cdot AE$ suggest we draw in?

%------------------
%-- Message coolbluealan ( user )
% circle through AED

%------------------
%-- Message bryanguo ( user )
% $\triangle EBC$

%------------------
%-- Message Trollyjones ( user )
% one around triangle EBC

%------------------
%-- Message Achilleas ( moderator )
This equation asks us to draw the circumcircle of $ADE$ and the circumcircle of $EBC$, since the powers of points $C$ with respect to the first ($CD \cdot CE$) and the power of $A$ with respect to the second ($AB \cdot AE$) are both part of this equation:

%------------------
%-- Message Achilleas ( moderator )



\begin{center}
\begin{asy}
import cse5;
import olympiad;
unitsize(4cm);

size(180); pathpen = black + linewidth(0.7);
pointpen = black; 
pen s = fontsize(8); 
pair O = origin, B=dir(-140), C=dir(-40);
pair D = (0.1,0.6);
pair A = intersectionpoint(B--(B + rotate(75)*(4*C-4*B)),D--( D + rotate(-75)*(4*C-4*D)));
pair E = intersectionpoint(D -- (D + D-C), A--(A + 5*(A-B)));
pair F = intersectionpoints(circumcircle(A,D,E), circumcircle(B,C,E))[1];
draw(MP("B",B,SW,s)--MP("C",C,SE,s)--MP("E",E,N,s)--cycle);
draw(C--MP("A",A,SW,s)--MP("D",D,rotate(-60)*E,s));
draw(circumcircle(A,D,E),mediumblue);
draw(circumcircle(B,E,C),red);

\end{asy}
\end{center}





%------------------
%-- Message Achilleas ( moderator )
What's the problem with plowing ahead with the power of point C with respect to the small circle or A with respect to the large one?

%------------------
%-- Message Lucky0123 ( user )
% We don't have anything to relate it to

%------------------
%-- Message xyab ( user )
% we're missing a point

%------------------
%-- Message Achilleas ( moderator )
We need some more points. As it stands, we have $CD \cdot CE = \text{what?}$

%------------------
%-- Message Lucky0123 ( user )
% Extend CA?

%------------------
%-- Message Achilleas ( moderator )
We can extend $CA$ to hit the smaller (blue) circle again at a point we'll call $F.$ What does this give us?

%------------------
%-- Message coolbluealan ( user )
% CD*CE=CA*CF

%------------------
%-- Message shausa ( user )
% $CD \cdot CE = CA \cdot CF$

%------------------
%-- Message bigmath ( user )
% CD * CE = AC * CF

%------------------
%-- Message chardikala2 ( user )
% $CD \cdot CE = CA \cdot CF$

%------------------
%-- Message Bimikel ( user )
% CA*CF=CD*CE

%------------------
%-- Message aidan0626 ( user )
% $CD\cdot CE=CA \cdot CF$

%------------------
%-- Message maxlangy ( user )
% Let CA intersect blue circle at F. Then $CD*CE = CA*CF$

%------------------
%-- Message mathlogic ( user )
% CD*CE=CF*CA

%------------------
%-- Message dxs2016 ( user )
% CD*CE = CA * CF

%------------------
%-- Message Lucky0123 ( user )
% $CA \cdot CF = CD \cdot CE $

%------------------
%-- Message mark888 ( user )
% $CA(CF)=CD(CE)$

%------------------
%-- Message christopherfu66 ( user )
% CD * CE = CA * CF

%------------------
%-- Message Ezraft ( user )
% $CD \cdot CE = CA \cdot CF$

%------------------
%-- Message ay0741 ( user )
% CA * CF = CD * CE

%------------------
%-- Message Gamingfreddy ( user )
% CD * CE = CA * CF

%------------------
%-- Message xyab ( user )
% $CA\cdot CF = CD\cdot CE$

%------------------
%-- Message MathJams ( user )
% CD*CE=AC*CF

%------------------
%-- Message Riya_Tapas ( user )
% CD * CE  = CA * CF

%------------------
%-- Message pike65_er ( user )
% CD*CE=CA*CF

%------------------
%-- Message ca981 ( user )
% CA x CF = CD x CE

%------------------
%-- Message trk08 ( user )
% CD*CE=CA*CF

%------------------
%-- Message bryanguo ( user )
% $CD \cdot CE = CA \cdot CF.$

%------------------
%-- Message Achilleas ( moderator )
This gives us $CD \cdot CE = CA \cdot CF.$ That makes us happy because it gets a $CA$ involved, which we need for the given equation. But what really gets us excited about this diagram?

%------------------
%-- Message aaravdodhia ( user )
% Maybe AC contains the intersection of the red and blue circles

%------------------
%-- Message Vishaln2024 ( user )
% is F the intersection of the two circles

%------------------
%-- Message Achilleas ( moderator )



\begin{center}
\begin{asy}
import cse5;
import olympiad;
unitsize(4cm);

size(180); pathpen = black + linewidth(0.7);
pointpen = black; 
pen s = fontsize(8); 
pair O = origin, B=dir(-140), C=dir(-40);
pair D = (0.1,0.6);
pair A = intersectionpoint(B--(B + rotate(75)*(4*C-4*B)),D--( D + rotate(-75)*(4*C-4*D)));
pair E = intersectionpoint(D -- (D + D-C), A--(A + 5*(A-B)));
pair F = intersectionpoints(circumcircle(A,D,E), circumcircle(B,C,E))[1];
draw(MP("B",B,SW,s)--MP("C",C,SE,s)--MP("E",E,N,s)--cycle);
draw(C--MP("F",F,SW,s)--MP("A",A,SW,s)--MP("D",D,rotate(-60)*E,s));
draw(circumcircle(A,D,E),mediumblue);
draw(circumcircle(B,E,C),red);

\end{asy}
\end{center}





%------------------
%-- Message Achilleas ( moderator )
It looks like $F$ is on the larger (red) circle, too. Interesting and surprising. Before spending a ton of time trying to prove that $F$ is on both circles, we should draw one or two more configurations. In each case, $F$ does indeed appear to be on both circles. (This is an example of the importance of drawing very precise diagrams -- it makes it much easier to discover such surprises.)

%------------------
%-- Message Achilleas ( moderator )
Now, back to our story. We have some options now. We can assume that $\angle ABC = \angle ADC$ and try to derive the equation regarding lengths, or start with the lengths and go after the angle equality. We'll do the latter right now (start with the lengths equation), since we used the lengths to inspire the circles in the first place. What's our first step? (Remember, we have to be careful not to assume $F$ is on the red circle until we prove that fact!!)

%------------------
%-- Message Achilleas ( moderator )
HInt: We found a new expression for $CD \cdot CE$. What do with it?

%------------------
%-- Message dxs2016 ( user )
% sub it in

%------------------
%-- Message bigmath ( user )
% plug it into the equation

%------------------
%-- Message xyab ( user )
% substitute it in the equation

%------------------
%-- Message vsar0406 ( user )
% Substitute it into the equation in the problem statement.

%------------------
%-- Message MathJams ( user )
% substitue it

%------------------
%-- Message MeepMurp5 ( user )
% Use the given equation

%------------------
%-- Message TomQiu2023 ( user )
% replace it

%------------------
%-- Message Achilleas ( moderator )
Plugging our new expression for $CD \cdot CE$ into the given equation, we have
$$ AC^2 = CA \cdot CF - AB \cdot AE . $$

%------------------
%-- Message Achilleas ( moderator )
So?

%------------------
%-- Message dxs2016 ( user )
% rearrange? get AC on one side?

%------------------
%-- Message Achilleas ( moderator )
Good idea! What do we get?

%------------------
%-- Message shausa ( user )
% $AF \cdot AC = AB \cdot AE$

%------------------
%-- Message Wangminqi1 ( user )
% $CA \cdot AF=AB \cdot AE$

%------------------
%-- Message bryanguo ( user )
% $AC \cdot AF = AB \cdot AE$

%------------------
%-- Message dxs2016 ( user )
% AB*AE = AC*AF

%------------------
%-- Message Gamingfreddy ( user )
% AC^2 = AC * (AC + AF) - AB * AE --> AC * AF - AB * AE = 0

%------------------
%-- Message MathJams ( user )
% CA*AF=BA*AE

%------------------
%-- Message bigmath ( user )
% AF*AC=AB*AE

%------------------
%-- Message Achilleas ( moderator )
We have $CA$ on both sides, so we can rearrange and collapse the equation a bit:
$$ AB \cdot AE = CA \cdot CF - AC^2 =  (CF - CA)\cdot AC = AF \cdot AC . $$

%------------------
%-- Message Achilleas ( moderator )
What does $AB \cdot AE = AF \cdot AC$ tell us?

%------------------
%-- Message bigmath ( user )
% so F is on the red circle

%------------------
%-- Message AOPS81619 ( user )
% That means that $F$ is on the red circle

%------------------
%-- Message xyab ( user )
% F is on the red circle

%------------------
%-- Message Gamingfreddy ( user )
% F is on circle EBC

%------------------
%-- Message ay0741 ( user )
% F is on the red circle by power of a point

%------------------
%-- Message TomQiu2023 ( user )
% EFBC is cyclic

%------------------
%-- Message coolbluealan ( user )
% FECB is cyclic

%------------------
%-- Message Bimikel ( user )
% F is on the same circle as B, C, and E

%------------------
%-- Message pritiks ( user )
% it is power of point on the red circle so F lies on the red circle

%------------------
%-- Message MeepMurp5 ( user )
% $BCEF$ is cyclic

%------------------
%-- Message Trollyjones ( user )
% F is on red circle because power of point

%------------------
%-- Message bryanguo ( user )
% $F$ is on red circle

%------------------
%-- Message ca981 ( user )
% BCEF cyclic

%------------------
%-- Message Yufanwang ( user )
% $F, B, E, C$ are on the red circle!


(Power of a point)

%------------------
%-- Message silver_maple ( user )
% power of a point holds, so F is on the red circle

%------------------
%-- Message sae123 ( user )
% F is on the red circle

%------------------
%-- Message Wangminqi1 ( user )
% $F$ is on the red circle

%------------------
%-- Message vsar0406 ( user )
% quadrilateral CBFE is cyclic

%------------------
%-- Message aaravdodhia ( user )
% Power of point A shows that points B, C, E, F are concyclic, so F is indeed on the red circle

%------------------
%-- Message SmartZX ( user )
% Point F lies on the red circle

%------------------
%-- Message Achilleas ( moderator )
From the converse of power of a point, this relationship tells us that $BFEC$ is cyclic.

%------------------
%-- Message Achilleas ( moderator )
How can we show that this implies that $\angle ABC = \angle ADC$?

%------------------
%-- Message Achilleas ( moderator )
Do you readily have an angle congruent to $\angle ABC$ from our work so far?

%------------------
%-- Message bryanguo ( user )
% $\angle ABC = \angle AFE$!

%------------------
%-- Message maxlangy ( user )
% $\angle AFE$

%------------------
%-- Message AOPS81619 ( user )
% $\angle EFA=\angle ABC$

%------------------
%-- Message aaravdodhia ( user )
% <CFE = <ABC

%------------------
%-- Message bryanguo ( user )
% note that $\angle ABC = \angle AFE$ because they lie on arc $CE$

%------------------
%-- Message sae123 ( user )
% $\angle EFC$

%------------------
%-- Message TomQiu2023 ( user )
% angle EFC

%------------------
%-- Message Trollyjones ( user )
% angle EFC

%------------------
%-- Message Trollface60 ( user )
% angle AFE

%------------------
%-- Message Ezraft ( user )
% $\angle CFE$ is congruent to $\angle ABC$

%------------------
%-- Message Bimikel ( user )
% angle EFA

%------------------
%-- Message mark888 ( user )
% angle CFE

%------------------
%-- Message mustwin_az ( user )
% $\angle AFE$

%------------------
%-- Message chardikala2 ( user )
% angle CFE

%------------------
%-- Message ay0741 ( user )
% angle CFE

%------------------
%-- Message SlurpBurp ( user )
% angle AFE

%------------------
%-- Message Achilleas ( moderator )
One angle-chase down!  How can we finish proving that $\angle ABC=\angle ADC$?

%------------------
%-- Message Achilleas ( moderator )
(justify your answer  )

%------------------
%-- Message Achilleas ( moderator )
(cyclic quads are easier to follow than similar triangles)

%------------------
%-- Message RP3.1415 ( user )
% $\angle ABC=\angle CFE=180^\circ - \angle EDA=\angle ADC$

%------------------
%-- Message SmartZX ( user )
% Angle AFE and Angle ADC are both supplementary of Angle EDA, so Angle ABC is congruent to angle ADC

%------------------
%-- Message sae123 ( user )
% $\angle ABC = \angle EFC.$ Then by arcs of the blue circle, $\angle EFC = 180^\circ - \angle EDA.$ Then $\angle EFC = 180^\circ - \angle EDA = \angle ADC$ and we are done.

%------------------
%-- Message dxs2016 ( user )
% we know from inscribed angles and cyclic quad BCEF that angle ABC = angle EFC. from quad ADEF being cyclic (blue circle), we get that angle EFC = anngle ADC from the opposite angle condition, thus angle ABC = angle ADC

%------------------
%-- Message AOPS81619 ( user )
% $EFAD$ is cyclic, so $\angle EFA+\angle EDA=180$, so $\angle ADC=180-\angle EDA=\angle EFA=\angle ABC$

%------------------
%-- Message silver_maple ( user )
% ADEF is obviously cyclic, so angle AFE is supplementary to angle ADE, so it's congruent to angle ADC

%------------------
%-- Message Ezraft ( user )
% $\angle AFE = \angle ADC$ as they are both supplementary to $\angle EDA$ ($AFED$ is cyclic); then $\angle ABC = \angle ADC$ as $\angle AFE = \angle ABC$

%------------------
%-- Message apple.xy ( user )
% since EFAD is cyclic, <EFA + <EDA = 180 degrees, and also <EDA and <ADC are supplementary, so <ADC = <EFA which = <ABC

%------------------
%-- Message raniamarrero1 ( user )
% angle AFE is supplementary to angle ADE since theyre in a cyclic quadrilateral and ADC is also supplementary to ADE so angle ADE = angle AFE = angle ABC

%------------------
%-- Message Achilleas ( moderator )
We have two circles, so we can just do some angle-chasing:
$$ \angle ABC = \angle EFC = 180^\circ - \angle EDA = \angle ADC. $$

%------------------
%-- Message Achilleas ( moderator )
So, are we finished with the problem?

%------------------
%-- Message vsar0406 ( user )
% no, we have to prove the other direction, I think

%------------------
%-- Message RP3.1415 ( user )
% No! There are two directions!!!

%------------------
%-- Message silver_maple ( user )
% we have to prove the converse too

%------------------
%-- Message aaravdodhia ( user )
% converse is pending to be proven

%------------------
%-- Message SlurpBurp ( user )
% we need to prove both directions

%------------------
%-- Message Achilleas ( moderator )
We're very close to finished. We have to prove that the $AC^2$ equation implies the angle equality and vice-versa. We have shown that the $AC^2$ equation shows the angle equality; how can we prove the converse?

%------------------
%-- Message sae123 ( user )
% trace backwards our steps

%------------------
%-- Message Achilleas ( moderator )
Are our steps reversible?

%------------------
%-- Message Ezraft ( user )
% yes

%------------------
%-- Message coolbluealan ( user )
% yes

%------------------
%-- Message ay0741 ( user )
% yes

%------------------
%-- Message Celwelf ( user )
% yes

%------------------
%-- Message vsar0406 ( user )
% yes

%------------------
%-- Message MathJams ( user )
% yes

%------------------
%-- Message bryanguo ( user )
% yes

%------------------
%-- Message pritiks ( user )
% yes

%------------------
%-- Message TomQiu2023 ( user )
% yes

%------------------
%-- Message mark888 ( user )
% Yup

%------------------
%-- Message Achilleas ( moderator )
We can note that all of our steps are "if and only if" in our original proof. Therefore, we can run them all backwards. If I had time on a test, I would probably write them all out in reverse (if only to make sure that I was correct that they could be run backwards!)

%------------------
%-- Message Achilleas ( moderator )
Here's what the whole proof looks like (I'd include a diagram, of course):

%------------------
%-- Message Achilleas ( moderator )
Let the circumcircles of $\triangle ADE$ and $\triangle BCE$ be $S_1$ and $S_2$, respectively. Let the other intersection of line $AC$ with $S_1$ be point $F.$ The power of point $C$ with respect to $S_1$ gives $CD \cdot CE = CA \cdot CF$, so we have
\begin{align*} CA^2 &= CD \cdot CE - AB \cdot AE\\ \iff CA^2 &= CA \cdot CF - AB \cdot AE \\ \iff CA \cdot CF -CA^2&= AB \cdot AE\\ \iff CA \cdot (CF-CA) &=AB \cdot AE\\ \iff CA \cdot FA &=AB \cdot AE . \end{align*}

%------------------
%-- Message Achilleas ( moderator )
We have $CA \cdot FA = AB \cdot AE$ if and only if $F$ is on $S_2$, we have $\angle EFA + \angle EDA =180^\circ$ if and only if $F$ is on $S_2$, so $\angle ADC = \angle EFA$ if and only if $F$ is on $S_2.$ From $S_1$, we have $\angle ABC = \angle EFA$, so $\angle ABC = \angle ADC$ if and only if $F$ is on $S_2$.

%------------------
%-- Message Achilleas ( moderator )
So, we have shown that both $CA^2 = (CD)(CE) - (AB)(AE)$ and $\angle ABC = \angle ADC$ if and only if $F$ is on $S_2$, so $CA^2 = (CD)(CE) - (AB)(AE)$ if and only if $\angle ABC = \angle ADC$.

%------------------
%-- Message Achilleas ( moderator )
We also have to deal with a few different configurations, but if you play with it a bit, you'll see that the proof holds for them all.

%------------------
%-- Message Achilleas ( moderator )
The next example is a hard one, so we'll take our time.

%------------------
%-- Message Achilleas ( moderator )
Point $K$ is outside circle $C$ and points $L$ and $N$ are on $C$ such that $KL$ and $KN$ are tangent to $C.$ Let $M$ be on ray $KN$ beyond $N$, and let $P$ be the second intersection of the circumcircle of $KLM$ and $C$. Let $Q$ be the foot of the perpendicular from $N$ to $ML$. Prove that $$ \angle MPQ = 2\angle KML. $$

%------------------
%-- Message xyab ( user )
% draw a diagram

%------------------
%-- Message Ezraft ( user )
% draw a diagram

%------------------
%-- Message TomQiu2023 ( user )
% Draw a diagram

%------------------
%-- Message Achilleas ( moderator )



\begin{center}
\begin{asy}
import cse5;
import olympiad;
unitsize(4cm);

size(180);
pathpen = black + linewidth(0.7);
pointpen = black;
pen s = fontsize(8);

size(180);
path p = Circle(origin,1);
pair C = origin;
pair D = point(p,40), E = point(p,-40),K=scale(1.2)*dir(0);
pair L = tangent(K,C,1,n=1);
pair N = tangent(K,C,1,n=2);
pair M = shift(K)*scale(2.8)*(N-K);
pair Q = foot(N,M,L);
path q = circumcircle(K,L,M);
pair P = intersectionpoints(q,p)[1];
draw(K--shift(L)*scale(1.2)*(L-K));
draw(K--shift(K)*scale(3.3)*(N-K));
draw(p, heavygreen);
draw(q, heavygreen);
dot(C);
draw(MP("C",C,NW,s)--MP("L",L,NE,s)--MP("K",K,E,s));
draw(L--MP("M",M,SE,s));
draw(C--MP("N",N,SE,s)--MP("Q",Q,scale(2)*WSW,s));
draw(K--M--MP("P",P,W,s)--Q--N);
draw(rightanglemark(N,Q,M,3));

\end{asy}
\end{center}





%------------------
%-- Message Achilleas ( moderator )
This is a very complicated diagram. Typically with very complicated problems, unless there are some very obvious useful observations we can make at the start, we want to do a little thinking about how we can prove what we want so we can focus our efforts better.

%------------------
%-- Message Achilleas ( moderator )
What's troubling about the angle equation that we want to show?

%------------------
%-- Message Lucky0123 ( user )
% There's a factor of 2 on the right hand

%------------------
%-- Message xyab ( user )
% its double the angle

%------------------
%-- Message coolbluealan ( user )
% it has a 2

%------------------
%-- Message TomQiu2023 ( user )
% We are showing one is twice the other, not that they are equal

%------------------
%-- Message J4wbr34k3r ( user )
% Double angles.

%------------------
%-- Message robertfeng ( user )
% there is a 2

%------------------
%-- Message Achilleas ( moderator )
We want to show $\angle MPQ = 2 \angle KML.$ Clearly we aren't going to be able to do this by just finding a cyclic quadrilateral and saying done, since that 2 in there doesn't work too well with just "this angle = that angle."

%------------------
%-- Message Achilleas ( moderator )
What are some strategies we might take to prove that $\angle MPQ =2 \angle KML$?

%------------------
%-- Message Achilleas ( moderator )
What might we see in the diagram that might be true and might help us?

%------------------
%-- Message nextgen_xing ( user )
% Draw an angle bisector?

%------------------
%-- Message Achilleas ( moderator )
Keep in mind that it is much easier to prove two angles are equal than to prove one angle is twice another - unless a problem is crawling with angle bisectors, we generally have many more tools to show one angle equals another.

%------------------
%-- Message Achilleas ( moderator )
Are there any angles in the diagram which might be equal to $2 \angle KML$ or $\angle MPQ/2$ which might help us with this problem?

%------------------
%-- Message Achilleas ( moderator )
Seeing $\angle MPQ/2$, we think about the bisector of angle $MPQ.$ What might we wonder about it?

%------------------
%-- Message pritiks ( user )
% where it intersects

%------------------
%-- Message SlurpBurp ( user )
% does it intersect MQ on circle C

%------------------
%-- Message Bimikel ( user )
% where does it intersect MQ and MK?

%------------------
%-- Message xyab ( user )
% does it meet the intersection of MQ with the smaller circle

%------------------
%-- Message MeepMurp5 ( user )
% It intersects the intersection of circle $PNL$ and line $ML$.

%------------------
%-- Message silver_maple ( user )
% does it intersect the circle at the same point that it intersects ML?

%------------------
%-- Message Trollyjones ( user )
% does it intersect MQ at the circle

%------------------
%-- Message Achilleas ( moderator )
We might wonder if it hits that intersection point of circle $C$ and $LM.$ This is definitely wishful thinking, but we have lots of circles we can use to chase angles around, so we draw in the segment from $P$ to the intersection of $LM$ and the smaller circle. We want to prove that this segment bisects $\angle MPQ.$

%------------------
%-- Message Achilleas ( moderator )



\begin{center}
\begin{asy}
import cse5;
import olympiad;
unitsize(4cm);

size(180);
pathpen = black + linewidth(0.7);
pointpen = black;
pen s = fontsize(8);
path p = Circle(origin,1);
pair C = origin;
pair D = point(p,40), E = point(p,-40),K=scale(1.2)*dir(0);
pair L = tangent(K,C,1,n=1);
pair N = tangent(K,C,1,n=2);
pair M = shift(K)*scale(2.8)*(N-K);
pair Q = foot(N,M,L);
path q = circumcircle(K,L,M);
pair P = intersectionpoints(q,p)[1];
pair S = intersectionpoints(L--M,p)[1];
draw(K--shift(L)*scale(1.2)*(L-K));
draw(K--shift(K)*scale(3.3)*(N-K));
draw(p, heavygreen);
draw(q, heavygreen);
dot(C);
draw(MP("C",C,NW,s)--MP("L",L,NE,s)--MP("K",K,E,s));
draw(L--MP("M",M,SE,s));
draw(C--MP("N",N,SE,s)--MP("Q",Q,scale(2)*WSW,s));
draw(K--M--MP("P",P,W,s)--Q--N);
draw(P--MP("S",S,WSW,s));
draw(rightanglemark(N,Q,M,3));

\end{asy}
\end{center}





%------------------
%-- Message Achilleas ( moderator )
(On the USAMO, you'll draw 1-2 more diagrams and test before going too far with this reasoning, but you'll see that it sure looks true. This is the point at which you should think 'ah-ha, I'm on to something!'  When you find something surprising like this in a complicated diagram, it almost always is an important step and merits a lot of attention.)

%------------------
%-- Message Achilleas ( moderator )
Now we have a little chasing we can do. We'd like to show that $\angle QPS = \angle MPS = \angle KML$. What should we probably add to this diagram if we want to do this?

%------------------
%-- Message coolbluealan ( user )
% intersection between MP and circle C

%------------------
%-- Message Achilleas ( moderator )
We should label the intersection of $PM$ and circle $C.$ What else?  How are we likely to get from our angles involving $S$ to $\angle KML$?

%------------------
%-- Message dxs2016 ( user )
% extend PS?

%------------------
%-- Message TomQiu2023 ( user )
% extend PS

%------------------
%-- Message Achilleas ( moderator )
We need to tie $\angle KML$ to the angles involving $S$, so we extend $PS$ to meet $KM$ at $R$ because we need to relate our angles involving $S$ to angles on the circle through $P$, $M$, and $K$.

%------------------
%-- Message Achilleas ( moderator )



\begin{center}
\begin{asy}
import cse5;
import olympiad;
unitsize(4cm);

size(180);
pathpen = black + linewidth(0.7);
pointpen = black;
pen s = fontsize(8);
path p = Circle(origin,1);
pair C = origin;
pair D = point(p,40), E = point(p,-40),K=scale(1.2)*dir(0);
pair L = tangent(K,C,1,n=1);
pair N = tangent(K,C,1,n=2);
pair M = shift(K)*scale(2.8)*(N-K);
pair Q = foot(N,M,L);
path q = circumcircle(K,L,M);
pair P = intersectionpoints(q,p)[1];
pair S = intersectionpoints(L--M,p)[1];
pair T = intersectionpoints(P--M,p)[0];
pair R = intersectionpoints(K--M,P--shift(P)*scale(2)*(S-P))[0];
draw(K--shift(L)*scale(1.2)*(L-K));
draw(K--shift(K)*scale(3.3)*(N-K));
draw(p, heavygreen);
draw(q, heavygreen);
dot(C);
draw(MP("C",C,NW,s)--MP("L",L,NE,s)--MP("K",K,E,s));
draw(L--MP("M",M,SE,s));
draw(C--MP("N",N,SE,s)--MP("Q",Q,scale(2)*WSW,s));
draw(K--M--MP("T",T,SW,s)--MP("P",P,W,s)--Q--N);
draw(P--MP("S",S,WSW,s)--MP("R",R,SE,s));
draw(rightanglemark(N,Q,M,3));

\end{asy}
\end{center}





%------------------
%-- Message Achilleas ( moderator )
Now what?  We'd like to show $\angle QPS = \angle MPS = \angle KML.$ We should be looking for ways to tie our right side $KML$ to our left (angles at $P$). Are there any angle equalities we can use to achieve this?

%------------------
%-- Message Achilleas ( moderator )
Hint: Cyclic quadrilaterals help.

%------------------
%-- Message apple.xy ( user )
% cyclic quadrilateral PLKM?

%------------------
%-- Message Achilleas ( moderator )
That's right! What do we get from this?

%------------------
%-- Message bryanguo ( user )
% $\angle KML = \angle KPL$

%------------------
%-- Message vsar0406 ( user )
% angles KML and KPL are equal, I think

%------------------
%-- Message Achilleas ( moderator )
We have $\angle KML = \angle KPL$ since they are inscribed in arc $KL$ of the circumcircle of $KLPM$.

%------------------
%-- Message Achilleas ( moderator )
Can we relate $\angle KPL$ to the angles at $P$ we want in any way?

%------------------
%-- Message Achilleas ( moderator )
We'd like to show either $\angle KPL = \angle MPS$ or $\angle KPL = \angle QPS$. Are there any angle equalities we can show to get either of these?

%------------------
%-- Message Achilleas ( moderator )
Is there an angle both $\angle KPL$ and $\angle MPS$ are part of?

%------------------
%-- Message apple.xy ( user )
% <LPM

%------------------
%-- Message xyab ( user )
% $\angle LPM$

%------------------
%-- Message dxs2016 ( user )
% angle LPM?

%------------------
%-- Message Trollyjones ( user )
% angle LPM

%------------------
%-- Message MathJams ( user )
% <LPM?

%------------------
%-- Message pritiks ( user )
% angle MPL

%------------------
%-- Message Lucky0123 ( user )
% $\angle MPL$

%------------------
%-- Message Riya_Tapas ( user )
% angle MPL

%------------------
%-- Message mark888 ( user )
% angle MPL

%------------------
%-- Message Bimikel ( user )
% angle MPL

%------------------
%-- Message Achilleas ( moderator )
Since $\angle KPL = \angle MPL - \angle MPK$ and $\angle MPS = \angle MPL - \angle SPL$, if we show that $\angle SPL = \angle MPK$, then we have $\angle KPL = \angle MPS$ (and thus $\angle KML = \angle KPL$). (This is a somewhat standard manipulation if you are trying to show two angles with the same vertex are equal.) How do we do it?

%------------------
%-- Message Achilleas ( moderator )
We should look for angles that are equal to either $\angle MPK$ or $\angle LPS$ that we might be able to use to tie the two together. Are there any?

%------------------
%-- Message dxs2016 ( user )
% angle MPK = angle MLK

%------------------
%-- Message apple.xy ( user )
% <MPK = <MLK?

%------------------
%-- Message ca981 ( user )
% ∠MLK = ∠MPK

%------------------
%-- Message Bimikel ( user )
% angle MPK equals angle MLK

%------------------
%-- Message Riya_Tapas ( user )
% angle MLK = angle MPK

%------------------
%-- Message Trollface60 ( user )
% angle MLK = angle MPK

%------------------
%-- Message Achilleas ( moderator )
$\angle MPK = \angle KLM$ from cyclic quadrilateral $MLKP$. Can we show $\angle KLM = \angle LPS$?

%------------------
%-- Message coolbluealan ( user )
% Yes because KL is tangent

%------------------
%-- Message MeepMurp5 ( user )
% yes, by the tangent chord  theorem

%------------------
%-- Message SlurpBurp ( user )
% they share the same arc on circle C

%------------------
%-- Message Gamingfreddy ( user )
% By theorem, since KL is tangent to circle C, angle KLM = angle LPS

%------------------
%-- Message ca981 ( user )
% Yes, KL is tangent

%------------------
%-- Message sae123 ( user )
% yes. Because $KL$ is tangent to $C$, we have $\angle MLK = \angle SLK = \angle SLP,$ by arcs on $C$.

%------------------
%-- Message Achilleas ( moderator )
$\angle KLM = \angle LPS$ since both are inscribed in arc $LS$ of circle $C$ (when you have tangents, remember that an angle formed by a chord and a tangent is an inscribed angle!).

%------------------
%-- Message Achilleas ( moderator )
Hence, $\angle MPK = \angle KLM = \angle LPS$, so
$$ \angle KPL = \angle MPL - \angle MPK = \angle MPL - \angle SPL = \angle MPS, $$ and since we have $\angle KML = \angle KPL$ already, we have $\angle KML = \angle MPS.$

%------------------
%-- Message Achilleas ( moderator )
We mark these angles equal (in general, you should mark surprising equal angles as you find them so you don't forget about them later in the problem when you need them. I unfortunately don't do that on all our slides as we go through the problem since it would take forever to make all the requisite diagrams).

%------------------
%-- Message Achilleas ( moderator )



\begin{center}
\begin{asy}
import cse5;
import olympiad;
unitsize(4cm);

import markers;
size(180);
pathpen = black + linewidth(0.7);
pointpen = black;
pen s = fontsize(8);
path p = Circle(origin,1);
pair C = origin;
pair D = point(p,40), E = point(p,-40),K=scale(1.2)*dir(0);
pair L = tangent(K,C,1,n=1);
pair N = tangent(K,C,1,n=2);
pair M = shift(K)*scale(2.8)*(N-K);
pair Q = foot(N,M,L);
path q = circumcircle(K,L,M);
pair P = intersectionpoints(q,p)[1];
pair S = intersectionpoints(L--M,p)[1];
pair T = intersectionpoints(P--M,p)[0];
pair R = intersectionpoints(K--M,P--shift(P)*scale(2)*(S-P))[0];
draw(K--shift(L)*scale(1.2)*(L-K));
draw(K--shift(K)*scale(3.3)*(N-K));
draw(p, heavygreen);
draw(q, heavygreen);
dot(C);
draw(MP("C",C,NW,s)--MP("L",L,NE,s)--MP("K",K,E,s));
draw(L--MP("M",M,SE,s));
draw(C--MP("N",N,SE,s)--MP("Q",Q,scale(2)*WSW,s));
draw(K--M--MP("T",T,SW,s)--MP("P",P,W,s)--Q--N);
draw(P--MP("S",S,WSW,s)--MP("R",R,SE,s));
draw(rightanglemark(N,Q,M,3));
markangle(n=1,radius=20,M,P,S);
markangle(n=1,radius=-15,S,M,R);

\end{asy}
\end{center}





%------------------
%-- Message sae123 ( user )
% now we need $\angle KML = \angle SPQ$?

%------------------
%-- Message Achilleas ( moderator )
That's right!

%------------------
%-- Message Achilleas ( moderator )
So, we've reduced our original problem to showing that $\angle KML = \angle QPS$ (or $\angle MPS = \angle QPS$).

%------------------
%-- Message Achilleas ( moderator )
What could we prove that would lead directly to either of these?

%------------------
%-- Message Achilleas ( moderator )
Hint: Cyclic quadrilaterals help.

%------------------
%-- Message MathJams ( user )
% show RQPM is cyclic

%------------------
%-- Message Wangminqi1 ( user )
% $PMRQ$ is cyclic

%------------------
%-- Message dxs2016 ( user )
% show QPMR cyclic?

%------------------
%-- Message bigmath ( user )
% PQRM is cyclic

%------------------
%-- Message sae123 ( user )
% prove $PQRM$ is cyclic.

%------------------
%-- Message Achilleas ( moderator )
If we could show that $PQRM$ is cyclic, we'd have $\angle QMR = \angle RPQ$ and we'd be done. So, it's time to start looking at those four points.

%------------------
%-- Message Achilleas ( moderator )
Do we see anything notable about these points that might help?

%------------------
%-- Message Achilleas ( moderator )
How is $S$ related to the quadrilateral that we just mentioned?

%------------------
%-- Message sae123 ( user )
% it is the intersection of diagonals

%------------------
%-- Message apple.xy ( user )
% intersection of the diagonals

%------------------
%-- Message maxlangy ( user )
% intersection of diagonals

%------------------
%-- Message mustwin_az ( user )
% its the intersection of its diagonals

%------------------
%-- Message xyab ( user )
% S is the intersection of the diagonals

%------------------
%-- Message Trollface60 ( user )
% S is the intersection of the diagonals of the quad

%------------------
%-- Message Riya_Tapas ( user )
% It's the intersection of the diagonals

%------------------
%-- Message mark888 ( user )
% S is the intersection of the diagonals

%------------------
%-- Message Achilleas ( moderator )
Point $S$ is the intersection of the diagonals of $PQRM$.

%------------------
%-- Message Achilleas ( moderator )
How about the equation $\angle KML = \angle MPS$ that we just found? What does it give us?

%------------------
%-- Message Lucky0123 ( user )
% $\triangle MRS$ is similar to $\triangle PRM$

%------------------
%-- Message Wangminqi1 ( user )
% $\triangle MSR \sim \triangle PMR$

%------------------
%-- Message Achilleas ( moderator )
It gives us $\triangle PRM \sim \triangle MRS.$ How does this help?

%------------------
%-- Message Achilleas ( moderator )
We have a bunch of angle equalities from our similarity, but it isn't clear how to take them to the end.

%------------------
%-- Message Achilleas ( moderator )
At this point playing with the problem you've probably made the mistake a dozen times of thinking that $P$, $Q$, and $N$ are collinear. When you find that you are making a mistake like this over and over again, it is almost always a good idea to redraw the diagram differently enough that you'll stop making that error:

%------------------
%-- Message Achilleas ( moderator )



\begin{center}
\begin{asy}
import cse5;
import olympiad;
unitsize(4cm);

import markers;
size(180);
pathpen = black + linewidth(0.7);
pointpen = black;
pen s = fontsize(8);
path p = Circle(origin,1);
pair C = origin;
pair D = point(p,40), E = point(p,-40),K=scale(1.2)*dir(0);
pair L = tangent(K,C,1,n=1);
pair N = tangent(K,C,1,n=2);
pair M = shift(K)*scale(2.95)*(N-K);
pair Q = foot(N,M,L);
path q = circumcircle(K,L,M);
pair P = intersectionpoints(q,p)[1];
pair S = intersectionpoints(L--M,p)[1];
pair T = intersectionpoints(P--M,p)[0];
pair R = intersectionpoints(K--M,P--shift(P)*scale(2)*(S-P))[0];
draw(K--shift(L)*scale(1.2)*(L-K));
draw(K--shift(K)*scale(3.3)*(N-K));
draw(p, heavygreen);
draw(q, heavygreen);
dot(C);
draw(MP("C",C,NW,s)--MP("L",L,NE,s)--MP("K",K,E,s));
draw(L--MP("M",M,SE,s));
draw(C--MP("N",N,SE,s)--MP("Q",Q,scale(2)*WSW,s));
draw(K--M--MP("T",T,SW,s)--MP("P",P,W,s)--Q--N);
draw(P--MP("S",S,WSW,s)--MP("R",R,SE,s));
draw(rightanglemark(N,Q,M,3));
markangle(n=1,radius=20,M,P,S);
markangle(n=1,radius=-15,S,M,R);

\end{asy}
\end{center}





%------------------
%-- Message Achilleas ( moderator )
If we're stuck going forward, we can always try running backwards again. Specifically, we look at $PQRM$ and think about ways to show it is cyclic. If we hit one that we might be able to use our triangle similarity on, we give that extra attention. Of all the pairs of potentially inscribed angles or potentially supplementary angles we could use, which stands out as a good candidate and why?

%------------------
%-- Message Achilleas ( moderator )
Hint: We already know that $\angle RMS=\angle RPM$.

%------------------
%-- Message dxs2016 ( user )
% mayble angle MPR and angle MQR?

%------------------
%-- Message Achilleas ( moderator )
$\angle RQM = \angle RPM$ is good candidate to aim for, because we might be able to get to it through $\angle RMS$, which we already know is equal to $\angle RPM$.

%------------------
%-- Message Achilleas ( moderator )
How might we show that $\angle RQM = \angle RMS$?

%------------------
%-- Message Achilleas ( moderator )
which one?

%------------------
%-- Message xyab ( user )
% show that RQ = RM

%------------------
%-- Message RP3.1415 ( user )
% Show $RQ=RM$

%------------------
%-- Message MTHJJS ( user )
% RQ = RM

%------------------
%-- Message sae123 ( user )
% $RQ = RM$?

%------------------
%-- Message Achilleas ( moderator )
If $\angle RQM = \angle RMS$, then $RQM$ is isosceles. Hence, if we show that $RQ = RM$, then we're set. That gets us thinking about lengths, so what should we take a look at?

%------------------
%-- Message dxs2016 ( user )
% PoP?

%------------------
%-- Message Achilleas ( moderator )
Of which point?

%------------------
%-- Message coolbluealan ( user )
% R

%------------------
%-- Message MTHJJS ( user )
% R?

%------------------
%-- Message mustwin_az ( user )
% R

%------------------
%-- Message apple.xy ( user )
% *R

%------------------
%-- Message dxs2016 ( user )
% R?

%------------------
%-- Message Achilleas ( moderator )
We should take a look at our similarity relationship again, and think about power of point $R$.

%------------------
%-- Message Achilleas ( moderator )
What does $\triangle PRM \sim \triangle MRS$ give us?

%------------------
%-- Message MathJams ( user )
% SR*PR=RM^2

%------------------
%-- Message Bimikel ( user )
% PR/MR=RM/RS=PM/MS

%------------------
%-- Message pritiks ( user )
% MR/PR = SR/RM

%------------------
%-- Message coolbluealan ( user )
% RS*RP=RM^2

%------------------
%-- Message dxs2016 ( user )
% RM/RS = PM/MS = PR/MR

%------------------
%-- Message Lucky0123 ( user )
% $\frac{SR}{RM} = \frac{RM}{PR}$

%------------------
%-- Message MTHJJS ( user )
% RM/PR = SR/RM

%------------------
%-- Message Trollface60 ( user )
% $RS \cdot RP = MR^2$

%------------------
%-- Message Ezraft ( user )
% $\frac{PR}{RM} = \frac{RM}{SR}$

%------------------
%-- Message Riya_Tapas ( user )
% RS/RM = MR/PR

%------------------
%-- Message mark888 ( user )
% $\frac{PR}{MR}=\frac{RM}{RS}=\frac{PM}{MS}$

%------------------
%-- Message sae123 ( user )
% $MR/PR = MS/PM = SR/RM$

%------------------
%-- Message Gamingfreddy ( user )
% SR/MR = MR/PR = MS/PM

%------------------
%-- Message dxs2016 ( user )
% RM^2 = PR*RS?

%------------------
%-- Message Riya_Tapas ( user )
% $MR^2 = PR\cdot{RS}$

%------------------
%-- Message aaravdodhia ( user )
% RM^2 = RS * RP

%------------------
%-- Message Achilleas ( moderator )
We obviously want to focus on $RM$, so we use that in both triangles and we have
$$ \dfrac{RM}{RP} = \dfrac{RS}{RM} \quad \text{or} \quad RM^2 = (RP)(RS). $$

%------------------
%-- Message Achilleas ( moderator )
That just plain looks like power of a point. We also have a circle already with $RP$ as a secant. Where does this lead us?

%------------------
%-- Message Lucky0123 ( user )
% But we also know that $RN^2 = SR * PR,$ so $RM = RN$

%------------------
%-- Message aaravdodhia ( user )
% so RM = RN

%------------------
%-- Message MathJams ( user )
% RM=RN

%------------------
%-- Message Achilleas ( moderator )
This takes us to $RN^2 = RP \cdot RS$ (power of point $R$ with respect to circle $C$), so $RN = RM.$ It's not $RM = RQ$, but it is close. (And we're happy because we've used the fact that $KM$ is tangent to circle $C$ at $N$.) How do we get to $RM = RQ$ from $RM = RN$?

%------------------
%-- Message Gamingfreddy ( user )
% Since NQM is a right triangle

%------------------
%-- Message sae123 ( user )
% then $R$ is the circumcenter of $QNM,$ making it equidistant from $Q$ and $M$.

%------------------
%-- Message pritiks ( user )
% use the right triangle QNM since the median = half the hypotenuse

%------------------
%-- Message coolbluealan ( user )
% R is the circumcenter since MQN is a right triangle

%------------------
%-- Message ay0741 ( user )
% Since MQN is a right triangle, QRN must be isoceles?

%------------------
%-- Message bryanguo ( user )
% R is the circumcenter of $\triangle MQN$

%------------------
%-- Message MeepMurp5 ( user )
% the length of the median is half of $MN$ because $\triangle MQN$ is right.

%------------------
%-- Message Gamingfreddy ( user )
% Since NQM is a right triangle and R is midpoint of NM, so R must be its circumcenter, so QR = RM

%------------------
%-- Message SmartZX ( user )
% R is the center of the circumcircle of MQN

%------------------
%-- Message xyab ( user )
% $\triangle MNQ$ circumcircle has center R because $\angle MQN$ is 90 degrees

%------------------
%-- Message Achilleas ( moderator )
Triangle $QMN$ is a right triangle. Point $R$ is the midpoint of the hypotenuse of $\triangle QMN$, so it is the center of the circumcircle of $\triangle QMN$. Hence, $RQ = RN = RM$.

%------------------
%-- Message Achilleas ( moderator )
Are we home?

%------------------
%-- Message MTHJJS ( user )
% yes we are 

%------------------
%-- Message maxben ( user )
% Yes

%------------------
%-- Message Yufanwang ( user )
% Yes!

%------------------
%-- Message coolbluealan ( user )
% yes

%------------------
%-- Message MathJams ( user )
% Yes!

%------------------
%-- Message pritiks ( user )
% yeah now we have to prove that angle MPQ = 2*angle KML

%------------------
%-- Message Achilleas ( moderator )
Now we're home: $\angle RQM = \angle RMQ = \angle RPM$, so $QRMP$ is cyclic. What angle equality does this give us?

%------------------
%-- Message aaravdodhia ( user )
% Yes! <RQM = <RPM and quad PQRM is cyclic, so <RPQ = <RMQ = <KML.

%------------------
%-- Message RP3.1415 ( user )
% $\angle NMQ=\angle RPQ$

%------------------
%-- Message Lucky0123 ( user )
% $\angle QMR = \angle QPR$

%------------------
%-- Message coolbluealan ( user )
% $\angle QMR=\angle QPS$

%------------------
%-- Message Ezraft ( user )
% $\angle RMS = \angle SPQ$

%------------------
%-- Message Gamingfreddy ( user )
% angle RMQ = angle RPQ

%------------------
%-- Message Trollface60 ( user )
% $\angle RMS = \angle RPQ$

%------------------
%-- Message Achilleas ( moderator )
So $\angle RPQ = \angle RMQ$. Hence, we have $$ \angle QPM = \angle QPR + \angle RPM = \angle KML + \angle KML = 2 \angle KML, $$as desired.

%------------------
%-- Message Achilleas ( moderator )
Quick question: What else can we infer from $RM^2 = RP \cdot RS$ that has nothing to do with any of the other points in the diagram?  Just focus on these 4.

%------------------
%-- Message Trollface60 ( user )
% RM is a tangent?

%------------------
%-- Message maxben ( user )
% RM is tangent to circle (PSM).

%------------------
%-- Message MTHJJS ( user )
% So MSP circumcircle is tangent to KM at M?

%------------------
%-- Message MTHJJS ( user )
% So triangle MSP's circumcircle is tangent to KM at M?

%------------------
%-- Message dxs2016 ( user )
% circumcircle of triangle SPM is tangent to KM at M?

%------------------
%-- Message Wangminqi1 ( user )
% $RM$ is tangent to the circumcircle of $\triangle PSM$

%------------------
%-- Message coolbluealan ( user )
% The circumcircle of PSM is tangent to RM

%------------------
%-- Message MeepMurp5 ( user )
% RM is tangent to $\odot (PSM)$

%------------------
%-- Message Celwelf ( user )
% the circumcircle of MSP has RM tangent to it

%------------------
%-- Message sae123 ( user )
% $RM$ is tangent to the circumcircle of $\triangle PSM$

%------------------
%-- Message AOPS81619 ( user )
% $RM$ is tangent to the circumcircle of $\triangle PSM$

%------------------
%-- Message Achilleas ( moderator )
By the converse of power of a point, we can use this to deduce that $RM$ is tangent to the circumcircle of $MSP$.

%------------------
%-- Message Achilleas ( moderator )
There are few points to take from this problem. First, note that we took almost no big leaps in this problem. Perhaps the first -- drawing $PR$ -- was the greatest, though this followed from our thinking about what might possibly lead us to a solution. The rest was a mix of going forwards and backwards, as you'll have to get proficient at to become an excellent geometer.

%------------------
%-- Message Achilleas ( moderator )
Second, the little step in the proof where we noted that an inscribed angle is equal to an angle between a tangent and a chord inscribed in the same arc is an important one to remember. It's a much less obvious equality than two inscribed angles and often overlooked.

%------------------
%-- Message Achilleas ( moderator )
Third, we have to avoid falling into the trap of running through a problem with angles and forgetting all about using lengths. While angle chasing is often easier and is definitely the place to start with problems involving angles, when we are stuck, we should look to our power of a point and similar triangles to use lengths to bail us out (though often we have to convince ourselves length will be useful first, as we did in noting that $RQ = RM$ would get us home if we could prove it).

%------------------
%-- Message Achilleas ( moderator )
Finally, similar triangles, power of a point, cyclic quadrilaterals. We used them all here.

%------------------
%-- Message Achilleas ( moderator )
Try this problem again on your own this weekend and see if you can work all the way through it.

%------------------
%-- Message Achilleas ( moderator )
Next week we will work on homothety, which you'll see can be a powerful tool to dramatically simplify problems that look as diabolical as the last problem we did tonight.

%------------------
%-- Message aaravdodhia ( user )
% Is there a slicker/shorter solution?

%------------------
%-- Message Achilleas ( moderator )
Perhaps one of you can find one slicker/shorter solution. 

%------------------
%-- Message Achilleas ( moderator )
If so, please post it on the message board.

%------------------
%-- Message Achilleas ( moderator )
Thank you all!

%------------------
%-- Message Achilleas ( moderator )
Have a wonderful weekend! I enjoy your posts on the message board either for homework problems or for olympiad geometry problems. Keep them coming!

%------------------
%-- Message Achilleas ( moderator )
Take care! 

%------------------
%-- Message TomQiu2023 ( user )
% Thanks!!

%------------------
%-- Message RP3.1415 ( user )
% Thanks for class!j

%------------------
%-- Message SmartZX ( user )
% Thank you!

%------------------
%-- Message aaravdodhia ( user )
% Thanks for class! 

%------------------
%-- Message bryanguo ( user )
% Thanks so much for class!! 

%------------------
%-- Message ww2511 ( user )
% thank you!

%------------------
%-- Message Gordonwong ( user )
% Thank you

%------------------
%-- Message MTHJJS ( user )
% You too Achilleas

%------------------
%-- Message dxs2016 ( user )
% thank you!

%------------------
%-- Message pritiks ( user )
% thanks!!!!! ))

%------------------
%-- Message mustwin_az ( user )
% Thank you

%------------------
%-- Message bigmath ( user )
% bye, thank you

%------------------
%-- Message Trollyjones ( user )
% thank you

%------------------
%-- Message sae123 ( user )
% thanks!

%------------------
%-- Message MathJams ( user )
% Thanks!

%------------------
%-- Message myltbc10 ( user )
% thank you!

%------------------
%-- Message Wangminqi1 ( user )
% Thank you!

%------------------
%-- Message Trollface60 ( user )
% Thank you!

%------------------
%-- Message MeepMurp5 ( user )
% Thank you!

%------------------
%-- Message Riya_Tapas ( user )
% Thank you for class!

%------------------
%-- Message Ezraft ( user )
% Thank you! 

%------------------
%-- Message aidan0626 ( user )
% Thank you!

%------------------
%-- Message Celwelf ( user )
% Thank youuu

%------------------
%-- Message Bimikel ( user )
% Thanks

%------------------
%-- Message Achilleas ( moderator )
No, I am still in Greece. It's 4:31am

%------------------
%-- Message MathJams ( user )
% oh woah

%------------------
%-- Message aaravdodhia ( user )
% You're great, sir. Thank you for doing this for us!

%------------------
%-- Message MathJams ( user )
% that's impressive

%------------------
%-- Message Yufanwang ( user )
% Thank you!

%------------------
%-- Message sae123 ( user )
% thanks for waking up to teach this class!

%------------------
%-- Message RP3.1415 ( user )
% Cool!

%------------------
%-- Message MathJams ( user )
% yeah it must be pretty hard, thanks!

%------------------
%-- Message Ezraft ( user )
% Thanks for waking up so early!

%------------------
%-- Message Ezraft ( user )
% This class is really fun

%------------------
%-- Message Lucky0123 ( user )
% Thank you!

%------------------
%-- Message Achilleas ( moderator )
By the way, last time I taught this class, people were sharing their solutions with those who have viewed the solutions only.

%------------------
%-- Message Ezraft ( user )
% I've had that in other classes, but how does that work with the writing problems?

%------------------
%-- Message Achilleas ( moderator )
(after the homework was due, of course)

%------------------
%-- Message Achilleas ( moderator )
(and again, only with those who have viewed the official solution)

%------------------
%-- Message Achilleas ( moderator )
I think it is a great way to learn from each other's solutions

%------------------
%-- Message Achilleas ( moderator )
I have a cool proof for the week 1 problem 1 which I will post, if no one has found the same.

%------------------
%-- Message Achilleas ( moderator )
(it is different from the official one)

%------------------
%-- Message Ezraft ( user )
% me too

%------------------
%-- Message vsar0406 ( user )
% oh cool

%------------------
%-- Message RP3.1415 ( user )
% cool

%------------------
%-- Message chardikala2 ( user )
% Thanks for the awesome class!

%------------------
%-- Message Achilleas ( moderator )
Okay..time for me to go to bed.

%------------------
%-- Message RP3.1415 ( user )
% lol good night

%------------------
%-- Message Ezraft ( user )
% good night/morning!

%------------------
%-- Message aaravdodhia ( user )
% Please sir. Good night!

%------------------
%-- Message vsar0406 ( user )
% good night! and see you all next Thursday!

%------------------
%-- Message Achilleas ( moderator )
Thanks, everyone!

%------------------
%-- Message Achilleas ( moderator )
Byeeee.... 

%------------------
