\section{Lesson Transcript}

%-- Message Achilleas ( moderator )
% Hi, everyone!

%------------------
%-- Message mark888 ( user )
% Hello!

%------------------
%-- Message MeepMurp5 ( user )
% Hi

%------------------
%-- Message Riya_Tapas ( user )
% hi! 

%------------------
%-- Message AOPS81619 ( user )
% Hi

%------------------
%-- Message Ezraft ( user )
% hello!

%------------------
%-- Message pike65_er ( user )
% helloo

%------------------
%-- Message xyab ( user )
% hi!

%------------------
%-- Message Achilleas ( moderator )
% \textbf{Olympiad GeometryWeek 10: Transformations}

%------------------
%-- Message Achilleas ( moderator )
Today we're going to talk about using geometric transformations to solve problems.

%------------------
%-- Message Achilleas ( moderator )
There are many types of transformations.  You've already learned about one.  Dilations (scaling figures up or down about a point) result in homothetic figures, which we've already studied.  In our last class, we'll discuss inversion.  In this class we'll add rotations and reflections to our tools.

%------------------
%-- Message Achilleas ( moderator )
Rotations are particularly useful in problems involving regular polygons, perpendicular lines, and midpoints.

%------------------
%-- Message Achilleas ( moderator )
What type of transformations might we use in problems involving equilateral triangles?

%------------------
%-- Message AOPS81619 ( user )
% Rotate by 60 degrees

%------------------
%-- Message MTHJJS ( user )
% rotation by 60 deg

%------------------
%-- Message pritiks ( user )
% a 60 degree rotation

%------------------
%-- Message razmath ( user )
% rotation by 60 degrees

%------------------
%-- Message RP3.1415 ( user )
% rotation by 60 degrees

%------------------
%-- Message Riya_Tapas ( user )
% $60^\circ$ rotations

%------------------
%-- Message Achilleas ( moderator )
Rotations of $60$ or $120$ degrees are useful in problems involving equilateral triangles.  For example, we can show that $ABC$ is equilateral by showing that a $60$ degree rotation about $A$ maps $B$ to $C.$

%------------------
%-- Message Achilleas ( moderator )
How about perpendicular lines?  What transformations should we use with them?

%------------------
%-- Message pritiks ( user )
% 90 degree rotation

%------------------
%-- Message christopherfu66 ( user )
% a 90 degree rotation

%------------------
%-- Message renyongfu ( user )
% 90 degree rotations?

%------------------
%-- Message Catherineyaya ( user )
% 90 degree rotation

%------------------
%-- Message B1002342 ( user )
% rotations of 90 degrees

%------------------
%-- Message Riya_Tapas ( user )
% $90^\circ$ degree rotation

%------------------
%-- Message Trollface60 ( user )
% rotation by 90 degrees

%------------------
%-- Message Bimikel ( user )
% rotate by 90 degrees

%------------------
%-- Message chardikala2 ( user )
% 90 degrees rotation?

%------------------
%-- Message mustwin_az ( user )
% rotation 90 degrees

%------------------
%-- Message smileapple ( user )
% 90 degree rotations

%------------------
%-- Message pike65_er ( user )
% 90 deg rotations?

%------------------
%-- Message ww2511 ( user )
% 90 degree rotation

%------------------
%-- Message AOPS81619 ( user )
% Rotate by 90 degrees or reflection?

%------------------
%-- Message mathlogic ( user )
% reflections

%------------------
%-- Message JacobGallager1 ( user )
% reflections?

%------------------
%-- Message Ezraft ( user )
% reflections, rotations by 90 degrees

%------------------
%-- Message razmath ( user )
% reflection across lines is pretty good

%------------------
%-- Message chardikala2 ( user )
% or reflections across some line or axis

%------------------
%-- Message dxs2016 ( user )
% reflections?

%------------------
%-- Message Achilleas ( moderator )
Naturally, rotations of $90$ degrees usually work well with perpendicular lines.

%------------------
%-- Message Achilleas ( moderator )
Reflections are also pretty good with perpendicular lines; if $A'$ is the reflection of $A$ over line $m,$ then $AA'$ is perpendicular to $m$ and $m$ bisects segment $AA'.$ Thus, perpendicular bisectors often make us consider reflections.

%------------------
%-- Message Achilleas ( moderator )
What kind of transformation can we use to work with midpoints of line segments?  (Say, if we want to show that a certain point was a midpoint.)

%------------------
%-- Message B1002342 ( user )
% reflections across that midpoint

%------------------
%-- Message RP3.1415 ( user )
% 180 degree rotation about the midpoint

%------------------
%-- Message ay0741 ( user )
% 180 rotation around midpoint

%------------------
%-- Message Achilleas ( moderator )
Reflection through a point is a $180$ degree rotation about the point.  Rotating a figure $180$ degrees around a midpoint maps one endpoint to another.

%------------------
%-- Message Achilleas ( moderator )
These concepts are pretty uselessly abstract until we try them on a few problems.  Before we dive into the problems, I'll introduce a little notation.  \textbf{This is not standard notation} - I just need something I can convey easily in class.

%------------------
%-- Message Achilleas ( moderator )
We write $\{P,60\}$ if we mean rotate $60$ degrees clockwise (cw) about $P.$ Similarly, $\{P,-60\}$ means rotate $60$ counterclockwise (ccw).

%------------------
%-- Message Achilleas ( moderator )
We write $\{AB\}$ if we mean reflect over $AB.$

%------------------
%-- Message Achilleas ( moderator )
We use $*$ to denote doing a series of transformations.  We read this right to left, though, so that $\{P,60\} * \{AB\}$ means reflect over $AB,$ then rotate $60$ degrees clockwise around $P.$ To apply this to a specific point, we write the point after the transformation in parentheses:

%------------------
%-- Message Achilleas ( moderator )
$\{P,60\} * \{AB\} (C) = D$ means that reflecting $C$ over $AB$ then rotating the result $60$ degrees clockwise around $P$ gives point $D.$

%------------------
%-- Message Achilleas ( moderator )
To make sure we have our notation down, take square $ABCD$ as shown in the diagram:

%------------------
%-- Message Achilleas ( moderator )



\begin{center}
\begin{asy}
import cse5;
import olympiad;


size(100);
draw(MP("A",dir(45),dir(45),fontsize(8))--MP("D",dir(45+90),dir(45+90),fontsize(8))--MP("C",dir(45+180),dir(45+180),fontsize(8))--MP("B",dir(45-90),dir(45-90),fontsize(8))--cycle);

\end{asy}
\end{center}





%------------------
%-- Message Achilleas ( moderator )
What is $\{AC\} (D)?$

%------------------
%-- Message RP3.1415 ( user )
% B

%------------------
%-- Message JacobGallager1 ( user )
% B

%------------------
%-- Message AOPS81619 ( user )
% $B$

%------------------
%-- Message MeepMurp5 ( user )
% $B$

%------------------
%-- Message coolbluealan ( user )
% B

%------------------
%-- Message dxs2016 ( user )
% B

%------------------
%-- Message myltbc10 ( user )
% B

%------------------
%-- Message Yufanwang ( user )
% B

%------------------
%-- Message Lucky0123 ( user )
% B

%------------------
%-- Message laura.yingyue.zhang ( user )
% B

%------------------
%-- Message Riya_Tapas ( user )
% $B$

%------------------
%-- Message pritiks ( user )
% B

%------------------
%-- Message ca981 ( user )
% B

%------------------
%-- Message J4wbr34k3r ( user )
% B

%------------------
%-- Message Wangminqi1 ( user )
% B

%------------------
%-- Message razmath ( user )
% B

%------------------
%-- Message B1002342 ( user )
% $B$

%------------------
%-- Message christopherfu66 ( user )
% $B$

%------------------
%-- Message Gamingfreddy ( user )
% {AC} (D) = B

%------------------
%-- Message mathlogic ( user )
% B

%------------------
%-- Message SlurpBurp ( user )
% $B$

%------------------
%-- Message renyongfu ( user )
% B

%------------------
%-- Message Bimikel ( user )
% B

%------------------
%-- Message Catherineyaya ( user )
% $B$

%------------------
%-- Message bigmath ( user )
% B

%------------------
%-- Message Ezraft ( user )
% $B$

%------------------
%-- Message Achilleas ( moderator )
The reflection of $D$ in $AC$ is $B,$ so $\{AC\} (D) = B.$

%------------------
%-- Message Achilleas ( moderator )
How about $\{C,-90\} (B)?$

%------------------
%-- Message RP3.1415 ( user )
% D

%------------------
%-- Message coolbluealan ( user )
% D

%------------------
%-- Message MeepMurp5 ( user )
% $D$

%------------------
%-- Message AOPS81619 ( user )
% $D$

%------------------
%-- Message pike65_er ( user )
% D

%------------------
%-- Message JacobGallager1 ( user )
% D

%------------------
%-- Message mathlogic ( user )
% D

%------------------
%-- Message Catherineyaya ( user )
% D

%------------------
%-- Message renyongfu ( user )
% D

%------------------
%-- Message mustwin_az ( user )
% D

%------------------
%-- Message Gamingfreddy ( user )
% D

%------------------
%-- Message Riya_Tapas ( user )
% $D$

%------------------
%-- Message mark888 ( user )
% D

%------------------
%-- Message Wangminqi1 ( user )
% D

%------------------
%-- Message ca981 ( user )
% D

%------------------
%-- Message B1002342 ( user )
% $D$

%------------------
%-- Message Lucky0123 ( user )
% D

%------------------
%-- Message pritiks ( user )
% D

%------------------
%-- Message Bimikel ( user )
% D

%------------------
%-- Message J4wbr34k3r ( user )
% D

%------------------
%-- Message christopherfu66 ( user )
% $D$

%------------------
%-- Message SlurpBurp ( user )
% $D$

%------------------
%-- Message smileapple ( user )
% $D$

%------------------
%-- Message Trollface60 ( user )
% D

%------------------
%-- Message Achilleas ( moderator )
Rotating $B$ $90$ degrees counterclockwise about $C$ gives $D,$ so $\{C,-90\} (B) = D.$

%------------------
%-- Message Achilleas ( moderator )
What is $\{AC\} * \{C,-90\} (B)?$

%------------------
%-- Message coolbluealan ( user )
% B

%------------------
%-- Message MeepMurp5 ( user )
% $B$

%------------------
%-- Message RP3.1415 ( user )
% B

%------------------
%-- Message MTHJJS ( user )
% B

%------------------
%-- Message mustwin_az ( user )
% B

%------------------
%-- Message pritiks ( user )
% B

%------------------
%-- Message mathlogic ( user )
% B

%------------------
%-- Message Riya_Tapas ( user )
% $B$

%------------------
%-- Message mark888 ( user )
% B

%------------------
%-- Message Bimikel ( user )
% B

%------------------
%-- Message razmath ( user )
% B.

%------------------
%-- Message Lucky0123 ( user )
% B

%------------------
%-- Message pike65_er ( user )
% B

%------------------
%-- Message ca981 ( user )
% B

%------------------
%-- Message AOPS81619 ( user )
% $B$

%------------------
%-- Message Gamingfreddy ( user )
% B

%------------------
%-- Message Achilleas ( moderator )
Well, $\{C,-90\} (B) = D,$ and $\{AC\} * D = B,$ so that's our answer.

%------------------
%-- Message Achilleas ( moderator )
What about $\{C,-90\} * \{AC\} (B)?$

%------------------
%-- Message Lucky0123 ( user )
% Its not one of the points on the figure

%------------------
%-- Message AOPS81619 ( user )
% It's not $A$, $B$, $C$, or $D$.

%------------------
%-- Message Achilleas ( moderator )
First, $\{AC\} (B) = D.$  Is $\{C,-90\} * D$ a point on the square?

%------------------
%-- Message razmath ( user )
% not any of the labelled points

%------------------
%-- Message Riya_Tapas ( user )
% Doesn't appear as a point on the diagram

%------------------
%-- Message mathlogic ( user )
% it's not on the square

%------------------
%-- Message Ezraft ( user )
% not one of $A, B, C, D$

%------------------
%-- Message Achilleas ( moderator )
No, it is not.

%------------------
%-- Message Lucky0123 ( user )
% Its the reflection of $B$ across $C$

%------------------
%-- Message myltbc10 ( user )
% reflection of B over CD

%------------------
%-- Message pritiks ( user )
% B reflected over DC

%------------------
%-- Message mustwin_az ( user )
% Reflection B across C

%------------------
%-- Message J4wbr34k3r ( user )
% No it's the reflection of B across C

%------------------
%-- Message AOPS81619 ( user )
% It's a point $P$ on line $BC$ such that $C$ is the midpoint of $BP$

%------------------
%-- Message MeepMurp5 ( user )
% the point $P$ such that $C$ is the midpoint of $BP$

%------------------
%-- Message dxs2016 ( user )
% reflection of B across DC

%------------------
%-- Message Catherineyaya ( user )
% the point E such that E is on line BC and C is the midpoint of BE

%------------------
%-- Message Riya_Tapas ( user )
% No, its the point $E$ on  the extension of $BC$ such that $EC=BC$

%------------------
%-- Message Achilleas ( moderator )
$\{C,-90\} * D$ is a point to the left of $C.$

%------------------
%-- Message Achilleas ( moderator )
Notice that $\{C,-90\} * \{AC\} (B)$ is not the same thing as $\{AC\} * \{C,-90\} (B).$ When you're applying several different transformations, it often matters what order you do them in.

%------------------
%-- Message Achilleas ( moderator )
Finally, what about $\{AC\} * \{A,-90\} * \{AD\} (B)?$

%------------------
%-- Message AOPS81619 ( user )
% $B$

%------------------
%-- Message coolbluealan ( user )
% B

%------------------
%-- Message razmath ( user )
% B

%------------------
%-- Message Wangminqi1 ( user )
% B

%------------------
%-- Message B1002342 ( user )
% $B$

%------------------
%-- Message Riya_Tapas ( user )
% $B$

%------------------
%-- Message MeepMurp5 ( user )
% $B$

%------------------
%-- Message Bimikel ( user )
% B

%------------------
%-- Message Gamingfreddy ( user )
% B

%------------------
%-- Message mathlogic ( user )
% B

%------------------
%-- Message MTHJJS ( user )
% B

%------------------
%-- Message dxs2016 ( user )
% B

%------------------
%-- Message mark888 ( user )
% B

%------------------
%-- Message Ezraft ( user )
% $B$

%------------------
%-- Message pritiks ( user )
% B

%------------------
%-- Message JacobGallager1 ( user )
% B

%------------------
%-- Message Trollface60 ( user )
% B

%------------------
%-- Message Catherineyaya ( user )
% B

%------------------
%-- Message pike65_er ( user )
% $B$

%------------------
%-- Message christopherfu66 ( user )
% $B$

%------------------
%-- Message Achilleas ( moderator )
Reflecting $B$ over $AD$ puts $B$ above $AD,$ then rotating that $90$ degrees counterclockwise takes the point to $D,$ and reflecting that over $AC$ brings us back home to $B,$ so:

%------------------
%-- Message Achilleas ( moderator )
$$\{AC\} * \{A,-90\} * \{AD\} (B) = B.$$

%------------------
%-- Message Achilleas ( moderator )
Note that the transformation $\{AC\} * \{A,-90\} * \{AD\}$ also maps $A,$ $C,$ and $D$ to themselves. In fact, it can be shown that it maps any point in the plane to itself (try to show this after class). Thus, this transformation is a long-winded way of saying 'do nothing'.  We usually call this 'do nothing' transformation the 'identity' and denote it $I.$  Thus $\{AC\} * \{A,-90\} * \{AD\} = I$.

%------------------
%-- Message Achilleas ( moderator )
Make sure you understand these notations.  If you do not, this class will be extremely confusing.

%------------------
%-- Message Achilleas ( moderator )
As one more test, what does $\{U, 20\} (X) = Z$ mean?

%------------------
%-- Message razmath ( user )
% Rotating X 20 degrees clockwise about U gives Z

%------------------
%-- Message dxs2016 ( user )
% rotating X 20 degrees cw around U gives the point Z

%------------------
%-- Message Catherineyaya ( user )
% Z is X rotated by 20 degrees clockwise about U

%------------------
%-- Message MeepMurp5 ( user )
% The rotation of $X$ $20^{\circ}$ clockwise about $U$ results in $Z$.

%------------------
%-- Message mathlogic ( user )
% rotating X about U 20 degrees clockwise gives point Z

%------------------
%-- Message mark888 ( user )
% rotating point $X$ clockwise $20$ degrees around point $U$ gives point $Z$.

%------------------
%-- Message AOPS81619 ( user )
% Rotate $X$ 20 degrees clockwise around $U$ and the result is $Z$

%------------------
%-- Message coolbluealan ( user )
% you get Z when you rotate X 20 degrees clockwise about U

%------------------
%-- Message pritiks ( user )
% rotation U 20 degrees CW around X gives us point Z

%------------------
%-- Message Ezraft ( user )
% $Z$ is $X$ rotated $20^{\circ}$ clockwise around $U$

%------------------
%-- Message Trollface60 ( user )
% rotating X 20 degrees clockwise about U gives Z

%------------------
%-- Message B1002342 ( user )
% $Z$ is the point obtained by a $20^{\circ}$ rotation of $X$ about $U$ in the clockwise direction

%------------------
%-- Message ca981 ( user )
% rotate X about point U by 20 degree clockwise, obtain point Z

%------------------
%-- Message pike65_er ( user )
% rotating point $X$ 20 degrees clockwise around $U$ gives point $Z$

%------------------
%-- Message christopherfu66 ( user )
% Rotating X twenty degrees clockwise about U results in Z.

%------------------
%-- Message smileapple ( user )
% Rotating $X$ about $U$ for $20^{\circ}$ clockwise will result in the point $Z$.

%------------------
%-- Message ay0741 ( user )
% X rotated 20 degrees clockwise around point U would map it to Z

%------------------
%-- Message Achilleas ( moderator )
$\{U, 20\} (X) = Z$ means that rotating $X$ $20$ degrees clockwise about $U$ results in point $Z.$ (Or 'the image of rotating $X$ $20$ degrees clockwise about $U$ is $Z$.)

%------------------
%-- Message Achilleas ( moderator )
\subsection{Properties of Rotations and Reflections}
There are three general properties of rotations and reflections that will be useful in problems involving rotation and reflection.  We won't prove all of these in class - I'll post some on the message board for you to prove.

%------------------
%-- Message Achilleas ( moderator )
\subsubsection{First Property}
Given intersecting lines $m$ and $n$, what is the result of reflecting a figure over $m$, then over $n$?    

%------------------
%-- Message Achilleas ( moderator )
For example, this combination takes quadrilateral $ABCD$ to $A'B'C'D'$ to $A''B''C''D''.$ What is the net transformation?

%------------------
%-- Message Achilleas ( moderator )



\begin{center}
\begin{asy}
import cse5;
import olympiad;


size(250);
pathpen = black + linewidth(0.7);
pointpen = black;
pen s = fontsize(8);

path scale(real s, pair D, pair E, real p) { return (point(D--E,p)+scale(s)*(-point(D--E,p)+D)--point(D--E,p)+scale(s)*(-point(D--E,p)+E));}

pair A = dir(120), B = dir(15), C = dir(-70), D = scale(.8)*dir(190);
pair P = (-1,-1), Q = (5,2), R = (5,-1), S = (-1,2), X = extension(P,Q,R,S);
pair AA = reflect(P,Q)*A, BB = reflect(P,Q)*B, CC = reflect(P,Q)*C, DD = reflect(P,Q)*D;
pair AAA = reflect(R,S)*AA, BBB = reflect(R,S)*BB, CCC = reflect(R,S)*CC, DDD = reflect(R,S)*DD;

draw(MP("A",A,N,s)--MP("B",B,E,s)--MP("C",C,SE,s)--MP("D",D,W,s)--cycle);
draw(MP("A'",AA,SE,s)--MP("B'",BB,NE,s)--MP("C'",CC,N,s)--MP("D'",DD,SW,s)--cycle,orange);
draw(MP("A''",AAA,E,s)--MP("B''",BBB,dir(-90),s)--MP("C''",CCC,W,s)--MP("D''",DDD,N,s)--cycle,heavygreen);
draw(P--Q^^R--S^^MP("X",X,scale(2)*dir(-90),s));

\end{asy}
\end{center}





%------------------
%-- Message MathJams ( user )
% A rotation about X?

%------------------
%-- Message Catherineyaya ( user )
% rotation about X

%------------------
%-- Message AOPS81619 ( user )
% Rotation around $X$?

%------------------
%-- Message Achilleas ( moderator )
The result is a rotation.

%------------------
%-- Message Achilleas ( moderator )
The rotation has center $X,$ and the angle of rotation is twice the angle between our two lines.

%------------------
%-- Message Achilleas ( moderator )
Let's prove this quickly.  Consider what happens to point $B.$  First, both transformations preserve the distance from $X,$ so it's unsurprising that $B''$ is a rotation of $B$ about $X.$  The question is the angle.

%------------------
%-- Message Achilleas ( moderator )
Imagine that $X$ is the origin, and the ray starting at $X$ and going to the upper left along the line is the positive x-axis.  We'll measure our angles counterclockwise from that ray.

%------------------
%-- Message Achilleas ( moderator )
Let $b$ be the initial angle of point $B,$ and let $r$ be the angle of the other line (doesn't matter which ray we use).  Then the first flip takes us from b to $r + (r-b) = 2r-b.$

%------------------
%-- Message Achilleas ( moderator )
The second takes us to the opposite of that, $b-2r$ (since we're reflecting across our "x-axis").  So the net effect is a clockwise rotation by $2r.$

%------------------
%-- Message Achilleas ( moderator )
\textbf{So the rule is this:} if line 2 is $r$ degrees clockwise from line $1,$ then reflecting over line $1$ and then line $2$ rotates you $2r$ clockwise.

%------------------
%-- Message Achilleas ( moderator )
What if the two lines $m$ and $m'$ are parallel? What is the composition of the two reflections in this case?

%------------------
%-- Message MeepMurp5 ( user )
% a translation?

%------------------
%-- Message B1002342 ( user )
% translation

%------------------
%-- Message ca981 ( user )
% pure translation

%------------------
%-- Message Yufanwang ( user )
% A translation

%------------------
%-- Message AOPS81619 ( user )
% A translation

%------------------
%-- Message Trollface60 ( user )
% a translation

%------------------
%-- Message Catherineyaya ( user )
% translation

%------------------
%-- Message mustwin_az ( user )
% translation

%------------------
%-- Message Wangminqi1 ( user )
% a translation

%------------------
%-- Message J4wbr34k3r ( user )
% A translation.

%------------------
%-- Message pritiks ( user )
% translation

%------------------
%-- Message Achilleas ( moderator )
A composition of reflections over two parallel lines is a translation. I'll leave this as an exercise.

%------------------
%-- Message Achilleas ( moderator )
\subsubsection{Second Property}
Now let's move to the second important property, involving rotations. How about rotating a figure about two different points $Y$ and $Z$ in succession?

%------------------
%-- Message Achilleas ( moderator )
For example, if we rotate around a point $Y$ by $40$ degrees counterclockwise, then rotate around a point $Z$ by $80$ degrees counterclockwise, then what is the net transformation?

%------------------
%-- Message razmath ( user )
% another rotation?

%------------------
%-- Message B1002342 ( user )
% a rotation

%------------------
%-- Message myltbc10 ( user )
% rotation 120 degrees counterclockwise

%------------------
%-- Message mark888 ( user )
% a rotation

%------------------
%-- Message MeepMurp5 ( user )
% another rotation I think

%------------------
%-- Message Achilleas ( moderator )



\begin{center}
\begin{asy}
import cse5;
import olympiad;


size(250);
pathpen = black + linewidth(0.7);
pointpen = black;
pen s = fontsize(8);

path scale(real s, pair D, pair E, real p) { return (point(D--E,p)+scale(s)*(-point(D--E,p)+D)--point(D--E,p)+scale(s)*(-point(D--E,p)+E));}

pair A = dir(120), B = dir(15), C = dir(-70), D = scale(.8)*dir(190);
pair X = (.5,1.3), Z = (1,3), Y = (-3,.6);
pair AA = Y + rotate(40)*(A-Y), BB = Y + rotate(40)*(B-Y), CC = Y + rotate(40)*(C-Y), DD = Y + rotate(40)*(D-Y);
pair AAA = Z + rotate(80)*(AA-Z), BBB = Z + rotate(80)*(BB-Z), CCC = Z + rotate(80)*(CC-Z), DDD = Z + rotate(80)*(DD-Z);

//dot(origin^^(1,1));
draw(MP("X",X,S,s)--MP("Y",Y,W,s)--MP("Z",Z,NE,s)--cycle);
draw(MP("A",A,N,s)--MP("B",B,E,s)--MP("C",C,SE,s)--MP("D",D,W,s)--cycle);
draw(MP("A'",AA,N,s)--MP("B'",BB,E,s)--MP("C'",CC,E,s)--MP("D'",DD,W,s)--cycle,orange);
draw(MP("A''",AAA,SW,s)--MP("B''",BBB,N,s)--MP("C''",CCC,NE,s)--MP("D''",DDD,SE,s)--cycle,heavygreen);

\end{asy}
\end{center}





%------------------
%-- Message Achilleas ( moderator )
In the diagram, $ABCD$ has been rotated $40$ degrees counterclockwise about $Y$ to $A'B'C'D'$ then $A'B'C'D'$ is rotated $80$ degrees counterclockwise about $Z$ to $A''B''C''D''.$  The result is a $40+80 = 120$ degree rotation about $X,$ where $X$ is the point such that $\angle ZYX = 40/2 = 20$ degrees and $\angle YZX = 80/2 = 40$ degrees.

%------------------
%-- Message Achilleas ( moderator )
More precisely, $X$ is obtained by taking the line $YZ$ and rotating $40/2=20$ clockwise, and (separately) $80/2=40$ counterclockwise, and taking the intersection.

%------------------
%-- Message Achilleas ( moderator )
I think it's intuitive that the net angle of rotation is the sum of the two individual angles, so we won't worry about explaining why that works.  As for the center:

%------------------
%-- Message Achilleas ( moderator )
The first rotation flips $X$ across $YZ;$ the second flips it back. So $X$ is ultimately fixed.

%------------------
%-- Message Achilleas ( moderator )
Hence, we could write:

%------------------
%-- Message Achilleas ( moderator )
$$\{Z,-80\} * \{Y,-40\} = \{X,-120\}.$$

%------------------
%-- Message Achilleas ( moderator )
We need to be able to know precisely the angles of this triangle, to identify  $X$.

%------------------
%-- Message Achilleas ( moderator )
(The two angles at  $Y$, $Z$ of the triangle are half of the angles of rotation.)

%------------------
%-- Message Achilleas ( moderator )
I will leave this proof as a message board problem.

%------------------
%-- Message Achilleas ( moderator )
This result is true for most angles of rotation. However, for certain angles of rotation (instead of $40$, $80$) the construction of $X$ won't work. When does that happen?

%------------------
%-- Message Lucky0123 ( user )
% Two rotations that add to 360 degrees

%------------------
%-- Message Achilleas ( moderator )
The case where this fails is when your two rotations of the line $YZ$ are parallel. Say the first rotation is $\alpha$ counterclockwise and the second $\beta$ counterclockwise. The two lines end up parallel if $-\alpha/2 = \beta/2 + 180k,$ or equivalently if $\alpha+\beta = 360k.$  So what happens in this case?

%------------------
%-- Message MathJams ( user )
% a translation instead?

%------------------
%-- Message AOPS81619 ( user )
% probably also translation

%------------------
%-- Message dxs2016 ( user )
% translation?

%------------------
%-- Message Achilleas ( moderator )
You just get a translation.  (I'll let you figure out for yourselves what the exact translation is.)

%------------------
%-- Message AOPS81619 ( user )
% translation is like rotation about an infinitely far away point

%------------------
%-- Message Achilleas ( moderator )
This is intuitive if you think of a translation as a rotation around the point at infinity. In this case the lines through $Y, Z$ are parallel, so they meet at infinity.

%------------------
%-- Message Achilleas ( moderator )
The important thing to remember is that the composition of $2$ rotations is a rotation, even if the $2$ rotations are not about the same point.  This holds if and only if the sum of the angles in the two rotations is not $360$ degrees, in which case the result is a translation.  Translations are essentially slides -- you slide the figure to another place without turning it or dilating it.

%------------------
%-- Message Achilleas ( moderator )
\subsubsection{Third Property}
Finally, if we have a rotation in which some point other than the center of the rotation is mapped to itself, what is the rotation?

%------------------
%-- Message B1002342 ( user )
% the identity

%------------------
%-- Message Riya_Tapas ( user )
% Identity rotation

%------------------
%-- Message razmath ( user )
% identity

%------------------
%-- Message Achilleas ( moderator )
A rotation that maps some point other than the center of the rotation to itself is the identity - it maps everything to itself.  (Translations also have a similar property - any translation that maps a point to itself is the identity.)

%------------------
%-- Message Achilleas ( moderator )
I'll post these three tools on the message board for proof.  At this point, we'll move on to some problems.

%------------------
%-- Message Achilleas ( moderator )
In general in Olympiads you can assert the aforementioned facts without proof.  However, I highly recommend you work through the proofs yourselves.

%------------------
%-- Message Achilleas ( moderator )
\subsection{Examples}
\begin{example}
Given the funky polygon in the diagram, construct an equilateral triangle $XYZ $ whose vertices are on the perimeter of the polygon.    
\end{example}

%------------------
%-- Message Achilleas ( moderator )



\begin{center}
\begin{asy}
import cse5;
import olympiad;


size(250);
pen s = fontsize(8);
pair A = (-43.74,-56.54), B = (-10.62,-18.4), C = (-78.32,3.7), D = (-31.81,-2.25), E = (-19.86,14.92), F = (33.74,26.3), G = (54.36,-23.12);
draw(MP("A",A,SW,s)--MP("B",B,dir(0),s)--MP("C",C,W,s)--MP("D",D,SE,s)--MP("E",E,NW,s)--MP("F",F,NE,s)--MP("G",G,dir(0),s)--cycle);

\end{asy}
\end{center}





%------------------
%-- Message Achilleas ( moderator )
What sort of transformation might we think of and why?

%------------------
%-- Message pritiks ( user )
% 60 degree rotations since we see equilateral triangle

%------------------
%-- Message AOPS81619 ( user )
% rotation by 60 degrees because it's equilateral triangles

%------------------
%-- Message Catherineyaya ( user )
% rotation by 60 degrees because equilateral triangles

%------------------
%-- Message MathJams ( user )
% rotation of 60 degrees, beacuse we are working with equilateral triangles

%------------------
%-- Message Ezraft ( user )
% rotation by $60^{\circ}$ because we have equilateral triangles

%------------------
%-- Message MeepMurp5 ( user )
% a rotation of $60^{\circ}$ clockwise or counterclockwise because those work well with equilateral triangles

%------------------
%-- Message B1002342 ( user )
% rotation of $60^{\circ}$ because all angles of an equilateral triangle are $60^{\circ}$

%------------------
%-- Message Achilleas ( moderator )
Seeing ``equilateral triangle," we think of $60$ degree rotations.  How will a $60$ degree rotation relate to this problem?

%------------------
%-- Message Achilleas ( moderator )
If we have $XYZ,$ then what is $\{X, -60\} (Y) $ if $XYZ$ is equilateral?

%------------------
%-- Message dxs2016 ( user )
% Z

%------------------
%-- Message coolbluealan ( user )
% Z

%------------------
%-- Message MathJams ( user )
% Z

%------------------
%-- Message myltbc10 ( user )
% Z

%------------------
%-- Message Trollface60 ( user )
% Z

%------------------
%-- Message razmath ( user )
% Z

%------------------
%-- Message Catherineyaya ( user )
% Z

%------------------
%-- Message MeepMurp5 ( user )
% $Z$

%------------------
%-- Message pritiks ( user )
% Z

%------------------
%-- Message Ezraft ( user )
% $Z$

%------------------
%-- Message RP3.1415 ( user )
% Z

%------------------
%-- Message ca981 ( user )
% Z

%------------------
%-- Message B1002342 ( user )
% $Z$

%------------------
%-- Message renyongfu ( user )
% Z

%------------------
%-- Message JacobGallager1 ( user )
% Z

%------------------
%-- Message ay0741 ( user )
% Z

%------------------
%-- Message smileapple ( user )
% $Z$

%------------------
%-- Message christopherfu66 ( user )
% $Z$

%------------------
%-- Message Achilleas ( moderator )
If we have $XYZ,$ we know that $\{X, -60\} (Y) = Z$ if $XYZ$ is equilateral. How does this help?

%------------------
%-- Message Achilleas ( moderator )
Say we have an $X. $ How can we use a rotation to find $Y$ and $Z?$

%------------------
%-- Message MathJams ( user )
% rotate the figure 60 deg about X and find the intersections of the new figure and the old one

%------------------
%-- Message Gamingfreddy ( user )
% rotate the polygon 60 degrees about X. The intersection(s) of the rotated and original polygons may be another vertex of XYZ.

%------------------
%-- Message coolbluealan ( user )
% rotate 60 degrees about X and find where the rotated diagram intersects the original diagram

%------------------
%-- Message Achilleas ( moderator )
Suppose we had $Y$ and $Z$ already. We know that $\{X, -60\} (Y) = Z,$ so if we pick a point $X$ on one of the sides and rotate our whole figure $60$ degree counterclockwise about it, the image, $Y',$ of $Y$ must coincide with our original $Z.$

%------------------
%-- Message Achilleas ( moderator )
$Y'$ must be on the rotated figure and $Z$ on the original figure, so this point must be on both the original and rotated figures.  (Any such point will do.)  Here's one possibility:

%------------------
%-- Message Achilleas ( moderator )



\begin{center}
\begin{asy}
import cse5;
import olympiad;


size(200);
pathpen = black + linewidth(0.7);
pointpen = black;
pen s = fontsize(8);

path scale(real s, pair D, pair E, real p) { return (point(D--E,p)+scale(s)*(-point(D--E,p)+D)--point(D--E,p)+scale(s)*(-point(D--E,p)+E));}
pair r(pair A, pair B, real d){return B = B+rotate(d)*(A-B);}

pair A = (-43.74,-56.54), B = (-10.62,-18.4), C = (-78.32,3.7), D = (-31.81,-2.25), E = (-19.86,14.92), F = (33.74,26.3), G = (54.36,-23.12);
pair X = point(A--G,.55);
pair Z = extension(E,D,r(F,X,60),r(G,X,60));
pair Y = r(Z,X,-60);
dot(MP("X",X,SE,s)^^MP("Y'=Z",Z,scale(2)*dir(-55),s));
draw(MP("A",A,SW,s)--MP("B",B,dir(0),s)--MP("C",C,W,s)--MP("D",D,SE,s)--MP("E",E,NW,s)--MP("F",F,NE,s)--MP("G",G,dir(0),s)--cycle);
draw(r(A,X,60)--r(B,X,60)--r(C,X,60)--r(D,X,60)--r(E,X,60)--r(F,X,60)--r(G,X,60)--cycle,mediumred);

//dot(MP("Y",Y,dir(0),s));
//draw(X--Y--Z--cycle,lightblue);

\end{asy}
\end{center}





%------------------
%-- Message Achilleas ( moderator )
We could have rotated clockwise, too.

%------------------
%-- Message Achilleas ( moderator )
How do we find $Y?$

%------------------
%-- Message Achilleas ( moderator )
(about what? direction?)

%------------------
%-- Message AOPS81619 ( user )
% Rotate $Z$ clockwise 60 degrees around $X$

%------------------
%-- Message Lucky0123 ( user )
% Rotate $Z$ 60 degrees clockwise about $X$

%------------------
%-- Message Catherineyaya ( user )
% rotate Y' 60 degrees clockwise about X

%------------------
%-- Message Achilleas ( moderator )
We know that rotating $Y'$ clockwise $60$ degrees around $X$ gives us a point on the original polygon; this is our $Y$ (i.e. $\{X,-60\} (Y') = Y$).  Thus, we draw $XY',$ then construct a $60$ degree angle on side $XY',$ producing $Y$ as shown:

%------------------
%-- Message Achilleas ( moderator )



\begin{center}
\begin{asy}
import cse5;
import olympiad;


size(200);
pathpen = black + linewidth(0.7);
pointpen = black;
pen s = fontsize(8);

path scale(real s, pair D, pair E, real p) { return (point(D--E,p)+scale(s)*(-point(D--E,p)+D)--point(D--E,p)+scale(s)*(-point(D--E,p)+E));}
pair r(pair A, pair B, real d){return B = B+rotate(d)*(A-B);}

pair A = (-43.74,-56.54), B = (-10.62,-18.4), C = (-78.32,3.7), D = (-31.81,-2.25), E = (-19.86,14.92), F = (33.74,26.3), G = (54.36,-23.12);
pair X = point(A--G,.55);
pair Z = extension(E,D,r(F,X,60),r(G,X,60));
pair Y = r(Z,X,-60);
dot(MP("X",X,SE,s)^^MP("Y'=Z",Z,scale(2)*dir(-55),s));
draw(MP("A",A,SW,s)--MP("B",B,dir(0),s)--MP("C",C,W,s)--MP("D",D,SE,s)--MP("E",E,NW,s)--MP("F",F,NE,s)--MP("G",G,dir(0),s)--cycle);
draw(r(A,X,60)--r(B,X,60)--r(C,X,60)--r(D,X,60)--r(E,X,60)--r(F,X,60)--r(G,X,60)--cycle,mediumred);

dot(MP("Y",Y,dir(0),s));
draw(X--Y--Z--cycle,lightblue);

\end{asy}
\end{center}





%------------------
%-- Message Achilleas ( moderator )
Each of the other points of intersection of the two polygons gives us another equilateral triangle solution:

%------------------
%-- Message Achilleas ( moderator )



\begin{center}
\begin{asy}
import cse5;
import olympiad;

pen red(real s){return rgb(255,255*s,255*s);}

//****************************************
pen original_funky_polygon = gray(.2); //higher the value, the lighter it gets
pen transform_funky_polygon = red(.8);
//****************************************

size(250);
pen s = fontsize(8);

path scale(real s, pair D, pair E, real p) { return (point(D--E,p)+scale(s)*(-point(D--E,p)+D)--point(D--E,p)+scale(s)*(-point(D--E,p)+E));}
pair r(pair A, pair B, real d){return B = B+rotate(d)*(A-B);}

pair A = (-43.74,-56.54), B = (-10.62,-18.4), C = (-78.32,3.7), D = (-31.81,-2.25), E = (-19.86,14.92), F = (33.74,26.3), G = (54.36,-23.12);
pair X = point(A--G,.60), Z = extension(E,D,r(F,X,60),r(G,X,60)), Y = r(Z,X,-60);
    pair P = extension(A,X,r(D,X,60),r(E,X,60)); draw(X--P--r(P,X,-60)--cycle,orange);
    pair Q = extension(A,B,r(D,X,60),r(E,X,60)); draw(X--Q--r(Q,X,-60)--cycle,green);
    pair R = extension(C,B,r(E,X,60),r(F,X,60)); draw(X--R--r(R,X,-60)--cycle,heavycyan);
    pair S = extension(C,D,r(F,X,60),r(E,X,60)); draw(X--S--r(S,X,-60)--cycle,magenta);
draw(MP("A",A,SW,s)--MP("B",B,dir(0),s)--MP("C",C,W,s)--MP("D",D,SE,s)--MP("E",E,NW,s)--MP("F",F,NE,s)--MP("G",G,dir(0),s)--cycle, original_funky_polygon);
draw(r(A,X,60)--r(B,X,60)--r(C,X,60)--r(D,X,60)--r(E,X,60)--r(F,X,60)--r(G,X,60)--cycle, transform_funky_polygon);

dot(MP("Y",Y,dir(0),s));
draw(X--Y--Z--cycle,lightblue);
dot(MP("X",X,SE,s)^^MP("Y'=Z",Z,scale(2)*dir(-55),s));

\end{asy}
\end{center}





%------------------
%-- Message Achilleas ( moderator )
One thing we glossed over in this problem was choosing $X.$  Depending on the figure, you might not be guaranteed that the rotation of the figure intersects the original figure for any $X. $ So to be fully rigorous, you would want to choose your $X$ carefully and prove that an intersection exists.

%------------------
%-- Message Achilleas ( moderator )
Next example:

%------------------
%-- Message Achilleas ( moderator )
In the diagram, $ABC$ is equilateral.  Point $R$ is on $AB,$ $P$ on $BC,$ and $Q$ on $AC$ such that $ARQ$ and $BRP$ are equilateral. Point $N$ is constructed such that $PQN$ is equilateral. Prove that $NR,$ $BQ,$ and $AP$ are equal in length.

%------------------
%-- Message Achilleas ( moderator )



\begin{center}
\begin{asy}
import cse5;
import olympiad;


size(150);
pathpen = black + linewidth(0.7);
pointpen = black;
pen s = fontsize(8);

path scale(real s, pair D, pair E, real p) { return (point(D--E,p)+scale(s)*(-point(D--E,p)+D)--point(D--E,p)+scale(s)*(-point(D--E,p)+E));}
pair r(pair A, pair B, real d){return B = B+rotate(d)*(A-B);}

real r = .4;

pair A = dir(210), B = dir(-30), C = dir(90);
pair R = point(A--B,.4), Q = r(R,A,60), P = r(R,B,-60);
pair N = r(P,Q,60), F = extension(N,R,A,P);

draw(MP("Q",Q,W,s)--MP("R",R,S,s)--MP("P",P,dir(30),s)--cycle--MP("N",N,dir(100),s)--P);
draw(MP("A",A,dir(210),s)--MP("B",B,dir(-30),s)--MP("C",C,dir(90),s)--cycle);
draw(N--R^^A--P^^Q--B,red);
dot(MP("F",F,scale(3)*dir(-60),s));

\end{asy}
\end{center}





%------------------
%-- Message Achilleas ( moderator )
Equilateral triangles galore.  What should we consider?

%------------------
%-- Message razmath ( user )
% rotation by 60 degrees

%------------------
%-- Message Riya_Tapas ( user )
% 60 degree rotations again

%------------------
%-- Message mark888 ( user )
% Rotations

%------------------
%-- Message pritiks ( user )
% 60 degree rotations

%------------------
%-- Message dxs2016 ( user )
% rotations by 60 deg

%------------------
%-- Message J4wbr34k3r ( user )
% Rotation 60 degrees.

%------------------
%-- Message Gamingfreddy ( user )
% rotations

%------------------
%-- Message Catherineyaya ( user )
% 60 degree rotation

%------------------
%-- Message mustwin_az ( user )
% rotation by 60 degrees

%------------------
%-- Message Ezraft ( user )
% rotations by $60^{\circ}$

%------------------
%-- Message Trollface60 ( user )
% rotate by 60 degrees

%------------------
%-- Message ca981 ( user )
% rotations

%------------------
%-- Message Achilleas ( moderator )
We'll consider rotations.

%------------------
%-- Message Achilleas ( moderator )
If we can map the segments to each other via rotation, we'll be able to show that they are equal in length.

%------------------
%-- Message Achilleas ( moderator )
How can we map $AP$ to $QB$? What rotation could we apply?

%------------------
%-- Message Achilleas ( moderator )
(you need to specify the center, the angle, and the direction)

%------------------
%-- Message razmath ( user )
% rotation at R by 60 degrees clockwise

%------------------
%-- Message Gamingfreddy ( user )
% Rotate AP about R by 60 degrees clockwise

%------------------
%-- Message coolbluealan ( user )
% rotate 60 degrees clockwise about R

%------------------
%-- Message B1002342 ( user )
% Rotation of $60^{\circ}$ clockwise about $R$?

%------------------
%-- Message Achilleas ( moderator )
$\{R,60\} (A) = Q$ and $\{R,60\} (P) = B,$ so $\{R, 60\} (AP) = QB;$ thus, $AP = QB.$

%------------------
%-- Message Achilleas ( moderator )
How can we map $BQ$ to $RN$? What rotation could we apply?

%------------------
%-- Message Achilleas ( moderator )
(you need to specify the center, the angle, and the direction)

%------------------
%-- Message MathJams ( user )
% Rotate about P 60 deg clockwise

%------------------
%-- Message Lucky0123 ( user )
% Rotate around $A$ by 60 degrees clockwise

%------------------
%-- Message razmath ( user )
% rotation at P by 60 degrees clockwise

%------------------
%-- Message Yufanwang ( user )
% Rotation aboutP by 60° clockwise

%------------------
%-- Message coolbluealan ( user )
% rotation 60 degrees clockwise about P

%------------------
%-- Message Trollface60 ( user )
% 60 degree cw rotation around P

%------------------
%-- Message Yufanwang ( user )
% Rotation about P by 60° clockwise

%------------------
%-- Message Catherineyaya ( user )
% rotate about P by 60 degrees clockwise

%------------------
%-- Message Ezraft ( user )
% rotation of $60^{\circ}$ colckwise about $P$

%------------------
%-- Message pritiks ( user )
% rotate BQ 60 degrees clockwise about P

%------------------
%-- Message Ezraft ( user )
% rotation of $60^{\circ}$ clockwise about $P$

%------------------
%-- Message B1002342 ( user )
% similar thing, rotate about $P$ $60^{\circ}$ clockwise

%------------------
%-- Message Gamingfreddy ( user )
% Rotation about P by 60 degrees clockwise

%------------------
%-- Message Achilleas ( moderator )
$\{P, 60\} (B) = R$ and $\{P, 60\} (Q) = N,$ so $\{P, 60\} (BQ) = RN$ and $BQ = RN.$

%------------------
%-- Message Achilleas ( moderator )
This is the sort of magic we'll be doing with rotations (and reflections) - these transformations preserve length, so if we can rotate or reflect one segment onto another, they have the same length.

%------------------
%-- Message Achilleas ( moderator )
As for concurrence: one way to prove it is to apply Miquel's theorem (which we mentioned in a previous class and is not hard to prove).  This says that the circumcircles of $AQR,$ $BPR,$ $CPQ$ are concurrent.  Now consider the intersection of $AP$ and $BQ.$  Let $F$ be their intersection.  Those two lines are related by a rotation of $60^\circ,$ so $\angle QFA=60^\circ.$  Then $F$ is on the circumcircle of $QRA.$  By the same logic it is on the circumcircle of $BPR.$  Therefore it is the Miquel Point for our configuration in triangle $ABC$. Since $N,C,P,F,Q$ are cocyclic, it follows that angle $NFQ$ equals $60^\circ$, thus $N,F,R$ are collinear (since angle $QFR$ is $120^\circ$).

%------------------
%-- Message Achilleas ( moderator )
Moving on:

%------------------
%-- Message Achilleas ( moderator )
\begin{example}
Squares of centers $P,$ $Q,$ and $R$ are constructed on the sides of a triangle $ABC$ outside the triangle.  Squares of centers $X,$ $Y,$ and $Z$ are constructed on the sides of the triangle $PQR$ inside the triangle.  Prove that $X,$ $Y,$ and $Z$ are the midpoints of the sides of triangle $ABC.$    
\end{example}

%------------------
%-- Message Achilleas ( moderator )
Before we get going on this problem, do I have to tackle each of the points $X,$ $Y,$ and $Z? $ In other words, do I have to prove that the center of the square constructed on $PQ$ is the midpoint of $BC,$ the center of the square constructed on $PR$ is the midpoint of $AC,$ etc., each individually?

%------------------
%-- Message dxs2016 ( user )
% perhaps just one, then everything follows similarly

%------------------
%-- Message Achilleas ( moderator )
No.  We only have to prove one.  The same proof will hold for all three since there's nothing special about any one side of $ABC.$

%------------------
%-- Message Achilleas ( moderator )
So, we draw our diagram for proving $X$ is the midpoint of $BC:$

%------------------
%-- Message Achilleas ( moderator )



\begin{center}
\begin{asy}
import cse5;
import olympiad;


size(250);
pathpen = black + linewidth(0.7);
pointpen = black;
pen s = fontsize(8);

path scale(real s, pair D, pair E, real p) { return (point(D--E,p)+scale(s)*(-point(D--E,p)+D)--point(D--E,p)+scale(s)*(-point(D--E,p)+E));}
pair r(pair A, pair B, real d){return B = B+rotate(d)*(A-B);}

pair B = dir(200), C = dir(-20), A = dir(110);
pair c1 = r(A,B,90), c2 = r(B,c1,90);
pair b1 = r(C,A,90), b2 = r(A,b1,90);
pair P = point(B--c2,.5), Q = point(C--b1,.5);

draw(A--B--c1--c2--A--c1--B--c2,heavygreen);
draw(A--C--b2--b1--A--b2--b1--C,heavygreen);
draw(MP("P",P,dir(70),s)--MP("Q",Q,dir(50),s)--r(P,Q,90)--r(Q,P,-90)--P--r(P,Q,90)^^Q--r(Q,P,-90),red);
MP("X",point(B--C,.5),scale(2)*S,s);
draw(MP("A",A,scale(2)*dir(100),s)--MP("B",B,S,s)--MP("C",C,S,s)--cycle);


\end{asy}
\end{center}





%------------------
%-- Message Achilleas ( moderator )
Why do we think of transformations?

%------------------
%-- Message Yufanwang ( user )
% So many squares

%------------------
%-- Message mark888 ( user )
% Because there are three squares.

%------------------
%-- Message Riya_Tapas ( user )
% Many $90^\circ$ angles

%------------------
%-- Message Catherineyaya ( user )
% lots of squares

%------------------
%-- Message Achilleas ( moderator )
We have squares, we have midpoints.  $90$ degree rotations can be useful in square problems, $180$ degree rotations useful in midpoint problems.  $90$ degree rotations.  $180$ degree rotations.  Hmmm. . .   What do we wonder?

%------------------
%-- Message AOPS81619 ( user )
% we proved that two 90 degree rotations makes a 180 degree rotation

%------------------
%-- Message Achilleas ( moderator )
We know that the result of two $90$ degree rotations is a $180$ degree rotation.  We think to look for a pair of useful $90$ degree rotations that we can prove result in a useful $180$ degree rotation.

%------------------
%-- Message Achilleas ( moderator )
What is the most likely candidate for the center of our $180$ degree rotation?

%------------------
%-- Message Achilleas ( moderator )
(we do not have $X$ yet)

%------------------
%-- Message SlurpBurp ( user )
% the midpoint of $BC$

%------------------
%-- Message Lucky0123 ( user )
% The midpoint of $BC$

%------------------
%-- Message mustwin_az ( user )
% Midpoint of BC

%------------------
%-- Message MTHJJS ( user )
% midpoint of BC?

%------------------
%-- Message Achilleas ( moderator )
We focus on our midpoint: Let $M$ be the midpoint of $BC;$ a $180$ degree rotation about $M$ takes $C$ to $B.$  What are we looking for now?

%------------------
%-- Message coolbluealan ( user )
% two rotations 90 degrees each that when combined also takes C to B

%------------------
%-- Message razmath ( user )
% two rotations which send B to C

%------------------
%-- Message SlurpBurp ( user )
% two rotations of $90^\circ$ that brings $B$ to $C$

%------------------
%-- Message Achilleas ( moderator )
We seek a couple $90$ degree rotations which together result in $\{M,180\}.$ $90$ degree rotations - we should focus on our squares.  What rotations do we want?

%------------------
%-- Message AOPS81619 ( user )
% Notice that $\{Q,-90\}*\{P,-90\}B=C$, and we can use our previous lemma I think

%------------------
%-- Message coolbluealan ( user )
% one with center Q 90 degrees clockwise and one with center P 90 degrees clockwise

%------------------
%-- Message mustwin_az ( user )
% Rotate C 90 degrees clockwise about Q to A then rotate A 90 degrees counterclockwise about P to B

%------------------
%-- Message ay0741 ( user )
% {P, 90} {Q, 90} (C) = B

%------------------
%-- Message Ezraft ( user )
% a rotation by $90^{\circ}$ clockwise about $Q,$ and a clockwise rotation by $90^{\circ}$ about $P$

%------------------
%-- Message Achilleas ( moderator )
We see that $\{Q,90\}$ takes $C$ to $A$ and $\{P,90\}$ takes $A$ to $ B.$  Therefore, what do we conclude?

%------------------
%-- Message Achilleas ( moderator )
Since $\{P,90\} * \{Q,90\} (C) = B,$ and we know that $\{P,90\} * \{Q,90\}$ is a $180$ degree rotation about some point, what do we have about $\{P, 90\} * \{Q,90\} ?$

%------------------
%-- Message Lucky0123 ( user )
% Its a 180 degrees rotation about the midpoint of BC

%------------------
%-- Message Catherineyaya ( user )
% {P, 90} * {Q, 90} = {M, 180}

%------------------
%-- Message Gamingfreddy ( user )
% {P, 90} * {Q, 90} is the same as {M, 180}

%------------------
%-- Message JacobGallager1 ( user )
% $\{P, 90 \} * \{Q, 90 \} = \{M, 180 \}$

%------------------
%-- Message Achilleas ( moderator )
We have $\{P, 90\} * \{Q,90\} = \{T,180\}$ for some point $T.$  The fact that $\{T,180\} (C) = B$ means that $T$ must be the midpoint of $BC$, so $T = M.$  Are we finished?

%------------------
%-- Message B1002342 ( user )
% the point $X$ for which $\angle XPQ = \angle XQP = 45^{\circ}$ is the center of the rotation, but this is also the rotation about the midpoint of $BC$

%------------------
%-- Message Ezraft ( user )
% it is equal to a rotation by $180^{\circ}$ about a point on the intersection of lines which form $45^{\circ}$ angles with $PQ,$ which is $X;$ Thus, $X$ is the midpoint of $BC$

%------------------
%-- Message Achilleas ( moderator )
Almost.  We can note that given $\{P, 90\} * \{Q,90\} = \{T,180\},$ we have $\angle TPQ = \angle TQP = 45$ degrees (from the information we discussed earlier about composition of rotations).  Since $T = M,$ we know that $MPQ$ is an isosceles right triangle.  Hence, $M$ is the center of the square constructed on side $PQ.$

%------------------
%-- Message Achilleas ( moderator )
In this problem we see the general technique of finding a group of transformations and showing that together they equal some other useful transformation.  This is often the key to breaking apart problems with transformations.

%------------------
%-- Message Achilleas ( moderator )
Next example:

%------------------
%-- Message Achilleas ( moderator )
\begin{example}
Trapezoid $ABCD$ has $AB \parallel CD.$  Point $P$ is on line $BC$ such that $P$ is not $B$ or $C$ and $M$ is the midpoint of $AB.$ $X$ is the intersection of line $PD$ and $AB,$ $Q$ the intersection of line $PM$ and $AC,$ and $Y$ the intersection of $DQ$ and $AB.$  Show that $M$ is the midpoint of $XY.$    
\end{example}

%------------------
%-- Message Achilleas ( moderator )



\begin{center}
\begin{asy}
import cse5;
import olympiad;


size(250);
pathpen = black + linewidth(0.7);
pointpen = black;
pen s = fontsize(8);

path scale(real s, pair D, pair E, real p) { return (point(D--E,p)+scale(s)*(-point(D--E,p)+D)--point(D--E,p)+scale(s)*(-point(D--E,p)+E));}
pair r(pair A, pair B, real d){return B = B+rotate(d)*(A-B);}

pair A = origin, B = (5,0), C = (4,2), D = (2,2), M = point(A--B,.5), P = C+scale(1.8)*(B-C), X = extension(D,P,A,B), Q = extension(P,M,A,C), Y = extension(D,Q,A,B);

draw(MP("A",A,dir(-160),s)--MP("B",B,dir(0),s)--MP("C",C,NE,s)--MP("D",D,N,s)--cycle);
draw(B--MP("P",P,dir(-60),s)--D--MP("Q",Q,NW,s)--P^^A--C^^Q--MP("Y",Y,S,s));
MP("M",M,dir(-120),s);
MP("X",X,dir(60),s);

\end{asy}
\end{center}





%------------------
%-- Message Achilleas ( moderator )
What sort of transformations are we likely to be looking at in this problem?

%------------------
%-- Message AOPS81619 ( user )
% reflections or rotation by 180 degrees

%------------------
%-- Message Ezraft ( user )
% reflections or $180^{\circ}$ rotations

%------------------
%-- Message pritiks ( user )
% 180 degree rotations since we're trying to show M is a midpoint

%------------------
%-- Message RP3.1415 ( user )
% 180 degree rotations

%------------------
%-- Message ca981 ( user )
% rotation about M by 180 degree ?

%------------------
%-- Message mustwin_az ( user )
% Rotations of 180 degrees

%------------------
%-- Message Achilleas ( moderator )
$M$ is the midpoint of $AB,$ so we know that $\{M,180\} (A) = B.$  What would we like to show?

%------------------
%-- Message dxs2016 ( user )
% {M,180}Y=X

%------------------
%-- Message AOPS81619 ( user )
% $\{M,180\}(Y)=X$

%------------------
%-- Message MeepMurp5 ( user )
% $ \{M, 180^{\circ} \}(Y) = X$

%------------------
%-- Message dxs2016 ( user )
% {M,180}(Y)=X

%------------------
%-- Message coolbluealan ( user )
% $\{M,180\}(Y)=X$

%------------------
%-- Message J4wbr34k3r ( user )
% {M, 180}(X)=Y

%------------------
%-- Message Lucky0123 ( user )
% {M,180}(X) = Y

%------------------
%-- Message B1002342 ( user )
% $\{M,180\}(Y) = X$

%------------------
%-- Message Catherineyaya ( user )
% {M, 180}(X)=Y

%------------------
%-- Message JacobGallager1 ( user )
% $\{M, 180 \}(X) = Y$

%------------------
%-- Message smileapple ( user )
% $\{M,180\}(X)=Y$

%------------------
%-- Message Wangminqi1 ( user )
% {M, 180}(Y)=X

%------------------
%-- Message mark888 ( user )
% {M,180}(X)=Y

%------------------
%-- Message mustwin_az ( user )
% {M,180}(X)=Y

%------------------
%-- Message Ezraft ( user )
% $\{ M, 180\}(X) = Y$

%------------------
%-- Message Achilleas ( moderator )
We would like to show that $\{M,180\} (Y) = X$ as well.

%------------------
%-- Message Achilleas ( moderator )
Thus, what are we hunting for?

%------------------
%-- Message Riya_Tapas ( user )
% A series of rotations that add up to $180^\circ$ that brings $X$ to $Y$

%------------------
%-- Message Achilleas ( moderator )
Do they have to be rotations?

%------------------
%-- Message Ezraft ( user )
% they could also be reflections

%------------------
%-- Message JacobGallager1 ( user )
% We're hunting for two transformations that when composed take $B$ to $A$, but also take $X to $Y\$

%------------------
%-- Message Achilleas ( moderator )
We're looking for a sequence of transformations that is equivalent to $\{M,180\},$ such that the sequence maps $Y$ to $X.$  How will we know we've found a sequence of transformations equivalent to $\{M,180\}?$  (We speak here of transformations which together make a rotation.)

%------------------
%-- Message AOPS81619 ( user )
% If it takes $A$ to $B$

%------------------
%-- Message Achilleas ( moderator )
What else?

%------------------
%-- Message AOPS81619 ( user )
% and it takes $M$ to $M$

%------------------
%-- Message MeepMurp5 ( user )
% it takes $M$ to $M$?

%------------------
%-- Message JacobGallager1 ( user )
% It fixes $M$

%------------------
%-- Message Achilleas ( moderator )
We'll know we've found a series of transformations equivalent to $\{M,180\}$ if $M$ is the only fixed point and the transformations all together map $A$ to $B.$

%------------------
%-- Message Achilleas ( moderator )
So, what other transformations might we use?

%------------------
%-- Message Achilleas ( moderator )
Are there any obvious rotations or reflections?

%------------------
%-- Message MathJams ( user )
% No?

%------------------
%-- Message Achilleas ( moderator )
There are no obvious rotations or reflections.  What other transformations might we consider?

%------------------
%-- Message Achilleas ( moderator )
Given that we have some parallel lines, which other transformation might be useful?

%------------------
%-- Message B1002342 ( user )
% homotheties

%------------------
%-- Message mustwin_az ( user )
% dilation?

%------------------
%-- Message SlurpBurp ( user )
% homothety, there seems to be similar triangles

%------------------
%-- Message pritiks ( user )
% dialations

%------------------
%-- Message myltbc10 ( user )
% dilation

%------------------
%-- Message coolbluealan ( user )
% homothety

%------------------
%-- Message ca981 ( user )
% Dilation ?

%------------------
%-- Message Gamingfreddy ( user )
% Homothety?

%------------------
%-- Message mark888 ( user )
% Dilation

%------------------
%-- Message Yufanwang ( user )
% Dilation

%------------------
%-- Message Achilleas ( moderator )
A dilation (homothety) might be useful.  Are there any potentially useful homotheties?

%------------------
%-- Message coolbluealan ( user )
% a homothety centered at P taking XB to DC

%------------------
%-- Message JacobGallager1 ( user )
% There is a homothety centered at $P$ which takes $XB$ to $DC$

%------------------
%-- Message Catherineyaya ( user )
% homothety centered at P mapping B to C, X to D

%------------------
%-- Message Wangminqi1 ( user )
% There is a homothety from $\triangle PXB$ to $\triangle PDC$ centered at $P$

%------------------
%-- Message MathJams ( user )
% Centered at P mapping C to B and D to X?

%------------------
%-- Message Ezraft ( user )
% the homothety about $P$ which maps $BX$ to $CD$

%------------------
%-- Message Achilleas ( moderator )
Triangle $PCD$ is homothetic to triangle $PBX$ with center $P.$

%------------------
%-- Message Achilleas ( moderator )
How about another one?

%------------------
%-- Message B1002342 ( user )
% homothety centered at $Q$ brining $AY$ to $CD$?

%------------------
%-- Message Ezraft ( user )
% $\triangle QAY$ is homothetic to $\triangle QCD$ about point $Q$

%------------------
%-- Message Gamingfreddy ( user )
% Triangle QDC is homothetic to triangle QYA with center Q

%------------------
%-- Message mark888 ( user )
% Triangle AYQ and CDQ is homothetic about center Q

%------------------
%-- Message Catherineyaya ( user )
% centered at Q mapping AY to CD

%------------------
%-- Message Achilleas ( moderator )
Triangle $QYA$ is homothetic to triangle $QDC$ with center $Q.$

%------------------
%-- Message Achilleas ( moderator )
What are we trying to build with these transformations?

%------------------
%-- Message Achilleas ( moderator )
We'd like to build a series of transformations that take $A$ to $B$ but leave $M$ unchanged.  Can we do it with these two homotheties?

%------------------
%-- Message Achilleas ( moderator )
(Let $\{J\}$ be the homothety that maps $QAY$ to $QCD$ and let $\{K\}$ be the homothety that takes $PCD$ to $PBX.$)

%------------------
%-- Message myltbc10 ( user )
% A goes to C and then to B

%------------------
%-- Message Achilleas ( moderator )
How?

%------------------
%-- Message Achilleas ( moderator )
(use the notation from this class)

%------------------
%-- Message JacobGallager1 ( user )
% Currently, $\{K \} * \{J \} (A) = B$

%------------------
%-- Message MathJams ( user )
% A under {J} goes to C, then under {K} goes to B

%------------------
%-- Message Ezraft ( user )
% $\{K\} * \{J\}(A) = B$

%------------------
%-- Message RP3.1415 ( user )
% {K}*{J}(A)=B

%------------------
%-- Message B1002342 ( user )
% $\{K\}*\{J\}(A) = B$

%------------------
%-- Message coolbluealan ( user )
% {J}(A)=C and {K}(C)=B so {K}*{J}(A)=B

%------------------
%-- Message renyongfu ( user )
% {K} * {J}(A) = B

%------------------
%-- Message dxs2016 ( user )
% {K}*{J}(A) =B

%------------------
%-- Message Achilleas ( moderator )
We see that homothety $\{J\}$ takes $A$ to $C,$ then homothety $\{K\}$ takes $C$ to $B.$  Hence $\{K\} * \{J\}(A) = B.$  This makes us happy.  What do we have to test?

%------------------
%-- Message B1002342 ( user )
% $\{K\}*\{J\}(M) = M$

%------------------
%-- Message Lucky0123 ( user )
% If {K} * {J} (M) = M

%------------------
%-- Message Achilleas ( moderator )
We have to make sure that $\{K\} * \{J\} (M) = M.$  Is it?

%------------------
%-- Message ay0741 ( user )
% yeah, just extend MQ until it meets line CD

%------------------
%-- Message Achilleas ( moderator )
$\{J\}$ (which takes $QAY$ to $QCD$) takes $M$ to a point on $CD$ that is on the extension of $PQ.$  Call this point $M'.$

%------------------
%-- Message Achilleas ( moderator )
What can we say about $\{K\} (M')?$

%------------------
%-- Message MathJams ( user )
% {K}(M') = M as desired

%------------------
%-- Message B1002342 ( user )
% $\{K\}(M') = M$

%------------------
%-- Message coolbluealan ( user )
% it is M

%------------------
%-- Message Ezraft ( user )
% $\{K\}(M') = M$

%------------------
%-- Message ca981 ( user )
% {K}(M')=M

%------------------
%-- Message Riya_Tapas ( user )
% It is point $M$

%------------------
%-- Message Achilleas ( moderator )
$\{K\}$ (which takes $PCD$ to $PBX$) takes $M'$ from a point on lines $CD$ and $PM$ to a point on lines $BX$ and $PM$ - that point is $M.$  Hence $\{K\} * \{J\} (M) = \{K\} (M') = M.$

%------------------
%-- Message Achilleas ( moderator )
What else do we want to test about $\{K\} * \{J\}?$

%------------------
%-- Message AOPS81619 ( user )
% It maps $Y$ to $X$

%------------------
%-- Message coolbluealan ( user )
% {K}*{J}(Y)=X

%------------------
%-- Message Riya_Tapas ( user )
% If {K}*{J}(Y)=X

%------------------
%-- Message MeepMurp5 ( user )
% Show $ \{K \} * \{ J \} (Y) = X$

%------------------
%-- Message JacobGallager1 ( user )
% We want to test that $\{K \} * \{J \} (Y) = X$

%------------------
%-- Message Achilleas ( moderator )
We wish to show that $\{K\} * \{J\} (Y) = X.$

%------------------
%-- Message Achilleas ( moderator )
What's $\{J\} (Y) ?$

%------------------
%-- Message Gamingfreddy ( user )
% D

%------------------
%-- Message Ezraft ( user )
% $D$

%------------------
%-- Message Riya_Tapas ( user )
% $D$

%------------------
%-- Message coolbluealan ( user )
% D

%------------------
%-- Message dxs2016 ( user )
% D

%------------------
%-- Message Wangminqi1 ( user )
% D

%------------------
%-- Message ca981 ( user )
% {J}(Y)=D

%------------------
%-- Message mark888 ( user )
% point $D$

%------------------
%-- Message MeepMurp5 ( user )
% $ \{J \}(Y) = D$

%------------------
%-- Message myltbc10 ( user )
% D

%------------------
%-- Message AOPS81619 ( user )
% $D$

%------------------
%-- Message JacobGallager1 ( user )
% $\{J \} (Y) = D$

%------------------
%-- Message Achilleas ( moderator )
$\{J\} (Y) = D,$ How about $\{K\} (D) ?$

%------------------
%-- Message coolbluealan ( user )
% X

%------------------
%-- Message Riya_Tapas ( user )
% $X$

%------------------
%-- Message Lucky0123 ( user )
% $X$

%------------------
%-- Message JacobGallager1 ( user )
% $\{K \} (D) = X$

%------------------
%-- Message ca981 ( user )
% {K} (D) = X

%------------------
%-- Message Wangminqi1 ( user )
% X

%------------------
%-- Message mustwin_az ( user )
% X

%------------------
%-- Message MathJams ( user )
% {K}(D) = X

%------------------
%-- Message MeepMurp5 ( user )
% $ \{ K \} (D) = X$

%------------------
%-- Message myltbc10 ( user )
% X

%------------------
%-- Message dxs2016 ( user )
% X

%------------------
%-- Message mark888 ( user )
% $X$

%------------------
%-- Message pritiks ( user )
% X

%------------------
%-- Message AOPS81619 ( user )
% $\{K\}(D)=X$

%------------------
%-- Message Ezraft ( user )
% $X$

%------------------
%-- Message renyongfu ( user )
% {K}(D) = X

%------------------
%-- Message Achilleas ( moderator )
$\{J\} (Y) = D, \{K\} (D) = X,$ so $\{K\} * \{J\} (Y) = \{K\} (D) = X$. Why are we are done?

%------------------
%-- Message Achilleas ( moderator )
In general, a composition of two homotheties of different centers is again a homothety.

%------------------
%-- Message Ezraft ( user )
% we have shown that this sequence of transformations is equivalent to a $180^{\circ}$ rotation about $M$

%------------------
%-- Message AOPS81619 ( user )
% Because our transformations must be equivalent to rotation of 180 degrees around $M$

%------------------
%-- Message Achilleas ( moderator )
We have two homotheties and the product of their scale factors is $-1.$  (We know this since $MA$ maps to $MB.$)  This means their composition preserves lengths.  Since $XM$ maps to $YM,$ $XM=YM.$

%------------------
%-- Message Achilleas ( moderator )
In general, a composition of two homotheties of different centers is again a homothety. The only exception to this rule is if the product of the scale factors of the homotheties is $1$, in which case their composition is a translation. In our case the product of the scale factors is $-1$, so the result is a length-preserving homothety which is just the reflection over $M$. We did not need these general facts in our proof though; all we used is that the composition is length preserving, since the product of the scale factors is $-1$.

%------------------
%-- Message Achilleas ( moderator )
These transformation proofs may look like magic, but they're really not.  We have a handful of indicators that suggest looking for a transformational approach - namely, midpoints, equilateral triangles, squares, homotheties.  From our initial indicators, the transformational solution, if there is one, is usually pretty short.

%------------------
%-- Message Achilleas ( moderator )
Therefore, when you hit a problem that you think might have a transformational solution, spend about 10 minutes or so looking for it.  Since the solution is usually pretty quick once you note that a transformational solution is possible, if you can't find it in 10 minutes or so, it's usually time to go after something else.

%------------------
%-- Message Achilleas ( moderator )
Next example:

%------------------
%-- Message Achilleas ( moderator )
\begin{example}
Angle $A$ is the smallest angle of triangle $ABC.$  Let point $U$ be on the circumcircle of triangle $ABC$ between points $B$ and $C$ (not the arc including $A$).  The perpendicular bisectors of $AB$ and $AC$ meet line $AU$ at $V$ and $W,$ respectively.  The lines $BV$ and $CW$ meet at $T.$  Show that $AU = TB + TC.$
    
\end{example}

%------------------
%-- Message Achilleas ( moderator )



\begin{center}
\begin{asy}
import cse5;
import olympiad;


size(250);
pathpen = black + linewidth(0.7);
pointpen = black;
pen s = fontsize(8);

path scale(real s, pair D, pair E, real p) { return (point(D--E,p)+scale(s)*(-point(D--E,p)+D)--point(D--E,p)+scale(s)*(-point(D--E,p)+E));}
pair r(pair A, pair B, real d){return B = B+rotate(d)*(A-B);}

pair A = dir(-20), B = dir(125), C = dir(205), U = point(Circle(origin,1),205), K = IPs(Circle(origin,1),point(A--B,.5)--bisectorpoint(A,B))[0];
pair M = IPs(Circle(origin,1),point(A--C,.5)--point(A--C,.5)+scale(2)*(bisectorpoint(A,C)-point(A--C,.5)))[0];
pair V = extension(K,rotate(180)*K,A,U), W = extension(M,rotate(180)*M,A,U);
pair T = extension(B,V,C,W);

draw(Circle(origin,1),heavygreen);
dot(MP("T",T,scale(2)*dir(-110),s)^^MP("V",V,scale(2)*dir(80),s)^^MP("W",W,scale(2)*dir(55),s)^^MP("U",U,dir(180),s));
draw(rightanglemark(M,point(A--C,.5),A,2)^^rightanglemark(rotate(180)*K,point(A--B,.5),A,2),red);
draw(scale(2.1,B,V,.13),arrow=ArcArrows(SimpleHead));
draw(scale(2.4,C,W,.13),arrow=ArcArrows(SimpleHead));
draw(scale(1.2,M,rotate(180)*M,.5),arrow=ArcArrows(SimpleHead));
draw(scale(1.2,K,rotate(180)*K,.5),arrow=ArcArrows(SimpleHead));
draw(U--MP("A",A,dir(-20),s)--MP("B",B,scale(2)*dir(170),s)--MP("C",C,scale(2)*dir(240),s)--A);
MP("m",M,dir(50),s);
MP("k",K,scale(2)*dir(15),s);


\end{asy}
\end{center}





%------------------
%-- Message Achilleas ( moderator )
Why try transformations?

%------------------
%-- Message Lucky0123 ( user )
% There are a lot of $90^\circ$ angles

%------------------
%-- Message dxs2016 ( user )
% perpendicular bisectors?

%------------------
%-- Message MathJams ( user )
% We have a lot of right angles, and a circle?

%------------------
%-- Message RP3.1415 ( user )
% because we have perpendicular bisectors so we could use reflection

%------------------
%-- Message Ezraft ( user )
% we have perpendicular bisectors, meaning $90^{\circ}$ angles and midpoints

%------------------
%-- Message Achilleas ( moderator )
We might think to try them just because we know that transformations will sometimes give a fast solution.

%------------------
%-- Message Achilleas ( moderator )
We have midpoints, but rotating $180$ degrees about the midpoints of $AB$ or $AC$ doesn't seem too helpful.

%------------------
%-- Message Achilleas ( moderator )
What other transformation might we consider?

%------------------
%-- Message RP3.1415 ( user )
% reflections!

%------------------
%-- Message MTHJJS ( user )
% reflection

%------------------
%-- Message Achilleas ( moderator )
Perpendicular bisectors sometimes offer quick solutions through reflection.  In this problem, how might reflections over $k$ and $m$ help?

%------------------
%-- Message Achilleas ( moderator )
What relationships can we find from reflections over $k$ or $m?$

%------------------
%-- Message Achilleas ( moderator )
For example, how about $\{m\} (A)$ and $\{k\} (A) ?$

%------------------
%-- Message Ezraft ( user )
% $\{k\}(A) = B, \{m\}(A) = C$

%------------------
%-- Message MathJams ( user )
% {k}(A) = B and {m} (A) = C

%------------------
%-- Message dxs2016 ( user )
% {m}(A) = C and {k}(A)=B

%------------------
%-- Message renyongfu ( user )
% {m}(A) = C and {k}(A) = B

%------------------
%-- Message B1002342 ( user )
% $\{m\}(A) = C$ and $\{k\}(A) = B$

%------------------
%-- Message RP3.1415 ( user )
% $\{ m \} (A)=C$ and $\{ k \} (A)=C$

%------------------
%-- Message Achilleas ( moderator )
We have $\{m\} (A) = C$ and $\{k\} (A) = B$.

%------------------
%-- Message Achilleas ( moderator )
Also, $\{m\} (AW) = CW$ (so $AW = CW$) and $\{k\} (AV) = BV.$

%------------------
%-- Message Achilleas ( moderator )
What point given in the problem have we not used?

%------------------
%-- Message B1002342 ( user )
% $T$

%------------------
%-- Message coolbluealan ( user )
% T

%------------------
%-- Message dxs2016 ( user )
% T

%------------------
%-- Message ca981 ( user )
% T

%------------------
%-- Message MeepMurp5 ( user )
% $T$

%------------------
%-- Message renyongfu ( user )
% T

%------------------
%-- Message Riya_Tapas ( user )
% $T$

%------------------
%-- Message Gamingfreddy ( user )
% T

%------------------
%-- Message Achilleas ( moderator )
We haven't used $T,$ but it's not too clear yet how to, since it doesn't map to anything obvious via reflection over $m$ or $k.$

%------------------
%-- Message Riya_Tapas ( user )
% $U$

%------------------
%-- Message MTHJJS ( user )
% U

%------------------
%-- Message Achilleas ( moderator )
We haven't used $U.$  How can we?

%------------------
%-- Message Achilleas ( moderator )
What's $\{m\}(U)$?

%------------------
%-- Message dxs2016 ( user )
% intersection of CW and circumcircle?

%------------------
%-- Message B1002342 ( user )
% $\{m\}(U) = CW\cap (ABC)$

%------------------
%-- Message mustwin_az ( user )
% Intersection of CW and circumcircle not C

%------------------
%-- Message coolbluealan ( user )
% where CW intersects the circle again

%------------------
%-- Message MathJams ( user )
% {m} (U ) is the interscetion of CW with the circle

%------------------
%-- Message ca981 ( user )
% on point where CW intersect circle

%------------------
%-- Message mark888 ( user )
% the intersection point of $CW$ with the circumcircle?

%------------------
%-- Message Achilleas ( moderator )
Let the extension of $CW$ hit the circle again at $E$ and the extension of $BV$ hit the circle again at $D.$

%------------------
%-- Message Achilleas ( moderator )



\begin{center}
\begin{asy}
import cse5;
import olympiad;


size(250);
pathpen = black + linewidth(0.7);
pointpen = black;
pen s = fontsize(8);

/* helper functions */
path scale(real s, pair D, pair E, real p) { return (point(D--E,p)+scale(s)*(-point(D--E,p)+D)--point(D--E,p)+scale(s)*(-point(D--E,p)+E));}
pair r(pair A, pair B, real d){return B = B+rotate(d)*(A-B);}

/* independent points */
pair A = dir(-20), B = dir(125), C = dir(205), U = point(Circle(origin,1),205);

pair K = IPs(Circle(origin,1),point(A--B,.5)--bisectorpoint(A,B))[0];
pair M = IPs(Circle(origin,1),point(A--C,.5)--point(A--C,.5)+scale(2)*(bisectorpoint(A,C)-point(A--C,.5)))[0];
pair V = extension(K,rotate(180)*K,A,U), W = extension(M,rotate(180)*M,A,U);
pair T = extension(B,V,C,W);
pair D = IPs(Circle(origin,1),scale(3,B,V,0))[1];
pair E = IPs(Circle(origin,1),scale(3,C,W,0))[0];

draw(Circle(origin,1),heavygreen);
dot(MP("T",T,scale(2)*dir(-110),s)^^MP("V",V,scale(2)*dir(80),s)^^MP("W",W,scale(2)*dir(55),s)^^MP("U",U,dir(180),s));
draw(rightanglemark(M,point(A--C,.5),A,2)^^rightanglemark(rotate(180)*K,point(A--B,.5),A,2),red);
draw(scale(2.1,B,V,.13),arrow=ArcArrows(SimpleHead));
draw(scale(2.4,C,W,.13),arrow=ArcArrows(SimpleHead));
draw(scale(1.2,M,rotate(180)*M,.5),arrow=ArcArrows(SimpleHead));
draw(scale(1.2,K,rotate(180)*K,.5),arrow=ArcArrows(SimpleHead));
draw(U--MP("A",A,dir(-20),s)--MP("B",B,scale(2)*dir(170),s)--MP("C",C,scale(2)*dir(240),s)--A);
MP("m",M,dir(50),s);
MP("k",K,scale(2)*dir(15),s);
MP("D",D,scale(2)*dir(-25),s);
MP("E",E,scale(2)*dir(-40),s);

\end{asy}
\end{center}





%------------------
%-- Message Achilleas ( moderator )
What do we have?

%------------------
%-- Message Achilleas ( moderator )
What's $\{m\} (AU) ?$

%------------------
%-- Message pritiks ( user )
% CE

%------------------
%-- Message Ezraft ( user )
% $CE$

%------------------
%-- Message coolbluealan ( user )
% CE

%------------------
%-- Message mustwin_az ( user )
% CE

%------------------
%-- Message dxs2016 ( user )
% CE

%------------------
%-- Message Wangminqi1 ( user )
% CE

%------------------
%-- Message B1002342 ( user )
% $\{m\}(AU) = CE$

%------------------
%-- Message myltbc10 ( user )
% CE

%------------------
%-- Message Achilleas ( moderator )
It's $\{m\} (AU) = CE.$ How about $\{k\} (AU) ?$

%------------------
%-- Message coolbluealan ( user )
% BD

%------------------
%-- Message B1002342 ( user )
% $BD$

%------------------
%-- Message Ezraft ( user )
% $BD$

%------------------
%-- Message Wangminqi1 ( user )
% BD

%------------------
%-- Message Gamingfreddy ( user )
% BD

%------------------
%-- Message myltbc10 ( user )
% BD

%------------------
%-- Message MathJams ( user )
% {k}(AU) = BD

%------------------
%-- Message Riya_Tapas ( user )
% $BD$

%------------------
%-- Message dxs2016 ( user )
% BD

%------------------
%-- Message mustwin_az ( user )
% BD

%------------------
%-- Message JacobGallager1 ( user )
% $\{k\}(AU) = BD$

%------------------
%-- Message pritiks ( user )
% BD

%------------------
%-- Message Achilleas ( moderator )
$\{k\} (AU) = BD.$ So?

%------------------
%-- Message ca981 ( user )
% BD = CE

%------------------
%-- Message JacobGallager1 ( user )
% $BD = AU = CE$

%------------------
%-- Message Achilleas ( moderator )
Thus, $AU = CE$ and $AU = BD,$ so $BD = CE.$  Can we use this to show $AU = TB + TC?$

%------------------
%-- Message coolbluealan ( user )
% we want CT=TD

%------------------
%-- Message B1002342 ( user )
% yes, if we can show $BT = TE$ or $CT = TD$ we are done

%------------------
%-- Message Achilleas ( moderator )
What kind of quadrilateral is $BCDE?$

%------------------
%-- Message Achilleas ( moderator )
(of course, it is cyclic.  )

%------------------
%-- Message Achilleas ( moderator )
(can you say more about it?)

%------------------
%-- Message J4wbr34k3r ( user )
% Isosceles trapezoid.

%------------------
%-- Message ay0741 ( user )
% also isoceles trapezoid

%------------------
%-- Message Ezraft ( user )
% its an isosceles trapezoid

%------------------
%-- Message ca981 ( user )
% isosceles trapezoid

%------------------
%-- Message leoouyang ( user )
% An isosceles trapezoid

%------------------
%-- Message MTHJJS ( user )
% isosceles trapezoid

%------------------
%-- Message Achilleas ( moderator )
$BD = EC$ implies that $BEDC$ is an isosceles trapezoid.  So what can we say about triangle $TCD?$

%------------------
%-- Message myltbc10 ( user )
% it is isosceles

%------------------
%-- Message Wangminqi1 ( user )
% It is isosceles

%------------------
%-- Message pritiks ( user )
% it is also isosceles

%------------------
%-- Message dxs2016 ( user )
% isosceles

%------------------
%-- Message ay0741 ( user )
% isosceles triangle

%------------------
%-- Message Gamingfreddy ( user )
% Triangle TCD is isosceles

%------------------
%-- Message ca981 ( user )
% isosceles triangle

%------------------
%-- Message AOPS81619 ( user )
% it is isosceles

%------------------
%-- Message renyongfu ( user )
% TCD is isosceles

%------------------
%-- Message coolbluealan ( user )
% isosceles

%------------------
%-- Message leoouyang ( user )
% It is isosceles and thus TC = TD

%------------------
%-- Message mark888 ( user )
% Isosceles triangle

%------------------
%-- Message Ezraft ( user )
% it is isosceles, so $TC = TD$ and we're done

%------------------
%-- Message Lucky0123 ( user )
% It's isosceles

%------------------
%-- Message MathJams ( user )
% isoscelese triangle!

%------------------
%-- Message B1002342 ( user )
% isosceles, because $\angle TDC = \angle TCD$ from looking at arcs

%------------------
%-- Message Achilleas ( moderator )
Triangle $TCD$ is isosceles. (Proof: $BEDC$ is an isosceles trapezoid.  Equal chords cut off equal arcs, so $\text{arc }(CE) =\text{arc }(DB),$ so $\text{arc } (BC) = \text{arc }(DE),$ so $\angle BDC = \angle DCE.$  Thus $TCD$ is isosceles and $TC = TD.$)

%------------------
%-- Message Achilleas ( moderator )
You can see this because $\angle TCD = \angle ECD = \angle BDC = \angle TDC.$

%------------------
%-- Message Achilleas ( moderator )
How do we finish?

%------------------
%-- Message MathJams ( user )
% TB+TC = TB+TD = BD = CE = AU

%------------------
%-- Message coolbluealan ( user )
% TC+TB=TD+TB=BD=AU

%------------------
%-- Message dxs2016 ( user )
% TB+TC=TD+TB=DB=EC=AU

%------------------
%-- Message ay0741 ( user )
% AU = BD = BT + TD = BT + TC

%------------------
%-- Message JacobGallager1 ( user )
% Because $\triangle TCD$ is iscoceles, we have $TC = TD$. Therefore, $BT + TC = BT + TD = BD = AU$ as desired

%------------------
%-- Message Achilleas ( moderator )
Hence $TB + TC = TB + TD = BD = AU,$ and we are done.

%------------------
%-- Message Achilleas ( moderator )
\subsection{Spiral Similarity}
One final transformation that we want to introduce is called a \textbf{spiral similarity}.

%------------------
%-- Message Achilleas ( moderator )
A spiral similarity is the composition of a homothety and a rotation, both having the same center.  Thus, a spiral similarity is specified by a center (of homothety and rotation), a scaling factor, and an angle of rotation. In this case it does not matter in which order we compose the two transformations, as the result is the same.

%------------------
%-- Message Yufanwang ( user )
% Don't we usually use spiral similarity with complex numbers?

%------------------
%-- Message Achilleas ( moderator )
% We do no have to use complex numbers for a spiral similarity.

%------------------
%-- Message Achilleas ( moderator )
% I guess you have seen an article or a handout doing it. 

%------------------
%-- Message Achilleas ( moderator )
In the following diagram, a spiral similarity takes $A$ to $A'$ and $B$ to $B'.$  We see that triangles $AOB$ and $A'OB'$ are similar.

%------------------
%-- Message Achilleas ( moderator )



\begin{center}
\begin{asy}
import cse5;
import olympiad;


size(200);
pen s = fontsize(8);

/* helper functions */
path scale(real s, pair D, pair E, real p) { return (point(D--E,p)+scale(s)*(-point(D--E,p)+D)--point(D--E,p)+scale(s)*(-point(D--E,p)+E));}
pair r(pair A, pair B, real d){return B = B+rotate(d)*(A-B);}

/* independent values */
real rot = 100, sca = 1.8, A_deg = 25, B_deg = 45, B_sca = .78;

pair A = dir(A_deg), B = scale(B_sca)*dir(B_deg);

pair AA = scale(sca)*rotate(rot)*A, BB = scale(sca)*rotate(rot)*B;

draw(arc(origin,.3,25,25+rot,true),arrow=ArcArrow(SimpleHead),linewidth(.5));
draw(arc(origin,.2,45,45+rot,true),arrow=ArcArrow(SimpleHead),linewidth(.5));
draw(MP("O",origin,S,s)--MP("A",A,E,s)--MP("B",B,N,s)--cycle);
draw(origin--MP("A'",AA,N,s)--MP("B'",BB,W,s)--cycle);

\end{asy}
\end{center}





%------------------
%-- Message Achilleas ( moderator )
Given points $A$, $B$, $A'$, and $B'$ in the plane, construct the center of the spiral similarity that takes $A$ to $A'$ and $B$ to $B'$.

%------------------
%-- Message Achilleas ( moderator )
As in other construction problems, let's start with a complete diagram, so we know what to look for.

%------------------
%-- Message Achilleas ( moderator )



\begin{center}
\begin{asy}
import cse5;
import olympiad;


import markers;
size(150);
pen s = fontsize(8);

/* helper functions */
path scale(real s, pair D, pair E, real p) { return (point(D--E,p)+scale(s)*(-point(D--E,p)+D)--point(D--E,p)+scale(s)*(-point(D--E,p)+E));}
pair r(pair A, pair B, real d){return B = B+rotate(d)*(A-B);}

real rotate_angle = 20;
pair O = (.8,.8);
pair A = origin, B = (.6,.05);

pair AA = r(A,O,rotate_angle), BB = r(B,O,rotate_angle);

draw(MP("A'",AA,S,s)--O--MP("B'",BB,SE,s)--cycle);
draw(MP("A",A,W,s)--MP("O",O,N,s)--MP("B",B,SE,s)--cycle);
markangle(radius=20,B,A,O,marker(markinterval(stickframe(n=1),true)));
markangle(radius=20,BB,AA,O,marker(markinterval(stickframe(n=1),true)));

\end{asy}
\end{center}





%------------------
%-- Message Achilleas ( moderator )
What do you see about triangles $OAB$ and $OA'B'$?

%------------------
%-- Message Ezraft ( user )
% they are similar

%------------------
%-- Message MeepMurp5 ( user )
% They're similar

%------------------
%-- Message tigerzhang ( user )
% They are similar

%------------------
%-- Message J4wbr34k3r ( user )
% Similar.

%------------------
%-- Message MathJams ( user )
% triangle OAB similar to triangle OA'B'

%------------------
%-- Message JacobGallager1 ( user )
% They are similar

%------------------
%-- Message Catherineyaya ( user )
% $\triangle OAB\sim\triangle OA'B'$

%------------------
%-- Message mustwin_az ( user )
% similar

%------------------
%-- Message coolbluealan ( user )
% they are similar

%------------------
%-- Message Achilleas ( moderator )
We see that triangles $OAB$ and $OA'B'$ are similar, so $\angle OAB = \angle OA'B'.$  Based on this observation, what construction might we try?

%------------------
%-- Message Gamingfreddy ( user )
% A circle?

%------------------
%-- Message Achilleas ( moderator )
Which circle?

%------------------
%-- Message Achilleas ( moderator )
(we are not given $O$)

%------------------
%-- Message pritiks ( user )
% circle through points A,A',B'

%------------------
%-- Message Wangminqi1 ( user )
% The circumcircle of $AA'B'$

%------------------
%-- Message Achilleas ( moderator )
How does the circumcircle of $AA'B'$ work?

%------------------
%-- Message coolbluealan ( user )
% extend AB and A'B' to meet at P and then draw the circumcircle of AA'P

%------------------
%-- Message JacobGallager1 ( user )
% Maybe let $P$ be the intersection between $BA'$ and $AB'$. Draw the circumcircles of $\triangle APB$ and $\triangle A'PB'$

%------------------
%-- Message Achilleas ( moderator )
Let $AB$ and $A'B'$ intersect at $P.$  Then points $O,$ $A,$ $A',$ and $P$ are concyclic.

%------------------
%-- Message Achilleas ( moderator )



\begin{center}
\begin{asy}
import cse5;
import olympiad;


import markers;
size(150);
pen s = fontsize(8);

/* helper functions */
path scale(real s, pair D, pair E, real p) { return (point(D--E,p)+scale(s)*(-point(D--E,p)+D)--point(D--E,p)+scale(s)*(-point(D--E,p)+E));}
pair r(pair A, pair B, real d){return B = B+rotate(d)*(A-B);}

real rotate_angle = 20;
pair O = (.8,.8);
pair A = origin, B = (.6,.05);

pair AA = r(A,O,rotate_angle), BB = r(B,O,rotate_angle);
pair P = extension(A,B,AA,BB);

draw(circumcircle(A,O,AA),heavygreen);
draw(B--MP("P",P,E,s)--BB);
draw(MP("A'",AA,S,s)--O--MP("B'",BB,S,s)--cycle);
draw(MP("A",A,W,s)--MP("O",O,N,s)--MP("B",B,S,s)--cycle);
markangle(radius=20,B,A,O,marker(markinterval(stickframe(n=1),true)));
markangle(radius=20,BB,AA,O,marker(markinterval(stickframe(n=1),true)));

\end{asy}
\end{center}





%------------------
%-- Message Achilleas ( moderator )
Since $\angle ABO = \angle A'B'O,$ we have that $\angle OBP = \angle OB'P.$  Therefore, points $O,$ $B,$ $B',$ and $P$ are also concyclic.

%------------------
%-- Message Achilleas ( moderator )



\begin{center}
\begin{asy}
import cse5;
import olympiad;


import markers;
size(150);
pen s = fontsize(8);

/* helper functions */
path scale(real s, pair D, pair E, real p) { return (point(D--E,p)+scale(s)*(-point(D--E,p)+D)--point(D--E,p)+scale(s)*(-point(D--E,p)+E));}
pair r(pair A, pair B, real d){return B = B+rotate(d)*(A-B);}

real rotate_angle = 20;
pair O = (.8,.8);
pair A = origin, B = (.6,.05);

pair AA = r(A,O,rotate_angle), BB = r(B,O,rotate_angle);
pair P = extension(A,B,AA,BB);

draw(circumcircle(A,O,AA));
draw(circumcircle(O,B,P));
draw(B--MP("P",P,E,s)--BB);
draw(MP("A'",AA,S,s)--O--MP("B'",BB,S,s)--cycle);
draw(MP("A",A,W,s)--MP("O",O,N,s)--MP("B",B,S,s)--cycle);
markangle(radius=20,B,A,O,marker(markinterval(stickframe(n=1),true)));
markangle(radius=20,BB,AA,O,marker(markinterval(stickframe(n=1),true)));

\end{asy}
\end{center}





%------------------
%-- Message Achilleas ( moderator )
So how can we construct point $O?$

%------------------
%-- Message B1002342 ( user )
% $P:= AB \cap C'B'$, then $O:= (AA'P)\cap (BB'P)$ where $O\neq P$

%------------------
%-- Message JacobGallager1 ( user )
% Draw the circumcircles of $\triangle ABP$ and $\triangle A'B'P$. The second intersection of these circles besides $P$ is $O$.

%------------------
%-- Message Achilleas ( moderator )
We can construct the center of the spiral similarity $O$ as follows: Let $P$ be the intersection of $AB$ and $A'B'.$  Construct the circumcircles of triangles $PAA'$ and $PBB'.$  Then these two circumcircles intersect at $P$ and $O.$

%------------------
%-- Message Achilleas ( moderator )
(The only time this fails will be if $AB \parallel A'B'.$  In that case, a spiral similarity with angle 0 might work (i.e., a pure homothety), or if also $AB=A'B',$ no spiral similarity works but a translation does.)

%------------------
%-- Message Achilleas ( moderator )
Since triangles $OAB$ and $OA'B'$ are similar, $OA/OA' = OB/OB'.$

%------------------
%-- Message Achilleas ( moderator )
It follows that $ OA/OB = OA'/OB'.$  In other words, triangles $OAA'$ and $OBB'$ are similar.

%------------------
%-- Message Achilleas ( moderator )
Therefore, the center of the spiral similarity that takes $AB$ to $A'B'$ is also the center of the spiral similarity that takes $AA'$ to $BB'.$

%------------------
%-- Message Achilleas ( moderator )
We now give a proof of Ptolemy's Theorem using spiral similarity.

%------------------
%-- Message Achilleas ( moderator )
Let $ABCD$ be a cyclic quadrilateral.

%------------------
%-- Message Achilleas ( moderator )



\begin{center}
\begin{asy}
import cse5;
import olympiad;


import markers;
size(150);
pen s = fontsize(8);

/* helper functions */
path scale(real s, pair D, pair E, real p) { return (point(D--E,p)+scale(s)*(-point(D--E,p)+D)--point(D--E,p)+scale(s)*(-point(D--E,p)+E));}
pair r(pair A, pair B, real d){return B = B+rotate(d)*(A-B);}

pair A = dir(125), B = dir(40), C = dir(-20), D = dir(200);

draw(Circle(origin,1));
draw(MP("A",A,dir(125),s)--MP("B",B,dir(40),s)--MP("C",C,dir(-20),s)--MP("D",D,dir(200),s)--A--C^^B--D);

\end{asy}
\end{center}





%------------------
%-- Message Achilleas ( moderator )
Consider the spiral similarity, centered at $A, $ that takes $C$ to $D.$  Let this spiral similarity take $B$ to $X,$ so triangles $ABC$ and $AXD$ are similar.

%------------------
%-- Message Achilleas ( moderator )



\begin{center}
\begin{asy}
import cse5;
import olympiad;


import markers;
size(150);
pen s = fontsize(8);

/* helper functions */
path scale(real s, pair D, pair E, real p) { return (point(D--E,p)+scale(s)*(-point(D--E,p)+D)--point(D--E,p)+scale(s)*(-point(D--E,p)+E));}
pair r(pair A, pair B, real d){return B = B+rotate(d)*(A-B);}

real A_deg = 125, B_deg = 40, C_deg = 340, D_deg = 200;
pair A = dir(A_deg), B = dir(B_deg), C = dir(C_deg), D = dir(200);
pair X = extension(D,B,A,r(B,A,(D_deg-C_deg)/2));
dot(X);

draw(Circle(origin,1));
draw(A--MP("X",X,SE,s));
draw(MP("A",A,dir(125),s)--MP("B",B,dir(40),s)--MP("C",C,dir(-20),s)--MP("D",D,dir(200),s)--A--C^^B--D);

\end{asy}
\end{center}





%------------------
%-- Message Achilleas ( moderator )
What do we notice about $X?$

%------------------
%-- Message RP3.1415 ( user )
% it lies on $DB$

%------------------
%-- Message MeepMurp5 ( user )
% it appears to lie on $BD$

%------------------
%-- Message coolbluealan ( user )
% it is on BD

%------------------
%-- Message Achilleas ( moderator )
Since triangles $ABC$ and $AXD$ are similar, $\angle ADX = \angle ACB.$  But $\angle ADB = \angle ACB.$  Therefore, $X$ lies on $BD.$

%------------------
%-- Message Achilleas ( moderator )
So what is $DX?$

%------------------
%-- Message Achilleas ( moderator )
(time for those ratios of corresponding side lengths)

%------------------
%-- Message B1002342 ( user )
% $DX = BC\cdot \frac{AD}{AC}$

%------------------
%-- Message Lucky0123 ( user )
% $DX = \frac{CB \cdot AD}{AC}$

%------------------
%-- Message dxs2016 ( user )
% DX=BC*AD/AC

%------------------
%-- Message Achilleas ( moderator )
$DX/BC = AD/AC,$ so $DX = BC \cdot  AD/AC.$

%------------------
%-- Message Achilleas ( moderator )
What else can we say about the diagram?

%------------------
%-- Message Achilleas ( moderator )
How about triangle $AXB$, for example?

%------------------
%-- Message Lucky0123 ( user )
% Its similar to triangle $ADC$

%------------------
%-- Message B1002342 ( user )
% AXB similar to ADC

%------------------
%-- Message myltbc10 ( user )
% it is similar to ADC

%------------------
%-- Message coolbluealan ( user )
% it is similar to $\triangle ADC$

%------------------
%-- Message Riya_Tapas ( user )
% Spiral similarity with center $A$, similar to $\triangle{ADC}$

%------------------
%-- Message Achilleas ( moderator )
From our observation above about spiral similarities, triangles $AXB$ and $ADC$ are also similar.

%------------------
%-- Message Achilleas ( moderator )
What is $BX?$

%------------------
%-- Message myltbc10 ( user )
% BX=CD*AB/AC

%------------------
%-- Message dxs2016 ( user )
% BX=DC*AB/AC

%------------------
%-- Message Achilleas ( moderator )
$BX/CD = AB/AC,$ so $BX = AB \cdot CD/AC.$

%------------------
%-- Message Achilleas ( moderator )
Then $BD = BX + DX = BC \cdot AD/AC + AB \cdot CD/AC.$

%------------------
%-- Message Achilleas ( moderator )
Next, what?

%------------------
%-- Message B1002342 ( user )
% multiply by AC

%------------------
%-- Message dxs2016 ( user )
% multiply both sides by AC

%------------------
%-- Message coolbluealan ( user )
% multiply both sides by AC

%------------------
%-- Message Gamingfreddy ( user )
% multiply both sides by AC

%------------------
%-- Message Lucky0123 ( user )
% $BD \cdot AC = AD \cdot BC + AB \cdot CD$

%------------------
%-- Message Achilleas ( moderator )
Therefore, $AC \cdot BD = AB \cdot CD + AD \cdot BC.$

%------------------
%-- Message Ezraft ( user )
% which is precisely Ptolemy's Theorem

%------------------
%-- Message Achilleas ( moderator )
% Yup! Awesome work everyone! 

%------------------
%-- Message Achilleas ( moderator )
% That's all for today!

%------------------
%-- Message Achilleas ( moderator )
% Thank you all! Have a wonderful week! See you next time! 

%------------------
%-- Message mark888 ( user )
% Thank you!

%------------------
%-- Message coolbluealan ( user )
% thanks!

%------------------
%-- Message Apollomindstorms ( user )
% Thanks, bye!

%------------------
%-- Message Ezraft ( user )
% thank you!

%------------------
%-- Message dxs2016 ( user )
% thanks!

%------------------
%-- Message Catherineyaya ( user )
% thank you!

%------------------
%-- Message Gamingfreddy ( user )
% Thank you!

%------------------
%-- Message laura.yingyue.zhang ( user )
% thank you!

%------------------
%-- Message Riya_Tapas ( user )
% Thank you for class!

%------------------
%-- Message christopherfu66 ( user )
% Thanks, bye!

%------------------
%-- Message Wangminqi1 ( user )
% Thank you!

%------------------
%-- Message Ezraft ( user )
% Thank you!

%------------------
%-- Message AOPS81619 ( user )
% thanks

%------------------
%-- Message B1002342 ( user )
% thank you !

%------------------
%-- Message smileapple ( user )
% thanks

%------------------
%-- Message MeepMurp5 ( user )
% Thank you!

%------------------
%-- Message mustwin_az ( user )
% thank you

%------------------
%-- Message JacobGallager1 ( user )
% Thank you!

%------------------
%-- Message MathJams ( user )
% thank you!

%------------------
%-- Message J4wbr34k3r ( user )
% TY!

%------------------
%-- Message RP3.1415 ( user )
% thanks for class

%------------------
%-- Message Achilleas ( moderator )
% Bye, everyone!

%------------------
