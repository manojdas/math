\documentclass[11pt,twoside]{scrartcl}
\usepackage{mdas}

\title{EGMO - 4.48}

\author{\TBD}
\date{\today}

\begin{document}
Given: Circle with center $O$ is tangent to $BC$ at $P$ and internally tangent to $\Gamma$ at $Q$.

\vspace{10pt}
Show that if $\angle{BAO} = \angle{CAO}$ then $\angle{PAO} = \angle{QAO}$.

\begin{center}
    \begin{asy}
        import cse5;
        import geometry;
        import olympiad;
        unitsize(6cm);

        pair A, B, C, P, Q, O, Ow, M, A1, R, S;
        path w, w1;
        // get this by trial and error for the center to be close to angle bisector
        real x = 0.42;

        Ow = (0,0);
        w = circle(Ow, 1);

        A = dir(145);
        B = dir(210);
        C = dir(330);

        // Center of arc
        M = dir (90);

        A1 = rotate((angle(C-A)-angle(B-A))*90/pi, A)*B;
        A1 = A + (A1 - A)*2;

        P = x*C + (1-x)*B;
        Q = P + (P - M);
        Q = intersectionpoint(w, Q--P);
        // O = (P + Q)/2;

        // Homothety scaling factor
        real r = distance(P,Q)/distance(M,Q);


        O = rotate(90,P)*B;
        O = P + (O - P)*r/distance(O, P);

        w1 = circle(O, r);

        R = O + (O - A);
        R = intersectionpoint(w, R--O);
        
        S = foot(Ow, B, C);

        dot(P);
        dot(Q);
        dot(O);
        dot(Ow, blue);

        draw(A--B--C--cycle);
        draw(w);
        draw(w1);

        draw(A--P^^A--O^^A--Q, black);

        // draw(A--A1, red, arrow=ArcArrow(SimpleHead));

        dot(M);
        dot(R);

        draw(M--Q, blue);
        draw(O--R, blue);
        draw(M--R, blue);

        draw(rightanglemark(Ow, S, P, 1), blue);

        label("$A$", A, N);
        label("$B$", B, SW);
        label("$C$", C, SE);

        label("$P$", P, N);
        label("$Q$", Q, S);
        label("$O$", O, SE);
        label("$R$", R, dir(R-O));
        label("$S$", S, NE);
        label("$M$", M, dir(90));

        label("$O_\Gamma$", Ow, SE);

        label(Label("$\Gamma$", Relative(0.2)), w);

    \end{asy}
\end{center}
\textbf{Hints}:
\begin{itemize}
    \item 220: Which quadrilateral is cyclic?
    \item 676: Complement Lemma 4.33 by extending $AO$ to meet $\Gamma$ again.
    \item 19: Use angle chasing to show that $APOQ$ is cyclic and we are done!
\end{itemize}
\end{document}