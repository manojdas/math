\documentclass[11pt,twoside]{scrartcl}
\usepackage{mdas}

\title{EGMO - 4.53}

\author{\TBD}
\date{\today}

\begin{document}


\begin{center}
    \begin{asy}
        import cse5;
        import geometry;
        import olympiad;
        unitsize(6cm);

        pair A, B, C, K, D, M, N, I, L, I_A, X, Y, Z;
        path w, w1, w2;
        real r;

        A = dir(110);
        B = dir(210);
        C = dir(330);

        w = incircle(A, B, C);
        I = incenter(A, B, C);

        K = tangent(B, I, inradius(A, B, C));
        D = foot(A, B, C);
        M = (A+D)/2;

        N = M + (M - K);
        N = intersectionpoint(w, M--N);
        w1 = circumcircle(B, C, N);

        w2 = excircle(B, C, A);
        I_A = excenter(B, C, A);
        r = exradius(B, C, A);

        X = intersectionpoint(B--C, w2);
        Y = intersectionpoint(K--I_A, w1);

        Z = (B+C)/2;


        dot(K);
        dot(I);
        dot(I_A);
        dot(M);
        dot(N);
        dot(X);
        dot(Y);

        draw(D--A--B--C--A);
        draw(w);
        draw(w1);
        draw(arc(I_A, r, 0, 180), blue);
        draw(K--N);
        draw(B--N--C);
        draw(rightanglemark(A, D, B, 1));
        draw(Y--Z, blue);
        
        draw(rightanglemark(K, Z, Y, 1), blue);

        draw(K--I_A, blue);
        draw(X--M, blue);


        draw(I_A--I, blue);
        draw(I_A--X^^I--K, blue);
        draw(B--I--C--I_A--B, blue);
        draw(rightanglemark(I_A, X, C, 1), blue);
        draw(rightanglemark(X, K, I, 1), blue);
        
        draw(rightanglemark(I, C, I_A, 1), blue);
        draw(rightanglemark(I_A, B, I, 1), blue);

        label("$A$", A, N);
        label("$B$", B, SW);
        label("$C$", C, SE);

        label("$D$", D, S);
        label("$K$", K, NW);
        label("$M$", M, NE);
        label("$M_A$", Z, SE);
        label("$N$", N, S);
        label("$I$", I, NE);
        label("$I_A$", I_A, S);
        label("$X$", X, NE);
        label("$T$", Y, NE);

        label(Label("$\Omega$", Relative(0.9)), w);
        label(Label("$(BCN)$", Relative(0.15)), w1);

    \end{asy}
\end{center}
Given: 
\begin{itemize}
    \item $\Omega$ incenter of acute triangle $ABC$ and is tangent to $BC$ at $K$
    \item $AD$ is an altitude of triangle $ABC$
    \item $M$ is the midpoint of $AD$
    \item $N$ is the common point of the circle $\Omega$ and $KM$
\end{itemize}
Prove $\Omega$ and $(BCN)$ are tangent to each other.

\textbf{Hints}:
\begin{itemize}
    \item 205: Which configurations come to mind?
    \item 634: Begin with Lemma 4.14 and 4.33.
    \item 450: Let ${\overline{KI_A}}$ meet the perpendicular bisector of $\overline{BC}$ at $T$. Show that $BNCT$ is cyclic.
    \item 177: Try using power of a point.
    \item 276: You can compute $KN$ using $I_AN \cdot I_AK = I_AI^2 - r^2$.
\end{itemize}
\end{document}